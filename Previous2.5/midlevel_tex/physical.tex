\section{Physical Data}
\label{s:physical}
\thispagestyle{plain}

\subsection{Data Arrays}
\label{s:dataarray}

\noindent
\textit{Node}: \texttt{DataArray\_t}

\begin{fctbox}
\textcolor{output}{\textit{ier}} = cg\_array\_write(\textcolor{input}{char *ArrayName}, \textcolor{input}{DataType\_t DataType}, & - w m \\
~~~~~~\textcolor{input}{int DataDimension}, \textcolor{input}{int *DimensionVector}, \textcolor{input}{void *Data}); & \\
\textcolor{output}{\textit{ier}} = cg\_narrays(\textcolor{output}{\textit{int *narrays}}); & r - m \\
\textcolor{output}{\textit{ier}} = cg\_array\_info(\textcolor{input}{int A}, \textcolor{output}{\textit{char *ArrayName}}, \textcolor{output}{\textit{DataType\_t *DataType}}, & r - m \\
~~~~~~\textcolor{output}{\textit{int *DataDimension}}, \textcolor{output}{\textit{int *DimensionVector}}); & \\
\textcolor{output}{\textit{ier}} = cg\_array\_read(\textcolor{input}{int A}, \textcolor{output}{\textit{void *Data}}); & r - m \\
\textcolor{output}{\textit{ier}} = cg\_array\_read\_as(\textcolor{input}{int A}, \textcolor{input}{DataType\_t DataType}, \textcolor{output}{\textit{void *Data}}); & r - m \\
\hline
call cg\_array\_write\_f(\textcolor{input}{ArrayName}, \textcolor{input}{DataType}, \textcolor{input}{DataDimension}, & - w m \\
~~~~~\textcolor{input}{DimensionVector}, \textcolor{input}{Data}, \textcolor{output}{\textit{ier}}) & \\
call cg\_narrays\_f(\textcolor{output}{\textit{narrays}}, \textcolor{output}{\textit{ier}}) & r - m \\
call cg\_array\_info\_f(\textcolor{input}{A}, \textcolor{output}{\textit{ArrayName}}, \textcolor{output}{\textit{\textit{DataType}}}, \textcolor{output}{\textit{DataDimension}}, & r - m \\
~~~~~\textcolor{output}{\textit{DimensionVector}}, \textcolor{output}{\textit{ier}}) & \\
call cg\_array\_read\_f(\textcolor{input}{A}, \textcolor{output}{\textit{Data}}, \textcolor{output}{\textit{ier}}) & r - m \\
call cg\_array\_read\_as\_f(\textcolor{input}{A}, \textcolor{input}{DataType}, \textcolor{output}{\textit{Data}}, \textcolor{output}{\textit{ier}}) & r - m \\
\end{fctbox}

\noindent
\textbf{\textcolor{input}{Input}/\textcolor{output}{\textit{Output}}}

\begin{Ventryi}{\texttt{DimensionVector}}\raggedright
\item [\texttt{narrays}]
      Number of \texttt{DataArray\_t} nodes under the current node.
      (\textcolor{output}{\textit{Output}})
\item [\texttt{A}]
      Data array index, where $1 \leq \text{\texttt{A}} \leq \text{\texttt{narrays}}$.
      (\textcolor{input}{Input})
\item [\texttt{ArrayName}]
      Name of the \texttt{DataArray\_t} node.
      (\textcolor{input}{Input} for \texttt{cg\_array\_write};
      \textcolor{output}{\textit{output}} for \texttt{cg\_array\_info})
\item [\texttt{DataType}]
      Type of data held in the \texttt{DataArray\_t} node.
      The admissible types are \texttt{Integer}, \texttt{RealSingle},
      \texttt{RealDouble}, and \texttt{Character}.
      (\textcolor{input}{Input} for \texttt{cg\_array\_write},
      \texttt{cg\_array\_read\_as}; \textcolor{output}{\textit{output}} for
      \texttt{cg\_array\_info})
\item [\texttt{DataDimension}]
      Number of dimensions.
      (\textcolor{input}{Input} for \texttt{cg\_array\_write};
      \textcolor{output}{\textit{output}} for \texttt{cg\_array\_info})
\item [\texttt{DimensionVector}]
      Number of data elements in each dimension.
      (\textcolor{input}{Input} for \texttt{cg\_array\_write};
      \textcolor{output}{\textit{output}} for \texttt{cg\_array\_info})
\item [\texttt{Data}]
      The data array.
      (\textcolor{input}{Input} for \texttt{cg\_array\_write};
      \textcolor{output}{\textit{output}} for \texttt{cg\_array\_read},
      \texttt{cg\_array\_read\_as})
\item [\texttt{ier}]
      Error status.
      (\textcolor{output}{\textit{Output}})
\end{Ventryi}

\newpage
\subsection{Data Class}
\label{s:dataclass}

\noindent
\textit{Node}: \texttt{DataClass\_t}

\begin{fctbox}
\textcolor{output}{\textit{ier}} = cg\_dataclass\_write(\textcolor{input}{DataClass\_t dataclass}); & - w m \\
\textcolor{output}{\textit{ier}} = cg\_dataclass\_read(\textcolor{output}{\textit{DataClass\_t *dataclass}}); & r - m \\
\hline
call cg\_dataclass\_write\_f(\textcolor{input}{dataclass}, \textcolor{output}{\textit{ier}}) & - w m \\
call cg\_dataclass\_read\_f(\textcolor{output}{\textit{dataclass}}, \textcolor{output}{\textit{ier}}) & r - m \\
\end{fctbox}

\noindent
\textbf{\textcolor{input}{Input}/\textcolor{output}{\textit{Output}}}

\begin{Ventryi}{\texttt{dataclass}}\raggedright
\item [\texttt{dataclass}]
      Data class for the nodes at this level.
      See below for the data classes currently supported in CGNS.
      (\textcolor{input}{Input} for \texttt{cg\_dataclass\_write};
      \textcolor{output}{\textit{output}} for \texttt{cg\_dataclass\_read})
\item [\texttt{ier}]
      Error status.
      (\textcolor{output}{\textit{Output}})
\end{Ventryi}
The data classes currently supported in CGNS are:

\begin{Ventryi}{\texttt{NormalizedByUnknownDimensional}}\raggedright
\item [\texttt{Dimensional}]
      Regular dimensional data.
\item [\texttt{NormalizedByDimensional}]
      Nondimensional data that is normalized by dimensional reference
      quantities.
\item [\texttt{NormalizedByUnknownDimensional}]
      All fields and reference data are nondimensional.
\item [\texttt{NondimensionalParameter}]
      Nondimensional parameters such as Mach number and lift coefficient.
\item [\texttt{DimensionlessConstant}]
      Constant such as $\pi$.
\end{Ventryi}
These classes are declared within typedef \texttt{DataClass\_t} in
\textit{cgnslib.h}, and as parameters in \textit{cgnslib\_f.h}.

\subsection{Data Conversion Factors}
\label{s:dataconversion}

\noindent
\textit{Node}: \texttt{DataConversion\_t}

\begin{fctbox}
\textcolor{output}{\textit{ier}} = cg\_conversion\_write(\textcolor{input}{DataType\_t DataType}, & - w m \\
~~~~~~\textcolor{input}{void *ConversionFactors}); & \\
\textcolor{output}{\textit{ier}} = cg\_conversion\_info(\textcolor{output}{\textit{DataType\_t *DataType}}); & r - m \\
\textcolor{output}{\textit{ier}} = cg\_conversion\_read(\textcolor{output}{\textit{void *ConversionFactors}}); & r - m \\
\hline
call cg\_conversion\_write\_f(\textcolor{input}{DataType}, \textcolor{input}{ConversionFactors}, \textcolor{output}{\textit{ier}}) & - w m \\
call cg\_conversion\_info\_f(\textcolor{output}{\textit{DataType}}, \textcolor{output}{\textit{ier}}) & r - m \\
call cg\_conversion\_read\_f(\textcolor{output}{\textit{ConversionFactors}}, \textcolor{output}{\textit{ier}}) & r - m \\
\end{fctbox}

\noindent
\textbf{\textcolor{input}{Input}/\textcolor{output}{\textit{Output}}}

\begin{Ventryi}{\texttt{ConversionOffset}}\raggedright
\item [\texttt{DataType}]
      Data type in which the conversion factors are recorded.
      Admissible data types for conversion factors are \texttt{RealSingle}
      and \texttt{RealDouble}.
      (\textcolor{input}{Input} for \texttt{cg\_conversion\_write};
      \textcolor{output}{\textit{output}} for \texttt{cg\_conversion\_info})
\item [\texttt{ConversionFactors}]
      Two-element array containing the scaling and offset factors.
      (\textcolor{input}{Input} for \texttt{cg\_conversion\_write};
      \textcolor{output}{\textit{output}} for \texttt{cg\_conversion\_read})
\item [\texttt{ier}]
      Error status.
      (\textcolor{output}{\textit{Output}})
\end{Ventryi}

The \texttt{DataConversion\_t} data structure contains factors to convert
the nondimensional data to ``raw'' dimensional data.
The scaling and offset factors are contained in the two-element array
\texttt{ConversionFactors}.
In pseudo-Fortran, the conversion process is as follows:

\begin{indlefttt}
ConversionScale  = ConversionFactors(1)
ConversionOffset = ConversionFactors(2)
Data(\textit{raw}) = Data(\textit{nondimensional})*ConversionScale + ConversionOffset
\end{indlefttt}

\subsection{Dimensional Units}
\label{s:dimensionalunits}

\noindent
\textit{Node}: \texttt{DimensionalUnits\_t}

\begin{fctbox}
\textcolor{output}{\textit{ier}} = cg\_units\_write(\textcolor{input}{MassUnits\_t mass}, \textcolor{input}{LengthUnits\_t length}, & - w m \\
~~~~~~\textcolor{input}{TimeUnits\_t time}, \textcolor{input}{TemperatureUnits\_t temperature}, & \\
~~~~~~\textcolor{input}{AngleUnits\_t angle}); & \\
\textcolor{output}{\textit{ier}} = cg\_unitsfull\_write(\textcolor{input}{MassUnits\_t mass}, \textcolor{input}{LengthUnits\_t length}, & - w m \\
~~~~~~\textcolor{input}{TimeUnits\_t time}, \textcolor{input}{TemperatureUnits\_t temperature}, & \\
~~~~~~\textcolor{input}{AngleUnits\_t angle}, \textcolor{input}{ElectricCurrentUnits\_t current}, & \\
~~~~~~\textcolor{input}{SubstanceAmountUnits\_t amount}, & \\
~~~~~~\textcolor{input}{LuminousIntensityUnits\_t intensity}); & \\
\textcolor{output}{\textit{ier}} = cg\_nunits(\textcolor{output}{\textit{int *nunits}}); & r - m \\
\textcolor{output}{\textit{ier}} = cg\_units\_read(\textcolor{output}{\textit{MassUnits\_t *mass}}, \textcolor{output}{\textit{LengthUnits\_t *length}}, & r - m \\
~~~~~~\textcolor{output}{\textit{TimeUnits\_t *time}}, \textcolor{output}{\textit{TemperatureUnits\_t *temperature}}, & \\
~~~~~~\textcolor{output}{\textit{AngleUnits\_t *angle}}); & \\
\textcolor{output}{\textit{ier}} = cg\_unitsfull\_read(\textcolor{output}{\textit{MassUnits\_t *mass}}, \textcolor{output}{\textit{LengthUnits\_t *length}}, & r - m \\
~~~~~~\textcolor{output}{\textit{TimeUnits\_t *time}}, \textcolor{output}{\textit{TemperatureUnits\_t *temperature}}, & \\
~~~~~~\textcolor{output}{\textit{AngleUnits\_t *angle}}, \textcolor{output}{\textit{ElectricCurrentUnits\_t *current}}, & \\
~~~~~~\textcolor{output}{\textit{SubstanceAmountUnits\_t *amount}}, & \\
~~~~~~\textcolor{output}{\textit{LuminousIntensityUnits\_t *intensity}}); & \\
\hline
call cg\_units\_write\_f(\textcolor{input}{mass}, \textcolor{input}{length}, \textcolor{input}{time}, \textcolor{input}{temperature}, \textcolor{input}{angle}, \textcolor{output}{\textit{ier}}) & - w m \\
call cg\_unitsfull\_write\_f(\textcolor{input}{mass}, \textcolor{input}{length}, \textcolor{input}{time}, \textcolor{input}{temperature}, \textcolor{input}{angle}, & - w m \\
~~~~~\textcolor{input}{current}, \textcolor{input}{amount}, \textcolor{input}{intensity}, \textcolor{output}{\textit{ier}}) & \\
call cg\_nunits\_f(\textcolor{output}{\textit{int *nunits}}) & r - m \\
call cg\_units\_read(\textcolor{output}{\textit{mass}}, \textcolor{output}{\textit{length}}, \textcolor{output}{\textit{time}}, \textcolor{output}{\textit{temperature}}, \textcolor{output}{\textit{angle}}, \textcolor{output}{\textit{ier}}) & r - m \\
call cg\_unitsfull\_read\_f(\textcolor{output}{\textit{mass}}, \textcolor{output}{\textit{length}}, \textcolor{output}{\textit{time}}, \textcolor{output}{\textit{temperature}}, \textcolor{output}{\textit{angle}}, & r - m \\
~~~~~\textcolor{output}{\textit{current}}, \textcolor{output}{\textit{amount}}, \textcolor{output}{\textit{intensity}}, \textcolor{output}{\textit{ier}}) & \\
\end{fctbox}

\noindent
\textbf{\textcolor{input}{Input}/\textcolor{output}{\textit{Output}}}

\begin{Ventryi}{\texttt{temperature}}\raggedright
\item [\texttt{mass}]
      Mass units.
      Admissible values are \texttt{Null}, \texttt{UserDefined},
      \texttt{Kilogram}, \texttt{Gram}, \texttt{Slug}, and
      \texttt{PoundMass}.
      (\textcolor{input}{Input} for \texttt{cg\_units\_write},
      \texttt{cg\_unitsfull\_write};
      \textcolor{output}{\textit{output}} for \texttt{cg\_units\_read},
      \texttt{cg\_unitsfull\_read})
\item [\texttt{length}]
      Length units.
      Admissible values are \texttt{Null}, \texttt{UserDefined},
      \texttt{Meter}, \texttt{Centimeter}, \texttt{Millimeter},
      \texttt{Foot}, and \texttt{Inch}.
      (\textcolor{input}{Input} for \texttt{cg\_units\_write},
      \texttt{cg\_unitsfull\_write};
      \textcolor{output}{\textit{output}} for \texttt{cg\_units\_read},
      \texttt{cg\_unitsfull\_read})
\item [\texttt{time}]
      Time units.
      Admissible values are \texttt{Null}, \texttt{UserDefined}, and
      \texttt{Second}.
      (\textcolor{input}{Input} for \texttt{cg\_units\_write},
      \texttt{cg\_unitsfull\_write};
      \textcolor{output}{\textit{output}} for \texttt{cg\_units\_read},
      \texttt{cg\_unitsfull\_read})
\item [\texttt{temperature}]
      Temperature units.
      Admissible values are \texttt{Null}, \texttt{UserDefined},
      \texttt{Kelvin}, \texttt{Celsius}, \texttt{Rankine}, and
      \texttt{Fahrenheit}.
      (\textcolor{input}{Input} for \texttt{cg\_units\_write},
      \texttt{cg\_unitsfull\_write};
      \textcolor{output}{\textit{output}} for \texttt{cg\_units\_read},
      \texttt{cg\_unitsfull\_read})
\item [\texttt{angle}]
      Angle units.
      Admissible values are \texttt{Null}, \texttt{UserDefined},
      \texttt{Degree}, and \texttt{Radian}.
      (\textcolor{input}{Input} for \texttt{cg\_units\_write},
      \texttt{cg\_unitsfull\_write};
      \textcolor{output}{\textit{output}} for \texttt{cg\_units\_read},
      \texttt{cg\_unitsfull\_read})
\item [\texttt{current}]
      Electric current units.
      Admissible values are \texttt{Null}, \texttt{UserDefined},
      \texttt{Ampere}, \texttt{Abampere}, \texttt{Statampere},
      \texttt{Edison}, and \texttt{auCurrent}.
      (\textcolor{input}{Input} for \texttt{cg\_unitsfull\_write};
      \textcolor{output}{\textit{output}} for \texttt{cg\_unitsfull\_read})
\item [\texttt{amount}]
      Substance amount units.
      Admissible values are \texttt{Null}, \texttt{UserDefined},
      \texttt{Mole}, \texttt{Entities}, \texttt{StandardCubicFoot}, and
      \texttt{StandardCubicMeter}.
      (\textcolor{input}{Input} for \texttt{cg\_unitsfull\_write};
      \textcolor{output}{\textit{output}} for \texttt{cg\_unitsfull\_read})
\item [\texttt{intensity}]
      Luminous intensity units.
      Admissible values are \texttt{Null}, \texttt{UserDefined},
      \texttt{Candela}, \texttt{Candle}, \texttt{Carcel},
      \texttt{Hefner}, and \texttt{Violle}.
      (\textcolor{input}{Input} for \texttt{cg\_unitsfull\_write};
      \textcolor{output}{\textit{output}} for \texttt{cg\_unitsfull\_read})
\item [\texttt{nunits}]
      Number of units used in the file (i.e., either 5 or 8).
      (\textcolor{output}{\textit{Output}})
\item [\texttt{ier}]
      Error status.
      (\textcolor{output}{\textit{Output}})
\end{Ventryi}

The supported units are declared within typedefs in \textit{cgnslib.h}
and as parameters in \textit{cgnslib\_f.h}.

When reading units data, either \texttt{cg\_units\_read} or
\texttt{cg\_unitsfull\_read} may be used, regardless of the number of
units used in the file.
If \texttt{cg\_unitsfull\_read} is used, but only five units are used
in the file, the returned values of \texttt{current}, \texttt{amount},
and \texttt{intensity} will be \texttt{Null}.

\newpage
\subsection{Dimensional Exponents}
\label{s:dimensionalexponents}

\noindent
\textit{Node}: \texttt{DimensionalExponents\_t}

\begin{fctbox}
\textcolor{output}{\textit{ier}} = cg\_exponents\_write(\textcolor{input}{DataType\_t DataType}, \textcolor{input}{void *exponents}); & - w m \\
\textcolor{output}{\textit{ier}} = cg\_expfull\_write(\textcolor{input}{DataType\_t DataType}, \textcolor{input}{void *exponents}); & - w m \\
\textcolor{output}{\textit{ier}} = cg\_nexponents(\textcolor{output}{\textit{int *nexponents}}); & r - m \\
\textcolor{output}{\textit{ier}} = cg\_exponents\_info(\textcolor{output}{\textit{DataType\_t *DataType}}); & r - m \\
\textcolor{output}{\textit{ier}} = cg\_exponents\_read(\textcolor{output}{\textit{void *exponents}}); & r - m \\
\textcolor{output}{\textit{ier}} = cg\_expfull\_read(\textcolor{output}{\textit{void *exponents}}); & r - m \\
\hline
call cg\_exponents\_write\_f(\textcolor{input}{DataType}, \textcolor{input}{exponents}, \textcolor{output}{\textit{ier}}) & - w m \\
call cg\_expfull\_write\_f(\textcolor{input}{DataType}, \textcolor{input}{exponents}, \textcolor{output}{\textit{ier}}) & - w m \\
call cg\_nexponents\_f(\textcolor{output}{\textit{nexponents}}, \textcolor{output}{\textit{ier}} ) & r - m \\
call cg\_exponents\_info\_f(\textcolor{output}{\textit{DataType}}, \textcolor{output}{\textit{ier}} ) & r - m \\
call cg\_exponents\_read\_f(\textcolor{output}{\textit{exponents}}, \textcolor{output}{\textit{ier}}) & r - m \\
call cg\_expfull\_read\_f(\textcolor{output}{\textit{exponents}}, \textcolor{output}{\textit{ier}}) & r - m \\
\end{fctbox}

\noindent
\textbf{\textcolor{input}{Input}/\textcolor{output}{\textit{Output}}}

\begin{Ventryi}{\texttt{exponents}}\raggedright
\item [\texttt{DataType}]
      Data type in which the exponents are recorded.
      Admissible data types for the exponents are \texttt{RealSingle}
      and \texttt{RealDouble}.
      (\textcolor{input}{Input} for \texttt{cg\_exponents\_write};
      \textcolor{output}{\textit{output}} for \texttt{cg\_exponents\_info})
\item [\texttt{exponents}]
      Exponents for the dimensional units for mass, length, time,
      temperature, angle, electric current, substance amount, and
      luminous intensity, in that order.
      (\textcolor{input}{Input} for \texttt{cg\_exponents\_write},
      \texttt{cg\_expfull\_write};
      \textcolor{output}{\textit{output}} for \texttt{cg\_exponents\_read},
      \texttt{cg\_expfull\_read})
\item [\texttt{nexponents}]
      Number of exponents used in the file (i.e., either 5 or 8).
      (\textcolor{output}{\textit{Output}})
\item [\texttt{ier}]
      Error status.
      (\textcolor{output}{\textit{Output}})
\end{Ventryi}

When reading exponent data, either \texttt{cg\_exponents\_read} or
\texttt{cg\_expfull\_read} may be used, regardless of the number of
exponents used in the file.
If \texttt{cg\_exponents\_read} is used, but all eight exponents are used
in the file, only the first five exponents are returned.
If \texttt{cg\_expfull\_read} is used, but only five exponents are used
in the file, the returned values of the exponents for electric current,
substance amount, and luminous intensity will be zero.
