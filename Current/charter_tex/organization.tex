\section{Organization/Bylaws}
\label{s:organization}
\thispagestyle{plain}

The CGNS Steering Committee is a voluntary organization that will
determine its own policies and internal structure, and will govern by
consensus whenever possible.
In the absence of consensus, a two-thirds majority of the Committee
members will be required to adopt changes to the standard, alter this
Charter, or take other official actions.

The CGNS Steering Committee will meet at a minimum of one time per year.
The time and location will be determined by consensus of the Committee,
and all members of the Committee will be notified in advance.

The members of the CGNS Steering Committee will appoint a Chairperson
whose responsibilities will include coordinating activities,
facilitating meetings and serving as a focal point for the Committee.
The Chairperson will be a member of the Committee, be elected by
consensus, and serve for a two-year term.
There is no limit on how many terms the Chairperson can be elected.
At the discretion of the Chairperson, a Vice-Chairperson may be
appointed by consensus of the Committee, to assist the Chairperson with
his or her duties.
The Vice-Chairperson will also be a member of the Committee.
The appointment of a secretary to maintain records will be at the
discretion of the Chairperson.

The CGNS Steering Committee may decide to suggest appropriate
contributions from its members.
The Steering Committee is not prohibited from charging membership fees;
the decision whether to do so, and the amount of the fees, lies within
the purview of the Steering Committee.

All parties are welcome to bring forward issues and participate in
development of the CGNS Standard, whether or not they are members of the
Steering Committee.

The decision whether to support the migration of the CGNS standard to
ISO/STEP, or any other organization, lies within the purview of the
Steering Committee.

\subsection{Representation}
\label{s:representation}

The CGNS Steering Committee will be made up of representatives from
specific institutions, rather than individuals.
Changes or additions to Steering Committee membership will be based on
potential contribution to the standard.
Membership on the Steering Committee will be limited to 30 institutions
that actively participate in the development, maintenance, distribution
and use of the CGNS Standard.
No more than 5 institutions may be related, i.e., have the same parent
organization.
Changes to the Membership (including the limit on the number of
institutions) will be determined by consensus, or if required, a
two-thirds majority of the existing Membership.

To help satisfy the duties of the Steering Committee as a whole, as
described in \autoref{s:mission}, the minimal responsibilities of each
individual Steering Committee member are to:
\begin{itemize*}
\item Attend as many telecons/meetings as possible, but not less than
      one per year
\item Read and send comments on proposals or other issues when asked to
      do so
\item Vote when asked to do so
\end{itemize*}
More active participation --- including support, software development,
and actively working to improve and promote CGNS --- is encouraged.

\newpage
The Steering Committee members as of 14 April 2011 are:

\begin{itemize*}
\item ADAPCO
\item ANSYS-CFX
\item ANSYS-ICEM CFD
\item Aerospatiale Matra -- Airbus
\item Boeing Commercial
\item Computational Engineering Solutions
\item Concepts NREC
\item Intelligent Light
\item NASA Glenn
\item NASA Langley
\item ONERA
\item Pointwise, Inc.
\item Pratt \& Whitney
\item Rolls-Royce / Allison
\item Stanford University
\item Stony Brook University
\item Tecplot, Inc.
\item TTC Technologies
\item University of Colorado
\item U. S. Air Force / AEDC
\end{itemize*}

\subsection{Standing Committees}
\label{s:committees}

The CGNS Steering Committee may constitute Standing Committees, in
an ongoing or temporary basis, to which it may delegate various
responsibilities.
The Standing Committees will report and make recommendations to the
Steering Committee who will retain the authority to act and make final
decisions.

\subsection{Software and Documentation Support Team}
\label{s:supportteam}

The CGNS Steering Committee will be responsible for selecting one or
more organizations to maintain and distribute existing documentation
and software, to develop and distribute new software resulting from
extensions to the standard, and to post or distribute meeting minutes
and other new documentation.

The organization(s) selected to maintain CGNS software will determine
the form of newly developed software and maintain compatibility with the
existing ADF Core and CGNS API.

The organization(s) selected to maintain CGNS Documentation will
be responsible for posting and maintaining on the web the Steering
Committee meeting minutes, Charter, and archive information.
