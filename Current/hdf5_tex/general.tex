%\section{SLL Node interface and General CGNS File Mapping Concepts}
\section{General CGNS \SLL Mapping Concepts}
\label{s:general}
\thispagestyle{plain}

This section describes the general philosophy underlying the use of
the \HDF tree structure by CGNS. \autoref{s:nodes} describes the exact
conventions for each type of data.

In \autoref{s:adfnodes}, we first describe the roles of the various
\HDF elements (i.e. \emph{groups} or \emph{attributes}) as they
are specifically applied within CGNS. \autoref{s:internal} describes
the overall layout of the tree structure itself.

\subsection{Use of \HDF elements in CGNS}
\label{s:adfnodes}

\autoref{s:adfnodestructure} described the general role of each of
the \HDF elements without reference to CFD.  Here we note any
additional information regarding their use within CGNS.

Attributes described in \ref*{s:nodeid} through \ref*{s:childtable} are
those recognized by both \HDF and CGNS. In \ref*{s:cardinality} through
\ref*{s:functions} we describe certain elements which are
derived from context, i.e., which the node possesses by virtue of its
location within a CGNS database. These notions, namely, Cardinality,
Parameters and Functions, are unique to CGNS.

\subsubsection{The Node ID}
\label{s:nodeid}

The Node ID is completely controlled by \HDF, and thus its role is
exactly the same for CGNS as it is for \HDF. CGNS software accesses the
Node ID only through calls to \HDF. \HDF itself guarantees that Node IDs
are unique and constant within any \HDF file (or collection of files)
while the file(s) are open.

% No user application should use the Node ID for its own purpose.

\subsubsection{The Node Name}
\label{s:nodename}

In CGNS, the Name may be left to the choice of the user, or it may be
specified by the SIDS. At the levels of the tree nearest the root,
the (end-)user is free to set the Name to distinguish among like
objects in the case at hand. For example, in a multizone problem, nodes
associated with different zones might be named ``\fort{UnderLeftWing}''
or ``\fort{AboveForwardFuselage}''. At this level, it is generally
not possible to identify a collection of names which are likely to
recur. This means that the naming of high level objects does not
require standardization, and the SIDS are silent regarding the naming
convention.

Because every \HDF node must be given a name when it is opened, default
names, based on the node Label, are provided by convention. The CGNS
Midlevel Library will record the default names if none is provided by
the user. The precise formula is given in the Label section below.

At levels of the tree farther from the root, the SIDS often specify
the name. There is, for example, a commonly encountered collection of
flow variables whose general meaning is widely understood. In this
case, standardizing the name conveys precise information. Thus the SIDS
specify, for instance, that a node containing static internal energy
per unit mass should have the Name ``\fort{EnergyInternal}''. Adherence
to these naming conventions guarantees uniform meaning of the data from
site to site and user to user. Of course, there may be a desire to
store quantities for which no naming convention is specified. In this
case any suitable name can be used, but there is no guarantee of proper
interpretation by anyone unaware of the choice.

By extension, a node name is a series of names separated by a
\emph{slash} '\texttt{/}' (like the POSIX file system names),
moreover '\texttt{/}' is the root name of the CGNS tree.

A CGNS Name can contain any printable ASCII character except the 
\emph{slash} '\texttt{/}' and the \emph{dot} '\texttt{.}' when
this \emph{dot} is the first character of the name.

\subsubsection{The Label}

Within CGNS, nearly all labels reflect C-style type definitions
(``typedefs'') specified by the SIDS, and end in the characters
``\fort{\_t}''. Some ``Leaf'' nodes (i.e. nodes that have no children)
do not represent higher level CGNS structures and therefore have labels
that do not follow the ``\fort{\_t}'' convention. At this writing, all
such nodes have the type \fort{int[]}, i.e., integer array, a type
already recognized in C, for which a separate type definition would be
artificial. Such nodes are generally located by the software through
their names, which are specified by the SIDS, rather than through their
labels.

The Label generally indicates the role of the data at and below the
node in the context of CFD. Nodes which are entry points to data for a
particular zone, for example, have the Label ``\fort{Zone\_t}''.

Parent nodes often have a number of children each containing data for
different instances of the same structure. Multiple children of the same
parent therefore often have the same Label. It is customary for software
to conduct searches which depend on the Label, e.g., to determine the
number of zones in a problem. The software will fail if the conventions
regarding Labels are not observed.

Labels are also used to build default node Names. These are derived from
the Label by dropping the characters ``\fort{\_t}'' and substituting the
smallest positive integer resulting in a unique name among children
of the same parent. For example, the first default Name for a node
of type \fort{Zone\_t} will be ``\fort{Zone1}''; the second will be
``\fort{Zone2}''; and so on.

\subsubsection{The Data Type}

Data Types are completely specified by the file mapping. Although \HDF
provides a number of types, in CGNS the only types used are \fort{MT}
(No Data), \fort{I4} (Integer), \fort{R4} and \fort{R8} (Real),
\fort{C1} (Character), and \fort{LK} (Link).

% IS THIS TRUE ??? IS IT HDF or SLL HERE ?
%
The specification of data types within the File Mapping allows for the
probability that files written under different circumstances may differ
in precision. The issue is complicated by the ability of \HDF to sense
the capabilities of the platform on which it is running and interpret or
record data accordingly. The general rule is that, although the user of
\HDF can specify the precision in which it is desired to read or write
the data, \HDF knows both the precision available at the source and the
precision acceptable to the destination and will mitigate accordingly.
Thus to specify the precison of real data as \fort{R4}, for example, has
no meaning unless both \fort{R4} and \fort{R8} are available. Therefore,
the generic specification ``\fort{DataType}'' is used to allow for all
possibilities.

For all integer data specified by the SIDS, \fort{I4} provides
sufficient precision. 

% === not sure to understand this sentence...
%Since ADF software handles conversion to the
%external environment, all integer data is specified by the File Mapping
%as \fort{I4}.

\subsubsection{The Number of Dimensions}

Whenever data is stored at a node, it is in the form of a single array
of elements of a single date type. Insofar as possible, the dimension
specified by CGNS is the natural underlying dimension; for example, a
rectangular array of pressures is stored with dimension equal to the
physical dimension of the problem.

There are situations in which this representation is not feasible. For
instance, a list of points which do not form a rectangular array in
physical space may be compacted into a one-dimensional array in \HDF.

Frequently the data is of type \fort{C1} (character data). In some
cases, the data holds additional information in the form of a name
specified by the SIDS, and in some cases holds user comment. All such
data is generally represented as a one-dimensional array (or list) of
characters.

\subsubsection{The Dimension Values}

These are used exactly as specified by \HDF. In the case of rectangular
arrays of physical data, the dimension values are set to the actual
sizes in physical space. Note that these sizes often depend on whether
the values are associated with grid nodes, cell centers or other
physical locations with respect to the grid. In any event, they refer to
the amount of data actually stored, not to any larger array from which
it may have been extracted.

In the case of list data, the dimension value is the length of the
list. Lists of characters may contain termination bytes such as
``\fort{\textbackslash n}''; by this means an entire document can be
stored in the data field.

\subsubsection{The Data}

CGNS imposes no conventions on the data itself beyond those specified
by \HDF. Note that it is a responsibility of the CGNS software to ensure
that the amount and type of stored data agrees with the specification of
the data type, number of dimensions, and dimension values.

\subsubsection{The Child Table}
\label{s:childtable}

The Child Table is completely controlled by \HDF, and thus its role is
exactly the same for CGNS as it is for \HDF. CGNS software accesses and
modifies the child table only through calls to \HDF.

In addition to the meaning of attributes of individual \HDF nodes,
the File Mapping specifies the relations between nodes in a CGNS
database. Consequently, the File Mapping determines what kinds of nodes
will lie in the child table.

% **** AH AH !!! ***********
%
It is important to reemphasize that \HDF provides no notion of order
among children. This means children can be identified only by their
names, labels and system-provided node IDs. In particular, the order
of a list of children returned by \HDF has nothing to do with the order
in which they were inserted in the file. However, the order returned is
consistent from call to call provided the file has not been closed and
the node structure has not been modifed.
%
% **** End of HA HA !!! ****

\subsubsection{Cardinality}
\label{s:cardinality}

The \emph{cardinality} of a CGNS node is the number of nodes of the same
label permitted at one point in the tree, i.e., as children of the same
parent. It consists of both lower and upper limits.

Since the notion of a CGNS database allows for work in progress, the
lower limit is generally zero (although the database may be of little
use until certain nodes are filled). The upper limit is usually either
one or many ($N$).

\subsubsection{Parameters}

CGNS relies on the fact that \SLL nodes cannot be found except by
following the pointers from their parents.  This means that software
accessing a node has had an opportunity to note all the data above that
node in the tree. Therefore, nodes do not repeat within themselves
information which is necessary for their interpretation but which is
available at a higher level.

A datum which is necessary for the proper interpretation of a
node but which is derived from its ancestors is referred to as a
\textit{structure parameter}.

\subsubsection{Functions}
\label{s:functions}

Occasionally the proper interpretation of a node depends on
an implicitly understood \emph{function} of its structure
parameters. Usually these relate to the actual amount of data
stored. Several of these functions are defined in the SIDS and
referenced in this document.

\subsection{CGNS Databases}
\label{s:cgns_databases}

\subsubsection{Definition of a CGNS Database}

By definition, a CGNS database is created when, within an \HDF file, a
node is created which conforms to the specifications given below for a
node of type ``\fort{CGNSBase\_t}''. This node is conceptually the root
of the CGNS database. Because it is created and controlled by the user,
it cannot be the root of the \HDF file.
Current CGNS conventions require that it be located directly below
the \HDF root node.

Further, by the mechanism of links, a CGNS database may span multiple
files. Thus there is no notion of a CGNS \emph{file}, only of a CGNS
\emph{database} implemented within one or more \emph{\HDF files}.

By virtue of its intended use, a CGNS database is dynamic in
that its content at any time reflects the current state of a CFD
problem of interest. For example, after the completion of a grid
generation procedure, a CGNS file may contain a grid but no boundary
conditions. Therefore, beyond the occurrence of a \fort{CGNSBase\_t}
node, there is no minimum content required in a CGNS database.

Conversely, there is no proscription against the inclusion, anywhere
within an \HDF file containing a CGNS database, of nodes of any form
whatsoever, provided only that their naming and labeling does not mimic
CGNS conventions. Such ``non-CGNS'' nodes, and those below them in the \HDF
tree, are not regarded as part of the CGNS database. CGNS software will
not detect the existence of non-CGNS nodes.

We may therefore take the following as a definition of a CGNS database:
\begin{quote}\itshape%
A CGNS database is a subtree of an \HDF file or files which is rooted at
a node with label ``\fort{CGNSBase\_t}'' and which conforms to the SIDS data
model as implemented by the SIDS-to-HDF File Mapping.
\end{quote}

\subsubsection{Location of CGNS Databases within \HDF Files}

An \HDF file may contain more than one \fort{CGNSBase\_t} node; i.e.,
there may be more than one CGNS database rooted within the same \HDF
file.
CGNS software accepts the \emph{name} of the desired database as an
argument, and will locate the correct \fort{CGNSBase\_t} node within the
specified \HDF file.
Obviously, each \fort{CGNSBase\_t} node in a single \HDF file must have a
unique name.

A CGNS database may link to CGNS nodes in the same or other \HDF
files. Thus, for example, a CGNS database may reference the grid from
another CGNS database without physically copying the the information.
In this case, the structure of the \HDF file into which the link is made
is invisible except below the node to which the link is made.

\subsubsection{File Management}

Beyond \textit{Open} and \textit{Close} neither \HDF nor CGNS provides
any file management facilities. The user is responsible for ensuring
that:

\begin{itemize}
\item The \HDF file containing the root of the required database is
      available and its permissions are properly set at runtime.
\item If links are made to other \HDF files, including any not under the
      user's direct control, these are also available at runtime.
\item No file is opened for writing by more than one program at a time.
\end{itemize}

It is possible, within CGNS, to protect files from inadvertent writing
by opening them as ``read only''.

\subsection{Internal Organization of a CGNS Database}
\label{s:internal}

\subsubsection{The \texttt{CGNSBase\_t} Node}

At the highest level of the tree defining a CGNS database there is
always a node labeled ``\fort{CGNSBase\_t}''. The name of this
node is user defined, and serves essentially as the name of the
database itself. This name is used by the CGNS software to open the
database.

\subsubsection{The \texttt{CGNSLibraryVersion\_t} Node}

An \HDF file may also contain other nodes below the root node beside
\fort{CGNSBase\_t}, but these are \emph{not} officially part of the
CGNS database and will not be recognized by most CGNS software.
One exception to this is a node called \fort{CGNSLibraryVersion\_t},
which is a child of the \HDF root node.
This node stores the version number of the CGNS standard with which
the file is consistent, and is created automatically when the file is
created or modified using the CGNS Mid-Level Library.
Officially, the CGNS version number is not a part of the CGNS
database (because it is not located below \fort{CGNSBase\_t}).
But because the Mid-Level Library software makes use of it, the node is
included in this document.

\subsubsection{Topological Basis of CGNS Database Organization}

Below the root, the organization of a CGNS database reflects the problem
topology.
Omitting detail, \autoref{f:example_hierarchy},
\hyperref[s:figures]{Appendix~\ref*{s:figures}} shows the overall
structure of the \HDF file.
Below the \HDF root node is the \fort{CGNSLibraryVersion\_t} node, and
one or more \fort{CGNSBase\_t} nodes.
Each \fort{CGNSBase\_t} node is the root of a CGNS database.

At the next level below a \fort{CGNSBase\_t} node are general
specifications which apply to the problem globally, such as reference
states, units, and so on.
At this level we also find a collection of nodes labeled
``\fort{Zone\_t}''.
The tree below each of these holds all the data local to one of the
various zones or subdomains which constitute the problem.

Beneath each \fort{Zone\_t} node there are nodes whose subtrees
store: the grid (labeled \fort{GridCoordinates\_t}); flowfields
(\fort{FlowSolution\_t}); boundary conditions (\fort{ZoneBC\_t});
information about the geometrical connection to other zones
(\fort{GridConnectivity\_t}); and information defining time-dependent
data. Below these there may be additional nodes containing yet more
geometrically local information. For example, under the \fort{ZoneBC\_t}
node there are nodes defining individual boundary conditions on portions
of faces of the zone (\fort{BC\_t}).

Certain types of nodes originally specified at a high level are
optionally repeated below. For example, immediately below a
\fort{Zone\_t} node we may find another \fort{ReferenceState\_t}
node (see \hyperref[s:figures]{Appendix~\ref*{s:figures}},
\autoref{f:Zone}). The CGNS convention is that such a node overrides
(for the associated portion of the topology only) any data found at a
higher level.

\subsubsection{Topics Not Currently Covered}
\label{s:notcovered}

No specification of the kind represented by this file mapping can ever
be complete. However, it is worth noting that there are certain entities
common in CFD which are not currently specified by the file mapping.

Within nodes of type \fort{FlowSolution\_t}, the current file mapping
permits the storage of fields of any number of dependent variables.
In addition to those whose names are specified in the SIDS the user may
add any desired quantities, naming them appropriately.
Names that are not currently codified in the SIDS will not be
common between practitioners without separate communication.

Obviously any sort of physical field could be stored in a
\fort{FlowSolution\_t} node. The
problem with using CGNS for such applications lies in the probable
need to specify additional physical information.  Standardizing this
information is tantamount to extending the SIDS and File Mapping to the
disciplines in question.

Similarly, if a reacting flow problem requires the specification of rate
tables or catalytic wall boundary conditions, extensions to the SIDS and
File mapping will be needed.
