\section{Background}
\label{s:background}
\thispagestyle{plain}

\subsection{History}
\label{s:history}

In the early 1980s, the PLOT3D data format gained acceptance as a de
\textit{facto} standard to enable the storage and exchange of CFD data
within analysis processes, and among collaborating organizations.
This initial CFD data standard today continues to be the most common
storage and exchange standard for CFD data, based on structured grids.

However, by the early 1990s several limitations in the PLOT3D standard
had become apparent.
Individual organizations were overcoming these limitations by defining
extensions to the PLOT3D standard to meet their needs.
These extensions were not coordinated among different organizations, and
therefore data stored in these extended formats generally could not be
utilized outside the originating organization.
Further, the PLOT3D standard had not anticipated several key trends in
CFD technology, such as unstructured grids, turbulence models based on
solutions of partial differential equations, and the need to include
chemical species concentrations as part of a CFD solution.
Also, the PLOT3D format, which was originally developed simply
to expedite post-processing (visualization) did not include
self-documenting features.
Therefore, it was necessary to rely on file-naming conventions or
external notes to maintain awareness of the flow conditions and analyzed
geometry of each PLOT3D data file.

The CGNS Data Standard was initially conceived in 1994 by
NASA,
Boeing, and then-McDonnell Douglas
teams working under the Integrated Wing Design element of NASA's
Advanced Subsonic Technology Program.
The objective of this work was to greatly reduce the time required to
design a transport wing.
Implicit in this goal was increased extensive use of Computational
Fluid Dynamics (CFD) and the possibility of collaborative analyses by
many organizations.

To achieve this vision, it was necessary to establish a common data
format suitable to meet the needs of production CFD tools in the mid- to
late-1990s.
This format would be used to enable interchange of data among different
CFD-related tools and different computing platforms, and to provide a
mechanism for archive and retrieval of CFD data.
The chief tools that were taken into consideration for this goal were
two structured-grid multi-block codes, OVERFLOW and TLNS3D.
The available data standard, the PLOT3D format, was increasingly proving
to be inadequate for this purpose.
Some of these shortcomings included:

\begin{itemize}
\item Requirement to read the entire file to retrieve any data.
\item No provision for multi-block connectivity data.
\item Requirement to convert to ASCII format to transfer data between
      dissimilar computing platforms.
\item Lack of self-documentation; descriptive information must be
      separately maintained outside the data file.
\end{itemize}

Several database options were considered by the NASA / Boeing /
McDonnell Douglas team during the period December 1994 to March 1995.
In March 1995, a decision was taken to build a new data standard called
CGNS (Complex Geometry Navier Stokes).
This standard was a ``clean sheet'' development, but it was heavily
influenced by the McDonnell Douglas
Common File Format
(CFF) standard, which had been established and deployed in 1989 and
significantly revised in 1992.

It should be noted that the CGNS data standard consists of two major
elements:

\begin{Ventryi}{\textit{Data Content and Format}}
   \item [\textit{Data Content and Format}]
         The definition of the intellectual content of the data to
         be represented in this standard, and the format of the
         representation in the standard-conforming data file.
   \item [\textit{Implementing Software}]
         Software packages developed to ease the process of establishing
         CGNS-compliant database references within an applications code.
\end{Ventryi}

In accepted standards contexts such as ISO / STEP, the ``standard''
consists only of the first item, a definition of the data content and
format.
In this regard, the CGNS development team went beyond the traditional
role in setting standards, by also developing software to easily
implement the standard in a code.
The implementing software, in turn, was developed in two layers:

\begin{itemize}
\item Low-level routines to perform elementary operations on the
      database, known as the \textit{ADF (Advanced Data Format) Library}.
      This low-level ADF library performs basic direct I/O operations on
      the file.
      It does not have any built-in knowledge of the data structure or
      the content of the data.
      The user must provide this knowledge; thus, a user who writes
      ADF calls must have a complete understanding of the CGNS data
      structures and content.
\item Higher-level routines to perform common operations required by a
      CFD code, known as the \textit{CGNS Mid-Level Library}.
      The CGNS Mid-Level Library is an Application Programming Interface
      (API) that allows the use of CGNS data files without any knowledge
      of the underlying data structures and file format.
      The person who writes code using this mid-level library needs only
      to have a general understanding of the standard data structure and
      content.
      The purpose of the mid-level library is to shield the user from
      the complexity of the basic ADF calls, and to ensure that the data
      are written in the proper structure to create a CGNS-compliant
      file.
      With the release of CGNS Version 2.4, the Mid-Level Library may be
      built using either ADF or HDF5 as the underlying database.
\end{itemize}

The data standards are controlled by two documents, which are available
at the CGNS Documentation Home Page, at
\url{http://www.grc.nasa.gov/WWW/cgns/}.
These key control documents are:

\begin{itemize*}
\item \textit{Standard Interface Data Structures (SIDS)}
\item \textit{SIDS-to-ADF File Mapping Manual}, or
      \textit{SIDS-to-HDF File Mapping Manual}
\end{itemize*}

The ADF library was developed during 1995, and the first large-scale
deployment was made by (then) McDonnell Douglas - St. Louis in November
1995, as part of an upgrade to the Common File Format system.
During 1995-97, the NASA - Boeing - McDonnell Douglas team focused on
adding content to the control documents, and laying out the requirements
of the mid-level library.

At a review in June 1997, the CGNS team (NASA, Boeing, and McDonnell
Douglas) determined that additional professional support would be
required to produce an adequate mid-level library.
Subcontracts were issued to the ICEM
CFD Engineering Company, in Berkeley, CA, following this decision.
ICEM CFD Engineering in effect became the lead organization for the
development of the mid-level library.
At this time, the acronym ``CGNS'' was re-defined to mean ``CFD General
Notation System'', which was more in keeping with the evolved goals of
this project.

An initial mid-level software library (version 1.0), which met the
original goals of structured multi-block analysis codes, was released
in May 1998.
At this time, a decision also was taken whereby NASA and Boeing
(McDonnell Douglas by this time had been absorbed by Boeing) would
relinquish all rights to ICEM CFD Engineering.
Concurrently, NASA and the informal CGNS committee determined that there
was no need for export authority, so the CGNS standard, the ADF and
mid-level library, and all supporting documentation could be distributed
worldwide as freeware.
Appropriate legal reviews and approvals were obtained at both NASA and
Boeing to validate this decision.

At meetings in March, May, and October 1998, the mid-level library was
extended to support a wide range of unstructured grid types.
The SIDS document defining the standard was modified, and extended
versions of the mid-level library were released at intervals in late
1998 and early 1999.
By May 1999, the extension to unstructured grids was released.

\subsection{Management}
\label{s:management}

Up to this time, all activities related to the development of the
standard, the implementing software, and the related documentation had
been coordinated and largely funded by NASA under the Advanced Subsonic
Technology Program.
In 1998, NASA took a decision that the Advanced Subsonic Technology
program would end on September 30, 1999, which was approximately one
year earlier than their original plan.
Further, NASA indicated that they would not be able to manage the
development of a standard or a software system such as CGNS, once it
ceased to be the focus of an ongoing NASA program.

At this time, a number of U.S. and international organizations had
established plans to use the CGNS standard and the ADF and
mid-level library, and in several cases they had begun implementation.
These organizations had a clear interest in the existence of an
organization to coordinate future use and extensions of the CGNS
standard and its supporting software and documentation.
Also during this same period (1998-99), The Boeing Company launched an
initiative to establish an ISO
standard for aerodynamic data, to be based on the CGNS standard.
However, in a best-case scenario CGNS will not become an ISO standard
until roughly 2005--2006, and acceptance of CGNS as an ISO standard is
not a certainty.
It became clear that CGNS needed to find an organizational home, to
coordinate its extension and utilization.

The organizations interested in the CGNS standard met in Hampton, VA,
on May 20, 1999 to discuss options for a CGNS management organization.
Out of this meeting, the CGNS Steering Committee was established.
This Steering Committee is a voluntary organization to coordinate the
further development and dissemination of the CGNS standard and its
supporting software and documentation.
In January 2000, the CGNS Steering Committee became an official
subcommittee under the purview of the American Institute of Aeronautics
and Astronautics (AIAA) Committee on Standards.
The AIAA also distributes the CGNS SIDS document as an AIAA Recommended
Practice.
However, this AIAA affiliation does not preclude the CGNS committee from
public dissemination of the SIDS and other CGNS documentation.

The following sections of this document present the vision of how the
CGNS Steering Committee will operate.
