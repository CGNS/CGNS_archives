\subsection{Miscellaneous Routines}
\label{s:subs_misc}

\fcthdra{Flush\_to\_Disk}{Flush the Data to the Disk}
\label{sub:Flush_to_Disk}

\begin{fctbox}
   {ADF\_Flush\_to\_Disk}
   {ADFFTD}
   {(ID,error\_return)}
\hline
Input  & const double ID    & real*8 ID \\
\hline
Output & int *error\_return & integer error\_return \\
\hline
\end{fctbox}

\begin{Ventryi}{\textit{error\_return}}
\item[\textit{ID}]
     The ID of the node to use.
\item[\textit{error\_return}]
     Error return code.
     (See \hyperref[s:errors]{Appendix~\ref*{s:errors}}.)
\end{Ventryi}

This routine, \fort{ADF\_Flush\_to\_Disk}, flushes data to disk; it is
used to force any modified information to be flushed to the physical
disk.
This ensures that data will not be lost if a program aborts.
The control of when to flush all data to disk is provided to the user
rather than flushing the data every time it is modified, which would
result in reduced performance.

\fcthdr{Database\_Garbage\_Collection}{Flush the Data to the Disk}
\label{sub:Database_Garbage_Collection}

\setlength{\savearrayrulewidth}{\arrayrulewidth}
\setlength{\arrayrulewidth}{0.8pt}
\noindent
\begin{tabularx}{\textwidth}{|>{\bfseries\columncolor{subcolor}}r%
   |>{\ttfamily\columncolor{subcolor}\hsize=1.05\hsize}X%
   |>{\ttfamily\columncolor{subcolor}\hsize=0.95\hsize}X%
   |}
\hline
\multicolumn{3}{|>{\columncolor{subcolor}}c|}{} \\
\multicolumn{3}{|>{\ttfamily\columncolor{subcolor}}l|}{ADF\_Database\_Garbage\_Collection (ID,error\_return)} \\
\multicolumn{3}{|>{\columncolor{subcolor}}c|}{} \\
\hline
Language &
   \multicolumn{1}{>{\bfseries\columncolor{subcolor}}c|}{C} &
   \multicolumn{1}{>{\bfseries\columncolor{subcolor}}c|}{Fortran} \\
\hline
Routine Name &
   \multicolumn{1}{>{\ttfamily\columncolor{subcolor}}c|}{ADF\_Database\_Garbage\_Collection} &
   \multicolumn{1}{>{\ttfamily\columncolor{subcolor}}c|}{ADFDGC} \\
\hline
Input  & const double ID    & real*8 ID \\
\hline
Output & int *error\_return & integer error\_return \\
\hline
\end{tabularx}
\setlength{\arrayrulewidth}{\savearrayrulewidth}

%\begin{fctbox}
%   {ADF\_Database\_Garbage\_Collection}
%   {ADFDGC}
%   {(ID,error\_return)}
%\hline
%Input  & const double ID    & real*8 ID \\
%\hline
%Output & int *error\_return & integer error\_return \\
%\hline
%\end{fctbox}

\begin{Ventryi}{\textit{error\_return}}
\item[\textit{ID}]
     The ID of the node to use.
\item[\textit{error\_return}]
     Error return code.
     (See \hyperref[s:errors]{Appendix~\ref*{s:errors}}.)
\end{Ventryi}

This routine, \fort{ADF\_Database\_Garbage\_Collection}, redistributes
the data within a file to use free space within the file.
This free space is not located at the end of the file, and it may have
been created during node deletions or other file operations.
Neighboring free spaces will be merged.

\noindent
\emph{Note:} For better file compaction, a utility could be written to
copy an ADF file into a newly created ADF file without wasted space.

\noindent
\emph{Note:} This routine is currently not implemented.

\fcthdr{Error\_Message}{Get a Description of the Error}
\label{sub:Error_Message}

\begin{fctbox}
   {ADF\_Error\_Message}
   {ADFERR}
   {(error\_code,error\_string)}
\hline
Input  & const int error\_code & integer error\_code \\
\hline
Output & char *error\_string   & character*(*) error\_string \\
\hline
\end{fctbox}

\begin{Ventryi}{\textit{error\_string}}
\item[\textit{error\_code}]
     The error return code from any ADF routine.
     (See \hyperref[s:errors]{Appendix~\ref*{s:errors}}.)
\item[\textit{error\_string}]
     An 80-byte description for the specified error.
     If \fort{ADF\_Error\_Message} cannot find a message corresponding
     to the input \fort{error\_code}, the error string ``\fort{Unknown
     error\#\textit{nnn}}'' will be returned.
\end{Ventryi}

This routine, \fort{ADF\_Error\_Message}, returns a textual error
message when given the error return code from any ADF routine.

\Example

See the \hyperlink{ex:Set\_Error\_State}{example for
\fort{ADF\_Set\_Error\_State}}.

\fcthdr{Set\_Error\_State}{Set the Error State Flag}
\label{sub:Set_Error_State}

\begin{fctbox}
   {ADF\_Set\_Error\_State}
   {ADFSES}
   {(error\_state,error\_return)}
\hline
Input  & const int error\_state & integer error\_state \\
\hline
Output & int *error\_return     & integer error\_return \\
\hline
\end{fctbox}

\begin{Ventryi}{\textit{error\_return}}
\item[\textit{error\_state}]
     Flag specifying the action to take when an error occurs; 0 to
     return the error code and continue, 1 to abort.
     The default is 0.
\item[\textit{error\_return}]
     Error return code.
     (See \hyperref[s:errors]{Appendix~\ref*{s:errors}}.)
\end{Ventryi}

This routine, \fort{ADF\_Set\_Error\_State}, sets the error state flag.
This flag controls the error handling convention for all ADF routines:
either return the error codes or print an error message and abort the
program on error.
The flag covers all open ADF files associated with the current program,
and it is not done on a file-by-file basis.
The default state for the ADF interface is to return error codes and not
abort.

\hypertarget{ex:Set\_Error\_State}{}
\Example

\begin{alltt}
   PROGRAM TEST
   C
         PARAMETER (MAXCHR=32)
   C
         CHARACTER*(MAXCHR) NODNAM
         CHARACTER*(80) MESS
   C
   C *** NODE IDS
   C
         REAL*8 RID,CID1,CID2
   C
   C *** OPEN DATABASE
   C
         CALL ADFDOPN('db.adf','NEW',' ',RID,IERR)
   C
   C *** CREATE 2 NODES
   C
         CALL ADFCRE(RID,'NODE 1',CID1,IERR)
         CALL ADFCRE(RID,'NODE 2',CID2,IERR)
   C
         CALL ADFGES(IDEFS,IERR)
         PRINT *,' DEFAULT ERROR STATE = ',IDEFS
   C
   C *** REQUEST NODE NAME FOR A NODE THAT DOES NOT EXIST
   C
         PRINT *,' *** ON ERROR CONTINUE'
         CALL ADFGNAM(CID3,NODNAM,IERR)
         CALL ADFERR(IERR,MESS)
         PRINT *,'     ADF ERROR OCURRED, MESSAGE: ',MESS
         PRINT *,' '
   C
   C *** SET ABORT ON ERROR FLAG
   C
         INEWS = 1
         CALL ADFSES(INEWS,IERR)
         PRINT *,' *** ABORT ON ERROR SET'
   C
   C *** REQUEST NODE NAME FOR A NODE THAT DOES NOT EXIST
   C
         CALL ADFGNAM(CID3,NODNAM,IERR)
         PRINT *,' HELLO WORLD'
   C
         STOP
         END
\end{alltt}

\noindent
The resulting output is:

\begin{alltt}
   DEFAULT ERROR STATE =            0
     *** ON ERROR CONTINUE
         ADF ERROR OCURRED, MESSAGE:
   ADF 10: ADF file index out of legal range.

     *** ABORT ON ERROR SET
   ADF 10: ADF file index out of legal range.
   ADF Aborted: Exiting
\end{alltt}

\fcthdr{Get\_Error\_State}{Get the Error State}
\label{sub:Get_Error_State}

\begin{fctbox}
   {ADF\_Get\_Error\_State}
   {ADFGES}
   {(error\_state,error\_return)}
\hline
Input  &  &  \\
\hline
Output & int *error\_state  & integer error\_state \\
       & int *error\_return & integer error\_return \\
\hline
\end{fctbox}

\begin{Ventryi}{\textit{error\_return}}
\item[\textit{error\_state}]
     Flag specifying the action to take when an error occurs; 0 to
     return the error code and continue, 1 to abort.
     The default is 0.
\item[\textit{error\_return}]
     Error return code.
     (See \hyperref[s:errors]{Appendix~\ref*{s:errors}}.)
\end{Ventryi}

This routine, \fort{ADF\_Get\_Error\_State}, returns the currently set
error state.

\Example

\begin{alltt}
   PROGRAM TEST
   C
   C *** NODE IDS
   C
         REAL*8 RID
   C
   C *** OPEN DATABASE
   C
         CALL ADFDOPN('db.adf','NEW',' ',RID,IERR)
   C
   C *** CHECK THE DEFAULT ERROR STATE
   C
         CALL ADFGES(IDEFS,IERR)
   C
   C *** SET THE ERROR STATE TO ABORT ON ERROR
   C
         CALL ADFSES(1,IERR)
   C
   C *** MAKE SURE STATE WAS SET AS DESIRED
   C
         CALL ADFGES(NDEFS,IERR)
   C
   C *** PRINT OUT RESULTS
   C
         PRINT *,' DEFAULT ERROR STATE = ',IDEFS
         PRINT *,' RESET ERROR STATE   = ',NDEFS
   C
         STOP
         END
\end{alltt}

\noindent
The resulting output is:

\begin{alltt}
   DEFAULT ERROR STATE =            0
   RESET ERROR STATE   =            1
\end{alltt}

\fcthdr{Database\_Version}{Get the Version Number of the ADF Library
That Created the ADF Database}

\label{sub:Database_Version}
\setlength{\savearrayrulewidth}{\arrayrulewidth}
\setlength{\arrayrulewidth}{0.8pt}
\noindent
\begin{tabularx}{\textwidth}{|>{\bfseries\columncolor{subcolor}}r%
   |>{\ttfamily\columncolor{subcolor}}X%
   |>{\ttfamily\columncolor{subcolor}}X%
   |}
\hline
\multicolumn{3}{|>{\columncolor{subcolor}}c|}{} \\
\multicolumn{3}{|>{\ttfamily\columncolor{subcolor}}l|}{ADF\_Database\_Version (root\_ID,version,creation\_date,modification\_date,} \\
\multicolumn{3}{|>{\ttfamily\columncolor{subcolor}}l|}{~~~~~~~~~~~~~~~~~~~~~~error\_return)} \\
\multicolumn{3}{|>{\columncolor{subcolor}}c|}{} \\
\hline
Language &
   \multicolumn{1}{>{\bfseries\columncolor{subcolor}}c|}{C} &
   \multicolumn{1}{>{\bfseries\columncolor{subcolor}}c|}{Fortran} \\
\hline
Routine Name &
   \multicolumn{1}{>{\ttfamily\columncolor{subcolor}}c|}{ADF\_Database\_Version} &
   \multicolumn{1}{>{\ttfamily\columncolor{subcolor}}c|}{ADFDVER} \\
\hline
Input  & constant double root\_ID & real*8 root\_ID \\
\hline
Output & char *version            & character*(*) version \\
       & char *creation\_date     & character*(*) creation\_date \\
       & char *modification\_date & character*(*) modification\_date \\
       & int *error\_return       & integer error\_return \\
\hline
\end{tabularx}
\setlength{\arrayrulewidth}{\savearrayrulewidth}

\begin{Ventryi}{\textit{modification\_date}}
\item[\textit{root\_ID}]
     The ID of the root node in the ADF file.
\item[\textit{version}]
     A 32-byte character string containing the version ID.
\item[\textit{creation\_date}]
     A 32-byte character string containing the creation date of the file.
\item[\textit{modification\_date}]
     A 32-byte character string containing the last modification date of
     the file.
\item[\textit{error\_return}]
     Error return code.
     (See \hyperref[s:errors]{Appendix~\ref*{s:errors}}.)
\end{Ventryi}

This routine, \fort{ADF\_Database\_Version}, returns the version number
of the ADF library that created an ADF database, the file creation date
and time, and the last modification date and time.
A modified ADF database will take on the version ID of the current ADF
library version if it is higher than the version indicated in the file.
On return, ``version'' contains a six-character string.
The meaning of the characters are described in detail in
\hyperref[s:versions]{Appendix~\ref*{s:versions}}.

\hypertarget{ex:Database\_Version}{}
\Example

\begin{alltt}
   PROGRAM TEST
   C
         PARAMETER (MAXCHR=32)
   C
         CHARACTER*(MAXCHR) NODNAM
         CHARACTER*(MAXCHR) CVER,LVER,CDATE,MDATE
         CHARACTER*(80) MESS
   C
   C *** NODE IDS
   C
         REAL*8 RID,CID1,CID2
   C
   C *** OPEN DATABASE
   C
         CALL ADFDOPN('db.adf','NEW',' ',RID,IERR)
   C
   C *** CREATE 2 NODES
   C
         CALL ADFCRE(RID,'NODE 1',CID1,IERR)
         CALL ADFCRE(RID,'NODE 2',CID2,IERR)
   C
         CALL ADFDVER(RID,CVER,CDATE,MDATE,IERR)
         CALL ADFLVER(LVER,IERR)
         PRINT *,' VERSION INFORMATION:'
         PRINT *,'    ADF LIBRARY USED FOR CREATION: ',CVER
         PRINT *,'    CREATION DATE                : ',CDATE
         PRINT *,'    MODIFICATION DATE            : ',MDATE
         PRINT *,'    ADF LIBRARY BEING USED       : ',LVER
   C
         STOP
         END
\end{alltt}

\noindent
The resulting output is:

\begin{alltt}
   VERSION INFORMATION:
      ADF LIBRARY USED FOR CREATION: ADF Database Version A01007
      CREATION DATE                : Thu Apr 24 15:41:55 1997
      MODIFICATION DATE            : Thu Apr 24 15:41:55 1997
      ADF LIBRARY BEING USED       : ADF Library  Version C01
\end{alltt}

\fcthdr{Library\_Version}{Get the Version Number of the ADF Library
That the Application Program is Currently Using}
\label{sub:Library_Version}

\begin{fctbox}
   {ADF\_Library\_Version}
   {ADFLVER}
   {(version,error\_return)}
\hline
Input  &  & \\
\hline
Output & char *version      & character*(*) version \\
       & int *error\_return & integer error\_return \\
\hline
\end{fctbox}

\begin{Ventryi}{\textit{error\_return}}
\item[\textit{version}]
     A 32-byte character string containing the ADF library version ID
     information.
\item[\textit{error\_return}]
     Error return code.
     (See \hyperref[s:errors]{Appendix~\ref*{s:errors}}.)
\end{Ventryi}

This routine, \fort{ADF\_Library\_Version}, gets the ADF library version
number.
This is the version number of the ADF library that the application
program is currently using.
For this routine, the format of the version ID is ``\fort{ADF Library
Version A\textit{XXXxxx}}''.

\Example

See the \hyperlink{ex:Database\_Version}{example for
\fort{ADF\_Database\_Version}}.
