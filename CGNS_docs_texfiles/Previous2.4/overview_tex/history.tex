\section{History and Current Status}
\label{s:history}
\thispagestyle{plain}

\subsection{Initial Development}

\enlargethispage{\baselineskip}
The initial development of CGNS took place from March 1995 through
May 1998, when Version 1.0 of the CGNS software was released.
The project originated through a series of meetings in 1994-1995 between
NASA, Boeing, and McDonnell Douglas, who were then working under the
Integrated Wing Design element of NASA's Advanced Subsonic Technology
Program.
The work being planned involved extensive use of CFD, and the
possibility of collaborative analyses by many organizations.
It was thus necessary to establish a common data format suitable to meet
the needs of production CFD tools in the mid- to late-1990s.
This format would be used to enable interchange of data among different
CFD-related tools and different computing platforms, and to provide a
mechanism for archive and retrieval of CFD data

At that time, the \textit{de facto} standard for CFD data was the file
format used by Plot3D, a flow visualization program developed at NASA
Ames.
However, the Plot3D format was developed to expedite post-processing,
and was never intended as a general-purpose standard for the storage
and exchange of CFD data.
By the early 1990s, as CFD became more sophisticated, the limitations of
the Plot3D format as a general-purpose standard had become apparent.
It had no provisions for storing data associated with modern CFD
technology, such as unstructured grids, turbulence models based on
solutions of partial differential equations, and flows with multiple
chemical species.

Individual organizations were overcoming these limitations by defining
extensions to the Plot3D format to meet their needs.
These extensions were not coordinated among different organizations, and
therefore data stored in these extended formats generally could not be
utilized outside the originating organization.
Further, the Plot3D format was not self-documenting, and it was
necessary to rely on file-naming conventions or external notes to
maintain awareness of the flow conditions and analyzed geometry of each
PLOT3D data file.

To overcome these limitations, several database options were considered
by the NASA~/ Boeing~/ McDonnell Douglas team from December 1994 to
March 1995.
In March 1995, the decision was made to build a new data standard called
CGNS (Complex Geometry Navier Stokes).
This standard was a ``clean sheet'' development, but it was heavily
influenced by the McDonnell Douglas Common File Format (CFF), which had
been established and deployed in 1989 and heavily revised in 1992.

Agreement was reached to develop CGNS at Boeing, under NASA Contract
NAS1-20267, with active participation by a team of CFD researchers from

\begin{itemize*}
   \item NASA Langley Research Center
   \item NASA Glenn Research Center
   \item NASA Ames Research Center
   \item McDonnell Douglas Corporation (now Boeing St. Louis)
   \item Boeing Commercial Airplane Group (Seattle) Aerodynamics
   \item Boeing Commercial Airplane Group (Seattle) Propulsion
   \item ICEM CFD Engineering Corporation
\end{itemize*}

The original development work itself occurred from 1995--1998, and
consisted of essentially four activities:

\begin{itemize*}
\item Development of the standalone database manager (the ADF Core)
\item Identification and layout of the CFD data (the SIDS and File
      Mapping)
\item Development of an API for use in applications codes (the
      Mid-Level Library)
\item Demonstration of system capability
\end{itemize*}

\subsubsection{Development of ADF}

As noted earlier, the decision to develop CGNS was made after an
examination of several existing formats.
Because of its hierarchical structure, the CFF (Common File Format) that
was then in use at McDonnell Douglas was judged the most compatible with
the SIDS.
However, CFF had known limitations, and it was decided to adopt the CFF
structure but recast it in a more general form.
This gave the developers control over the software and facilitated the
development of a robust, portable, reliable, and flexible database
manager, which could be distributed freely within the system.

These low-level routines were developed during 1995, and make up the
ADF (Advanced Data Format) Core.
ADF was written and extensively tested at Boeing by a small subteam
familiar with standard software development practices, and it may now be
considered a stable product.

\subsubsection{Identification and layout of the CFD data}

Another task, undertaken in parallel with the development of ADF, was to
identify the data associated with CFD and establish a means of encoding
it within the ADF format.
This task was the primary concern of the national team, with Boeing
proposing the standards and the remainder of the team serving to
critique, test, and improve them.
This began with the development of the SIDS and its language.
The hierarchical nature of the SIDS was used to suggest the SIDS-to-ADF
File Mapping conventions, which were developed slightly after the SIDS
but along the same track.

\subsubsection{Development of an API}

At a review in June 1997, the CGNS team (NASA, Boeing, and McDonnell
Douglas) determined that additional support would be required to produce
an adequate Mid-Level Library.
Subcontracts were issued to the ICEM CFD Engineering Company, in
Berkeley, CA, following this decision.
ICEM CFD in effect became the lead organization for the development of
the Mid-Level Library.

Also at this time, the acronym ``CGNS'' was re-defined to mean ``CFD
Generalized Notation System'', which was more in keeping with the evolved
goals of this project.

Version 1.0 of the Mid-Level Library, which met the original goal of
supporting structured multi-block analysis codes, was released in May
1998.
This set of higher-level routines enables a user to implement the CFD
standards defined by the SIDS, without specific knowledge of the File
Mapping or the ADF Core.
This activity was crucial to the acceptance and implementation of CGNS
among the CFD community.

NASA and the informal CGNS committee determined that there was no
need for export authority so the CGNS standard, the ADF Core and the
Mid-Level Library, and all supporting documentation could be distributed
worldwide as freeware.
Appropriate legal reviews and approvals were obtained at both NASA and
Boeing to validate this decision.

\subsubsection{Demonstration of system capability}

\paragraph{Prototypes.}
To test the CGNS data recording and retrieval mechanisms and to
gain experience with the installation of the software into common
applications codes, the developers modified two CFD solvers (namely,
NPARC and TLNS3D) to operate in the CGNS environment.
In addition, NASA Langley Research Center modified CFL3D similarly.
These modifications were made early in the project and predate the
Mid-Level Library and some later portions of the SIDS.

Separate data cases were prepared for each of the prototypes.
Each prototype proved capable of starting, exiting, and restarting its
data case as expected.
In addition, in all three cases, transfer of the CGNS data between
workstation (SGI) and mainframe (CRAY) platforms was successfully
demonstrated.

Dissimilarity of the \emph{content} of the data required by NPARC
and the other two codes (i.e., nodal vs. cell-centered data) prevented
restarting NPARC from TLNS3D/CFL3D data and vice versa.
However, this type of restart was demonstrated between TLNS3D and CFL3D.

These prototypes were limited in certain ways and, therefore, were not
suitable as a final implementation of their respective capabilities in
CGNS.
The principal limitations are:

\begin{itemize}
\item The prototypes implement only the grid coordinates, flow solution,
      boundary condition, and 1-to-1 interface connectivity portions of
      the CGNS data specification.
\item No attempt was made to modify the internal structure of any of
      the codes in order to improve compatibility with CGNS data
      organization.
\item The prototypes accessed the ADF files primarily using routines
      at the ADF Core level.
      There were some higher-level routines included, and these, in many
      cases, suggested the content of the CGNS Mid-Level Libraries.
      But the high-level prototype routines often intermingled
      ADF functions with CGNS functions and, in some cases, were
      code-specific or dependent on internal directives to make them so.
      They were thus less broadly applicable than the current Mid-Level
      Library routines.
\item These codes exercised only a portion of the CGNS boundary
      condition specifications. 
\item The prototypes sometimes incorporated extra nodes into their ADF
      files to carry code-specific data.
      This practice arose partially from the lack of complete CGNS data
      specifications at the time the prototypes were written, but it
      resulted also, in part, because the current code input structure
      required a different (but equivalent) form of the CGNS data, and
      the developers opted to duplicate it within the CGNS database.
\end{itemize}

The general lesson learned from the construction of the prototypes
was that professional programmers had no conceptual difficulty in
implementing CGNS at the ADF Core level.
But the resulting code was cumbersome, and the development of the
Mid-Level Library was needed to facilitate dissemination of CGNS among
those disinclined to work with the ADF Core.

\paragraph{System Demonstrator.}
As the CGNS effort progressed, an additional activity was undertaken
to demonstrate the capability of the mature system.
Known as the ``system demonstrator'', this tested the ability of CGNS to
transfer data seamlessly between applications that had never operated
together before.
The system demonstrator used most of that portion of the currently
existing SIDS that has been included in the Mid-Level Library.

The geometry chosen for the test was a high-lift configuration known as
the trapezoidal wing.
This is a multielement airfoil with a full-span slat and flap, and a
generic fuselage.

Three separate CGNS-compatible application codes were involved in the
system demonstrator.

\begin{Ventryic}{OVERFLOW}
\item [GMAN]
      A pre-processor and grid generation tool developed by Boeing
      St. Louis
\item [OVERFLOW]
      A CFD flow solver developed by NASA Ames
\item [Visual3]
      A flow visualization tool from ICEM CFD, originally developed at
      MIT
\end{Ventryic}

The system demonstrator consisted of the following tasks:

\begin{enumerate}
\item The grid was generated using NASA Ames grid tools and written
      as a Plot3D file.
\item The grid file was sent to Boeing St. Louis, where it was processed
      by a locally-modified version of GMAN.
      GMAN calculated the grid connectivity information, and wrote the
      grid and connectivity data into a CGNS database.
      Boundary conditions were also added to the CGNS database at this
      time.
\item The CGNS file was returned to NASA Ames, where it was read by the
      newly-modified OVERFLOW code.
      (Some iteration was necessary here, because the definition of
      overset holes used by GMAN differs from that normally expected by
      OVERFLOW.
      The CGNS file was intact, and served, as intended, to highlight the
      discrepancy.)
      OVERFLOW computed the flow field, writing the results into the
      CGNS database.
\item The CGNS file was next sent to ICEM CFD.
      There, the CGNS database was read and displayed using Visual3.
\end{enumerate}

The system demonstrator involved significant cross-platform transfer
among various workstation and mainframe computing facilities.
The results indicated that universal data exchange was well supported by
CGNS.

\subsection{Subsequent Development}

\subsubsection{Software}

Since the release of Version 1.0 of the CGNS software in May 1998,
several improvements and extensions have been made to the SIDS, the File
Mapping, and the Mid-Level Library.

\begin{itemize}
\item Version 1.1, released in June 1999, added support for
      unstructured grids, and for associating CAD geometric entities
      with grid surfaces.
\item Version 2.0, released in December 2000, added support for
      moving and/or deforming grids, and for iterative/time-dependent
      data.
\item Version 2.1, released in May 2002, added support for
      user-defined data arrays, chemistry, and linked nodes.
\item Version 2.2, released in May 2003, added support for axisymmetry
      information, rotating coordinates, special properties associated
      with particular grid connectivity patches, such as periodicity or
      averaging, special properties associated with particular boundary
      condition patches, such as wall functions and bleed, and gravity.
\item Version 2.3, released in Jan 2004, restored the capability to
      specify a boundary condition patch using \fort{ElementRange} or
      \fort{ElementList}.
\item Version 2.4, released in Aug 2005, added support for describing
      electric field, magnetic field, and conductivity models used
      in electromagnetic flows, specification of units for electric
      current, substance amount, and luminous intensity, more flexible
      specification of boundary condition locations, and additional
      user-defined data.
      Also new in Version 2.4 is the ability to choose, at build time,
      either ADF or HDF5 as the underlying database manager.
\item Version 2.5, released in Sep 2007, added Mid-Level Library
      functions to check file validity, configure some internal CGNS
      library options, and provide alternate ways to access a node.
      Some changes were also made in the use of functions for partial
      writes of coordinate, element, and solution data.
      Support was also added for building the CGNS library as a DLL under
      Windows.
\end{itemize}
A more detailed list of revisions to the CGNS software is available at
the CGNS web site.

ICEM CFD served as the focal point for CGNS software development through
the release of Version 2.0, plus the changes in Version 2.5, using
internal company resources.
New features in Version 2.1 and 2.4 were added by Intelligent Light,
with funding from NASA Langley.
Eagle Aeronautics, with additional participation by ICEM CFD, added the
new features in Version 2.2, again with funding from NASA Langley.
Support for HDF as the underlying database in Version 2.4 was added
by personnel from ONERA, ICEM CFD, and the U. S. Air Force Arnold
Engineering Development Center.

\subsubsection{Management and Support}

In 1998 NASA announced that the Advanced Subsonic Technology program,
which had funded the initial CGNS development, would end in September
1999.
Several organizations interested in the continued development of CGNS
met in May 1999 to discuss options.
The decision was made to create the CGNS Steering Committee, a voluntary
public forum made up of international representatives from government
and private industry.

The Steering Committee is responsible for coordinating the further
development and dissemination of the CGNS standard and its supporting
software and documentation.
In Jan 2000, the CGNS Steering Committee became a Sub-committee of the
AIAA CFD Committee on Standards.
Additional details about the mission and responsibilities of the
Steering Committee, and its organization, are in the \textit{CGNS Steering
Committee Charter}.

The basic CGNS documentation has of course been updated to reflect the
software changes described above.
In addition, all the documentation has been converted to LaTeX (used
to create PDF versions for printing), and to HTML (for interactive
use).
A new document, \textit{A User's Guide to CGNS}, was made available in
October 2001, and is very useful for new and prospective users of the
Mid-Level Library.

To encourage communication between CGNS users, a mailing list called
CGNStalk was created in Oct 2000.
Instructions for subscribing are available at the CGNS web site.

\subsection{Current Status}

Since the initial release of the Mid-Level Library in May 1998,
interest in CGNS has continued to grow throughout the global CFD
community.
The system is being used by engineers and scientists in academia,
industry, and government.
By January 2002, 591 users from more than 25 countries had formally
registered at the CGNS web site.
As of September 2007, the CGNStalk mailing list had 224 participants
from 20 different countries and at least 79 different organizations.

\subsubsection{Software}

The current ``production'' release of CGNS is Version 2.5, released in
September 2007.

Several extensions to CGNS have been formally proposed.
Documentation supporting these extensions, and information on their
current status, is available via the ``Proposals for Extensions'' link at
the CGNS web site.

\subsubsection{Standards}

Between 1999 and 2002, an effort was spearheaded by Boeing to establish
an ISO-STEP standard for the representation, storage, and exchange of
digital data in fluid dynamics based on the CGNS standard.
Unfortunately, the effort had to be curtailed because of budget
problems.
It was subsequently decided that an existing ISO standard on finite
element solid mechanics would be rewritten and submitted to include CGNS
as well as an integrated engineering analysis framework.

As part of its role as a sub-committee of the AIAA CFD Committee
on Standards, the CGNS Steering Committee is also involved in the
development of the AIAA Recommended Practice for the storage of CFD
data.
The Recommended Practice consists of the CGNS Standard Interface
Data Structures (SIDS) document, reformatted to conform to AIAA's
requirements.
The current AIAA Recommended Practice (corresponding to CGNS Version
2.4) is available at the
\href{http://www.aiaa.org/content.cfm?pageid=363&id=1657}{AIAA
Online Store}, and as a
\href{http://www.grc.nasa.gov/www/cgns/sids/aiaa/R_101A_2005.pdf}{PDF file}
(1.01M, 200 pages) at the CGNS Documentation web site.

\subsection{Noteworthy Contributions}

The entire CGNS team deserves credit for its accomplishments.
The project might easily be cited as a prime example of effective
voluntary cooperation between government and industry.
Every member has contributed ideas, enthusiasm, and improvements, and
enabled us to represent a cross-section of knowledge and practice from
the CFD community.

It would not be practical to outline each member's contributions to a
project of this complexity.
Nevertheless, it is appropriate to note the contributions of a few
individuals who dedicated special effort during the crucial stages
leading to the initial release of CGNS in May 1998.

\begin{itemize}
\item Steve Allmaras conceived, pursued, and completed the SIDS and
      developed the language in which it is written.
\item Tom Dickens, Matt Smith, and Wayne Jones designed the ADF Core.
\item Tom Dickens wrote the ADF Core.
\item Dan Owen maintained and tuned the ADF Core after Tom left the group.
\item Chuck Keagle designed and executed stringent testing procedures
      for the ADF Core; that is to say, he did everything he could think
      of to break it.
\item Diane Poirier designed and wrote the CGNS Midlevel Library and
      drafted the original proposals for unstructured grid.
      She and Alan Magnuson drafted the proposals for CAD-to-geometry
      specifications.
\item Gary Shurtleff wrote the TLNS3D and NPARC prototypes and provided
      us with what was, for a long time, the only examples of working
      CGNS software.
\item Chris Rumsey served as liaison with NASA Langley, responding in
      detail to requests for criticism and improvements and testing the
      system by writing the CFL3D prototype.
\item Cetin Kiris converted the OVERFLOW code to CGNS, twice.
\item Wayne Jones tested our abstractions with real code, which found
      its way into the ADF Midlevel Library and the CGNS Toolkit.
\item Matt Smith wrote much of the File Mapping and ADF Core documents
      and, with knowledge of both CFD and software design, brought good
      sense to bear on proposals that needed it.
\item Ray Cosner's retrospective vantage point made him a reliable
      supporter who was often able to move things forward when progress
      slowed.
\item Susan Jacob supplied initial guidance as a Productivity+ coach and
      got us all moving in (more or less) the same direction.
\item Doug McCarthy led the project to completion.
\item Shay Gould made the documentation intelligible and presentable.
\end{itemize}

\noindent
And, a particularly special mention:

\begin{itemize}
\item Ben Paul secured the initial funding and shepherded the project
      through the early phases of the contract.
      We will never forget the many meetings Ben held to make sure that
      he thought that he knew that we thought that we knew what we needed
      to do, and that we were doing it.
      His patience and good-natured perseverance could not have been
      replaced.
\end{itemize}
