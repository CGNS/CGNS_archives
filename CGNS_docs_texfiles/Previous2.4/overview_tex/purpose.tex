\section{Purpose and Scope}
\label{s:purpose}
\thispagestyle{plain}

The general purpose of CGNS is to provide a standard for recording and
recovering computer data associated with the numerical solution of the
equations of fluid dynamics.

CGNS consists of a collection of conventions, and software implementing
those conventions, for the storage and retrieval of CFD (Computational
Fluid Dynamics) data.
The system consists of two parts: (1) a standard format for recording
the data, and (2) software that reads, writes, and modifies data in that
format.
The format is a conceptual entity established by the documentation, and
is intended to be general, portable, expandable, and durable.
The software is a physical product supplied to enable developers to
access and produce data recorded in that format.
All CGNS software is completely free and open to anyone. 

The CGNS standard, applied through the use of the supplied software, is
intended to:

\begin{itemize*}
   \item facilitate the exchange of CFD data
   \begin{itemize*}
      \item between sites
      \item between applications codes
      \item across computing platforms
   \end{itemize*}
   \item stabilize the archiving of CFD data
\end{itemize*}

The principal target of CGNS is data normally associated with
compressible viscous flow (i.e., the Navier-Stokes equations), but the
standard is also applicable to subclasses such as Euler and potential
flows.
The CGNS standard addresses the following types of data.

\begin{itemize*}
\item Structured, unstructured, and hybrid grids
\item Flow solution data, which may be nodal, cell-centered,
      face-centered, or edge-centered
\item Multizone interface connectivity, both abutting and overset
\item Boundary conditions
\item Flow equation descriptions, including the equation of state,
      viscosity and thermal conductivity models, turbulence models,
      and multi-species chemistry models
\item Time-dependent flow, including moving and deforming grids
\item Dimensional units and nondimensionalization information
\item Reference states
\item Convergence history
\item Association to CAD geometry definitions
\end{itemize*}

Much of the standard and the software is applicable to computational
field physics in general.
Disciplines other than fluid dynamics would need to augment the data
definitions and storage conventions, but the fundamental database
software, which provides platform independence, is not specific to fluid
dynamics.
