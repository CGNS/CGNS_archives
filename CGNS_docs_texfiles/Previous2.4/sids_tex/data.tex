\section{Data-Array Structure Definitions}
\label{s:data}
\thispagestyle{plain}

This section defines the structure type |DataArray_t| for describing
data arrays.  This general-purpose structure is used to declare
data arrays and scalars throughout the CGNS hierarchy.  It is used
to describe grid coordinates, flow-solution data, governing flow
parameters, boundary-condition data, and other information.  For most
of these different types of CFD data, we have also established a
list of standardized identifiers for entities of type |DataArray_t|.
For example, |Density| is used for data arrays containing static
density.  The list of standardized data-name identifiers is provided in
\hyperref[s:dataname]{Appendix~\ref*{s:dataname}}.

We address five classes of data with the |DataArray_t| structure type:
\begin{enumerate}
\item[(a)] dimensional data (e.g., velocity in units of m/s);
\item[(b)] nondimensional data normalized by dimensional reference
           quantities;
\item[(c)] nondimensional data with associated nondimensional reference
           quantities;
\item[(d)] nondimensional parameters (e.g., Reynolds number, pressure
           coefficient);
\item[(e)] pure constants (e.g., $\pi$, $e$).
\end{enumerate}
The first two of these classes often occur within the same test case,
where each piece of data is either dimensional itself or normalized by a
dimensional quantity.  The third data class is typical of a completely
nondimensional test case, where all field data and reference quantities
are nondimensional.  The forth class, nondimensional parameters, are
universal in CFD, although not always consistently defined.  The
individual components of nondimensional parameters may be data from any
of the first three classes.

Each of the five classes of data requires different information to
describe dimensional units or normalization associated with the data.
These requirements are reflected in the structure definition for
|DataArray_t|.

The remainder of this section is as follows: the structure type
|DataArray_t| is first defined.  Then the class identification and data
manipulation is discussed in \autoref{s:data_manip} for each of the five data
classes.  Finally, examples of |DataArray_t| entities are presented in
\autoref{s:data_example}.

\subsection{Definition: \texttt{DataArray\_t}}
\label{s:DataArray}

|DataArray_t| describes a multi-dimensional data array of given
type, dimensionality and size in each dimension.  The data may be dimensional,
nondimensional or pure constants.  Qualifiers are provided to describe
dimensional units or normalization information associated with the data.
\begin{alltt}
  DataArray\_t< DataType, int Dimension, int[Dimension] DimensionValues > :=
    \{
    List( Descriptor\_t Descriptor1 ... DescriptorN ) ;                      (o)
 
    Data( DataType, Dimension, DimensionValues ) ;                          (r)

    DataClass\_t DataClass ;                                                 (o)
    
    DimensionalUnits\_t DimensionalUnits ;                                   (o)

    DimensionalExponents\_t DimensionalExponents ;                           (o)
    
    DataConversion\_t DataConversion ;                                       (o)
    \} ;
\end{alltt}

\begin{notes}
\item
 Default names for the |Descriptor_t| list are as shown; 
 users may choose other legitimate names.  Legitimate names must be unique 
 within a given instance of |DataArray_t| and shall not include 
 the names |DataClass|, |DimensionalUnits|, |DimensionalExponents|, or
 |DataConversion|.
\item
 |Data()| is the only required field for |DataArray_t|.
\end{notes}

|DataArray_t| requires three structure parameters: |Dimension| is the 
dimensionality of the data array; |DimensionValues| is an array of length
|Dimension| that contains the size of the data arrays in each dimension;
and |DataType| is the data type of the data stored.  |DataType| will usually
be |real|, but other data types are permissible. 

The optional entities \texttt{DataClass}, \texttt{DimensionalUnits},
\texttt{DimensionalExponents} and \texttt{DataConversion} provide
information on dimensional units and normalization associated with the
data.  The function of these qualifiers is provided in the next section.

This structure type is formulated to describe an array of scalars.
Therefore, for vector quantities (e.g., the position vector or the velocity
vector), separate structure entities are required for each component of the
vector.  For example, the cartesian coordinates of a 3-D grid are described
by three separate |DataArray_t| entities: one for $x$, one for $y$ and one
for $z$ (see \hyperref[ex:grid1]{Example~\ref*{ex:grid1}}).

\subsubsection{Definition: \texttt{DataConversion\_t}}
\label{s:DataConversion}

|DataConversion_t| contains conversion factors for recovering raw
dimensional data from given nondimensional data.
These conversion factors are typically associated with nondimensional
data that is normalized by dimensional reference quantities.
\begin{alltt}
  DataConversion\_t :=
    \{
    real ConversionScale ;                                                  (r)
    
    real ConversionOffset ;                                                 (r)
    \} ;
\end{alltt}
Given a nondimensional piece of data, |Data(nondimensional)|, the
conversion to ``raw'' dimensional form is:
\begin{alltt}
  Data(raw) = Data(nondimensional)*ConversionScale + ConversionOffset
\end{alltt}
These conversion factors are further described in \autoref{s:data_normbydim}. 

\subsection{Data Manipulation}
\label{s:data_manip}

The optional entities of |DataArray_t| provide information for manipulating
the data, including changing units or normalization.  This section describes
the rules under which these optional entities operate and the specific
manipulations that can be performed on the data.

Within a given instance of |DataArray_t|, the class of data and all
information required for manipulations may be completely and precisely
specified by the entities |DataClass|, |DimensionalUnits|,
|DimensionalExponents| and |DataConversion|.
|DataClass| identifies the class of data and governs the manipulations
that can be performed.
Each of the five data classes is treated separately in the subsequent
sections.

The entities |DataClass| and |DimensionalUnits| serve special functions in
the CGNS hierarchy.  If |DataClass| is absent from a given instance of
|DataArray_t|, then its value is determined from ``global'' data.  This global
data may be set at any level of the CGNS hierarchy with the data set at the
lowest level taking precedence.  |DimensionalUnits| may be similarly set by
global data.  The rules for determining the appropriate set of global data to
apply is further detailed in \autoref{s:precedence}.

This alternate functionality provides a measure of economy in describing
dimensional units or normalization within the hierarchy.  Examples that
make use of global data are presented in \autoref{s:grid_example} and
\autoref{s:flow_example} for grid and flow solution data.  The complete
two-zone case of \hyperref[s:twozone]{Appendix~\ref*{s:twozone}} also
depicts this alternate functionality.

\subsubsection{Dimensional Data}
\label{s:data_dim}

If |DataClass = Dimensional|, the data is dimensional.
The optional qualifiers |DimensionalUnits| and |DimensionalExponents|
describe dimensional units associated with the data.
These qualifiers are provided to specify the system of dimensional units
and the dimensional exponents, respectively.
For example, if the data is the $x$-component of velocity, then
|DimensionalUnits| will state that the pertinent dimensional units are,
say, |Meter| and |Second|; |DimensionalExponents| will specify the
pertinent dimensional exponents are |LengthExponent| |=| |1| and
|TimeExponent| |=| |-1|.
Combining the information gives the units m/s.
Examples showing the use of these two qualifiers are provided in
\autoref{s:data_example}.

If |DimensionalUnits| is absent, then the appropriate set of dimensional
units is obtained from ``global'' data.  The rules for determining
this appropriate set of global dimensional units are presented in
\autoref{s:precedence}.

If |DimensionalExponents| is absent, then the appropriate dimensional
exponents can be determined by convention if the specific instance of
|DataArray_t| corresponds to one of the standardized data-name identifiers
listed in \hyperref[s:dataname]{Appendix~\ref*{s:dataname}}.
Otherwise, the exponents are unspecified.
We strongly recommend inclusion of the |DimensionalExponents| qualifier
whenever the data is dimensional and the instance of |DataArray_t| is not
among the list of standardized identifiers.

\subsubsection{Nondimensional Data Normalized by Dimensional Quantities}
\label{s:data_normbydim}

If |DataClass = NormalizedByDimensional|, the data is nondimensional and is
normalized by dimensional reference quantities.  All optional entities in
|DataArray_t| are used.  |DataConversion| contains factors to convert the 
nondimensional data to ``raw'' dimensional data; these factors are
|ConversionScale| and |ConversionOffset|.  The conversion process is as
follows:
\begin{alltt}
  Data(raw) = Data(nondimensional)*ConversionScale + ConversionOffset
\end{alltt}
where |Data(nondimensional)| is the original nondimensional data, and
|Data(raw)| is the converted raw data.  This converted raw data is
dimensional, and the optional qualifiers |DimensionalUnits| and
|DimensionalExponents| describe the appropriate dimensional units and
exponents.  Note that |DimensionalUnits| and |DimensionalExponents| also
describe the units for |ConversionScale| and |ConversionOffset|.

If \texttt{DataConversion} is absent, the equivalent defaults are
\texttt{ConversionScale = 1} and \texttt{ConversionOffset = 0}.
If either \texttt{DimensionalUnits} or \texttt{DimensionalExponents} is absent,
follow the rules described in the previous section.

Note that functionally there is little difference between these first two
data classes (\fort{DataClass = Dimensional} and
\fort{NormalizedByDimensional}).
In the first case the data is dimensional, and in the second, the converted
raw data is dimensional.
Also, the equivalent defaults for \fort{DataConversion} produce no changes
in the data; hence, it is almost the same as stating the original data is
dimensional.

\subsubsection{Nondimensional Data Normalized by Unknown Dimensional Quantities}
\label{s:data_normbyunkdim}

If |DataClass = NormalizedByUnknownDimensional|, the data is nondimensional
and is normalized by some unspecified dimensional quantities.  This type of
data is typical of a completely nondimensional test case, where all field
data and all reference quantities are nondimensional.  

Only the |DimensionalExponents| qualifier is used in this case,
although it is expected that this qualifier will be seldom utilized in
practice.  For entities of |DataArray_t| that are not among the list of
standardized data-name identifiers, the qualifier could provide useful
information by defining the exponents of the dimensional form of the
nondimensional data.

Rather than providing qualifiers to describe the normalization of the data,
we instead dictate that all data of type |NormalizedByUnknownDimensional|
in a given database be nondimensionalized consistently.  This is done by
picking one set of mass, length, time and temperature scales and normalizing
all appropriate data by these scales.  We describe this process in detail in
the following.  \hyperref[s:twozone]{Appendix~\ref*{s:twozone}} also shows
a completely nondimensional database where consistent normalization is used
throughout.

The practice of nondimensionalization within flow solvers and other
application codes is quite popular.  The problem with this practice
is that to manipulate the data from a given code, one must often know
the particulars of the nondimensionalization used.  This largely
results from what we call inconsistent normalization---more than
the minimum required scales are used to normalize data within the
code.  For example, in the OVERFLOW flow solver, the following
nondimensionalization is used:
$$
\begin{array}{c@{\qquad}c@{\qquad}c}
 \tilde{x} = x/L, & \tilde{u} = u/c_\infty, & \tilde{\rho} = \rho/\rho_\infty, \\ 
 \tilde{y} = y/L, & \tilde{v} = v/c_\infty, & \tilde{p} = p/(\rho_\infty c_\infty^2), \\
 \tilde{z} = z/L, & \tilde{w} = w/c_\infty, & \tilde{\mu} = \mu/\mu_\infty,
\end{array}
$$
where $(x,y,z)$ are the cartesian coordinates, $(u,v,w)$ are the
cartesian components of velocity, $\rho$ is static density, $p$ is
static pressure, $c$ is the static speed of sound, and $\mu$ is the
molecular viscosity.
In this example, tilde quantities $(\:\tilde{}\:)$ are nondimensional
and all others are dimensional.
Four dimensional scales are used for normalization: $L$ (a unit length),
$\rho_\infty$, $c_\infty$ and $\mu_\infty$.
However, only three fundamental dimensional units are represented: mass,
length and time.  The extra normalizing scale leads to inconsistent
normalization.  The primary consequence of this is additional
nondimensional parameters, such as Reynolds number, appearing in
the nondimensionalized governing equations where none are found in
the original dimensional equations.  Many definitions, including
skin friction coefficient, also have extra terms appearing in the
nondimensionalized form.  This adds unnecessary complication to any data
or equation manipulation associated with the flow solver.

Consistent normalization avoids many of these problems.  Here the
number of scales used for normalization is the same as the number
of fundamental dimensional units represented by the data.  Using
consistent normalization, the resulting nondimensionalized form of
equations and definitions is identical to their original dimensional
formulations.  One piece of evidence to support this assertion is that
it is not possible to form any nondimensional parameters from the set of
dimensional scales used for normalization.

An important fallout of consistent normalization is that the actual
scales used for normalization become immaterial for all data
manipulation processes.  To illustrate this consider the following
nondimensionalization procedure: let $M$ (mass), $L$ (length) and $T$
(time) be arbitrary dimensional scales by which all data is normalized
(neglect temperature data for the present).  The nondimensional data
follows:
$$
\begin{array}{c@{\qquad}c@{\qquad}c}
 x' = x/L, & u' = u/(L/T), & \rho' = \rho/(M/L^3), \\
 y' = y/L, & v' = v/(L/T), & p' = p/(M/(L T^2)), \\ 
 z' = z/L, & w' = w/(L/T), & \mu' = \mu/(M/(L T)), 
\end{array}
$$
where primed quantities are nondimensional and all others are dimensional.

Consider an existing database where all field data and all reference
data is nondimensional and normalized as shown.  Assume the database has
a single reference state given by,
$$
\begin{array}{c@{\qquad}c@{\qquad}c}
 x'_\refer = x_\refer/L, & u'_\refer = u_\refer/(L/T), & \rho'_\refer = \rho_\refer/(M/L^3), \\ 
 y'_\refer = y_\refer/L, & v'_\refer = v_\refer/(L/T), & p'_\refer = p_\refer/(M/(L T^2)) \\
 z'_\refer = z_\refer/L, & w'_\refer = w_\refer/(L/T), & \mu'_\refer = \mu_\refer/(M/(L T)).
\end{array}
$$
If a user wanted to change the nondimensionalization of grid-point
pressures, the procedure is straightforward.
Let the desired new normalization be given by
$p''_{ijk} = p_{ijk} / (\rho_\refer c_\refer^2)$, where all terms on
the right-hand-side are \emph{dimensional}, and as such they are unknown
to the database user.
However, the desired manipulation is possible using only nondimensional
data provided in the database,
$$
\begin{array}{cl}
 p''_{ijk} & \ds {}\equiv p_{ijk} / (\rho_\refer c_\refer^2) \\[.02in]
           & \ds {}= {p_{ijk} \over M/(L T^2)} {M/L^3 \over \rho_\refer}
	             \left[ L/T \over c_\refer \right]^2 \\
 \noalign{\smallskip} \\
           & \ds {}= p'_{ijk} / (\rho'_\refer (c'_\refer)^2)
\end{array}
$$
Thus, the desired renormalization is possible using the database's
nondimensional data as if it were actually dimensional.  There is,
in fact, a high degree of equivalence between dimensional data and
consistently normalized nondimensional data.  The procedure shown in
this example should extend to any desired renormalization, provided the
needed reference-state quantities are included in the database.

This example points out two stipulations that we now dictate for 
data in the class \texttt{NormalizedByUnknownDimensional},
\begin{enumerate}
\item[(a)]
All nondimensional data within a given database that has 
\texttt{DataClass = NormalizedByUnknownDimensional}
shall be consistently normalized.
\item[(b)]
Any nondimensional reference state appearing in a database should
be sufficiently populated with reference quantities to allow for
renormalization procedures.
\end{enumerate}
The second of these stipulations is somewhat ambiguous, but good
practice would suggest that a flow solver, for example, should output
to the database enough static and/or stagnation reference quantities to
sufficiently define the state.

\hyperref[s:twozone]{Appendix~\ref*{s:twozone}} shows an example of a
well-populated reference state.

With these two stipulations, we contend the following:
\begin{itemize}
\item
The dimensional scales used to nondimensionalize all data are
immaterial, and there is no need to identify these quantities in a CGNS
database.
\item
The dimensional scales need not be reference-state quantities provided
in the database.  For example, a given database could contain freestream
reference state conditions, but all the data is normalized by sonic
conditions (which are not provided in the database).
\item
All renormalization procedures can be carried out treating the data as if
it were dimensional with a consistent set of units.
\item
Any application code that internally uses consistent normalization
can use the data provided in a CGNS database without modification or
transformation to the code's internal normalization.
\end{itemize}

Before ending this section, we note that the OVERFLOW flow solver
mentioned above (or any other application code that internally uses
inconsistent normalization) could easily read and write data to a
nondimensional CGNS database that conforms to the above stipulations.
On output, the code could renormalize data so it is consistently
normalized.
Probably, the easiest method would be to remove the molecular viscosity
scale ($\mu_\infty$), and only use $L$, $\rho_\infty$ and $c_\infty$ for
all normalizations (recall these are dimensional scales).
The only change from the above example would be the nondimensionalization
of viscosity, which would become,
$\tilde{\tilde{\mu}} = \mu / (\rho_\infty c_\infty L).$
%$$
% \tilde{\tilde{\mu}} = \mu / (\rho_\infty c_\infty L).
%$$
The code could then output all field data as,
$$
\begin{array}{c@{\qquad}c@{\qquad}c}
 \tilde{x}_{ijk} = x_{ijk}/L, & \tilde{u}_{ijk} = u_{ijk}/c_\infty, & \tilde{\rho}_{ijk} = \rho_{ijk}/\rho_\infty, \\
 \tilde{y}_{ijk} = y_{ijk}/L, & \tilde{v}_{ijk} = v_{ijk}/c_\infty, & \tilde{p}_{ijk} = p_{ijk}/(\rho_\infty c_\infty^2), \\[.02in]
 \tilde{z}_{ijk} = z_{ijk}/L, & \tilde{w}_{ijk} = w_{ijk}/c_\infty, & \tilde{\tilde{\mu}}_{ijk} = \mu_{ijk} / (\rho_\infty c_\infty L),
\end{array}
$$
and output the freestream reference quantities,
$$
\begin{array}{c@{\qquad}c@{\qquad}c}
 \tilde{u}_\infty = u_\infty/c_\infty,     & \tilde{\rho}_\infty = \rho_\infty/\rho_\infty = 1, \\
 \tilde{v}_\infty = v_\infty/c_\infty,     & \tilde{p}_\infty = p_\infty/(\rho_\infty c_\infty^2) = 1/\gamma, \\[.02in]
 \tilde{w}_\infty = w_\infty/c_\infty,     & \tilde{\tilde{\mu}}_\infty = \mu_\infty / (\rho_\infty c_\infty L) \sim O(1/Re), \\
 \tilde{c}_\infty = c_\infty/c_\infty = 1, & \tilde{L} = L/L = 1,
\end{array}
$$
where $\gamma$ is the specific heat ratio (assumes a perfect gas) and
$Re$ is the Reynolds number.

On input, the flow solver should be able to recover its internal
normalizations from the data in a nondimensional CGNS database by
treating the data as if it were dimensional.

\subsubsection{Nondimensional Parameters}

If |DataClass = NondimensionalParameter|, the data is a nondimensional
parameter (or array of nondimensional parameters).  Examples include Mach
number, Reynolds number and pressure coefficient.  These parameters are
prevalent in CFD, although their definitions tend to vary between different
application codes.  A list of standardized data-name identifiers for
nondimensional parameters is provided in
\hyperref[s:dataname_nondim]{Appendix~\ref*{s:dataname_nondim}}.

We distinguish nondimensional parameters from other data classes by
the fact that they are \emph{always} dimensionless.  In a completely
nondimensional database, they are distinct in that their normalization
is not necessarily consistent with other data.

Typically, the |DimensionalUnits|, |DimensionalExponents| and
|DataConversion| qualifiers are not used for nondimensional parameters;
although, there are a few situations where they may be used (these are
discussed below).
Rather than rely on optional qualifiers to describe the normalization,
we establish the convention that \emph{any nondimensional parameters
should be accompanied by their defining scales}; this is further
discussed in \hyperref[s:dataname_nondim]{Appendix~\ref*{s:dataname_nondim}}.
An example is Reynolds number defined as $Re = V L_R / \nu$, where $V$,
$L_R$ and $\nu$ are velocity, length, and viscosity scales, respectively.
Note that these defining scales may be dimensional or nondimensional data.
We establish the data-name identifiers \texttt{Reynolds},
\texttt{Reynolds\_Velocity}, \texttt{Reynolds\_Length} and
\texttt{Reynolds\_ViscosityKinematic} for the Reynolds number and its
defining scales.
Anywhere an instance of |DataArray_t| is found with the identifier
|Reynolds|, there should also be entities for the defining scales.
An example of this use for Reynolds number is given in
\autoref{s:data_example}.

In certain situations, it may be more convenient to use the optional
qualifiers of |DataArray_t| to describe the normalization used
in nondimensional parameters.  These situations must satisfy two
requirements:  First, the defining scales are dimensional; and second,
the nondimensional parameter is a normalization of a single ``raw'' data
quantity and it is clear what this raw data is.  Examples that satisfy
this second constraint are pressure coefficient, where the raw data is
static pressure, and lift coefficient, where the raw data is the lift
force.  Conversely, Reynolds number is a parameter that violates the
second requirement---there are three pieces of raw data rather than one
that make up $Re$.  For nondimensional parameters that satisfy these two
requirements, the qualifiers |DimensionalUnits|, |DimensionalExponents|
and |DataConversion| may be used as in \autoref{s:data_normbydim} to
recover the raw dimensional data.

\subsubsection{Dimensionless Constants}

If |DataClass = DimensionlessConstant|, the data is a constant (or array
of constants) with no associated dimensional units.  The |DimensionalUnits|,
|DimensionalExponents| and |DataConversion| qualifiers are not used.

\subsection{Data-Array Examples}
\label{s:data_example}

This section presents five examples of data-array entities and
illustrates the use of optional information for describing dimensional
and nondimensional data.

\begin{example}{One-Dimensional Data Array, Constants}
\label{ex:data1}

A one-dimensional array of integers; the array is the integers from 1 to 10.
The data is pure constants.
\begin{alltt}
  !  DataType = int
  !  Dimension = 1
  !  DimensionValues = 10
  DataArray\_t<int, 1, 10> Data1 =
    \{\{
    Data(int, 1, 10) = [1, 2, 3, 4, 5, 6, 7, 8, 9, 10] ;
    
    DataClass\_t DataClass = DimensionlessConstant ;
    \}\} ;
\end{alltt}
The structure parameters for |DataArray_t| state the data is an
one-dimensional integer array of length ten.  The value of |DataClass|
indicates the data is unitless constants.
\end{example}

\begin{example}{Two-Dimensional Data Array, Pressures}
\label{ex:data2}

A two-dimensional array of pressures with size $11\times 9$ given by the
array |P(i,j)|.
The data is dimensional with units of N/m\tsup{2} (i.e., kg/(m-s\tsup{2})).
Note that |Pressure| is the data-name identifier for static pressure.
\begin{alltt}
  !  DataType = real
  !  Dimension = 2
  !  DimensionValues = [11,9]
  DataArray\_t<real, 2, [11,9]> Pressure =
    \{\{
    Data(real, 2, [11,9]) = ((P(i,j), i=1,11), j=1,9) ;
    
    DataClass\_t DataClass = Dimensional ;

    DimensionalUnits\_t DimensionalUnits =
      \{\{
      MassUnits        = Kilogram ;
      LengthUnits      = Meter ;
      TimeUnits        = Second ;
      TemperatureUnits = Null ;
      AngleUnits       = Null ;
      \}\} ;
	
    DimensionalExponents\_t DimensionalExponents =
      \{\{
      MassExponent        = +1 ;
      LengthExponent      = -1 ;
      TimeExponent        = -2 ;
      TemperatureExponent =  0 ;
      AngleExponent       =  0 ;
      \}\} ;
    \}\} ;
\end{alltt}
From the data-name identifier conventions presented in
\hyperref[s:dataname]{Appendix~\ref*{s:dataname}}, |Pressure| has a
floating-point data type; hence, the appropriate structure parameter
for |DataArray_t| is |real|.  

The value of |DataClass| indicates that the data is dimensional, and
both the dimensional units and dimensional exponents are provided.
\texttt{DimensionalUnits} specifies that the units are kilograms,
meters, and seconds, and \texttt{DimensionalExponents} specifies the
appropriate exponents for pressure.
Combining the information gives pressure as kg/(m-s\tsup{2}).
\texttt{DimensionalExponents} could have been defaulted, since the
dimensional exponents are given in
\hyperref[s:dataname]{Appendix~\ref*{s:dataname}} for the data-name
identifier |Pressure|.

Note that FORTRAN multidimensional array indexing is used to store the data;
this is reflected in the FORTRAN-like implied do-loops for |P(i,j)|.
\end{example}

\begin{example}{Three-Dimensional Data Array, Nondimensional Static Enthalpy}
\label{ex:data3}

A 3-D array of size $33\times 9\times 17$ containing nondimensional static 
enthalpy.  The data is normalized by freestream velocity as follows:
$$
 \bar{h}_{i,j,k} = {h_{i,j,k} \over q^2_{\rm ref}},
$$
where $\bar{h}_{i,j,k}$ is nondimensional static enthalpy.  The freestream
velocity is dimensional with a value of 10 m/s.  
\begin{alltt}
  !  DataType = real
  !  Dimension = 3
  !  DimensionValues = [33,9,17]
  DataArray\_t<real, 3, [33,9,17]> Enthalpy =
    \{\{
    Data(real, 3, [33,9,17]) = (((H(i,j,k), i=1,33), j=1,9), k=1,17) ;
    
    DataClass\_t DataClass = NormalizedByDimensional ;
    
    DataConversion\_t DataConversion =
      \{\{
      real ConversionScale  = 100 ;
      real ConversionOffset = 0 ;
      \}\} ;
      
    DimensionalUnits\_t DimensionalUnits =
      \{\{
      MassUnits        = Null ;
      LengthUnits      = Meter ;
      TimeUnits        = Second ;
      TemperatureUnits = Null ;
      AngleUnits       = Null ;
      \}\} ;
	
    DimensionalExponents\_t DimensionalExponents =
      \{\{
      MassExponent        =  0 ;
      LengthExponent      = +2 ;
      TimeExponent        = -2 ;
      TemperatureExponent =  0 ;
      AngleExponent       =  0 ;
      \}\} ;
    \}\} ;
\end{alltt}
From \hyperref[s:dataname]{Appendix~\ref*{s:dataname}}, the identifier for
static enthalpy is |Enthalpy| and its data type is |real|.

The value of |DataClass| indicates that the data is nondimensional and
normalized by a dimensional reference quantity.  |DataConversion| provides
the conversion factors for recovering the raw static enthalpy, which has
units of m\tsup{2}/s\tsup{2} as indicated by |DimensionalUnits| and
|DimensionalExponents|.  Note that |DimensionalExponents| could have been
defaulted using the conventions for the data-name identifier |Enthalpy|.
\end{example}

\begin{example}{Three-Dimensional Data Array, Nondimensional Database}
\label{ex:data4}

The previous example for nondimensional enthalpy is repeated for a
completely nondimensional database.
\begin{alltt}
  !  DataType = real
  !  Dimension = 3
  !  DimensionValues = [33,9,17]
  DataArray\_t<real, 3, [33,9,17]> Enthalpy =
    \{\{
    Data(real, 3, [33,9,17]) = (((H(i,j,k), i=1,33), j=1,9), k=1,17) ;
    
    DataClass\_t DataClass = NormalizedByUnknownDimensional ;
    \}\} ;
\end{alltt}
The value of |DataClass| indicates the appropriate class.  
\end{example}

\begin{example}{Data Arrays for Reynolds Number}
\label{ex:data5}

Reynolds number of $1.554 \!\times\! 10^{6}$ based on a velocity scale of
10 m/s, a length scale of 2.3 m and a kinematic viscosity scale of
$1.48 \!\times\! 10^{-5}$ m\tsup{2}/s.
Assume the database has globally set the dimensional units to kilograms,
meters, and seconds, and the global default data class to dimensional
(|DataClass = Dimensional|).
\begin{alltt}
  !  DataType = real
  !  Dimension = 1
  !  DimensionValues = 1
  DataArray\_t<real, 1, 1> Reynolds =
    \{\{
    Data(real, 1, 1) = 1.554e+06 ;
    
    DataClass\_t DataClass = NondimensionalParameter ;
    \}\} ;

  DataArray\_t<real, 1, 1> Reynolds\_Velocity =
    \{\{
    Data(real, 1, 1) = 10. ;
    \}\} ;

  DataArray\_t<real, 1, 1> Reynolds\_Length =
    \{\{
    Data(real, 1, 1) = 2.3 ;
    \}\} ;

  DataArray\_t<real, 1, 1> Reynolds\_ViscosityKinematic =
    \{\{
    Data(real, 1, 1) = 1.48e-05 ;
    \}\} ;
\end{alltt}
|Reynolds| contains the value of the Reynolds number, and the value of its
|DataClass| qualifier designates it as a nondimensional parameter.
By conventions described in
\hyperref[s:dataname_nondim]{Appendix~\ref*{s:dataname_nondim}},
the defining scales are contained in the associated entities
|Reynolds_Velocity|, |Reynolds_Length|, and |Reynolds_ViscosityKinematic|.
Since each of these entities contain no qualifiers, global information
is used to decipher that they are all dimensional with mass, length, and
time units of kilograms, meters, and seconds.
The structure parameters for each |DataArray_t| entity state that they
contain a real scalar.

If a user wanted to convey the dimensional units of the defining scales
using optional qualifiers of |DataArray_t|, then the last three entities in
this example would have a form similar to that in
\hyperref[ex:data2]{Example~\ref*{ex:data2}}.
\end{example}
