\section{Error Handling Routines}
\label{s:errors}

\begin{fctbox}
\textcolor{output}{\textit{ier}} = cgio\_error\_message(\textcolor{output}{\textit{char *error\_msg}}); & - - - \\
void cgio\_error\_code(\textcolor{output}{\textit{int *errcode}}, \textcolor{output}{\textit{int *file\_type}}); & - - - \\
void cgio\_error\_exit(\textcolor{input}{const char *msg}); & - - - \\
void cgio\_error\_abort(\textcolor{input}{int abort\_flag}); & - - - \\
\hline
call cgio\_error\_message\_f(\textcolor{output}{\textit{error\_msg}}, \textcolor{output}{\textit{ier}}) & - - - \\
call cgio\_error\_code\_f(\textcolor{output}{\textit{errcode}}, \textcolor{output}{\textit{file\_type}}) & - - - \\
call cgio\_error\_exit\_f(\textcolor{input}{msg}) & - - - \\
call cgio\_error\_abort\_f(\textcolor{input}{abort\_flag}) & - - - \\
\end{fctbox}

\noindent
\textbf{\textcolor{input}{Input}/\textcolor{output}{\textit{Output}}}

\begin{Ventryi}{\texttt{abort\_flag}}\raggedright
\item [\texttt{error\_msg}]
      Error message from CGIO or the underlying database manager.
\item [\texttt{errcode}]
      The last error code from CGIO or the underlying database manager.
\item [\texttt{file\_type}]
      Where the last error was encountered. \texttt{CGIO\_FILE\_NONE} for an
      error coming from CGIO, else the type of database.
\item [\texttt{msg}]
      An additional message to print, which prefixes the error message
      before exiting. This may be \texttt{NULL} or an empty string, in
      which case it is not printed.
\item [\texttt{abort\_flag}]
      Abort on error flag.
\item [\texttt{ier}]
      Error status.
\end{Ventryi}

\subsection{Function Descriptions}

\subsubsection{\texttt{cgio\_error\_message}} \label{error_message}
    \noindent
    Gets the error message for the last error encountered, and returns
    it in \texttt{error\_msg}, Maximum length of the error message is
    \texttt{CGIO\_MAX\_ERROR\_LENGTH} (80). In C, \texttt{error\_msg} should
    be dimensioned at least 81 in the calling routine
    to allow for the terminating '0'. The function returns the error
    code corresponding to the error message.

\subsubsection{\texttt{cgio\_error\_code}} \label{error_code}
    \noindent
    Returns the last error code in \texttt{errcode} and where is was
    generated in \texttt{file\_type}. If the error code is < 0, then
    the error is from the CGIO library, and \texttt{file\_type} will be
    \texttt{CGIO\_FILE\_NONE}, otherwise \texttt{file\_type} will be the
    type of database.

\subsubsection{\texttt{cgio\_error\_exit}} \label{error_exit}
    \noindent
    Prints \texttt{msg} and any error message to \textit{stderr} and exits.
    The exit code will be \texttt{abort\_flag} if it is set, else -1.
    If \texttt{msg} is \texttt{NULL} or
    an empty string, then it is not printed.

\subsubsection{\texttt{cgio\_error\_abort}} \label{error_abort}
    \noindent
    Sets the flag to abort (exit) when an error is encountered. If
    \texttt{abort\_flag} is non-zero, then an error in the CGIO routines
    or database managers will cause \texttt{cgio\_error\_exit} to be called.
    The exceptions are
    \texttt{cgio\_is\_supported} (\autoref{is_supported}),
    \texttt{cgio\_check\_file} (\autoref{check_file}), and
    \texttt{cgio\_is\_link} (\autoref{is_link}). These routines
    will not cause an abort on an error.

\subsection{Error Messages} \label{s:messages}

\setlength{\LTleft}{0pt}
\setlength{\LTright}{0pt}
\settowidth{\tmplength}{\bold{Code}}
\setlength{\Pwidth}{\linewidth-4\tabcolsep-\tmplength}

\begin{longtable}{c >{\raggedright\arraybackslash}p{\Pwidth}}
\caption[CGIO Errors]{\textbf{CGIO Errors}}
\label{t:cgioerrors}
\\ \hline\hline \\*[-2ex]
\bold{Code} & \bold{Error Message}
\\*[1ex] \hline\hline \\*[-2ex]
\endfirsthead

\multicolumn{2}{l}{{\bfseries \autoref{t:cgioerrors}: CGIO Errors} (\emph{Continued})}
\\*[1ex] \hline\hline \\*[-2ex]
\bold{Code} & \bold{Error Message}
\\*[1ex] \hline\hline \\*[-2ex]
\endhead

\\*[-2ex]\hline
\multicolumn{2}{r}{\emph{Continued on next page}} \\
\endfoot
\\*[-2ex] \hline\hline
\endlastfoot
0   & no error \\
-1  & invalid cgio index \\
-2  & malloc/realloc failed \\
-3  & unknown file open mode \\
-4  & invalid file type \\
-5  & filename is NULL or empty \\
-6  & character string is too small \\
-7  & file was not found \\
-8  & pathname is NULL or empty \\
-9  & no match for pathname \\
-10 & error opening file for reading \\
-11 & file opened in read-only mode \\
-12 & NULL or empty string \\
-13 & invalid configure option \\
-14 & rename of tempfile file failed \\
-15 & too many open files \\
-16 & dimensions exceed that for a 32-bit integer
\end{longtable}

\begin{longtable}{c >{\raggedright\arraybackslash}p{\Pwidth}}
\caption[ADF/HDF5 Errors]{\textbf{ADF/HDF5 Errors}}
\label{t:dberrors}
\\ \hline\hline \\*[-2ex]
\bold{Code} & \bold{Error Message}
\\*[1ex] \hline\hline \\*[-2ex]
\endfirsthead

\multicolumn{2}{l}{{\bfseries \autoref{t:dberrors}: ADF/HDF5 Errors} (\emph{Continued})}
\\*[1ex] \hline\hline \\*[-2ex]
\bold{Code} & \bold{Error Message}
\\*[1ex] \hline\hline \\*[-2ex]
\endhead

\\*[-2ex]\hline
\multicolumn{2}{r}{\emph{Continued on next page}} \\
\endfoot
\\*[-2ex] \hline\hline
\endlastfoot
1   & Integer number is less than a given minimum value \\
2   & Integer value is greater than given maximum value \\
3   & String length of zero of blank string detected \\
4   & String length longer than maximum allowable length \\
5   & String length is not an ASCII-Hex string \\
6   & Too many ADF files opened \\
7   & ADF file status was not recognized \\
8   & ADF file open error \\
9   & ADF file not currently opened \\
10  & ADF file index out of legal range \\
11  & Block/offset out of legal range \\
12  & A string pointer is null \\
13  & FSEEK error \\
14  & FWRITE error \\
15  & FREAD error \\
16  & Internal error: Memory boundary tag bad \\
17  & Internal error: Disk boundary tag bad \\
18  & File Open Error: NEW - File already exists \\
19  & ADF file format was not recognized \\
20  & Attempt to free the RootNode disk information \\
21  & Attempt to free the FreeChunkTable disk information \\
22  & File Open Error: OLD - File does not exist \\
23  & Entered area of unimplemented code \\
24  & Subnode entries are bad \\
25  & Memory allocation failed \\
26  & Duplicate child name under a parent node \\
27  & Node has no dimensions \\
28  & Node's number of dimensions is not in legal range \\
29  & Specified child is not a child of the specified parent \\
30  & Data-Type is too long \\
31  & Invalid Data-Type \\
32  & A pointer is null \\
33  & Node had no data associated with it \\
34  & Error zeroing out of memory \\
35  & Requested data exceeds actual data available \\
36  & Bad end value \\
37  & Bad stride values \\
38  & Minimum value is greater than maximum value \\
39  & The format of this machine does not match a known signature \\
40  & Cannot convert to or from an unknown native format \\
41  & The two conversion formats are equal; no conversion done \\
42  & The data format is not supported on a particular machine \\
43  & File close error \\
44  & Numeric overflow/underflow in data conversion \\
45  & Bad start value \\
46  & A value of zero is not allowable \\
47  & Bad dimension value \\
48  & Error state must be either a 0 (zero) or a 1 (one) \\
49  & Dimensional specifications for disk and memory are unequal \\
50  & Too many link levels are used; may be caused by a recursive link \\
51  & The node is not a link. It was expected to be a link. \\
52  & The linked-to node does not exist \\
53  & The ADF file of a linked node is not accessible \\
54  & A node ID of 0.0 is not valid \\
55  & Incomplete data when reading multiple data blocks \\
56  & Node name contains invalid characters \\
57  & ADF file version incompatible with this library version \\
58  & Nodes are not from the same file \\
59  & Priority stack error \\
60  & Machine format and file format are incomplete \\
61  & Flush error \\
62  & The node ID pointer is NULL \\
63  & The maximum size for a file exceeded \\
64  & Dimensions exceed that for a 32-bit integer \\
70  & H5Glink:soft link creation failed \\
71  & Node attribute doesn't exist \\
72  & H5Aopen:open of node attribute failed \\
73  & H5Iget\_name:failed to get node path from ID \\
74  & H5Gmove:moving a node group failed \\
75  & H5Gunlink:node group deletion failed \\
76  & H5Gopen:open of a node group failed \\
77  & H5Dget\_space:couldn't get node dataspace \\
78  & H5Dopen:open of the node data failed \\
79  & H5Dextend:couldn't extend the node dataspace \\
80  & H5Dcreate:node data creation failed \\
81  & H5Screate\_simple:dataspace creation failed \\
82  & H5Acreate:node attribute creation failed \\
83  & H5Gcreate:node group creation failed \\
84  & H5Dwrite:write to node data failed \\
85  & H5Dread:read of node data failed \\
86  & H5Awrite:write to node attribute failed \\
87  & H5Aread:read of node attribute failed \\
88  & H5Fmount:file mount failed \\
89  & Can't move a linked-to node \\
90  & Can't change the data for a linked-to node \\
91  & Parent of node is a link \\
92  & Can't delete a linked-to node \\
93  & File does not exist or is not a HDF5 file \\
94  & unlink (delete) of file failed \\
95  & couldn't get file index from node ID \\
96  & H5Tcopy:copy of existing datatype failed \\
97  & H5Aget\_type:couldn't get attribute datatype \\
98  & H5Tset\_size:couldn't set datatype size \\
99  & routine not implemented \\
100 & H5L: Link target is not an HDF5 external link \\
101 & HDF5: No external link feature available \\
102 & HDF5: Internal problem with objinfo \\
103 & HDF5: No value for external link \\
104 & HDF5: Cannot unpack external link \\
106 & HDF5: Root descriptor is NULL \\
107 & dimensions need transposed - open in modify mode \\
108 & invalid configuration option
\end{longtable}
