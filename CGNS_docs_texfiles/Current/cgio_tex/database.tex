\section{Database-Level Routines}
\label{s:database}

\begin{fctbox}
\textcolor{output}{\textit{ier}} = cgio\_is\_supported(\textcolor{input}{int file\_type}); & - - - \\
\textcolor{output}{\textit{ier}} = cgio\_check\_file(\textcolor{input}{const char *filename}, \textcolor{output}{\textit{int *file\_type}}); & - - - \\
\textcolor{output}{\textit{ier}} = cgio\_open\_file(\textcolor{input}{const char *filename}, \textcolor{input}{int file\_mode}, & r w m \\
~~~~~~~\textcolor{input}{int file\_type}, \textcolor{output}{\textit{int *cgio\_num}}); & \\
\textcolor{output}{\textit{ier}} = cgio\_close\_file(\textcolor{input}{int cgio\_num}); & r w m \\
\textcolor{output}{\textit{ier}} = cgio\_get\_file\_type(\textcolor{input}{int cgio\_num}, \textcolor{output}{\textit{int *file\_type}}); & r w m \\
\textcolor{output}{\textit{ier}} = cgio\_get\_root\_id(\textcolor{input}{int cgio\_num}, \textcolor{output}{\textit{double *rootid}}); & r w m \\
\hline
call cgio\_is\_supported\_f(\textcolor{input}{file\_type}, \textcolor{output}{\textit{ier}}) & - - - \\
call cgio\_check\_file\_f(\textcolor{input}{filename}, \textcolor{output}{\textit{file\_type}}, \textcolor{output}{\textit{ier}}) & - - - \\
call cgio\_open\_file\_f(\textcolor{input}{filename}, \textcolor{input}{file\_mode}, \textcolor{input}{file\_type}, \textcolor{output}{\textit{cgio\_num}}, \textcolor{output}{\textit{ier}}) & r w m \\
call cgio\_close\_file\_f(\textcolor{input}{cgio\_num}, \textcolor{output}{\textit{ier}}) & r w m \\
call cgio\_get\_file\_type\_f(\textcolor{input}{cgio\_num}, \textcolor{output}{\textit{file\_type}}, \textcolor{output}{\textit{ier}}) & r w m \\
call cgio\_get\_root\_id\_f(\textcolor{input}{cgio\_num}, \textcolor{output}{\textit{rootid}}, \textcolor{output}{\textit{ier}}) & r w m \\
\end{fctbox}

\noindent
\textbf{\textcolor{input}{Input}/\textcolor{output}{\textit{Output}}}

\begin{Ventryi}{\texttt{file\_type}}\raggedright
\item [\texttt{file\_type}]
      Type of database file. acceptable values are \texttt{CGIO\_FILE\_NONE},
      \texttt{CGIO\_FILE\_ADF}, \texttt{CGIO\_FILE\_HDF5} and
      \texttt{CGIO\_FILE\_ADF2}.
\item [\texttt{filename}]
      Name of the database file, including path name if necessary.
      There is no limit on the length of this character variable.
\item [\texttt{file\_mode}]
      Mode used for opening the file.
      The supported modes are \texttt{CGIO\_MODE\_READ},
      \texttt{CGIO\_MODE\_WRITE}, and \texttt{CGIO\_MODE\_MODIFY}.
\item [\texttt{cgio\_num}]
      Indentifier for the open database file.
\item [\texttt{rootid}]
      Ndeo identifier for the root node of the database.
\item [\texttt{ier}]
      Error status.
\end{Ventryi}

\subsection{Function Descriptions}

\subsubsection{\texttt{cgio\_is\_supported}} \label{is_supported}
    \noindent
    Determines if the database type given by \texttt{file\_type} is supported
    by the library. Retuns 0 if supported, else \texttt{CGIO\_ERR\_FILE\_TYPE}
    if not. \texttt{CGIO\_FILE\_ADF} is always supported;
    \texttt{CGIO\_FILE\_HDF5} is supported if the library was built with HDF5;
    and \texttt{CGIO\_FILE\_ADF2} is supported when built in 32-bit mode.

\subsubsection{\texttt{cgio\_check\_file}} \label{check_file}
    \noindent
    Checks the file \texttt{filename} to determine if it is a valid database.
    If so, returns 0 and the type of database in \texttt{file\_type},
    otherwise returns an error code and \texttt{file\_type} will be set
    to \texttt{CGIO\_FILE\_NONE}.

\subsubsection{\texttt{cgio\_open\_file}} \label{open_file}
    \noindent
    Opens a database file of the specified type and mode. If successfull,
    returns 0, and the database identifier in \texttt{cgio\_num}, otherwise
    returns an error code. The database identifier is used to access the
    database in subsequent function calls.

    \noindent
    The mode in which the database is opened is given by \texttt{file\_mode},
    which may take the value \texttt{CGIO\_MODE\_READ}, \texttt{CGIO\_MODE\_WRITE},
    or \texttt{CGIO\_MODE\_MODIFY}. New databases should be opened with
    \texttt{CGIO\_MODE\_WRITE}, while existing databases are opened with either
    \texttt{CGIO\_MODE\_READ} (for read-only access) or
    \texttt{CGIO\_MODE\_MODIFY} (for read/write access).
 
    \noindent
    A specific database type may be specified by \texttt{file\_type}, which may
    be one of \texttt{CGIO\_FILE\_NONE}, \texttt{CGIO\_FILE\_ADF},
    \texttt{CGIO\_FILE\_HDF5}, or \texttt{CGIO\_FILE\_ADF2}. When opening a database
    in write mode, \texttt{CGIO\_FILE\_NONE} indicates that the default database
    type should be used, otherwise the specified database type will be opened.
    When opening in read or modify mode, \texttt{CGIO\_FILE\_NONE} indicates that
    any database type is acceptable, otherwise if the database type does
    not match that given by \texttt{file\_type} an error will be retuned.

\subsubsection{\texttt{cgio\_close\_file}} \label{close_file}
    \noindent
    Closes the database given by \texttt{cgio\_num}. Returns 0 for success,
    else an error code.

\subsubsection{\texttt{cgio\_get\_file\_type}} \label{get_file_type}
    \noindent
    Gets the type of the database given by \texttt{cgio\_num}.
    Returns 0 and the type in \texttt{file\_type} if successfull,
    else an error code.

\subsubsection{\texttt{cgio\_get\_root\_id}} \label{get_root_id}
    \noindent
    Gets the unique node identifier for the root node in the database given
    by \texttt{cgio\_num}. Returns 0 and the identifier in \texttt{rootid}
    if successfull, else an error code.


