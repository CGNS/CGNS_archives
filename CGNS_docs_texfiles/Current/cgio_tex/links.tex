\section{Link Management Routines}
\label{s:links}

\begin{fctbox}
\textcolor{output}{\textit{ier}} = cgio\_is\_link(\textcolor{input}{int cgio\_num}, \textcolor{input}{double id}, \textcolor{output}{\textit{int *link\_len}}); & r w m \\
\textcolor{output}{\textit{ier}} = cgio\_link\_size(\textcolor{input}{int cgio\_num}, \textcolor{input}{double id}, \textcolor{output}{\textit{int *file\_len}}, & r w m \\
~~~~~~~\textcolor{output}{\textit{int *name\_len}}); & \\
\textcolor{output}{\textit{ier}} = cgio\_create\_link(\textcolor{input}{int cgio\_num}, \textcolor{input}{double pid}, \textcolor{input}{const char *name}, & - w m \\
~~~~~~~\textcolor{input}{const char *filename}, \textcolor{input}{const char *name\_in\_file}, \textcolor{output}{\textit{double *id}}); & \\
\textcolor{output}{\textit{ier}} = cgio\_get\_link(\textcolor{input}{int cgio\_num}, \textcolor{input}{double id}, \textcolor{output}{\textit{char *filename}}, & r w m \\
~~~~~~~\textcolor{output}{\textit{char *name\_in\_file}}); & \\
\hline
call cgio\_is\_link\_f(\textcolor{input}{cgio\_num}, \textcolor{input}{id}, \textcolor{output}{\textit{link\_len}}, \textcolor{output}{\textit{ier}}) & r w m \\
call cgio\_link\_size\_f(\textcolor{input}{cgio\_num}, \textcolor{input}{id}, \textcolor{output}{\textit{file\_len}}, \textcolor{output}{\textit{name\_len}}, \textcolor{output}{\textit{ier}}) & r w m \\
call cgio\_create\_link\_f(\textcolor{input}{cgio\_num}, \textcolor{input}{pid}, \textcolor{input}{name}, \textcolor{input}{filename}, \textcolor{input}{name\_in\_file}, & - w m \\
~~~~~\textcolor{output}{\textit{id}}, \textcolor{output}{\textit{ier}}) & \\
call cgio\_get\_link\_f(\textcolor{input}{cgio\_num}, \textcolor{input}{id}, \textcolor{output}{\textit{filename}}, \textcolor{output}{\textit{name\_in\_file}}, \textcolor{output}{\textit{ier}}) & r w m \\
\end{fctbox}

\noindent
\textbf{\textcolor{input}{Input}/\textcolor{output}{\textit{Output}}}

\begin{Ventryi}{\texttt{name\_in\_file}}\raggedright
\item [\texttt{cgio\_num}]
      Indentifier for the open database file.
\item [\texttt{id}]
      Node identifier.
\item [\texttt{pid}]
      Parent node identifier.
\item [\texttt{link\_len}]
      Total length of the link information (\texttt{file\_len + name\_len}).
\item [\texttt{file\_len}]
      Length of the name of the linked-to file.
      This will be 0 if this is an internal link.
\item [\texttt{name\_len}]
      Length of the pathname of the linked-to node.
\item [\texttt{name}]
      Name of the link node.
\item [\texttt{filename}]
      Name of the linked-to file. If creating an internal link, then this
      should be \texttt{NULL} or an empty string. When reading an internal
      link, this will be returned as an empty string.
\item [\texttt{name\_in\_file}]
      Pathname of the linked-to node.
\item [\texttt{ier}]
      Error status.
\end{Ventryi}

\subsection{Function Descriptions}

\subsubsection{\texttt{cgio\_is\_link}} \label{is_link}
    \noindent
    Determines if the node indentified by \texttt{id} in the database given by
    \texttt{cgio\_num} is a link or not. The function returns 0 if successfull,
    else an error code. If this node is a link, then the total length of the
    linked-to file and node information in returned in \texttt{link\_len}.
    If the node is not a link, \texttt{link\_len} will be 0.

\subsubsection{\texttt{cgio\_link\_size}} \label{link_size}
    \noindent
    Gets the size of the linked-to file name in \texttt{file\_len} and the
    node pathname length in \texttt{name\_len} for the node identified by
    \texttt{id} in the database given by \texttt{cgio\_num}. The function
    returns 0 for success, else an error code. If this is an internal link
    (link to a node in the same database), then \texttt{file\_len} will be
    returned as 0.

\subsubsection{\texttt{cgio\_create\_link}} \label{create_link}
    \noindent
    Creates a link node as a child of the parent node identified by \texttt{pid}
    in the database given by \texttt{cgio\_num}. The name of the node is given
    by \texttt{name}, the name of the linked-to file by \texttt{filename}, and
    the pathname to the linked-to node by \texttt{name\_in\_file}. If this is an
    internal link (link to a node in the same database), then \texttt{filename}
    should be defined as \texttt{NULL} or an empty string. The function
    returns 0 and the indentifier of the new node in \texttt{id} on success,
    otherwise an error code is returned.

\subsubsection{\texttt{cgio\_get\_link}} \label{get_link}
    \noindent
    Gets the link information for the node identified by \texttt{id} in the
    database given by \texttt{cgio\_num}. If successfull, the function returns
    0 and the linked-to file name in \texttt{filename} and the node pathname
    in \texttt{name\_in\_file}. These strings are '0'-terminated, and thus
    should be dimensioned at least (\texttt{file\_len + 1}) and
    (\texttt{name\_len + 1}), respectively If this is an internal link
    (link to a node in the same database), then \texttt{filename} will be an
    empty string. The maximum length for a file name is given by
    \texttt{CGIO\_MAX\_FILE\_LENGTH} (1024) and for a link pathname by
    \texttt{CGIO\_MAX\_LINK\_LENGTH} (4096).

