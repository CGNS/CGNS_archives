\section{Data Structure Management Routines}
\label{s:structure}

\begin{fctbox}
\textcolor{output}{\textit{ier}} = cgio\_create\_node(\textcolor{input}{int cgio\_num}, \textcolor{input}{double pid}, \textcolor{input}{const char *name}, & - w m \\
~~~~~~~\textcolor{output}{\textit{double *id}}); & \\
\textcolor{output}{\textit{ier}} = cgio\_new\_node(\textcolor{input}{int cgio\_num}, \textcolor{input}{double pid}, \textcolor{input}{const char *name}, & - w m \\
~~~~~~~\textcolor{input}{const char *label}, \textcolor{input}{const char *data\_type}, \textcolor{input}{int ndims}, & \\
~~~~~~~\textcolor{input}{const cgsize\_tt *dims}, \textcolor{input}{const void *data}, \textcolor{output}{\textit{double *id}}); & \\
\textcolor{output}{\textit{ier}} = cgio\_delete\_node(\textcolor{input}{int cgio\_num}, \textcolor{input}{double pid}, \textcolor{input}{double id}); & - w m \\
\textcolor{output}{\textit{ier}} = cgio\_move\_node(\textcolor{input}{int cgio\_num}, \textcolor{input}{double pid}, \textcolor{input}{double id}, & - w m \\
~~~~~~~\textcolor{input}{double new\_pid}); & \\
\textcolor{output}{\textit{ier}} = cgio\_number\_children(\textcolor{input}{int cgio\_num}, \textcolor{input}{double id}, \textcolor{output}{\textit{int *num\_child}}); & r w m \\
\textcolor{output}{\textit{ier}} = cgio\_children\_names(\textcolor{input}{int cgio\_num}, \textcolor{input}{double id}, \textcolor{input}{int start}, & r w m \\
~~~~~~~\textcolor{input}{int max\_ret}, \textcolor{input}{int name\_len}, \textcolor{output}{\textit{int *num\_ret}}, \textcolor{output}{\textit{char *child\_names}}); & \\
\textcolor{output}{\textit{ier}} = cgio\_children\_ids(\textcolor{input}{int cgio\_num}, \textcolor{input}{double id}, \textcolor{input}{int start}, & r w m \\
~~~~~~~\textcolor{input}{int max\_ret}, \textcolor{output}{\textit{int *num\_ret}}, \textcolor{output}{\textit{char *child\_ids}}); & \\
\hline
call cgio\_create\_node\_f(\textcolor{input}{cgio\_num}, \textcolor{input}{pid}, \textcolor{input}{name}, \textcolor{output}{\textit{id}}, \textcolor{output}{\textit{ier}}) & - w m \\
call cgio\_new\_node\_f(\textcolor{input}{cgio\_num}, \textcolor{input}{pid}, \textcolor{input}{name}, \textcolor{input}{label}, \textcolor{input}{data\_type}, \textcolor{input}{ndims}, & - w m \\
~~~~~\textcolor{input}{dims}, \textcolor{input}{data}, \textcolor{output}{\textit{id}}, \textcolor{output}{\textit{ier}}); & \\
call cgio\_delete\_node\_f(\textcolor{input}{cgio\_num}, \textcolor{input}{pid}, \textcolor{input}{id}, \textcolor{output}{\textit{ier}}) & - w m \\
call cgio\_move\_node\_f(\textcolor{input}{cgio\_num}, \textcolor{input}{pid}, \textcolor{input}{id}, \textcolor{input}{new\_pid}, \textcolor{output}{\textit{ier}}) & - w m \\
call cgio\_number\_children\_f(\textcolor{input}{cgio\_num}, \textcolor{input}{id}, \textcolor{output}{\textit{num\_child}}, \textcolor{output}{\textit{ier}}) & r w m \\
call cgio\_children\_names\_f(\textcolor{input}{cgio\_num}, \textcolor{input}{id}, \textcolor{input}{start}, \textcolor{input}{max\_ret}, \textcolor{input}{name\_len}, & r w m \\
~~~~~\textcolor{output}{\textit{num\_ret}}, \textcolor{output}{\textit{child\_names}}, \textcolor{output}{\textit{ier}}) & \\
call cgio\_children\_ids\_f(\textcolor{input}{cgio\_num}, \textcolor{input}{id}, \textcolor{input}{start}, \textcolor{input}{max\_ret}, & r w m \\
~~~~~\textcolor{output}{\textit{num\_ret}}, \textcolor{output}{\textit{child\_ids}}, \textcolor{output}{\textit{ier}}) & \\
\end{fctbox}

\noindent
\textbf{\textcolor{input}{Input}/\textcolor{output}{\textit{Output}}}

\begin{Ventryi}{\texttt{child\_names}}\raggedright
\item [\texttt{cgio\_num}]
      Database identifier.
\item [\texttt{pid}]
      Parent node identifier.
\item [\texttt{id}]
      Node identifier.
\item [\texttt{name}]
      Node name (max length 32).
\item [\texttt{label}]
      Node label (max length 32).
\item [\texttt{data\_type}]
      Type of data contained in the node. One of ``MT'', ``I4'', ``I8'',
      ``U4'', ``U8'', ``R4'', ``C1'', or ``B1''.
\item [\texttt{ndims}]
      Number of dimensions for the data (max 12).
\item [\texttt{dims}]
      Data dimension values (\texttt{ndims} values).
\item [\texttt{data}]
      Data array to be stored with the node.
\item [\texttt{new\_pid}]
      New parent node identifier under which the node is to be moved.
\item [\texttt{num\_child}]
      Number of children of the specified node.
\item [\texttt{start}]
      Starting index for returned child names or ids
      (1 <= \texttt{start} <= \texttt{num\_child}).
\item [\texttt{max\_ret}]
      Maximum child names or ids to be returned
      (1 <= \texttt{max\_ret} <= \texttt{num\_child-start+1}).
\item [\texttt{name\_len}]
      Length reserved for each returned child name.
\item [\texttt{num\_ret}]
      Number of returned values of child names or identifiers.
\item [\texttt{child\_names}]
      Child node names (\texttt{num\_ret} values). This array should
      be dimensioned at least (\texttt{name\_len * max\_ret}).
\item [\texttt{child\_ids}]
      Child node identifiers (\texttt{num\_ret} values). This array
       should be dimensioned at least (\texttt{max\_ret}).
\item [\texttt{ier}]
      Error status.
\end{Ventryi}

\subsection{Function Descriptions}

\subsubsection{\texttt{cgio\_create\_node}} \label{create_node}
    \noindent
    Creates a new empty node in the database given by \texttt{cgio\_num}
    as a child of the node identified by \texttt{pid}. The name of the new
    node is given by \texttt{name}, and must not already exist as a child
    of the parent node. The node will contain no label, dimensions,
    or data. Use the \hyperlink{s:node}{Node Management Routines} to
    change the properties of the node, and the
    \hyperlink{s:dataio}{Data I/O Routines} to add data. Returns 0 and the
    identifier of the new node in \texttt{id} on success, else an error code.

\subsubsection{\texttt{cgio\_new\_node}} \label{new_node}
    \noindent
    Creates a new node in the database given by \texttt{cgio\_num}
    as a child of the node identified by \texttt{pid}. The name of the new
    node is given by \texttt{name}, and must not already exist as a child
    of the parent node. The node label is given by \texttt{label},
    the type of data by \texttt{data\_type}, the dimensions of the data by
    \texttt{ndims} and \texttt{dims}, and the data to write to the node
    by \texttt{data}. This is equivalent to calling the routines:
    \hyperlink{create_node}{\texttt{cgio\_create\_node}},
    \hyperlink{set_label}{\texttt{cgio\_set\_label}}.
    \hyperlink{set_dimensions}{\texttt{cgio\_set\_dimensions}}, and
    \hyperlink{write_all_data}{\texttt{cgio\_write\_all\_data}}
    Returns 0 and the identifier of the new node in \texttt{id} on success,
    else an error code.

\subsubsection{\texttt{cgio\_delete\_node}} \label{delete_node}
    \noindent
    Deletes the node identified by \texttt{id} below the parent node identified
    by \texttt{pid} in the database given by \texttt{cgio\_num}. All children
    of the deleted node will also be deleted unless a link is encountered.
    The link node will be deleted but nothing below it.
    Returns 0 on success, else an error code.

\subsubsection{\texttt{cgio\_move\_node}} \label{move_node}
    \noindent
    Moves the node indentified by \texttt{id} below the parent node identified
    by \texttt{pid} to below the new parent node identified by \texttt{new\_pid}
    in the database given by \texttt{cgio\_num}. A node by the same name as
    that that for \texttt{id} must not already exist under \texttt{new\_pid}.
    A node may only be moved if it and the parent nodes all reside in the
    sane physical database. Returns 0 on success, else an error code.

\subsubsection{\texttt{cgio\_number\_children}} \label{number_children}
    \noindent
    Gets the number of children of the node identified by \texttt{id} in the
    database given by \texttt{cgio\_num}, Returns 0 and the number of children
    in \texttt{num\_child} on success, else an error code.

\subsubsection{\texttt{cgio\_children\_names}} \label{children_names}
    \noindent
    Gets the names of the children of the node identified by \texttt{id} in
    the database given by \texttt{cgio\_num}. The starting index for the
    array of names is given by \texttt{start}, and the maximum number of
    names to return by \texttt{max\_ret}. Both \texttt{start} and \texttt{max\_ret}
    should be between 1 and \texttt{num\_child}, inclusively. The size reserved
    for each name in \texttt{child\_names} is given by \texttt{name\_len}.
    The array \texttt{child\_names} should be dimensioned at least
    (\texttt{name\_len * max\_ret}). Since node names are limited to a length
    of \texttt{CGIO\_MAX\_NAME\_LENGHT} (32), \texttt{name\_len} should be at
    least 32 to ensure the returned names are mot truncated. In C, an
    additional byte should be added to \texttt{name\_len} allow for the
    terminating '0' for each name. If successfull, the function returns 0;
    the actual number of returned names is given by \texttt{num\_ret},
    and the array of names in \texttt{child\_names}. In C, the names are
    '0'-terminated within each name field. In Fortran, any unused space
    is padded with blanks (space character).

\subsubsection{\texttt{cgio\_children\_ids}} \label{children_ids}
    \noindent
    Gets the node identifiers of the children of the node identified by
    \texttt{id} in the database given by \texttt{cgio\_num}.
    The starting index for the array of ids is given by \texttt{start},
    and the maximum ids to return by \texttt{max\_ret}. Both \texttt{start}
    and \texttt{max\_ret} should be between 1 and \texttt{num\_child}, inclusively.
    The array \texttt{child\_ids} should be dimensioned at least
    (\texttt{max\_ret}). If successfull, the function returns 0; the
    actual number of returned ids is given by \texttt{num\_ret}, and the
    array of identifiers in \texttt{child\_ids}.

