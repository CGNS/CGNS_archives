\section{Sample Fortran Program}
\label{s:sampleFortran}
\thispagestyle{plain}

The following Fortran program builds the ADF file shown in
\autoref{f:example} on p.~\pageref*{f:example}.

\begin{verbatim}
         PROGRAM TEST
   C
   C     SAMPLE ADF TEST PROGRAM TO BUILD ADF FILES ILLUSTRATED
   C     IN THE EXAMPLE DATABASE FIGURE
   C
         PARAMETER (MAXCHR=32)
   C
         CHARACTER*(MAXCHR) TSTLBL,DTYPE
         CHARACTER*(MAXCHR) FNAM,PATH
   C
         REAL*8 RID,PID,CID,TMPID,RIDF2
         REAL A(4,3),B(4,3)
         INTEGER IC(6),ID(6)
         INTEGER IERR
         INTEGER IDIM(2),IDIMA(2),IDIMC,IDIMD
   C
         DATA A /1.1,2.1,3.1,4.1,
        X        1.2,2.2,3.2,4.2,
        X        1.3,2.3,3.3,4.3/
         DATA IDIMA /4,3/
   C
         DATA IC /1,2,3,4,5,6/
         DATA IDIMC /6/
   C
   C     SET ERROR FLAG TO ABORT ON ERROR
   C
         CALL ADFSES(1,IERR)
   C
   C *** 1.) OPEN 1ST DATABASE (ADF_FILE_TWO.ADF)
   C     2.) CREATE THREE NODES AT FIRST LEVEL
   C     3.) PUT LABEL ON NODE F3
   C     4.) PUT DATA IN F3
   C     5.) CREATE TWO NODES BELOW F3
   C     6.) CLOSE DATABASE
   C
         CALL ADFDOPN('adf_file_two.adf','NEW',' ',RID,IERR)
         RIDF2 = RID
         CALL ADFCRE(RID,'F1',TMPID,IERR)
         CALL ADFCRE(RID,'F2',TMPID,IERR)
         CALL ADFCRE(RID,'F3',PID,IERR)
         CALL ADFSLB(PID,'LABEL ON NODE F3',IERR)
         CALL ADFPDIM(PID,'R4',2,IDIMA,IERR)
         CALL ADFWALL(PID,A,IERR)
   C
         CALL ADFCRE(PID,'F4',CID,IERR)
   C
         CALL ADFCRE(PID,'F5',CID,IERR)
   C
         CALL ADFDCLO(RID,IERR)
   C
   C *** 1.) OPEN 2ND DATABASE
   C     2.) CREATE NODES
   C     3.) PUT DATA IN N13
   C
         CALL ADFDOPN('adf_file_one.adf','NEW',' ',RID,IERR)
   C
   C     THREE NODES UNDER ROOT
   C
         CALL ADFCRE(RID,'N1',TMPID,IERR)
         CALL ADFCRE(RID,'N2',TMPID,IERR)
         CALL ADFCRE(RID,'N3',TMPID,IERR)
   C
   C     THREE NODES UNDER N1 (TWO REGULAR AND ONE LINK)
   C
         CALL ADFGNID(RID,'N1',PID,IERR)
         CALL ADFCRE(PID,'N4',TMPID,IERR)
         CALL ADFLINK(PID,'L3','adf_file_two.adf','/F3',TMPID,IERR)
         CALL ADFCRE(PID,'N5',TMPID,IERR)
   C
   C     TWO NODES UNDER N4
   C
         CALL ADFGNID(PID,'N4',CID,IERR)
         CALL ADFCRE(CID,'N6',TMPID,IERR)
         CALL ADFCRE(CID,'N7',TMPID,IERR)
   C
   C     ONE NODE UNDER N6
   C
         CALL ADFGNID(RID,'/N1/N4/N6',PID,IERR)
         CALL ADFCRE(PID,'N8',TMPID,IERR)
   C
   C     THREE NODES UNDER N3
   C
         CALL ADFGNID(RID,'N3',PID,IERR)
         CALL ADFCRE(PID,'N9',TMPID,IERR)
         CALL ADFCRE(PID,'N10',TMPID,IERR)
         CALL ADFCRE(PID,'N11',TMPID,IERR)
   C
   C     TWO NODES UNDER N9
   C
         CALL ADFGNID(PID,'N9',CID,IERR)
         CALL ADFCRE(CID,'N12',TMPID,IERR)
         CALL ADFCRE(CID,'N13',TMPID,IERR)
   C
   C     PUT LABEL AND DATA IN N13
   C
         CALL ADFSLB(TMPID,'LABEL ON NODE N13',IERR)
         CALL ADFPDIM(TMPID,'I4',1,IDIMC,IERR)
         CALL ADFWALL(TMPID,IC,IERR)
   C
   C     TWO NODES UNDER N10
   C
         CALL ADFGNID(RID,'/N3/N10',PID,IERR)
         CALL ADFLINK(PID,'L1',' ','/N3/N9/N13',TMPID,IERR)
         CALL ADFCRE(PID,'N14',TMPID,IERR)
   C
   C     TWO NODES UNDER N11
   C
         CALL ADFGNID(RID,'/N3/N11',PID,IERR)
         CALL ADFLINK(PID,'L2',' ','/N3/N9/N13',TMPID,IERR)
         CALL ADFCRE(PID,'N15',TMPID,IERR)
   C
   C *** READ AND PRINT DATA FROM NODES
   C     1.) NODE F5 THROUGH LINK L3
   C
         CALL ADFGNID(RID,'/N1/L3',PID,IERR)
         CALL ADFGLB(PID,TSTLBL,IERR)
         CALL ADFGDT(PID,DTYPE,IERR)
         CALL ADFGND(PID,NUMDIM,IERR)
         CALL ADFGDV(PID,IDIM,IERR)
         CALL ADFRALL(PID,B,IERR)
         PRINT *,' NODE F3 THROUGH LINK L3:'
         PRINT *,'   LABEL       = ',TSTLBL
         PRINT *,'   DATA TYPE   = ',DTYPE
         PRINT *,'   NUM OF DIMS = ',NUMDIM
         PRINT *,'   DIM VALS    = ',IDIM
         PRINT *,'   DATA:'
         WRITE(*,100)((B(J,I),I=1,3),J=1,4)
     100 FORMAT(5X,3F10.2)
   C
   C     2.) N13
   C
         CALL ADFGNID(RID,'N3/N9/N13',PID,IERR)
         CALL ADFGLB(PID,TSTLBL,IERR)
         CALL ADFGDT(PID,DTYPE,IERR)
         CALL ADFGND(PID,NUMDIM,IERR)
         CALL ADFGDV(PID,IDIMD,IERR)
         CALL ADFRALL(PID,ID,IERR)
         PRINT *,' '
         PRINT *,' NODE N13:'
         PRINT *,'   LABEL       = ',TSTLBL
         PRINT *,'   DATA TYPE   = ',DTYPE
         PRINT *,'   NUM OF DIMS = ',NUMDIM
         PRINT *,'   DIM VALS    = ',IDIMD
         PRINT *,'   DATA:'
         WRITE(*,200)(ID(I),I=1,6)
     200 FORMAT(5X,6I6)
   C
   C     3.) N13 THROUGH L1
   C
         CALL ADFGNID(RID,'N3/N10/L1',TMPID,IERR)
         CALL ADFGLB(TMPID,TSTLBL,IERR)
         CALL ADFRALL(TMPID,ID,IERR)
         PRINT *,' '
         PRINT *,' NODE N13 THROUGH LINK L1:'
         PRINT *,'   LABEL       = ',TSTLBL
         PRINT *,'   DATA:'
         WRITE(*,200)(ID(I),I=1,6)
   C
   C     4.) N13 THROUTH L2
   C
         CALL ADFGNID(RID,'N3/N11/L2',CID,IERR)
         CALL ADFGLB(CID,TSTLBL,IERR)
         CALL ADFRALL(CID,ID,IERR)
         PRINT *,' '
         PRINT *,' NODE N13 THROUGH LINK L2:'
         PRINT *,'   LABEL       = ',TSTLBL
         PRINT *,'   DATA:'
         WRITE(*,200)(ID(I),I=1,6)
   C
   C     PRINT LIST OF CHILDREN UNDER ROOT NODE
   C
         CALL PRTCLD(RID)
   C
   C     PRINT LIST OF CHILDREN UNDER N3
   C
         CALL ADFGNID(RID,'N3',PID,IERR)
         CALL PRTCLD(PID)
   C
   C     REOPEN ADF_FILE_TWO AND GET NEW ROOT ID
   C
         CALL ADFDOPN('adf_file_two.adf','OLD',' ',RID,IERR)
         PRINT *,' '
         PRINT *,' COMPARISON OF ROOT ID: '
         PRINT *,' ADF_FILE_TWO.ADF ORIGINAL ROOT ID = ',RIDF2
         PRINT *,' ADF_FILE_TWO.ADF NEW ROOT ID      = ',RID
   C
         STOP
         END
   C
   C ************* SUBROUTINES ****************
   C
         SUBROUTINE PRTCLD(PID)
   C
   C *** PRINT TABLE OF CHILDREN GIVEN A PARENT NODE-ID
   C
         PARAMETER (MAXCLD=10)
         PARAMETER (MAXCHR=32)
         REAL*8 PID
         CHARACTER*(MAXCHR) NODNAM,NDNMS(MAXCLD)
         CALL ADFGNAM(PID,NODNAM,IERR)
         CALL ADFNCLD(PID,NUMC,IERR)
         WRITE(*,120)NODNAM,NUMC
     120 FORMAT(/,' PARENT NODE NAME = ',A,/,
        X       '     NUMBER OF CHILDREN = ',I2,/,
        X       '     CHILDREN NAMES:')
         NLEFT = NUMC
         ISTART = 1
   C     --- TOP OF DO-WHILE LOOP
     130 CONTINUE
            CALL ADFCNAM(PID,ISTART,MAXCLD,LEN(NDNMS),
        X                NUMRET,NDNMS,IERR)
            WRITE(*,140)(NDNMS(K),K=1,NUMRET)
     140    FORMAT(8X,A)
            NLEFT = NLEFT - MAXCLD
            ISTART = ISTART + MAXCLD
         IF (NLEFT .GT. 0) GO TO 130
         RETURN
         END
\end{verbatim}

\noindent
The resulting output is:

\begin{verbatim}
   NODE F3 THROUGH LINK L3:
     LABEL       = LABEL ON NODE F3
     DATA TYPE   = R4
     NUM OF DIMS = 2
     DIM VALS    = 4 3
     DATA:
             1.10      1.20      1.30
             2.10      2.20      2.30
             3.10      3.20      3.30
             4.10      4.20      4.30

   NODE N13:
     LABEL       = LABEL ON NODE N13
     DATA TYPE   = I4
     NUM OF DIMS = 1
     DIM VALS    = 6
     DATA:
            1      2      3      4      5      6

   NODE N13 THROUGH LINK L1:
     LABEL       = LABEL ON NODE N13
     DATA:
            1      2      3      4      5      6

   NODE N13 THROUGH LINK L2:
     LABEL       = LABEL ON NODE N13
     DATA:
            1      2      3      4      5      6

   PARENT NODE NAME = ADF MotherNode
       NUMBER OF CHILDREN =  3
       CHILDREN NAMES:
          N1
          N2
          N3

   PARENT NODE NAME = N3
       NUMBER OF CHILDREN =  3
       CHILDREN NAMES:
          N9
          N10
          N11

   COMPARISON OF ROOT ID:
     ADF_FILE_TWO.ADF ORIGINAL ROOT ID = 1.2653021189994324-320
     ADF_FILE_TWO.ADF NEW ROOT ID      = 4.7783097267391979-299
\end{verbatim}
