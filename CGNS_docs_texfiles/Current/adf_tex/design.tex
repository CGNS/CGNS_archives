\section{ADF Design Considerations}
\label{s:design}
\thispagestyle{plain}

This section provides a summary of the design considerations that were
used in the construction of the ADF software library.

\subsection{ADF File Header Information}

Every ADF file has a header section that contains information about the
file itself.
The following information from this header is available to the user:

\begin{itemize}
\item ADF version number of the library that created the database.
\item File creation date and time.
\item File modification date and time.
\item Data format used in database (IEEE big endian, IEEE little endian,
      etc.)
\end{itemize}

[Under data format, ``endian'' refers to the ordering of bytes in a
multi-byte number.
Big endian is a computer architecture in which, within a given
multi-byte numeric representation, the most significant byte has the
lowest address (the word is stored ``big-end-first'').
Little endian is a computer architecture in which, within a given 16- or
32-bit word, bytes at lower addresses have lower significance (the word
is stored ``little-end-first'').]

\subsection{ADF File Optimizations}

To optimize the performance of ADF, the following techniques have been
incorporated into the ADF software to enhance performance of the file:

\begin{itemize}
\item Use block-based, unbuffered (raw mode) I/O where available, with
      block sizes of 4096 bytes.
\item Align medium to large data chunks on block boundaries.
\item If the data size is equal to or greater than 2048, then
      \begin{itemize}
      \item align data to the next block and
      \item add extra to free (garbage) lists
      \end{itemize}
\item Avoid, where possible, small- to medium-sized chunks of data that
      span block boundaries.
\item Align, where possible, to the next block and add extra to free
      (garbage) lists.
\end{itemize}

Allow data space to grow by linking data chunks.
It is possible to increase the last dimension; doing so will extend the
data.
However, internal dimensional changes will corrupt the existing data.

The pointer table for child nodes will also contain the child node names.

\subsection{ADF File Portability}

To address code portability and future needs, the following design
decisions were made:

\begin{itemize}
\item Use larger than 32-bit file pointers to allow for files larger than
      4 Gigabytes.
      (C routine \fort{lseek} may not handle this, but ADF files should
      allow it.)
\item Use 48-bit pointers within a block of data.
      32-bit pointers to 4096-byte blocks and a 16-bit pointer to a
      position within the block.
      This allows for files with 4 gigabyte blocks, in other words
      $2^{32} \times 4096 = 17.5922 + 12$ bytes or 17.59 Tera bytes.
\item The ID pointers will be 64-bit coded IDs.
      Users may use double data type (\fort{real*8}).
      This is parceled as follows: 32 bits for block number, 12 bits for
      block offset, and 20 bits for file identifier.
\item Encode the integer information, other than data, in ASCII,
      Hex-based notation.
\end{itemize}

\subsection{ADF File Error Checking}

Error checking has been implemented for the ADF file and is summarized
here.
Each item in the ADF file will have item-specific boundary tags
surrounding it to provide file-based corruption checking.
For variable-sized items, the associated boundary tags will include
file-based size information.
Information will be written to the disk in a sequence that will not
allow corrupt files.
For example, when adding a new child node, the complete child
information will be written before the parent's child-table is updated.
An ADF-core subroutine for downloading data to disk will be provided.

\subsection{ADF Source Code Considerations}

The ADF library of source code will incorporate the use of UNIX ``what''
strings for the ADF version number and also RCS versioning information
in the source code and in the object code.
The source code is written in portable ANSI C, using POSIX-defined system
calls.

\subsection{ADF Node Header Information}

The following information is contained in an ADF node header:

\begin{itemize}
\item node boundary tag
\item name (32 characters)
\item label (32 characters)
      \begin{itemize}
      \item number of subnodes (\fort{num\_sub\_nodes}) is an integer
            (ASCII, Hex encoded in file).
      \item address pointer of the subnodes indicated by the variable.
      \end{itemize}
\item \fort{sub\_nodes}, which are pointers (ASCII, Hex encoded in file)
      to a table (the ``child table'') of file pointers and names for each
      of the node's children.
      Note that the child name information is redundant and included for
      performance.
\item data type is specified in \fort{data\_type} (32 characters allowed).
\item dimensionality of the data, called \fort{num\_dimensions}.
      It is ASCII, Hex encoded in the file.
\item dimension values are listed in the integer array of 12,
      \fort{dimension\_values}.
      The dimension values are Hex encoded.
\item integer value for the number of data chunks is found in
      \fort{num\_data\_chunks} (ASCII, Hex encoded).
\item data address, which is as follows:
      \begin{itemize}
      \item if \fort{num\_data\_chunks} $= 1$, then a file pointer to data
      \item if \fort{num\_data\_chunks} $> 1$, then a file pointer to
            a table of \fort{num\_data\_chunks} file pointers and
            associated data sizes
      \end{itemize}
%
%      \settowidth{\tmplengtha}{if \fort{num\_data\_chunks}}
%      \settowidth{\tmplengthb}{\ $= 1,\mbox{\ }$}
%      \setlength{\Pwidth}{\linewidth-2\tabcolsep-\tmplengtha-\tmplengthb}
%      \begin{tabular}{l @{}l @{\raggedright\arraybackslash}p{\Pwidth}}
%      if \fort{num\_data\_chunks} & $\ = 1,\mbox{\ }$ &
%                                    then a file pointer to data \\
%                                  & $\ > 1,\mbox{\ }$ &
%                                    then a file pointer to a table
%                                    of \fort{num\_data\_chunks} file
%                                    pointers and associated data sizes
%      \end{tabular}
\item ending node boundary tag.
\end{itemize}

\subsection{Fortran Character Array Portability Concerns}
\label{s:design:fortran}

Fortran character arrays are different from any other array type because
they inherently include declared length information.
Abstractly, they are a compound type: an array and an integer.
The ANSI standards do not specify the implementation mechanism for
handling this data type and so it is left to the vendor.
As one might expect, vendors have devised different policies.
This is particularly evident in how argument lists are created and used.
The matter is further complicated when writing functions in other
languages that are to be called from Fortran.

To keep the interfaces simple and to keep the Fortran and C data I/O
calls similar (as opposed to having separate data I/O functions for
character data), ADF suggests abiding by the following rules (these are
required for Cray T90-mode users):

\begin{itemize}
\item Do not pass character arrays as the actual data arguments to any
      ADF read or write function unless that node has been defined with a
      data type of \fort{C1}.
\item If a node has been defined as data type \fort{C1}, then pass
      character arrays only as the actual data arguments to all ADF read
      and write function.
\end{itemize}

\emph{Note to Cray T90-mode users:} The above rules must be followed.
In addition, the given node must be available and have its data type
correctly defined.
Error handling is not possible otherwise, and ADF will abort or fail
regardless of how the error state flag is set.

\subsection{Integer 64 Data Type Portability Concerns}

For portability reasons, it is suggested that the use of the \fort{I8}
data type be restricted to a 64-bit environment.

\subsection{Compound Data Types Portability Concerns}

For the transportability reasons discussed below, use of compound data
types is not recommended.

When using compound data types (e.g., with structures in C), it is
important to be aware of data alignment issues.
If one is not careful, the actual size of the structure in memory may be
larger than the sum of the individual members.
The total size depends on the order and word boundary alignment
requirements of the specific data types.
This is platform and compiler dependent and not handled by ADF.
In order to provide the greatest portability (at least up to 64-bit
environments), it is recommended that

\begin{itemize}
\item 8-byte data types (\fort{I8}, \fort{R8}) be aligned on 8 byte
      boundaries, and
\item data types smaller than the word size be padded to a size equal to
      the word size.
\end{itemize}

So a 4-byte data type (e.g., \fort{I4[1]}) needs ``padding'' (e.g.,
\fort{I4[2]}) if it is to be followed by an 8-byte data type.
And assuming a word size of 4 bytes, all \fort{C1}-data-type elements
need dimension values of multiples of 4 bytes (e.g., \fort{C1[4]},
\fort{C1[8]}, etc.).
To be even more careful, size everything in multiples of 8 bytes,
for example, ``\fort{C1[8]}, \fort{I4[2]}, \fort{C1[16]},
\fort{R4[6]}, \fort{R8[5]}, \fort{I8[1]}''.

For a given architecture and compiler, and taking into consideration the
restrictions given above, compound data types should work.
It is more portable and highly recommended that users write out the
individual components of a structure into separate nodes.
It would probably be best to copy the individual components of a list of
structures into an appropriate array type and write the temporary array
out using the write-all or write-strided routines.
