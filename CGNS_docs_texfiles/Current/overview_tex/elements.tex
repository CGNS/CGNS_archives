\section{Elements and Documentation}
\label{s:elements}
\thispagestyle{plain}

\subsection{Introduction}

CGNS concerns itself with the recording and retrieval of data associated
with the computation of fluid flows.
Included are such structures as grids, flowfields, boundary conditions,
and zone connectivity information.
CGNS ``understands'' this data in the sense that it contains conventions
for recording it based on its structure and its role in CFD.

The underlying design of CGNS is that of a common database that is
accessed (read, written, or modified) by a collection of ``applications''
programs such as solvers, grid generators, and postprocessors.

CGNS itself does not synthesize, modify, or interpret the data it
stores.
The applications create, edit, or display the data; CGNS is limited to
recording and retrieving it.
Each application's program accesses the data directly using CGNS
function calls installed in the application by its developer.
The applications are not regarded as part of CGNS itself.

CGNS is passive.
It does not initiate action and cannot ``push'' information into the
applications codes or ``pull'' information out.
Rather, the codes must request the information they seek and store the
information they produce.
The applications must be launched by a user who organizes the location
and content of the database.
The process and sequence of events remain under user control.
Thus CGNS facilitates, but does not incorporate, the development of
batch or interactive environments designed to control the CFD process.

The elements of CGNS address all activities associated with the
storage of the data on external media and its movement to and from the
applications programs.
These elements include the following:

\begin{itemize}
\item The Standard Interface Data Structures (SIDS), which specify the
      intellectual content of CFD data and the conventions that govern
      naming and terminology.
\item The SIDS File Mapping,
      which specify the exact location where the CFD
      data defined by the SIDS is to be stored within a database file.
\item The Database Manager, consisting
      of both a file format specification and its I/O software, which
      handles the actual reading and writing of data from and to
      external storage media.
\end{itemize}

The following sections discuss in more detail the roles of the CGNS
elements and introduce their documentation.

\subsection{Structure of a CGNS Database}

In this section, the conceptual structure of a CGNS database, and the
nodes from which it is built, are discussed. This describes the way
in which the CGNS software ``sees'' the database, not necessarily the way
in which it is implementated. The details of the implementation are
left to the underlying database manager.

A CGNS database consists of a collection of elements called nodes.
These nodes are arranged in a tree structure that is logically similar
to a UNIX file system.
The nodes are said to be connected in a ``child-parent'' relationship
according to the following simple rules:

\begin{enumerate}
\item Each node may have any number of child nodes.
\item Except for one node, called the root, each node is the child of
      exactly one other node, called its parent.
\item The root node has no parent.
\end{enumerate}

\subsubsection{Structure of a Node}

Each node has exactly the same internal structure.
The entities associated with each node are the following:

\begin{itemize*}
\item Node Identifier (ID)
\item Name
\item Label
\item Data Type
\item Dimension
\item Dimension Values
\item Data
\item Child Table
\end{itemize*}

\paragraph{Node Identifier.}
The Node ID is a floating point number assigned by the system when the
database is opened or created.
Applications may record the ID and use it to return directly to the
corresponding node when required.
The Node ID is valid only while the file is open; subsequent openings of
the same file may be expected to yield different IDs.

\paragraph{Name.}
The Name field holds a character string chosen by the user or specified
by the SIDS to identify the particular instance of the data being
recorded.

\paragraph{Label.}
The Label, also a character string, is specified by the CGNS mapping
conventions and identifies the kind of data being recorded.
For example, a node with label \fort{Zone\_t} may record (at and below
it) information on the zone with Name ``UnderWing.''
No node may have more than one child
with the same name, but the CGNS mapping conventions commonly specify
many children with the same label.
For some nodes, the mapping conventions specify that the name
field has significance for the meaning of the data (e.g.,
\fort{EnthalpyStagnation}).
Although the user may specify another name, these ``paper'' conventions
serve the transfer of data between users and between applications.
These names and their meanings are established by the SIDS.

\paragraph{Data Type, Dimension, Dimension Values, Data.}
Nodes may or may not contain data.
For those that do, CGNS specifies a single array whose type (integer,
etc.), dimension, and size are recorded in the Data Type, Dimension, and
Dimension Value fields, respectively.
The mapping conventions specify some nodes that serve to establish the
tree structure and point to further data below but contain no data
themselves.
For these nodes, the Data Type is \fort{MT}, and the other fields are
empty. A Link to another node within the current or an external
database is indicated by a Data Type of \fort{LK}.

\paragraph{Child Table.}
The Child Table contains a list of the node's children.
It is maintained by the database manager as children are created and deleted.

\subsubsection{High-Level Organization of the CGNS Database}

For a full specification of the location of CFD data in the
CGNS database, the user should see the \textit{SIDS File Mapping}
document.
For convenience, we summarize the high-level structure below.

A CGNS database consists of a tree of nodes implemented as all or part
of one or more database files.
All information is identified by and accessed through a single node in
one of these files.
There is thus no notion of a CGNS \textit{file}, only of a CGNS
\textit{database} implemented within one or more \textit{ADF (or HDF5)
files}.

By definition, the root node of a CGNS database has the Label
\fort{CGNSBase\_t}.
The Name of the database can be specified by the user and is stored in
the ``Name'' field of the \fort{CGNSBase\_t} node.
Current CGNS conventions require that the \fort{CGNSBase\_t} node be
located directly below a ``root node'' in the database file
identified by the name ``\fort{/}''.

An database file may contain multiple CGNS databases, and thus
multiple \texttt{CGNSBase\_t} nodes.
However, each node labeled \fort{CGNSBase\_t} in a single
file must have a unique name.
The user or application must know the name of the file containing
the entry-level node and, if there is more than one node labeled
\fort{CGNSBase\_t} in that file, the name of the database as well.

Below the \fort{CGNSBase\_t} node, the mapping conventions specify a
subnode for each zone.
This node has label \fort{Zone\_t}.
Its Name refers to the particular zone whose characteristics are
recorded at and below the node, such as ``UnderWing.''
In general, names can be specified by the user, but defaults are
specified for nodes that the user does not choose to name.
For the \fort{Zone\_t} nodes, the defaults are \fort{Zone1},
\fort{Zone2}, and so forth, in order of creation.
A similar convention for default names applies elsewhere.
It is impossible to create a node without a name (or with a name of zero
length).
The CGNS Mid-Level Library conforms to the default convention.

Below each zone node will be found nodes for the grid, flowfield,
boundary conditions, and connectivity information; these, in turn, are
parents of nodes specifying extent, spatial location, and so on.

The file mapping specifies that one or more ``Descriptor'' nodes may be
inserted anywhere in the file.
Descriptor nodes are used to record textual information regarding the
file contents.
The size of Descriptor nodes is unlimited, so entire documents could be
named and stored within the data field if desired.
Descriptors are intended to store human-readable text, and they are not
processed by any supplied CGNS software (except, of course, that the
text may be stored and retrieved).

It is possible, by using the linking capability of CGNS, for a child of
any node to be a node in another database file, or elsewhere within
the same file.
This mechanism enables one database to share a grid, for example, with
another database without duplicating the information.

\subsection{Standard Interface Data Structures (SIDS)}

The establishment of a standard for storing CFD-related information
requires a detailed specification of the content and meaning of the data
to be stored.
For example, it is necessary to state the meaning of the words ``boundary
condition'' in a form sufficiently concrete to be recorded precisely, and
yet sufficiently flexible to embrace current and future practice.
The \textit{Standard Interface Data Structures}
(SIDS) document describes this ``intellectual content'' of CFD-related
data in detail.

An exact description of the intellectual content is required not only
to define the precise form of the data but also to guarantee that the
meaning of the data is consistently interpreted by practitioners.
Thus the SIDS include a collection of naming conventions that
specify the precise meaning of nomenclature (e.g., the strings
\fort{DensityStatic} and \fort{BCWallViscous}).

The SIDS are written in a self-contained C-like descriptive language.
SIDS data structures are defined in a hierarchical manner in which more
complex entities are built from simpler ones.
These structures are closely reflected in CGNS-compliant
files: simple entities are often stored in single nodes, while more
complex structures are stored in entire subtrees.

\subsection{SIDS File Mapping}

Because of the generality of the tree structure, there are
many conceivable means of encoding CFD data.
But for any application to access, say, the boundary conditions for zone
``UnderWing'', requires a single convention with regard to where in the
file that data has been stored.
The \textit{SIDS File Mapping} document, sometimes referred to
as the ``File Mapping,'' establishes the precise node, and properties
of that node, where each piece of CGNS data should be recorded.
The CGNS Mid-Level Library relies on the File Mapping to locate
CFD-related data within the file.

The mapping provides locations for an extensive set of CFD
data.
Most applications will make use of only a small subset of this data.
Further, inasmuch as applications are viewed as editors that are in the
process of building the database, most of them are intended for use on
incomplete data sets.
Therefore, it is not required that all the data elements specified by
the CGNS conventions be complete in order for a database to be CGNS
compliant.
The user must ensure that the current state of the database will support
whatever application he may launch.
Of course, the application should gracefully handle any absence or
deficiency of data.

CGNS conventions do not specify the following:

\begin{itemize*}
\item the use the applications programs may make of the data
\item the means by which the applications programs modify the data
\item the form in which the data is stored internal to an application
\end{itemize*}

\noindent
The validity, accuracy and completeness of the data are determined
entirely by the applications software.

The tree structure also makes it possible for applications
to ignore data for which they have no use.
(In fact, they cannot even discover the data's existence without a
specific inquiry.)
Therefore, it is permissible for a file containing a CGNS
database to contain additional nodes not specified by the mapping.
Such nodes will be disregarded by software not prepared to use them.
However, if data essential to the CFD process is stored in a manner
not consistent with CGNS conventions, that data becomes invisible and
therefore useless to other applications.

Note that the SIDS serve not only to facilitate the mapping of data onto
the file structure but also to standardize the meaning of
the recorded data.
Thus there are two kinds of conventions operative within CGNS.
Adherence to the File Mapping conventions guarantees that the software
will be able to find and read the data.
Adherence to the SIDS guarantees uniformity of meaning among users and
between applications.
The \textit{SIDS File Mapping} document establishes the
context of CGNS for a database manager; the SIDS define the
nomenclature, content, and meaning of the stored data.

The File Mapping generally avoids the storage of redundant data.
Sometimes an application may require an alternate (but intellectually
equivalent) form of the data; in such cases it is recommended that the
alternate form be prepared at the time of use and kept separate from the
CGNS data.
This avoids habitual reliance on the alternate form, which would
invalidate the standard.

\subsection{Database Manager}

A database manager contains the I/O software,
which handles the actual reading and writing
of data from and to external storage media. It must conform, at least
in context, to that specified by the
\textit{SIDS File Mapping} document,
and provide a minimal number of data access routines
(referred to as core routines).
In principle, it is possible to install CGNS I/O into an application
using only these core routines.
However, such an approach would require the installer to access the data
at a very fundamental level and would result in lengthy sequences of
core function calls.
Therefore, the CGNS system also includes a \textit{Mid-Level Library},
an API (Application Programming Interface) that contains
additional routines intended to facilitate higher-level access to the
data.
These are CFD-knowledgeable routines suitable for direct installation
into applications codes.

The CGNS software was originally developed around
\textit{ADF (Advanced Data Format)}
as it's database manager, thus much of the concepts and structures
of CGNS originated from there.

In version 2.4 of the CGNS software,
\textit{HDF5 (Hierarchical Data Format} was
introduced as an alternative database manager. At that time, either
ADF or HDF5 (but not both) was selectable at build time.

It should be noted that because of HDF5's parallel and compression      
capability as well as its support, the CGNS Steering Committee has made 
the decision to slowly transition (beginning in 2005) to HDF5 as the    
official data storage mechanism.                                        
However, ADF will continue to be available for use, with the CGNS
mid-level library capable of (1) using either format and (2) translating
back and forth between the two.

Beginning with CGNS version 3.0,
both ADF and HDF5 are supported concurrently and transparently
by CGNS. To facilitate this, a new set of core routines, described in
the \textit{CGIO User's Guide}, have been
developed as a replacement to the individual ADF and HDF5 core routines.
These allow general access to the low-level I/O, irrespective of the
underlying database manager.

\subsection{Mid-Level Library, or API}

The CGNS Mid-Level Library, or Applications Programming Interface
(API), is one of the most directly visible parts of CGNS, and it is of
particular interest to applications code developers.
It consists of a set of routines that are designed to allow applications
to access CGNS data according to the role of the data in CFD.
Unlike the ADF (or HDF5) Core, routines in the CGNS Mid-Level Library
``understand'' the SIDS-defined CFD data structures and the File
Mapping.
This enables applications developers to insert CGNS I/O into their
applications without having detailed knowledge of the File Mapping.
For instance, an application might use CGNS mid-level calls to retrieve
all boundary conditions for a given zone.

The \textit{CGNS Mid-Level Library} document contains complete
descriptions and usage instructions for all mid-level routines.
As with the ADF (or HDF5) Core, all calls are provided in both C and
Fortran.

\subsection{Documentation}

The five CGNS elements described above are documented individually,
as follows:

\begin{itemize*}
\item \textit{Standard Interface Data Structures}
\item \textit{SIDS File Mapping Manual}
\item \textit{Mid-Level Library}
\item \textit{CGIO User's Guide}
\item \textit{ADF Implementation}
\item \textit{HDF5 Implementation}
\end{itemize*}

\noindent
In addition, the following documentation is also recommended:

\begin{itemize*}
\item \textit{CGNS Overview and Entry-Level Document} (this document)
\item ``The CGNS System'', AIAA Paper 98-3007
\item ``Advances in the CGNS Database Standard for Aerodynamics and
      CFD'', AIAA Paper 2000-0681
\item ``CFD General Notation System (CGNS): Status and Future
      Directions'', AIAA Paper 2002-0752
\end{itemize*}

\noindent
The specific documents of interest vary with the level of intended use
of CGNS.

\subsubsection{Prospective Users}

Prospective users are presumably unfamiliar with CGNS.
They will probably wish to begin with the current Overview document, or,
if they require more detailed information, the AIAA papers listed above.
Beyond that, most will find a quick read of the
\textit{SIDS File Mapping Manual}
enlightening as to the logical form of the contents of CGNS files.
Browsing the figures in the File
Mapping Manual, as well as the SIDS
itself, will provide some feel for the scope of the system.
The \textit{User's Guide to CGNS}, and
the \textit{CGNS Mid-Level Library}
document, should give an indication of what might be required to
implement CGNS in a given application.
Prospective users should probably not concern themselves with the
details of ADF (or HDF5).

\subsubsection{End Users}

The end user is the practitioner of CFD who generates the grids, runs
the flow codes and/or analyzes the results.
For this user, a scan of this Overview document will sufficiently
explain the overall workings of the system.
This includes end user responsibilities for matters not governed by
CGNS, such as the maintenance of files and directories.
The end user will also find useful the
\textit{User's Guide to CGNS}, as
well as those portions of the SIDS
which deal with standard data names.
The AIAA papers listed above may also be useful if more details about
the capabilities of CGNS are desired.

\subsubsection{Applications Code Developers}

The applications code developer builds or maintains code to support the
various sub-processes encountered in CFD, e.g., grid generation, flow
solution, post-processing, or flow visualization.
The code developer must be able to install CGNS compliant I/O.
The most convenient method for doing so is to utilize the CGNS Mid-Level
Library.
The \textit{User's Guide to CGNS} is the
starting point for learning to use the Mid-Level Library to create and
use CGNS files.
The \textit{CGNS Mid-Level Library}
document itself should also be considered essential.
This library of routines will perform the most common I/O operations in
a CGNS-compliant manner.
However, even when the Mid-Level Library suffices to implement all
necessary I/O, an understanding of the
file mapping and SIDS will be useful.
It will likely be necessary to consult the
SIDS to determine the precise meaning
of the nomenclature.

Applications code developers wishing to read or write data
that isn't supported by the Mid-Level Library, will have
to use the CGIO low-level routines to access the underlying database
manager directly. The \textit{CGIO User's Guide} documents these
routines in detail.

\subsubsection{CGNS System Developers}

CGNS System development can be kept somewhat compartmentalized.
Developers responsible for maintenance of the ADF Core, or the building
of supplements to the ADF Core, need not concern themselves with
documentation other than the \textit{ADF User Guide}.
(Development and maintenance of HDF5 is under the purview of NCSA, so
has no relevance here.)
System developers wishing to add to the CGNS Mid-Level Library will need
all the documents.
Theoretical developments, such as extensions to the SIDS, may possibly
be undertaken with a knowledge of the SIDS alone, but such contributions
must also be added to the File Mapping before they can be implemented.
