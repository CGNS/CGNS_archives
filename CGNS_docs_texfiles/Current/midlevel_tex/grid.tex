\section{Grid Specification}
\label{s:grid}
\thispagestyle{plain}

\subsection{Zone Grid Coordinates}
\label{s:gridcoordinates}

\noindent
\textit{Node}: \texttt{GridCoordinates\_t}

\texttt{GridCoordinates\_t} nodes are used to describe grids associated
with a particular zone.
The original grid must be described by a \texttt{GridCoordinates\_t} node
named \texttt{GridCoordinates}.
Additional \texttt{GridCoordinates\_t} nodes may be used, with user-defined
names, to store grids at multiple time steps or iterations.
In addition to the discussion of the \texttt{GridCoordinates\_t} node in
the \href{../sids/sids.pdf}{SIDS} and \href{../filemap/filemap.pdf}{File
Mapping} manuals, see the discussion of the \texttt{ZoneIterativeData\_t}
and \texttt{ArbitraryGridMotion\_t} nodes in the SIDS manual.

\begin{fctbox}
\textcolor{output}{\textit{ier}} = cg\_grid\_write(\textcolor{input}{int fn}, \textcolor{input}{int B}, \textcolor{input}{int Z}, \textcolor{input}{char *GridCoordName}, & - w m \\
~~~~~~\textcolor{output}{\textit{int *G}}); & \\
\textcolor{output}{\textit{ier}} = cg\_ngrids(\textcolor{input}{int fn}, \textcolor{input}{int B}, \textcolor{input}{int Z}, \textcolor{output}{\textit{int *ngrids}}); & - w m \\
\textcolor{output}{\textit{ier}} = cg\_grid\_read(\textcolor{input}{int fn}, \textcolor{input}{int B}, \textcolor{input}{int Z}, \textcolor{input}{int G}, \textcolor{output}{\textit{char *GridCoordName}}); & r - m \\
\hline
call cg\_grid\_write\_f(\textcolor{input}{fn}, \textcolor{input}{B}, \textcolor{input}{Z}, \textcolor{input}{GridCoordName}, \textcolor{output}{\textit{G}}, \textcolor{output}{\textit{ier}}) & - w m \\
call cg\_ngrids\_f(\textcolor{input}{fn}, \textcolor{input}{B}, \textcolor{input}{Z}, \textcolor{output}{\textit{ngrids}}, \textcolor{output}{\textit{ier}}) & - w m \\
call cg\_grid\_read\_f(\textcolor{input}{fn}, \textcolor{input}{B}, \textcolor{input}{Z}, \textcolor{input}{G}, \textcolor{output}{\textit{GridCoordName}}, \textcolor{output}{\textit{ier}}) & r - m \\
\end{fctbox}

\noindent
\textbf{\textcolor{input}{Input}/\textcolor{output}{Output}}

\noindent (Note that for Fortran calls, all integer arguments are integer*4 in 32-bit mode and integer*8 in 64-bit mode.
See ``64-bit Fortran Portability and Issues" section.)

\begin{Ventryi}{\texttt{GridCoordinateName}}\raggedright
\item [\texttt{fn}]
      CGNS file index number.
      (\textcolor{input}{Input})
\item [\texttt{B}]
      Base index number, where $1 \leq \text{\texttt{B}} \leq \text{\texttt{nbases}}$.
      (\textcolor{input}{Input})
\item [\texttt{Z}]
      Zone index number, where $1 \leq \text{\texttt{Z}} \leq \text{\texttt{nzones}}$.
      (\textcolor{input}{Input})
\item [\texttt{G}]
      Grid index number, where $1 \leq \text{\texttt{G}} \leq \text{\texttt{ngrids}}$.
      (\textcolor{input}{Input} for \texttt{cg\_grid\_read};
      \textcolor{output}{\textit{output}} for \texttt{cg\_grid\_write})
\item [\texttt{ngrids}]
      Number of \texttt{GridCoordinates\_t} nodes for zone \texttt{Z}.
      (\textcolor{output}{\textit{Output}})
\item [\texttt{GridCoordinateName}]
      Name of the \texttt{GridCoordinates\_t} node.
      Note that the name ``\texttt{GridCoordinates}'' is reserved for the
      original grid and must be the first \texttt{GridCoordinates\_t}
      node to be defined.
      (\textcolor{input}{Input} for \texttt{cg\_grid\_write};
      \textcolor{output}{\textit{output}} for \texttt{cg\_grid\_read})
\item [\texttt{ier}]
      Error status.
      (\textcolor{output}{\textit{Output}})
\end{Ventryi}

The above functions are applicable to any \texttt{GridCoordinates\_t} node.

\begin{fctbox}
\textcolor{output}{\textit{ier}} = cg\_coord\_write(\textcolor{input}{int fn}, \textcolor{input}{int B}, \textcolor{input}{int Z}, \textcolor{input}{DataType\_t datatype}, & - w m \\
~~~~~~\textcolor{input}{char *coordname}, \textcolor{input}{void *coord\_array}, \textcolor{output}{\textit{int *C}}); & \\
\textcolor{output}{\textit{ier}} = cg\_coord\_partial\_write(\textcolor{input}{int fn}, \textcolor{input}{int B}, \textcolor{input}{int Z}, & - w m \\
~~~~~~\textcolor{input}{DataType\_t datatype}, \textcolor{input}{char *coordname}, \textcolor{input}{cgsize\_t *range\_min}, & \\
~~~~~~\textcolor{input}{cgsize\_t *range\_max}, \textcolor{input}{void *coord\_array}, \textcolor{output}{\textit{int *C}}); & \\
\textcolor{output}{\textit{ier}} = cg\_ncoords(\textcolor{input}{int fn}, \textcolor{input}{int B}, \textcolor{input}{int Z}, \textcolor{output}{\textit{int *ncoords}}); & r - m \\
\textcolor{output}{\textit{ier}} = cg\_coord\_info(\textcolor{input}{int fn}, \textcolor{input}{int B}, \textcolor{input}{int Z}, \textcolor{input}{int C}, \textcolor{output}{\textit{DataType\_t *datatype}}, & r - m \\
~~~~~~\textcolor{output}{\textit{char *coordname}}); & \\
\textcolor{output}{\textit{ier}} = cg\_coord\_read(\textcolor{input}{int fn}, \textcolor{input}{int B}, \textcolor{input}{int Z}, \textcolor{input}{char *coordname}, & r - m \\
~~~~~~\textcolor{input}{DataType\_t datatype}, \textcolor{input}{cgsize\_t *range\_min}, \textcolor{input}{cgsize\_t *range\_max}, & \\
~~~~~~\textcolor{output}{\textit{void *coord\_array}}); & \\
\hline
call cg\_coord\_write\_f(\textcolor{input}{fn}, \textcolor{input}{B}, \textcolor{input}{Z}, \textcolor{input}{datatype}, \textcolor{input}{coordname}, \textcolor{input}{coord\_array}, \textcolor{output}{\textit{C}}, & - w m \\
~~~~~\textcolor{output}{\textit{ier}}) & \\
call cg\_coord\_partial\_write\_f(\textcolor{input}{fn}, \textcolor{input}{B}, \textcolor{input}{Z}, \textcolor{input}{datatype}, \textcolor{input}{coordname}, \textcolor{input}{range\_min}, & - w m \\
~~~~~\textcolor{input}{range\_max}, \textcolor{input}{coord\_array}, \textcolor{output}{\textit{C}}, \textcolor{output}{\textit{ier}}) & \\
call cg\_ncoords\_f(\textcolor{input}{fn}, \textcolor{input}{B}, \textcolor{input}{Z}, \textcolor{output}{\textit{ncoords}}, \textcolor{output}{\textit{ier}}) & r - m \\
call cg\_coord\_info\_f(\textcolor{input}{fn}, \textcolor{input}{B}, \textcolor{input}{Z}, \textcolor{input}{C}, \textcolor{output}{\textit{datatype}}, \textcolor{output}{\textit{coordname}}, \textcolor{output}{\textit{ier}}) & r - m \\
call cg\_coord\_read\_f(\textcolor{input}{fn}, \textcolor{input}{B}, \textcolor{input}{Z}, \textcolor{input}{coordname}, \textcolor{input}{datatype}, \textcolor{input}{range\_min}, & r - m \\
~~~~~\textcolor{input}{range\_max}, \textcolor{output}{\textit{coord\_array}}, \textcolor{output}{\textit{ier}}) & \\
\end{fctbox}

\noindent
\textbf{\textcolor{input}{Input}/\textcolor{output}{\textit{Output}}}

\noindent (Note that for Fortran calls, all integer arguments are integer*4 in 32-bit mode and integer*8 in 64-bit mode.
See ``64-bit Fortran Portability and Issues" section.)

\begin{Ventryi}{\texttt{coord\_array}}\raggedright
\item [\texttt{fn}]
      CGNS file index number.
      (\textcolor{input}{Input})
\item [\texttt{B}]
      Base index number, where $1 \leq \text{\texttt{B}} \leq \text{\texttt{nbases}}$.
      (\textcolor{input}{Input})
\item [\texttt{Z}]
      Zone index number, where $1 \leq \text{\texttt{Z}} \leq \text{\texttt{nzones}}$.
      (\textcolor{input}{Input})
\item [\texttt{C}]
      Coordinate array index number, where $1 \leq \text{\texttt{C}} \leq \text{\texttt{ncoords}}$.
      (\textcolor{input}{Input} for \texttt{cg\_coord\_info};
      \textcolor{output}{\textit{output}} for \texttt{cg\_coord\_write})
\item [\texttt{ncoords}]
      Number of coordinate arrays for zone \texttt{Z}.
      (\textcolor{output}{\textit{Output}})
\item [\texttt{datatype}]
      Data type in which the coordinate array is written.
      Admissible data types for a coordinate array are \texttt{RealSingle}
      and \texttt{RealDouble}.
      (\textcolor{input}{Input} for \texttt{cg\_coord\_write},
      \texttt{cg\_coord\_partial\_write}, \texttt{cg\_coord\_read};
      \textcolor{output}{\textit{output}} for \texttt{cg\_coord\_info})
\item [\texttt{coordname}]
      Name of the coordinate array.
      It is strongly advised to use the SIDS nomenclature conventions
      when naming the coordinate arrays to insure file compatibility.
      (\textcolor{input}{Input} for \texttt{cg\_coord\_write},
      \texttt{cg\_coord\_partial\_write}, \texttt{cg\_coord\_read};
      \textcolor{output}{\textit{output}} for \texttt{cg\_coord\_info})
\item [\texttt{range\_min}]
      Lower range index (eg., \texttt{imin, jmin, kmin}).
      (\textcolor{input}{Input})
\item [\texttt{range\_max}]
      Upper range index (eg., \texttt{imax, jmax, kmax}).
      (\textcolor{input}{Input})
\item [\texttt{coord\_array}]
      Array of coordinate values for the range prescribed.
      (\textcolor{input}{Input} for \texttt{cg\_coord\_write};
      \texttt{cg\_coord\_partial\_write}, \textcolor{output}{\textit{output}} for \texttt{cg\_coord\_read})
\item [\texttt{ier}]
      Error status.
      (\textcolor{output}{\textit{Output}})
\end{Ventryi}

The above functions are applicable \emph{only} to the 
\texttt{GridCoordinates\_t} node named \texttt{GridCoordinates}, used
for the original grid in a zone.
Coordinates for additional \texttt{GridCoordinates\_t} nodes in a zone
must be read and written using the \texttt{cg\_array\_\textit{xxx}} functions
described in \autoref{s:dataarray}.

When writing, the function \texttt{cg\_coord\_write} will
automatically write the full range of coordinates (i.e., the entire
\texttt{coord\_array}).
The function \texttt{cg\_coord\_partial\_write} may be used to write
only a subset of \texttt{coord\_array}.
When using the partial write, any existing data as defined by
\texttt{range\_min} and \texttt{range\_max} will be overwritten by the
new values.
All other values will not be affected.

The function \texttt{cg\_coord\_read} returns the coordinate array
\texttt{coord\_array}, for the range prescribed by \texttt{range\_min} and
\texttt{range\_max}.
The array is returned to the application in the data type requested in
\texttt{datatype}.
This data type does not need to be the same as the one in which the
coordinates are stored in the file.
A coordinate array stored as double precision in the CGNS file can be
returned to the application as single precision, or vice versa.

In Fortran, when using \texttt{cg\_coord\_read\_f} to read 2D or 3D
coordinates, the extent of each dimension of \texttt{coord\_array} must
be consistent with the requested range.
When reading a 1D solution, the declared size can be larger than the
requested range.
For example, for a 2D zone with $100 \times 50$ vertices, if
\texttt{range\_min} and \texttt{range\_max} are set to (11,11) and (20,20)
to read a subset of the coordinates, then \texttt{coord\_array} must be
dimensioned (10,10).
If \texttt{coord\_array} is declared larger (e.g., (100,50)) the
indices for the returned coordinates will be wrong.

\newpage
\subsection{Element Connectivity}
\label{s:elements}

\noindent
\textit{Node}: \texttt{Elements\_t}

\begin{fctbox}
\textcolor{output}{\textit{ier}} = cg\_section\_write(\textcolor{input}{int fn}, \textcolor{input}{int B}, \textcolor{input}{int Z}, \textcolor{input}{char *ElementSectionName}, & - w m \\
~~~~~~\textcolor{input}{ElementType\_t type}, \textcolor{input}{cgsize\_t start}, \textcolor{input}{cgsize\_t end}, \textcolor{input}{int nbndry}, & \\
~~~~~~\textcolor{input}{cgsize\_t *Elements}, \textcolor{output}{\textit{int *S}}); & \\
\textcolor{output}{\textit{ier}} = cg\_section\_partial\_write(\textcolor{input}{int fn}, \textcolor{input}{int B}, \textcolor{input}{int Z}, & - w m \\
~~~~~~\textcolor{input}{char *ElementSectionName}, \textcolor{input}{ElementType\_t type}, \textcolor{input}{cgsize\_t start}, & \\
~~~~~~\textcolor{input}{cgsize\_t end}, \textcolor{input}{int nbndry}, \textcolor{output}{\textit{int *S}}); & \\
\textcolor{output}{\textit{ier}} = cg\_elements\_partial\_write(\textcolor{input}{int fn}, \textcolor{input}{int B}, \textcolor{input}{int Z}, \textcolor{input}{int S}, & - w m \\
~~~~~~\textcolor{input}{cgsize\_t start}, \textcolor{input}{cgsize\_t end}, \textcolor{input}{cgsize\_t *Elements}); & \\
\textcolor{output}{\textit{ier}} = cg\_parent\_data\_write(\textcolor{input}{int fn}, \textcolor{input}{int B}, \textcolor{input}{int Z}, \textcolor{input}{int S}, & - w m \\
~~~~~~\textcolor{input}{cgsize\_t *ParentData}); & \\
\textcolor{output}{\textit{ier}} = cg\_parent\_data\_partial\_write(\textcolor{input}{int fn}, \textcolor{input}{int B}, \textcolor{input}{int Z}, \textcolor{input}{int S}, & - w m \\
~~~~~~\textcolor{input}{cgsize\_t start}, \textcolor{input}{cgsize\_t end}, \textcolor{input}{cgsize\_t *ParentData}); & \\
\textcolor{output}{\textit{ier}} = cg\_nsections(\textcolor{input}{int fn}, \textcolor{input}{int B}, \textcolor{input}{int Z}, \textcolor{output}{\textit{int *nsections}}); & r - m \\
\textcolor{output}{\textit{ier}} = cg\_section\_read(\textcolor{input}{int fn}, \textcolor{input}{int B}, \textcolor{input}{int Z}, \textcolor{input}{int S}, & r - m \\
~~~~~~\textcolor{output}{\textit{char *ElementSectionName}}, \textcolor{output}{\textit{ElementType\_t *type}}, \textcolor{output}{\textit{cgsize\_t *start}}, & \\
~~~~~~\textcolor{output}{\textit{cgsize\_t *end}}, \textcolor{output}{\textit{int *nbndry}}, \textcolor{output}{\textit{int *parent\_flag}}); & \\
\textcolor{output}{\textit{ier}} = cg\_ElementDataSize(\textcolor{input}{int fn}, \textcolor{input}{int B}, \textcolor{input}{int Z}, \textcolor{input}{int S}, & r - m \\
~~~~~~\textcolor{output}{\textit{cgsize\_t *ElementDataSize}}); & \\
\textcolor{output}{\textit{ier}} = cg\_ElementPartialSize(\textcolor{input}{int fn}, \textcolor{input}{int B}, \textcolor{input}{int Z}, \textcolor{input}{int S}, & r - m \\
~~~~~~\textcolor{input}{cgsize\_t start}, \textcolor{input}{cgsize\_t end}, \textcolor{output}{\textit{cgsize\_t *ElementDataSize}}); & \\
\textcolor{output}{\textit{ier}} = cg\_elements\_read(\textcolor{input}{int fn}, \textcolor{input}{int B}, \textcolor{input}{int Z}, \textcolor{input}{int S}, & r - m \\
~~~~~~\textcolor{output}{\textit{cgsize\_t *Elements}}, \textcolor{output}{\textit{cgsize\_t *ParentData}}); & \\
\textcolor{output}{\textit{ier}} = cg\_elements\_partial\_read(\textcolor{input}{int fn}, \textcolor{input}{int B}, \textcolor{input}{int Z}, \textcolor{input}{int S}, & r - m \\
~~~~~~\textcolor{input}{cgsize\_t start}, \textcolor{input}{cgsize\_t end}, \textcolor{output}{\textit{cgsize\_t *Elements}}, & \\
~~~~~~\textcolor{output}{\textit{cgsize\_t *ParentData}}); & \\
\textcolor{output}{\textit{ier}} = cg\_npe(\textcolor{input}{ElementType\_t type}, \textcolor{output}{\textit{int *npe}}); & r w m \\
\end{fctbox}

\begin{fctbox}
call cg\_section\_write\_f(\textcolor{input}{fn}, \textcolor{input}{B}, \textcolor{input}{Z}, \textcolor{input}{ElementSectionName}, \textcolor{input}{type}, \textcolor{input}{start}, \textcolor{input}{end}, & - w m \\
~~~~~\textcolor{input}{nbndry}, \textcolor{input}{Elements}, \textcolor{output}{\textit{S}}, \textcolor{output}{\textit{ier}}) & \\
call cg\_section\_partial\_write\_f(\textcolor{input}{fn}, \textcolor{input}{B}, \textcolor{input}{Z}, \textcolor{input}{ElementSectionName}, \textcolor{input}{type}, & - w m \\
~~~~~\textcolor{input}{start}, \textcolor{input}{end}, \textcolor{input}{nbndry}, \textcolor{output}{\textit{S}}, \textcolor{output}{\textit{ier}}) & \\
call cg\_elements\_partial\_write\_f(\textcolor{input}{fn}, \textcolor{input}{B}, \textcolor{input}{Z}, \textcolor{input}{S}, & - w m \\
~~~~~~\textcolor{input}{start}, \textcolor{input}{end}, \textcolor{input}{Elements}, \textcolor{output}{\textit{ier}}); & \\
call cg\_parent\_data\_write\_f(\textcolor{input}{fn}, \textcolor{input}{B}, \textcolor{input}{Z}, \textcolor{input}{S}, \textcolor{input}{ParentData}, \textcolor{output}{\textit{ier}}) & - w m \\
call cg\_parent\_data\_partial\_write\_f(\textcolor{input}{fn}, \textcolor{input}{B}, \textcolor{input}{Z}, \textcolor{input}{S}, \textcolor{input}{start}, & - w m \\
~~~~~\textcolor{input}{end}, \textcolor{input}{ParentData}, \textcolor{output}{\textit{ier}}) & \\
call cg\_nsections\_f(\textcolor{input}{fn}, \textcolor{input}{B}, \textcolor{input}{Z}, \textcolor{output}{\textit{nsections}}, \textcolor{output}{\textit{ier}}) & r - m \\
call cg\_section\_read\_f(\textcolor{input}{fn}, \textcolor{input}{B}, \textcolor{input}{Z}, \textcolor{input}{S}, \textcolor{output}{\textit{ElementSectionName}}, \textcolor{output}{\textit{type}}, & r - m \\
~~~~~\textcolor{output}{\textit{start}}, \textcolor{output}{\textit{end}}, \textcolor{output}{\textit{nbndry}}, \textcolor{output}{\textit{parent\_flag}}, \textcolor{output}{\textit{ier}}) & \\
call cg\_ElementDataSize\_f(\textcolor{input}{fn}, \textcolor{input}{B}, \textcolor{input}{Z}, \textcolor{input}{S}, \textcolor{output}{\textit{ElementDataSize}}, \textcolor{output}{\textit{ier}}) & r - m \\
call cg\_ElementPartialSize\_f(\textcolor{input}{fn}, \textcolor{input}{B}, \textcolor{input}{Z}, \textcolor{input}{S}, \textcolor{input}{start}, \textcolor{input}{end}, \textcolor{output}{\textit{ElementDataSize}}, & r - m \\
~~~~~\textcolor{output}{\textit{ier}}) & \\
call cg\_elements\_read\_f(\textcolor{input}{fn}, \textcolor{input}{B}, \textcolor{input}{Z}, \textcolor{input}{S}, \textcolor{output}{\textit{Elements}}, \textcolor{output}{\textit{ParentData}}, \textcolor{output}{\textit{ier}}) & r - m \\
call cg\_elements\_partial\_read\_f(\textcolor{input}{fn}, \textcolor{input}{B}, \textcolor{input}{Z}, \textcolor{input}{S}, \textcolor{input}{start}, \textcolor{input}{end}, \textcolor{output}{\textit{Elements}}, & r - m \\
~~~~~\textcolor{output}{\textit{ParentData}}, \textcolor{output}{\textit{ier}}) & \\
call cg\_npe\_f(\textcolor{input}{type}, \textcolor{output}{\textit{npe}}, \textcolor{output}{\textit{ier}}) & r w m \\
\end{fctbox}

\noindent
\textbf{\textcolor{input}{Input}/\textcolor{output}{\textit{Output}}}

\noindent (Note that for Fortran calls, all integer arguments are integer*4 in 32-bit mode and integer*8 in 64-bit mode.
See ``64-bit Fortran Portability and Issues" section.)

\begin{Ventryi}{\texttt{ElementSectionName}}\raggedright
\item [\texttt{fn}]
      CGNS file index number.
      (\textcolor{input}{Input})
\item [\texttt{B}]
      Base index number, where $1 \leq \text{\texttt{B}} \leq \text{\texttt{nbases}}$.
      (\textcolor{input}{Input})
\item [\texttt{Z}]
      Zone index number, where $1 \leq \text{\texttt{Z}} \leq \text{\texttt{nzones}}$.
      (\textcolor{input}{Input})
\item [\texttt{ElementSectionName}]
      Name of the \texttt{Elements\_t} node.
      (\textcolor{input}{Input} for \texttt{cg\_section\_write};
      \textcolor{output}{\textit{output}} for \texttt{cg\_section\_read})
\item [\texttt{type}]
      Type of element.
      See the eligible types for \texttt{ElementType\_t} in \autoref{s:typedefs}.
      (\textcolor{input}{Input} for \texttt{cg\_section\_write}, \texttt{cg\_npe};
      \textcolor{output}{\textit{output}} for \texttt{cg\_section\_read})
\item [\texttt{start}]
      Index of first element in the section.
      (\textcolor{input}{Input} for
      \texttt{cg\_section\_write},
      \texttt{cg\_section\_partial\_write},
      \texttt{cg\_parent\_data\_partial\_write},
      \texttt{cg\_ElementPartialSize},
      \texttt{cg\_elements\_partial\_write};
      \textcolor{output}{\textit{output}} for
      \texttt{cg\_section\_read})
\item [\texttt{end}]
      Index of last element in the section.
      (\textcolor{input}{Input} for
      \texttt{cg\_section\_write},
      \texttt{cg\_section\_partial\_write},
      \texttt{cg\_parent\_data\_partial\_write},
      \texttt{cg\_ElementPartialSize},
      \texttt{cg\_elements\_partial\_write};
      \textcolor{output}{\textit{output}} for
      \texttt{cg\_section\_read})
\item [\texttt{nbndry}]
      Index of last boundary element in the section.
      Set to zero if the elements are unsorted.
      (\textcolor{input}{Input} for \texttt{cg\_section\_write};
      \textcolor{output}{\textit{output}} for \texttt{cg\_section\_read})
\item [\texttt{nsections}]
      Lower range index (eg., \texttt{imin, jmin, kmin}).
      (\textcolor{output}{\textit{Output}})
\item [\texttt{S}]
      Element section index, where $1 \leq \text{\texttt{S}} \leq \text{\texttt{nsections}}$.
      (\textcolor{input}{Input} for \texttt{cg\_parent\_data\_write},
      \texttt{cg\_section\_read}, \texttt{cg\_ElementDataSize},
      \texttt{cg\_elements\_read};
      \textcolor{output}{\textit{output}} for \texttt{cg\_section\_write})
\item [\texttt{parent\_flag}]
      Flag indicating if the parent data are defined.
      If the parent data exist, \texttt{parent\_flag} is set to 1;
      otherwise it is set to 0.
      (\textcolor{output}{\textit{Output}})
\item [\texttt{ElementDataSize}]
      Number of element connectivity data values.
      (\textcolor{output}{\textit{Output}})
\item [\texttt{Elements}]
      Element connectivity data.
      (\textcolor{input}{Input} for \texttt{cg\_section\_write};
      \textcolor{output}{\textit{output}} for \texttt{cg\_elements\_read})
\item [\texttt{ParentData}]
      For boundary or interface elements, this array contains
      information on the cell(s) and cell face(s) sharing the element.
      If you do not need to read the \texttt{ParentData} when reading
      the \texttt{ElementData}, you may set the value to \texttt{NULL}.
      (\textcolor{output}{\textit{Output}})
\item [\texttt{npe}]
      Number of nodes for an element of type \texttt{type}.
      (\textcolor{output}{\textit{Output}})
\item [\texttt{ier}]
      Error status.
      (\textcolor{output}{\textit{Output}})
\end{Ventryi}

It is important to note that each element under a given \texttt{Zone\_t} -- including
all cells, faces, edges, boundary elements, etc. -- must have a unique element index
number. The numbering should be consecutive (i.e., no gaps).
This global numbering system insures that each and every element
within a zone is uniquely identified by its number.

If the specified \texttt{Elements\_t} node doesn't yet exist,
it may be created using either \texttt{cg\_sec\-tion\_write} or
\texttt{cg\_section\_partial\_write}.
The function \texttt{cg\_section\_write} writes the full range as
indicated by \texttt{start} and \texttt{end}
and supplied by the element connectivity array \texttt{Elements}. The 
\texttt{cg\_section\_partial\_write} function will create the element 
section data for the range start to end with the element data intialized 
to $0$. To add elements to the section, use \texttt{cg\_elements\_partial\_write} 
and parent data (it it exists) using \texttt{cg\_parent\_data\_partial\_write}. 
Both of these functions will replace the data for the range as indicated by 
\texttt{start} and \texttt{end} with the new values. In most cases, the data is 
not duplicated in the mid-level library, but written directly from the user data 
to disk. The exception to this is in the case of \texttt{MIXED}, 
\texttt{NGON\_n}, and \texttt{NFACE\_n} element sets. Since the size of the 
element connectivity array is not known directly, the MLL will keep a copy 
of the data in memory for the partial writes.

The function \texttt{cg\_elements\_read} returns all of the element
connectivity and parent data.
Specified subsets of the element connectivity and parent data may be
read using \texttt{cg\_elements\_partial\_read}.

\subsection{Axisymmetry}
\label{s:axisymmetry}

\noindent
\textit{Node}: \texttt{Axisymmetry\_t}

\begin{fctbox}
\textcolor{output}{\textit{ier}} = cg\_axisym\_write(\textcolor{input}{int fn}, \textcolor{input}{int B}, \textcolor{input}{float *ReferencePoint}, & - w m \\
~~~~~~\textcolor{input}{float *AxisVector}); & \\
\textcolor{output}{\textit{ier}} = cg\_axisym\_read(\textcolor{input}{int fn}, \textcolor{input}{int B}, \textcolor{output}{\textit{float *ReferencePoint}}, & r - m \\
~~~~~~\textcolor{output}{\textit{float *AxisVector}}); & \\
\hline
call cg\_axisym\_write\_f(\textcolor{input}{fn}, \textcolor{input}{B}, \textcolor{input}{ReferencePoint}, \textcolor{input}{AxisVector}, \textcolor{output}{\textit{ier}}) & - w m \\
call cg\_axisym\_read\_f(\textcolor{input}{fn}, \textcolor{input}{B}, \textcolor{output}{\textit{ReferencePoint}}, \textcolor{output}{\textit{AxisVector}}, \textcolor{output}{\textit{ier}}) & r - m \\
\end{fctbox}

\noindent
\textbf{\textcolor{input}{Input}/\textcolor{output}{\textit{Output}}}

\noindent (Note that for Fortran calls, all integer arguments are integer*4 in 32-bit mode and integer*8 in 64-bit mode.
See ``64-bit Fortran Portability and Issues" section.)

\begin{Ventryi}{\texttt{ReferencePoint}}\raggedright
\item [\texttt{fn}]
      CGNS file index number.
      (\textcolor{input}{\textit{Input}})
\item [\texttt{B}]
      Base index number, where $1 \leq \text{\texttt{B}} \leq \text{\texttt{nbases}}$.
      (\textcolor{input}{\textit{Input}})
\item [\texttt{ReferencePoint}]
      Origin used for defining the axis of rotation.
      (In Fortran, this is an array of Real*4 values.)
      (\textcolor{input}{Input} for \texttt{cg\_axisym\_write};
      \textcolor{output}{\textit{output}} for \texttt{cg\_axisym\_read})
\item [\texttt{AxisVector}]
      Direction cosines of the axis of rotation, through the reference
      point.
      (In Fortran, this is an array of Real*4 values.)
      (\textcolor{input}{Input} for \texttt{cg\_axisym\_write};
      \textcolor{output}{\textit{output}} for \texttt{cg\_axisym\_read})
\item [\texttt{ier}]
      Error status.
      (\textcolor{output}{\textit{Output}})
\end{Ventryi}

This node can only be used for a bi-dimensional model, i.e.,
\texttt{PhysicalDimension} must equal two.

\subsection{Rotating Coordinates}
\label{s:rotatingcoordinates}

\noindent
\textit{Node}: \texttt{RotatingCoordinates\_t}

\begin{fctbox}
\textcolor{output}{\textit{ier}} = cg\_rotating\_write(\textcolor{input}{float *RotationRateVector}, & - w m \\
~~~~~~\textcolor{input}{float *RotationCenter}); & \\
\textcolor{output}{\textit{ier}} = cg\_rotating\_read(\textcolor{output}{\textit{float *RotationRateVector}}, & r - m \\
~~~~~~\textcolor{output}{\textit{float *RotationCenter}}); & \\
\hline
call cg\_rotating\_write\_f(\textcolor{input}{RotationRateVector}, \textcolor{input}{RotationCenter}, \textcolor{output}{\textit{ier}}) & - w m \\
call cg\_rotating\_read\_f(\textcolor{output}{\textit{RotationRateVector}}, \textcolor{output}{\textit{RotationCenter}}, \textcolor{output}{\textit{ier}}) & r - m \\
\end{fctbox}

\noindent
\textbf{\textcolor{input}{Input}/\textcolor{output}{\textit{Output}}}

\noindent (Note that for Fortran calls, all integer arguments are integer*4 in 32-bit mode and integer*8 in 64-bit mode.
See ``64-bit Fortran Portability and Issues" section.)

\begin{Ventryi}{\texttt{RotationRateVector}}\raggedright
\item [\texttt{RotationRateVector}]
      Components of the angular velocity of the grid about the center
      of rotation.
      (In Fortran, this is an array of Real*4 values.)
      (\textcolor{input}{Input} for \texttt{cg\_rotating\_write};
      \textcolor{output}{\textit{output}} for \texttt{cg\_rotating\_read})
\item [\texttt{RotationCenter}]
      Coordinates of the center of rotation.
      (In Fortran, this is an array of Real*4 values.)
      (\textcolor{input}{Input} for \texttt{cg\_rotating\_write};
      \textcolor{output}{\textit{output}} for \texttt{cg\_rotating\_read})
\item [\texttt{ier}]
      Error status.
      (\textcolor{output}{\textit{Output}})
\end{Ventryi}
