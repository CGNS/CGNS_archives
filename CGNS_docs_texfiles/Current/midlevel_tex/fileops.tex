\section{File Operations}
\label{s:fileops}
\thispagestyle{plain}

\subsection{Opening and Closing a CGNS File}
\label{s:openclose}

\begin{fctbox}
\textcolor{output}{\textit{ier}} = cg\_open(\textcolor{input}{char *filename}, \textcolor{input}{int mode}, \textcolor{output}{\textit{int *fn}}); & r w m \\
\textcolor{output}{\textit{ier}} = cg\_version(\textcolor{input}{int fn}, \textcolor{output}{\textit{float *version}});         & r w m \\
\textcolor{output}{\textit{ier}} = cg\_close(\textcolor{input}{int fn});                           & r w m \\
\textcolor{output}{\textit{ier}} = cg\_is\_cgns(\textcolor{input}{const char *filename}, \textcolor{output}{\textit{int *file\_type}});                           & r w m \\
\textcolor{output}{\textit{ier}} = cg\_save\_as(\textcolor{input}{int fn}, \textcolor{input}{const char *filename}, \textcolor{input}{int file\_type}, & r w m \\
~~~~~~\textcolor{input}{int follow\_links}); & \\
\textcolor{output}{\textit{ier}} = cg\_set\_file\_type(\textcolor{input}{int file\_type});                           & r w m \\
\textcolor{output}{\textit{ier}} = cg\_get\_file\_type(\textcolor{input}{int fn}, \textcolor{output}{\textit{int *file\_type}});                           & r w m \\
\hline
call cg\_open\_f(\textcolor{input}{filename}, \textcolor{input}{mode}, \textcolor{output}{\textit{fn}}, \textcolor{output}{\textit{ier}})          & r w m \\
call cg\_version\_f(\textcolor{input}{fn}, \textcolor{output}{\textit{version}}, \textcolor{output}{\textit{ier}})              & r w m \\
call cg\_close\_f(\textcolor{input}{fn}, \textcolor{output}{\textit{ier}})                         & r w m \\
call cg\_is\_cgns\_f(\textcolor{input}{filename}, \textcolor{output}{\textit{file\_type}}, \textcolor{output}{\textit{ier}})                         & r w m \\
call cg\_save\_as\_f(\textcolor{input}{fn}, \textcolor{input}{filename}, \textcolor{input}{file\_type}, \textcolor{input}{follow\_links}, \textcolor{output}{\textit{ier}})                         & r w m \\
call cg\_set\_file\_type\_f(\textcolor{input}{file\_type}, \textcolor{output}{\textit{ier}})                         & r w m \\
call cg\_get\_file\_type\_f(\textcolor{input}{fn}, \textcolor{output}{\textit{file\_type}}, \textcolor{output}{\textit{ier}})                         & r w m \\
\end{fctbox}

\noindent
\textbf{\textcolor{input}{Input}/\textcolor{output}{\textit{Output}}}

\noindent (Note that for Fortran calls, all integer arguments are integer*4 in 32-bit mode and integer*8 in 64-bit mode.
See ``64-bit Fortran Portability and Issues" section.)

\begin{Ventryi}{\texttt{follow\_links}}\raggedright
\item [\texttt{filename}]
      Name of the CGNS file, including path name if necessary.
      There is no limit on the length of this character variable.
      (\textcolor{input}{Input})
\item [\texttt{mode}]
      Mode used for opening the file.
      The modes currently supported are \texttt{CG\_MODE\_READ},
      \texttt{CG\_MODE\_WRITE}, and \texttt{CG\_MODE\_MODIFY}.
      (\textcolor{input}{Input})
\item [\texttt{fn}]
      CGNS file index number.
      (\textcolor{input}{Input} for \texttt{cg\_version} and
      \texttt{cg\_close}; \textcolor{output}{\textit{output}} for
      \texttt{cg\_open})
\item [\texttt{version}]
      CGNS version number.
      (\textcolor{output}{\textit{Output}})
\item [\texttt{file\_type}]
      Type of CGNS file.  This will typically be either \texttt{CG\_FILE\_ADF} or
      \texttt{CG\_FILE\_HDF5} depending on the underlying file format.
      (\textcolor{input}{Input} for \texttt{cg\_save\_as} and
      \texttt{cg\_set\_file\_type}; \textcolor{output}{\textit{output}} for
      \texttt{cg\_get\_file\_type})  \\
      However, note that when built in 32-bit, there is also an option
      to create a Version 2.5 CGNS file by setting the file type to
      \texttt{CG\_FILE\_ADF2}.
\item [\texttt{follow\_links}]
      This flag determines whether links are left intact when saving a CGNS 
      file. If non-zero, then the links will be removed and the data associated 
      with the linked files copied to the new file.
      (\textcolor{input}{Input})
\item [\texttt{ier}]
      Error status.
      (\textcolor{output}{\textit{Output}})
\end{Ventryi}

The function \texttt{cg\_open} must always be the first one called.
It opens a CGNS file for reading and/or writing and returns an index
number \texttt{fn}.
The index number serves to identify the CGNS file in subsequent
function calls.
Several CGNS files can be opened simultaneously.
The current limit on the number of files opened at once depends on the
platform.
On an SGI workstation, this limit is set at 100 (parameter
\texttt{FOPEN\_MAX} in \textit{stdio.h}).

The file can be opened in one of the following modes:

\begin{Ventryic}{\texttt{CG\_MODE\_MODIFY}}
\item [\texttt{CG\_MODE\_READ}]
      Read only mode.
\item [\texttt{CG\_MODE\_WRITE}]
      Write only mode.
\item [\texttt{CG\_MODE\_MODIFY}]
      Reading and/or writing is allowed.
\end{Ventryic}

When the file is opened, if no \texttt{CGNSLibraryVersion\_t} node is
found, a default value of 1.05 is assumed for the CGNS version number.
Note that this corresponds to an old version of the CGNS standard, that
doesn't include many data structures supported by the current standard.

The function \texttt{cg\_close} must always be the last one called.
It closes the CGNS file designated by the index number \texttt{fn} and
frees the memory where the CGNS data was kept.
When a file is opened for writing, \texttt{cg\_close} writes all the
CGNS data in memory onto disk prior to closing the file.
Consequently, if is omitted, the CGNS file is not written properly.

In order to reduce memory usage and improve execution speed,
large arrays such as grid coordinates or flow solutions are not actually
stored in memory.
Instead, only basic information about the node is kept,
while reads and writes of the data is directly to and from the
application's memory. An attempt is also made to do the same with
unstructured mesh element data.

The function \texttt{cg\_is\_cgns} may be used to determine if a file is a 
CGNS file or not, and the type of file (\texttt{CG\_FILE\_ADF} or \texttt{CG\_FILE\_HDF5}). 
If the file is a CGNS file, \texttt{cg\_is\_cgns} returns \texttt{CG\_OK}, 
otherwise \texttt{CG\_ERROR} is returned and \texttt{file\_type} is set to 
\texttt{CG\_FILE\_NONE}. 

The CGNS file identified by \texttt{fn} may be saved to a different filename 
and type using \texttt{cg\_save\_as}. In order to save as an HDF5 file, the 
library must have been built with HDF5 support. ADF support is always built.
The function 
\texttt{cg\_set\_file\_type} sets the default file type for
newly created CGNS files. The function 
\texttt{cg\_get\_file\_type} returns the file type for the CGNS file identified by 
\texttt{fn}. If the CGNS library is built as 32-bit, the additional
file type, \texttt{CG\_FILE\_ADF2}, is available. This allows creation
of a 2.5 compatible CGNS file.

\newpage
\subsection{Configuring CGNS Internals}
\label{s:configure}

\begin{fctbox}
\textcolor{output}{\textit{ier}} = cg\_configure(\textcolor{input}{int option}, \textcolor{input}{void *value}); & r w m \\
\textcolor{output}{\textit{ier}} = cg\_error\_handler(\textcolor{input}{void (*)(int, char *)}); & r w m \\
\textcolor{output}{\textit{ier}} = cg\_set\_compress(\textcolor{input}{int compress}); & r w m \\
\textcolor{output}{\textit{ier}} = cg\_get\_compress(\textcolor{output}{\textit{int *compress}}); & r w m \\
\textcolor{output}{\textit{ier}} = cg\_set\_path(\textcolor{input}{const char *path}); & r w m \\
\textcolor{output}{\textit{ier}} = cg\_add\_path(\textcolor{input}{const char *path}); & r w m \\
\hline
call cg\_set\_compress\_f(\textcolor{input}{compress}, \textcolor{output}{\textit{ier}})          & r w m \\
call cg\_get\_compress\_f(\textcolor{output}{\textit{compress}}, \textcolor{output}{\textit{ier}})          & r w m \\
call cg\_set\_path\_f(\textcolor{input}{path}, \textcolor{output}{\textit{ier}})          & r w m \\
call cg\_add\_path\_f(\textcolor{input}{path}, \textcolor{output}{\textit{ier}})          & r w m \\
\end{fctbox}

\noindent
\textbf{\textcolor{input}{Input}/\textcolor{output}{\textit{Output}}}

\noindent (Note that for Fortran calls, all integer arguments are integer*4 in 32-bit mode and integer*8 in 64-bit mode.
See ``64-bit Fortran Portability and Issues" section.)

\begin{Ventryi}{\texttt{option}}\raggedright
\item [\texttt{option}]
      The option to configure, currently one of
      \texttt{CG\_CONFIG\_ERROR}, \texttt{CG\_CONFIG\_COMPRESS},
      \texttt{CG\_CONFIG\_SET\_PATH}, \texttt{CG\_CONFIG\_ADD\_PATH},
      or \texttt{CG\_CONFIG\_HDF5\_COMPRESS}
      as defined in \textit{cgnslib.h}.
      (\textcolor{input}{Input})
\item [\texttt{value}]
      The value to set, type cast as \texttt{void *}.
      (\textcolor{input}{Input})
\item [\texttt{compress}]
      CGNS compress (rewrite) setting).
      (\textcolor{input}{Input} for \texttt{cg\_set\_compress}; \textcolor{output}{\textit{output}} for
      \texttt{cg\_get\_compress})
\item [\texttt{path}]
      Pathname to search for linked to files when opening a file with external links.
      (\textcolor{input}{Input})
\item [\texttt{ier}]
      Error status.
      (\textcolor{output}{\textit{Output}})
\end{Ventryi}

The function \texttt{cg\_configure} allows certain CGNS library internal
options to be configured.
The currently supported options and expected values are:

\begin{Ventryi}{\texttt{CG\_CONFIG\_SET\_PATH}}
\item [\texttt{CG\_CONFIG\_ERROR}]
      This allows an error call-back function to be defined by the
      user.
      The value should be a pointer to a function to receive the error.
      The function is defined as \texttt{void err\_callback(int is\_error,
      char *errmsg)}, and will be called for errors and warnings.
      The first argument, \texttt{is\_error}, will be 0 for warning
      messages, 1 for error messages, and $-1$ if the program is
      going to terminate (i.e., a call to \texttt{cg\_error\_exit()}).
      The second argument is the error or warning message.
      If this is defined, warning and error messages will go to the
      function, rather than the terminal.
      A \texttt{value} of \texttt{NULL} will remove the call-back
      function.
\item [\texttt{CG\_CONFIG\_COMPRESS}]
      When a CGNS file is closed after being opened in modify mode, the
      normal operation of the CGNS library is to rewrite the file if
      there is unused space.
      This happens when nodes have been rewritten or deleted.
      Setting \texttt{value} to 0 will prevent the library from
      rewriting the file, and setting it to 1 will force the rewrite.
      The default value is $-1$.
\item [\texttt{CG\_CONFIG\_SET\_PATH}]
      Sets the search path for locating linked-to files.
      The argument \texttt{value} should be a character string
      containing one or more directories, formatted the same as for the
      \texttt{PATH} environment variable.
      This will replace any current settings.
      Setting \texttt{value} to \texttt{NULL} will remove all paths.
\item [\texttt{CG\_CONFIG\_ADD\_PATH}]
      Adds a directory, or list of directories, to the linked-to file
      search path.
      This is the same as \texttt{CG\_CONFIG\_SET\_PATH}, but adds to
      the path instead of replacing it.
\item [\texttt{CG\_CONFIG\_HDF5\_COMPRESS}]
      Sets the compression level for data written from HDF5. The default 
      is no compression. Setting \texttt{value} to $-1$, will use the default 
      compression level of $6$. The acceptable values are $0$ to $9$, 
      corresponding to gzip compression levels.
\end{Ventryi}

The routines \texttt{cg\_error\_handler}, \texttt{cg\_set\_compress}, 
\texttt{cg\_set\_path}, and \texttt{cg\_add\_path} are convenience functions 
built on top of \texttt{cg\_configure}.

There is no Fortran counterpart to function \texttt{cg\_configure} or 
\texttt{cg\_error\_handler}. 

\textit{Note}: The HDF5 implementation does not support search paths
for linked files. The links need to be either absolute or relative
pathnames. As a result, it is recommended that the search path options
not be used as they may be removed in future versions.

\subsection{Interfacing with CGIO}
\label{s:cgiointer}

\begin{fctbox}
\textcolor{output}{\textit{ier}} = cg\_get\_cgio(\textcolor{input}{int fn}, \textcolor{output}{\textit{int *cgio\_num}}); & r w m \\
\textcolor{output}{\textit{ier}} = cg\_root\_id(\textcolor{input}{int fn}, \textcolor{output}{\textit{double *rootid}}); & r w m \\
\hline
call cg\_get\_cgio\_f(\textcolor{input}{fn}, \textcolor{output}{\textit{cgio\_num}}, \textcolor{output}{\textit{ier}}) & r w m \\
call cg\_root\_id\_f(\textcolor{input}{fn}, \textcolor{output}{\textit{rootid}}, \textcolor{output}{\textit{ier}}) & r w m \\
\end{fctbox}

\noindent
\textbf{\textcolor{input}{Input}/\textcolor{output}{\textit{Output}}}

\noindent (Note that for Fortran calls, all integer arguments are integer*4 in 32-bit mode and integer*8 in 64-bit mode.
See ``64-bit Fortran Portability and Issues" section.)

\begin{Ventryi}{\texttt{cgio\_num}}\raggedright
\item [\texttt{fn}]
      CGNS file index number.
      (\textcolor{input}{\textit{Input}})
\item [\texttt{cgio\_num}]
      CGIO indentifier for the CGNS file.
      (\textcolor{output}{\textit{Output}})
\item [\texttt{rootid}]
      Root node identifier for the CGNS file.
      (\textcolor{output}{\textit{Output}})
\item [\texttt{ier}]
      Error status.
      (\textcolor{output}{\textit{Output}})
\end{Ventryi}

These allow for the use of the low-level CGIO functions in conjunction
with the Mid Level Library. The function \texttt{cg\_get\_cgio} returns
the CGIO database identifier for the specified CGNS file,
which is used in the CGIO routines. The root node identifier for the
CGNS file is returned by \texttt{cg\_root\_id}. 

