\section{Standard and Software Governing Principles}
\label{s:principles}
\thispagestyle{plain}

\subsection{Distribution}
\label{s:distribution}

This section describes the policy governing the distribution of the
CGNS standard and software to the engineering and scientific community
at large.
By definition, the CGNS \textit{standard} refers to the
Standard Interface Data Structures
(SIDS) definitions, the SIDS File Mapping, and the CGNS Mid-Level
Library structure (API), as well as all documentation.
The CGNS \textit{software} refers to the ADF Core source code, the CGNS
Library source code, and the ADF and HDF5 database manager implementations.
The CGNS \textit{software} may also include sample programs
demonstrating the application and use of the CGNS and ADF libraries,
as well as some utility programs to assist with the implementation and
analysis of CGNS-based files and systems.

Implementation and maintenance of the CGNS distribution policy is the
responsibility of the CGNS Steering Committee.
The distribution policy dictates that both the CGNS standard and the
CGNS software are publicly available, and that the standard and software
itself are free of charge.
The CGNS software may be used for any purpose, including commercial
applications, and may be altered and redistributed, subject to the
restrictions described in the CGNS License (see the
\hyperref[s:license]{Appendix}).

It is the responsibility of the CGNS Steering Committee to enable
distribution mechanisms that comply with the following principles:

\begin{itemize}
\item The CGNS standard (documentation and definitions) will be publicly
      available at no more than the cost of distribution.
\item The CGNS software (ADF core and CGNS Library), including source
      code, will also be available at no more than the cost of
      distribution.
\item The CGNS API (Mid-Level Library), including source code will be
      similarly available.
\item Development, sale, and licensing of proprietary packages based on
      CGNS that perform substantive operations on the data, beyond the
      I/O performed by the API, are encouraged.
      Such packages must abide by the restrictions described in the
      CGNS License.
\item The sale of services designed to assist in the conversion of
      existing software to the CGNS standard is acceptable.
\item The voluntary contribution of software that performs operations on
      CGNS data is encouraged.
\item The CGNS Steering Committee will provide mechanisms for the
      accumulation and distribution of contributed software, but will not
      be responsible for the function of contributed software.
\item Contributed software does not become part of the CGNS Standard,
      that is, either the SIDS or the API, without the approval of the
      CGNS Steering Committee.
\item The Steering Committee may agree to support or endorse additional
      utility software.
\item The Steering Committee will \emph{not} endorse third party
      software.
\end{itemize}

\subsection{Changes or Additions to the Standard}
\label{s:changes}

CGNS is a standard that has been developed with the key concepts of
flexibility and extendibility in mind.
The standard can accommodate the majority of CFD data quantities in
practical usage today; however, some additional capabilities are still
being implemented.
It is also understood that in the future other additional capabilities
will need to be implemented as well.
For these reasons, a process for adding to or modifying the existing
CGNS standard is necessary.

To address a particular need or deficiency in CGNS, a proposal for a
potential change to the standard first must be made.
A Technical Team will prepare all proposals.
A Technical Team may voluntarily submit the proposal, or a Technical
Team may be specifically appointed by the Steering Committee to author
the proposal.

A primary requirement of all proposals for modifications will be to
support and maintain code compatibility.
No additions or changes to the CGNS standard will be adopted which
invalidate existing software or data.

Prior to adoption, the Technical Team must present all proposals in
an open and public forum.
Included with the proposal, a draft of the necessary changes to the
SIDS and File Mapping must be provided by the team introducing the
modifications.
The open forum will then review the proposal, identify any possible
shortcomings, and suggest alternatives or improvements.

After the proposal has been presented and deliberated upon, only the
Steering Committee has final authority of approval and may elect to
do one of three things.
First, the Steering Committee may vote by consensus (or a two-thirds
majority if necessary) to accept the proposal as is, and thus the
changes are approved for implementation.
If such approval does not occur, the Steering Committee may still feel
there is merit to the proposal, and may choose to defer acceptance of
the proposal under the provision that specific changes be made.
Finally, the Steering Committee may deem there is little merit in the
proposed changes to CGNS, and reserves the right to reject the proposal
outright.
Whatever the disposition of the proposal, individual organizations may
implement \texttt{UserDefined} functions, provided that they adhere to
the conventions and standards as defined in the SIDS.
