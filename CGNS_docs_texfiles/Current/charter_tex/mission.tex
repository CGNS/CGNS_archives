\section{Mission/Vision/Responsibilities}
\label{s:mission}
\thispagestyle{plain}

The mission of the CGNS Steering Committee is to ensure the continuation
of the CFD General Notation System.
The survival of a standard depends entirely on its level of use.
Therefore the CGNS Steering Committee must aim at providing a standard
that is widely accepted by the CFD community.

Several elements must be satisfied to ensure the acceptance of the
CGNS standard.
The most obvious asset is that CGNS must be useful.
Not only must it answer the current needs for the recording of fluid
dynamics data, but it must also follow the changes in requirements as
CFD progresses.
A second important element is that CGNS must be easy to implement.
The CGNS Mid-Level Library (or
Application Programming Interface, API) must be user-friendly and
well documented, and online
support must be available for all users at all times.
The standard must also be easily accessible, meaning that all the
sources, binaries, documents and any other pertinent information must be
available to anyone without restrictions.
Finally, it is of utmost importance that CGNS retains its public nature,
encouraging contributions from all users.

The Steering Committee has the responsibility to oversee that the
CGNS standard remains useful, accessible, easy to use, and preserves
its public nature.
This implies multiple activities, which can be subdivided in the
following groups:

\subsection{Ensure the maintenance of the existing software, documentation and web site}
\label{s:maintenance}

The CGNS Steering Committee is responsible for appointing a prime
source, and overseeing the prime source activities.
The Steering Committee must ensure that the prime source maintains the
existing software, documentation and web site.
This includes, but is not limited to:

\begin{itemize}
\item correcting/updating the documentation if necessary
\item fixing any reported software bug
\item collecting a list of CGNS users via the web site
\item keeping the web site up to date with the latest versions of the
      documentation and
      software
\item informing the user base of new releases and major software problems
\item posting proposals for
      new features or modifications to the CGNS standard on the web site
      and collect comments from the user base
\item maintaining a distribution site for
      contributed software
      utilities which utilize the CGNS standard
\end{itemize}

\subsection{Provide mechanisms for the evolution of the standard}
\label{s:evolution}

The CGNS Steering Committee has the responsibility to support and even
encourage the evolution of the standard in order for CGNS to remain
useful.
Therefore, the committee must solicit technical support and ``in-kind''
contributions.
In addition, the Steering Committee must follow the policies described
in \autoref{s:changes} of this document
regarding the collection and evaluation of technical
proposals.

\subsection{Promote the acceptance of CGNS}
\label{s:acceptance}

The CGNS Steering Committee has the responsibility to promote the
acceptance of CGNS throughout the CFD community.
This can be achieved through various means, including word of mouth,
advertising, business articles, and presentations at conferences and
technical meetings.

\subsection{Provide mechanism for answering questions and exchanging ideas}
\label{s:services}

Electronic mail constitutes the main point of contact between CGNS
users and CGNS developers.
Therefore, the CGNS Steering Committee must maintain an electronic mail
forum, to which users can post questions, answer questions, and exchange
ideas.
Members of the CGNS Steering Committee and/or appointed qualified
persons will respond to the posted questions on the forum.

\subsection{Determine the means by which the CGNS activities are supported}
\label{s:support}

The CGNS Steering Committee has the obligation to determine the means by
which all CGNS activities are supported.
The Committee is also responsible for identifying and obtaining sources
of funding, if appropriate.
Finally the CGNS Steering Committee has the responsibility to distribute
the tasks and funds to the most appropriate candidate, in the best
interests of CGNS.
