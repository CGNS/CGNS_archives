\pdfbookmark[1]{Abstract}{abstract}
\addtocontents{toc}{\protect\contentsline {section}{Abstract}{\thepage}{abstract}}
\hypertarget{abstract}{}
\section*{Abstract}
\thispagestyle{plain}

The CFD General Notation System (CGNS) is a standard for recording and recovering computer
data associated with the numerical solution of the equations of fluid dynamics. The intent is to
facilitate the exchange of CFD data between sites, between applications codes, and across
computing platforms, and to stabilize the archiving of CFD data.
The CGNS system consists of a collection of conventions, and software implementing those
conventions, for the storage and retrieval of CFD data. It consists of two parts: (1) a standard
format for recording the data, and (2) software that reads, writes, and modifies data in that format.
The format is a conceptual entity established by the documentation; the software is a physical
product supplied to enable developers to access and produce data recorded in that format.
The standard format, or paper convention, part of CGNS consists of two fundamental pieces. The
first, known as the Standard Interface Data Structures, is described in this Recommended
Practice. It defines the intellectual content of the information to be stored. The second, known as
the File Mapping, defines the exact location in a CGNS file where the data is to be stored.
