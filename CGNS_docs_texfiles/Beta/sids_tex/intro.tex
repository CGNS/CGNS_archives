\section{Introduction}
\thispagestyle{plain}

CGNS (CFD General Notation System) is a collection of conventions,
along with software implementing those conventions, for the storage and
retrieval of CFD (computational fluid dynamics) data.
The CGNS system is designed to facilitate the exchange of data between
sites and applications, as well as to help stabilize the archiving of
fluid dynamic data.
In today's environment, it is important in many technical arenas to
maintain detailed records of scientific computations.
CGNS was designed to help promote a long-lasting and extensible standard
for this purpose.
Many companies and institutions choose to adopt the CGNS standard in
order to increase productivity by (1) reducing the time required to
translate between data created and used by different applications, and
(2) increasing the quality, longevity, and re-usability of archived
data.

The CGNS standard is a conceptual entity established by the
documentation.
The CGNS software is a physical product supplied to enable writing and
reading data according to this standard.
All CGNS software is completely free and open to anyone.
By using the supplied software, it is relatively easy for users to
adhere to most of the standard described in detail in this document.

The CGNS project originated during 1994
through a series of meetings that addressed improved transfer of NASA
technology to industry. A principal impediment in this process was
the disparity in I/O formats employed by various flow codes, grid
generators, and other utilities, and CGNS was conceived as a means to
promote ``plug-and-play'' CFD.  Agreement was reached to develop CGNS
at Boeing, under NASA Contract NAS1-20267, with active participation
by a team of CFD researchers from NASA's Langley, Lewis (now Glenn),
and Ames Research Centers, McDonnell Douglas Corporation (now part of
Boeing), and Boeing Commercial Airplane Group.  This team, which was
joined by ICEM CFD Engineering Corporation of Berkeley, California in
1997, undertook the core of the development. However, in the spirit of
creating a completely open and broadly accepted standard, all interested
parties were encouraged to participate; the US Air Force and Arnold
Engineering Development Center were notably present. From the beginning,
the purpose was to develop a system that could be distributed freely,
including all documentation, software and source code. This goal has now
been fully realized; further, control of CGNS has
been completely transferred to a public forum known as the CGNS Steering
Committee.

The principal target is the data normally associated with compressible
viscous flow (i.e., the Navier-Stokes equations), but the standard is
also applicable to subclasses such as Euler and potential flows. The
initial release addressed multi-zone grids, flow fields, boundary
conditions, and zone-to-zone connection information, as well as a number
of auxiliary items, such as non-dimensionalization, reference states,
and equation set specifications. Extensions incorporated since then
include unstructured mesh, connections to geometry definition,
time-dependent flow, and support for multiple species and chemistry.

It is worth noting that extensibility is a fundamental design
characteristic of the system, which in principal could be used for
other disciplines of computational field physics, such as acoustics or
electromagnetics, given the willingness of the cognizant scientific
community to define the conventions.

The standard format, or paper convention, part of CGNS consists of
two fundamental pieces. The first, known as the Standard Interface
Data Structures (SIDS), describes in detail the intellectual content
of the information to be stored. It defines, for example, the precise
meaning of a ``boundary condition''.
The second, known as the SIDS File Mapping defines
the exact location in a CGNS file where the data is to be stored.

The implementation, or software, part of CGNS likewise consists of two
separate entities.
CGNS files are read and written by a stand-alone database manager,
either ADF (Advanced Data Format) or HDF (Hierarchical Data Format).
The database manager implements a tree-like data structure, as a
binary file.
Since the format of this file is completely controlled by the database
manager, and since ADF and HDF are both written in ANSI C (Fortran
wrappers are provided), these files and the database manager itself are
portable to any environment which supports ANSI C.
Both ADF and HDF are available separately and constitute useful tools
for the storage of large quantities of scientific data.

The underlying database manager, however, implements no knowledge of
CFD or of the File Mapping.
To simplify access to CGNS files, a second layer of software known as
the Mid-Level Library is provided.
This layer is in effect an API, or Application Programming Interface for
CFD.
The API incorporates knowledge of the CFD data structures, their meaning
and their location in the file, enabling applications such as flow codes
and grid generators to access the data in familiar terms.
The API is therefore the piece of the CGNS system most visible to
applications developers.
Like the ADF and HDF database managers, the Mid-Level Library is written
in ANSI C; all public API routines have Fortran counterparts.

This document presents the formal definition of the Standard Interface
Data Structures (SIDS). \autoref{s:design} presents the major design
philosophies used to develop the CGNS database and the encoding of
this database into the SIDS; this section also provides an overview
of the database hierarchy. \autoref{s:conv} describes the C-like
nomenclature conventions used to define the SIDS.  This section also
gives the conventions for structured grid indexing and unstructured
element numbering, and the nomenclature for multizone interfaces.
Low-level building-block structures are defined in \autoref{s:build};
these structures are used to define all higher-level structures.
Structures for defining data arrays, including dimensional-units and
nondimensional information, are presented in \autoref{s:data}.  The
top levels of the CGNS hierarchy are next defined in \autoref{s:topo}.
The following sections then fill out the remainder of the hierarchy:
\autoref{s:gridflow} defines the grid-coordinate, elements, and
flow-solution structures; \autoref{s:connectivity} defines the
multizone interface connectivity structures; \autoref{s:BC} defines
boundary-condition structures; \autoref{s:floweqn} defines structures
for describing governing flow equations; \autoref{s:timedep} defines
structures related to time-dependent flows; and \autoref{s:misc}
contains miscellaneous structures.  Two appendices complete the
document. \hyperref[s:dataname]{Appendix~\ref*{s:dataname}} provides
naming conventions for data contained within the CGNS database, and
\hyperref[s:twozone]{Appendix~\ref*{s:twozone}} contains a complete SIDS
description of a structured-grid two-zone test case.

\subsection{Major Differences from Previous CGNS Versions}
\label{s:differences}
The following items represent noteworthy alterations and additions to
the SIDS starting with the August 1999 draft document.
(Note that some of these changes --- notably those for unstructured
zones, family, and geometry reference --- have existed previously in
separate documents, but are now being merged officially into the SIDS;
the data structures themselves are not ``new.'')

\subsubsection{Version 2.0, Beta 1}
The following changes were made for Version 2.0, Beta 1.

\begin{itemize}
\item The capability for recording unstructured zones has been 
      added to the SIDS.
      (These changes occur throughout the document, although some
      specific items are listed below.)
\item The values \texttt{UserDefined} and \texttt{Null} are now allowed for
      all enumeration types (throughout document).
\item The following nodes are now defined (some of these also
      include additional new children sub-nodes): \texttt{Family\_t}
      (\autoref{s:Family}), \texttt{Elements\_t} (\autoref{s:Elements}),
      \texttt{ZoneType\_t} (\autoref{s:Zone}), \texttt{FamilyName\_t}
      (\autoref{s:Zone}), \texttt{GeometryReference\_t}
      (\autoref{s:GeometryReference}), \texttt{FamilyBC\_t}
      (\autoref{s:FamilyBC}).
\item Under \texttt{CGNSBase\_t}, the \texttt{IndexDimension} is no longer
      recorded; it has been replaced by \texttt{CellDimension} and
      \texttt{PhysicalDimension} (\autoref{s:CGNSBase}).
\item Under \texttt{Zone\_t}, the optional parameter
      \texttt{VertexSizeBoundary} has been added for unstructured zones
      (\autoref{s:Zone}).
\item The method for general connectivity (\texttt{GridConnectivity\_t}) 
      has been altered.
      It now requires the use of either (a) \texttt{PointListDonor} (an
      integer, for \texttt{Abutting1to1} only) or (b) \texttt{CellListDonor}
      (an integer) plus \texttt{InterpolantsDonor} (a real)
      (\autoref{s:GridConnectivity}).
\item The \texttt{GridLocation\_t} parameter has been moved up one level
      (from \texttt{BCDataSet\_t} to \texttt{BC\_t}).
      Thus, for example, if the boundary conditions are defined at
      vertices (the default), then any associated dataset information
      must also be specified at vertices (\autoref{s:BCdefn} and
      \autoref{s:BCDataSet}).
\item The data-name identifier \texttt{LengthReference}
      has been added (\autoref{s:ReferenceState} and
      \hyperref[s:dataname_flow]{Appendix~\ref*{s:dataname_flow}}).
\item The $\nu_t$ parameter has been renamed
      \texttt{ViscosityEddyKinematic}, and a new parameter
      \texttt{ViscosityEddy}, representing $\mu_t$, has been defined
      (\hyperref[s:dataname_flow]{Appendix~\ref*{s:dataname_flow}}).
\end{itemize}

\subsubsection{Version 2.0, Beta 2}
The following changes were made for Version 2.0, Beta 2.

\begin{itemize}
\item The following data structures related to time-dependent flow
      have been added:
      \texttt{BaseIterativeData\_t} (\autoref{s:BaseIterativeData}),
      \texttt{ZoneIterativeData\_t} (\autoref{s:ZoneIterativeData}),
      \texttt{RigidGridMotion\_t} (\autoref{s:RigidGridMotion}),
      \texttt{ArbitraryGridMotion\_t} (\autoref{s:ArbitraryGridMotion}).
\end{itemize}

\subsubsection{Version 2.1, Beta 1}
The following changes were made for Version 2.1, Beta 1.

\begin{itemize}
\item A node type
      \texttt{UserDefinedData\_t} (\autoref{s:UserDefinedData})
      is added for the storage of arbitrary user defined data in
      \texttt{Descriptor\_t} and \texttt{DataArray\_t} children without the
      restrictions or implicit meanings imposed on these node types at
      other node locations.
\item Support for multi-species flows and chemistry has been added.
      New gas models have been added to the \texttt{GasModelType\_t}
      enumeration (\autoref{s:GasModel}), and
      \texttt{ThermalRelaxationModel\_t} and \texttt{ChemicalKineticsModel\_t}
      data structures have been added for describing the 
      thermal relaxation and chemical kinetics models
      (\autoref{s:ThermalRelaxationModel} and
      \autoref{s:ChemicalKineticsModel}).
      Additional flow solution data-name identifiers are included
      (\hyperref[s:dataname_flow]{Appendix~\ref*{s:dataname_flow}}).
\end{itemize}

\subsubsection{Version 2.2, Beta 1}
The following changes were made for Version 2.2, Beta 1.

\begin{itemize}
\item \texttt{Axisymmetry\_t} and \texttt{RotatingCoordinates\_t} 
      nodes have been added, allowing the recording of data relevant
      to axisymmetric flows and rotating coordinates
      (\autoref{s:Axisymmetry} and \autoref{s:RotatingCoordinates}).
\item A \texttt{Gravity\_t} node has been added for storage of the
      gravitational vector (\autoref{s:Gravity}).
\item A \texttt{GridConnectivityProperty\_t} node has been added, allowing
      the recording of special properties associated with particular
      connectivity patches, such as periodic interfaces, or interfaces
      where the data is to be averaged in some way prior to passing it
      to a neighboring interface (\autoref{s:GridConnectivityProperty}).
\item A \texttt{BCProperty\_t} node has been added, allowing the recording
      of special properties associated with particular boundary
      condition patches, such as wall function or bleed regions
      (\autoref{s:BCProperty}).
\item Additional flow solution data-name identifiers are included
      for variables in rotating coordinate systems
      (\hyperref[s:dataname_flow]{Appendix~\ref*{s:dataname_flow}}).
\end{itemize}

\subsubsection{Version 2.3}
The following changes were made for Version 2.3.

\begin{itemize}
\item \texttt{ElementRange} and \texttt{ElementList} have been added to the
      \texttt{BC\_t} data structure.
      \texttt{ElementRange} or \texttt{ElementList} may now be used
      to define a boundary condition patch by specifying face indices,
      instead of using \texttt{PointRange} or \texttt{PointList} with
      \texttt{GridLocation} set to \texttt{FaceCenter}.
      The use of \texttt{PointRange} or \texttt{PointList} to define a
      boundary condition patch hasn't changed.
      They may be used to define a boundary condition patch by specifying
      either vertex or face indices.

      When \texttt{PointRange} or \texttt{PointList} is used, the choice
      between vertex or face indices is determined by the value of
      \texttt{GridLocation\_t}.
      When \texttt{ElementRange} or \texttt{ElementList} is used,
      \texttt{GridLocation\_t} is ignored.
      (\autoref{s:BCdefn}).
\end{itemize}

\subsubsection{Version 2.4}
The following changes were made for Version 2.4.

\begin{itemize}
\item \texttt{GridLocation\_t}, \texttt{PointRange}, and \texttt{PointList}
      have been added to the \texttt{BCDataSet\_t} data structure,
      allowing boundary conditions to be specified at locations
      different from those used to defined the BC patch.
      (E.g., a BC patch may be defined using vertices, with boundary
      conditions applied at face centers.)
      (\autoref{s:BCDataSet}).
\item Data structures have been added to \texttt{FlowEquationSet\_t} for
      describing the electric field, magnetic field, and conductivity
      models used for electromagnetic flows.
      Corresponding recommended data-name identifiers have also been
      added.
      (\autoref{s:FlowEquationSet} and \autoref{s:EM}).
\item \texttt{RotatingCoordinates\_t} has been added to the
      \texttt{Family\_t} data structure.
      (\autoref{s:Family}).
\item A \texttt{BCDataSet\_t} list has been added to the
      \texttt{FamilyBC\_t} data structure, allowing specification of
      boundary condition data arrays for CFD families.
      (\autoref{s:FamilyBC}).
\item \texttt{GridLocation\_t}, \texttt{PointRange}, \texttt{PointList},
      \texttt{FamilyName\_t}, \texttt{UserDefinedData\_t}, and
      \texttt{Ordinal} have been added to the \texttt{UserDefinedData\_t}
      data structure.
      (\autoref{s:UserDefinedData}).
\item The \texttt{DimensionalUnits\_t} and \texttt{DimensionalExponents\_t}
      structures have been expanded to include units for electric
      current, substance amount, and luminous intensity.
      (\autoref{s:DimensionalUnits}).
\item The capability to include rind data with unstructured grids
      has been added, and \texttt{Rind\_t} has been added to the
      \texttt{Elements\_t} structure, allowing specification of
      connectivity information for rind elements.
      (\autoref{s:Rind}).
\end{itemize}

\subsubsection{Version 2.5}
No changes were made to the data structures for Version 2.5.

\subsubsection{Version 3.1}
The following changes were made for Version 3.1

\begin{itemize}
\item Added a \texttt{PYRA\_13} element in the description of the unstructured grid 
      element numbering conventions for pyramid elements
      (\autoref{s:unstructgrid_3d}).
\item Modified the description of the \texttt{Elements\_t} structure to incorporate 
      the new definition of the \texttt{NGON\_n} element type, and add the 
      \texttt{NFACE\_n} element type. Also added examples illustrating their use, 
      including for polyhedral elements
      (\autoref{s:Elements}).
\item References to \texttt{Null} were changed to \texttt{xxxxNull}, and references to 
      \texttt{UserDefined} were changed to \texttt{xxxxUserDefined}, as appropriate. 
      Also made sure that the \texttt{xxxxNull} is listed first, and the 
      \texttt{xxxxUserDefined} is listed second.
\item Added section Model Type Structure Definition
      (\autoref{s:ModelType}).
\item Added \texttt{ZoneSubRegion\_t} (CPEX\footnote{CPEX descriptions can be found
      at http://cgns.sourceforge.net/Proposals.html; last accessed 3/14/2013.}
      0030) (\autoref{s:ZoneSubRegion}). 
      Made changes associated with CPEX 
      0031: (1) added CellDimension to several places in \texttt{Zone\_t}, (2) added 
      new ParentElements and ParentElementsPosition in \texttt{Elements\_t}, 
      (3) added CellDimension and revamped usage of PointList and PointRange in 
      \texttt{FlowSolution\_t}, (4) added CellDimension to several places in 
      \texttt{ZoneBC\_t}, (5) added CellDimension and revamped usage of PointList 
      and PointRange in \texttt{BC\_t}, (6) changes in \texttt{BCDataSet\_t}, 
      (7) added CellDimension and revamped usage of PointList and PointRange in 
      \texttt{DiscreteData\_t}, (8) changes in \texttt{FamilyBC\_t}, (9) creation 
      of new \texttt{FamilyBCDataSet\_t}, (10) changes in Zone Grid Connectivities, 
      and (11) and addition of \texttt{ZoneGridConnectivityPointers} and 
      \texttt{ZoneSubRegionPointers} in \texttt{ZoneIterativeData\_t}. 
      A note regarding CPEX 0031:  The use of ElementList and ElementRange has been
      deprecated in favor of PointList and PointRange, as described in CPEX 0031 and
      in relevant sections of this document.
\item Added clarity that if rotating about more than 1 axis, then it is done in a 
      particular order in \texttt{Periodic\_t} and in Data-Name Identifiers for 
      Rigid Grid Motion. 
      (\autoref{t:id_rigidgrid})
\item Modified table in \autoref{s:BCdefn} to include CellCenter GridLocation.
\end{itemize}

\subsubsection{Version 3.2}
The following changes were made for Version 3.2

\begin{itemize}
\item Added AdditionalFamilyName\_t under BC\_t, Zone\_t, ZoneSubRegion\_t, and
      UserDefinedData\_t; and added
      FamilyName\_t under Family\_t (a hierarchy of families is now possible); according
      to CPEX 0033 and 0034.
\item Added new cubic elements in \autoref{s:unstructgrid}, according to CPEX 0036.
\end{itemize}
