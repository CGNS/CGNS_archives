\section{Boundary Conditions}
\label{s:BC}
\thispagestyle{plain}

This section is an attempt to unify boundary-condition specifications
within Navier-Stokes codes.  The structures and conventions developed
are a compromise between simplicity and generality.  It is imperative
that they be easy to use initially, but that they are general enough to
provide future flexibility and extensibility.

This section may be somewhat daunting initially.  It is suggested that
the reader refer to the several, well-explained examples presented in
\autoref{s:BCexample} during study of the following sections to help
resolve any questions and confusions that might arise.

The difficulty with boundary conditions is that there is such a wide
variety used, and even a single boundary-condition equation is often
implemented differently in different codes.  Some boundary conditions,
such as a symmetry plane, are fairly well defined.  Other boundary
conditions are much looser in their definition and implementation.
An inflow boundary is a good example.  It is generally accepted how
many solution quantities should be specified at an inflow boundary
(from mathematical well-posedness arguments), but what those quantities
are will change with the class of flow problems (e.g., internal flows
vs.~external flows), and they will also change from code to code.

An additional difficulty for CFD analysis is that in some situations
different boundary-condition equations are applied depending on local
flow conditions.  Any boundary where the flow can change from inflow
to outflow or supersonic to subsonic is a candidate for flow-dependent
boundary-condition equations.

These difficulties have molded the design of our boundary-condition
specification structures and conventions.  We define boundary-condition
types (\autoref{s:BCType}) that establish the equations to be enforced.
However, for those more loosely defined boundary conditions, such as
inflow/outflow, the boundary-condition type merely establishes general
guidelines on the equations to be imposed.  Augmenting (and superseding)
the information provided by the boundary-condition type is precisely
defined boundary-condition solution data.  We rely on our conventions
for data-name identifiers to identify the exact quantities involved in
the boundary conditions; these data-name identifier conventions are
presented in \hyperref[s:dataname]{Appendix~\ref*{s:dataname}}.

One flexibility that is provided by this approach is that
boundary-condition information can easily be built during the course
of an analysis.  For example, during grid-generation phases minimal
information (e.g., the boundary-condition type) may be given.  Then
prior to running of the flow solver, more specific boundary-condition
information, such as Dirichlet or Neumann data, may be added to the
database.

An additional flexibility provided by the structures of this section
is that both uniform and non-uniform boundary-condition data can be
described within the same framework.

We realize that most current codes allow little or no flexibility in
choosing solution quantities to specify for a given boundary-condition
type.  We also realize the coding effort involved with checking for
consistency between I/O specifications and internal boundary-condition
routines.  To make these boundary-condition structures more
palatable initially, we adopt the convention that if no solution
quantities are specified for a given boundary-condition type, then
the code is free to enforce any appropriate boundary condition (see
\autoref{s:BC_specdata}).

Note that there are no boundary-condition structures defined for
abutting or overset interfaces, unless they involve cases of symmetry or
degeneracy.
In other words, it is a CGNS design intent that a given zone boundary
segment or location should at most be defined (or covered) by either a
boundary condition or a multizone interface connectivity, but not by
both.
There is also no separate boundary-condition structure for periodic
boundary conditions (i.e., when a zone interfaces with itself).
Both of these situations are addressed by the interface connectivity
data structures described in \autoref{s:connectivity}.

In the sections to follow, the definitions of boundary-condition
structures are first presented in \autoref{s:BCstruct} through
\autoref{s:BCProperty}.  Boundary-condition types are then discussed
in detail in \autoref{s:BCType}, including a description of the
boundary-condition equations to be enforced for each type; this section
also describes the distinction between boundary-condition types that
impose a set of equations regardless of local flow conditions and those
that impose different sets of boundary-condition equations depending
on the local flow solution.  The rules for matching boundary-condition
types and the appropriate sets of boundary-condition equations are
next discussed in \autoref{s:BCType_assoc}.  Details of specifying
data to be imposed in boundary-condition equations are provided in
\autoref{s:BC_specdata}.  Finally, \autoref{s:BCexample} presents
several examples of boundary conditions.

\subsection{Boundary Condition Structures Overview}
\label{s:BCstruct}

Prior to presenting the detailed boundary condition structures, we give
a brief overview of the hierarchy used to describe boundary conditions.

Boundary conditions are classified as either fixed or flow-dependent.
Fixed boundary conditions enforce a given set of boundary-condition
equations regardless of flow conditions; whereas, flow-dependent
boundary conditions enforce different sets of boundary-condition
equations depending on local flow conditions.  We incorporate both
fixed and flow-dependent boundary conditions into a uniform framework.
This allows all boundary conditions to be described in a similar
manner.  We consider this functionally superior to separately treating
fixed and flow-dependent boundary conditions, even though the latter
allows a simpler description mechanism for fixed boundary conditions.
The current organization also makes sense considering the fact that
flow-dependent boundary conditions are composed of multiple sets of
fixed boundary conditions.

% bctree.tex	BC hierarchical tree figure

\begin{figure}
\begin{picture}(6.5,3.0)(-3.25,-3.0)
% \put(-3.25,-3.0){\framebox(6.5,3.0){}}

 %  CGNS database
 \put(-0.3,-0.3){\framebox(0.6,0.3){BC}}
 \put(0,-0.3){\circle*{0.04}}

 %  BC patch spec
 \put(0,-0.3){\line(-4,-1){2.39}}
 \put(-2.4,-0.9){%
 \begin{picture}(0,0)
  \put(0,0){\circle*{0.04}}
  \put(-0.5,-0.4){\framebox(1.0,0.4){\shortstack{BC patch \\ specification}}}
 \end{picture}}

 %  BC type
 \put(0,-0.3){\line(-5,-3){0.99}}
 \put(-1,-0.9){%
 \begin{picture}(0,0)
  \put(0, 0  ){\circle*{0.04}}
  \put(-0.45,-0.4){\framebox(0.9,0.4){\shortstack{BC type \\ (compound)}}}
 \end{picture}}

 %  BC data set (1)
 \put(0,-0.3){\line(1,-2){0.295}}
 \put(0.3,-0.9){%
 \begin{picture}(0,0)
  \put(0, 0  ){\circle*{0.04}}
  \put(0,-0.3){\circle*{0.04}}
  \put(-0.6,-0.3){\framebox(1.2,0.3){BC data set (1)}}

  %  BC type for data-set 1
  \put(0,-0.3){\line(-5,-2){1.49}}
  \put(-1.5,-0.9){%
  \begin{picture}(0,0)
   \put(0, 0  ){\circle*{0.04}}
   \put(-0.45,-0.4){\framebox(0.9,0.4){\shortstack{BC type \\ (simple)}}}
  \end{picture}}

  %  Dirichlet data for data-set 1
  \put(0,-0.3){\line(0,-1){0.59}}
  \put(0.0,-0.9){%
  \begin{picture}(0,0)
   \put(0, 0  ){\circle*{0.04}}
   \put(0,-0.3){\circle*{0.04}}
   \put(-0.7,-0.3){\framebox(1.4,0.3){Dirichlet BC data}}
   
   \put(-0.55,-1.2){\framebox(0.3,0.3){$\alpha$}} 
   \put( 0.25,-1.2){\framebox(0.3,0.3){$\beta$}} 

   \put(0,-0.3){\line(-2,-3){0.39}}
   \put(0,-0.3){\line( 2,-3){0.39}}
  
   \put(-0.4,-0.9){\circle*{0.04}}
   \put( 0.4,-0.9){\circle*{0.04}}
  \end{picture}}

  %  Neumann data for data-set 1
  \put(0,-0.3){\line(3,-1){1.79}}
  \put(+1.8,-0.9){%
  \begin{picture}(0,0)
   \put(0, 0  ){\circle*{0.04}}
   \put(0,-0.3){\circle*{0.04}}
   \put(-0.7,-0.3){\framebox(1.4,0.3){Neumann BC data}}

   \put(-0.55,-1.2){\framebox(0.3,0.3){$\gamma$}} 
   \put( 0.25,-1.2){\framebox(0.3,0.3){$\delta$}} 

   \put(0,-0.3){\line(-2,-3){0.39}}
   \put(0,-0.3){\line( 2,-3){0.39}}
  
   \put(-0.4,-0.9){\circle*{0.04}}
   \put( 0.4,-0.9){\circle*{0.04}}
  \end{picture}}
 \end{picture}}

 % BC data set (N)
 \put(0,-0.3){\line(4,-1){2.39}}
 \put(+2.4,-0.9){%
 \begin{picture}(0,0)
  \put(0, 0  ){\circle*{0.04}}
  \put(0,-0.3){\circle*{0.04}}
  \put(-0.6,-0.3){\framebox(1.2,0.3){BC data set (N)}}
  \put(0,-0.3){\line(-5,-2){0.25}}
  \put(0,-0.3){\line( 0,-1){0.1}}
  \put(0,-0.3){\line( 3,-1){0.3}}
 \end{picture}}

 %  ...
 \put(+1.3,-0.9){%
 \begin{picture}(0,0)
  \put(0,-0.15){\makebox(0,0){$\cdots$}}
 \end{picture}}
\end{picture}
\caption{Hierarchy for Boundary Condition Structures}
\label{f:bctree}
\end{figure}


\autoref{f:bctree} depicts the hierarchy used for prescribing a single
boundary condition.
Each boundary condition includes a type that describes the general
equations to enforce, a patch specification, and a collection of data
sets.
The minimum required information for any boundary condition is the patch
specification and the boundary-condition type (indicated by ``BC type
(compound)'' in the figure).
This minimum information is similar to that used in many existing
flow solvers.

Generality in prescribing equations to enforce and their associated
boundary-condition data is provided in the optional data sets.  Each
data set contains all boundary condition data required for a given
fixed or simple boundary condition.  Each data set is also tagged
with a boundary-condition type.  For fixed boundary conditions,
the hierarchical tree contains a single data set, and the two
boundary-condition types shown in \autoref{f:bctree} are identical.
Flow-dependent or compound boundary conditions contain multiple
data sets, each to be applied separately depending on local flow
conditions.  The compound boundary-condition type describes the
general flow-dependent boundary conditions, and each data set contains
associated simple boundary-condition types.  For example, a farfield
boundary condition would contain four data sets, where each applies
to the different combinations of subsonic and supersonic inflow and
outflow.  Boundary-condition types are described in \autoref{s:BCType}
and \autoref{s:BCType_assoc}.

Within a single data set, boundary condition data is grouped by
equation type into Dirichlet and Neumann data.  The lower leaves of
\autoref{f:bctree} show data for generic flow-solution quantities
$\alpha$ and $\beta$ to be applied in Dirichlet conditions, and data
for $\gamma$ and $\delta$ to be applied in Neumann boundary conditions.
|DataArray_t| entities are employed to store these data and to identify
the specific flow variables they are associated with.

In situations where the data sets (or any information contained therein)
are absent from a given boundary-condition hierarchy, flow solvers
are free to impose any appropriate boundary conditions.  Although
not pictured in \autoref{f:bctree}, it is also possible to specify
the reference state from which the flow solver should extract the
boundary-condition data.

\subsection{Zonal Boundary Condition Structure Definition: \texttt{ZoneBC\_t}}
\label{s:ZoneBC}

All boundary-condition information pertaining to a given
zone is contained in the |ZoneBC_t| structure.
\begin{alltt}
  ZoneBC\_t< int CellDimension, int IndexDimension, int PhysicalDimension > :=
    \{
    List( Descriptor\_t Descriptor1 ... DescriptorN ) ;                      (o)

    List( BC\_t<CellDimension, IndexDimension, int PhysicalDimension> 
          BC1 ... BCN ) ;                                                   (o)

    ReferenceState\_t ReferenceState ;                                       (o)

    DataClass\_t DataClass ;                                                 (o)
                
    DimensionalUnits\_t DimensionalUnits ;                                   (o)

    List( UserDefinedData\_t UserDefinedData1 ... UserDefinedDataN ) ;       (o)
    \} ;
\end{alltt}

\begin{notes}
\item
 Default names for the \fort{Descriptor\_t}, \fort{BC\_t}, and
 \fort{UserDefinedData\_t}
 lists are as shown; users may choose other legitimate names.
 Legitimate names must be unique within a given instance of
 \fort{ZoneBC\_t} and shall not include the names \fort{DataClass},
 \fort{DimensionalUnits}, or \fort{ReferenceState}.
\item
 All lists within a \fort{ZoneBC\_t} structure entity may be empty.
\end{notes}

\fort{ZoneBC\_t} requires three structure parameters, \fort{CellDimension},
\fort{IndexDimension} and \fort{PhysicalDimension}, which are passed
onto all \fort{BC\_t} substructures.

Boundary-condition information for a single patch is contained in the
\fort{BC\_t} structure.
All boundary-condition information pertaining to a given zone is
contained in the list of \fort{BC\_t} structure entities.
If a zone contains $N$ boundary-condition patches, then $N$ (and
only $N$) separate instances of \fort{BC\_t} must be provided in the
\fort{ZoneBC\_t} entity for the zone.
That is, each boundary-condition patch must be represented by a single
\fort{BC\_t} entity.

Reference data applicable to all boundary conditions of a zone is
contained in the |ReferenceState| structure.  |DataClass| defines the
zonal default for the class of data contained in the boundary conditions
of a zone.  If the boundary conditions contain dimensional data,
|DimensionalUnits| may be used to describe the system of dimensional
units employed.  If present, these three entities take precedence of
all corresponding entities at higher levels of the hierarchy.  These
precedence rules are further discussed in \autoref{s:precedence}.

Reference-state data is useful for situations where boundary-condition
data is not provided, and flow solvers are free to enforce any
appropriate boundary condition equations.  The presense of
|ReferenceState| at this level or below specifies the appropriate
flow conditions from which the flow solver should extract its
boundary-condition data.
For example, when computing an external flowfield around an airplane, an
engine nozzle exit is often simulated by imposing a stagnation pressure
boundary condition (or some other stagnation quantity) different from
freestream.
The nozzle-exit
stagnation quantities could be specified in an instance of
|ReferenceState| at this level or below in lieu of providing explicit
Dirichlet or Neumann data (see \autoref{s:BC_specdata}).

The \fort{UserDefinedData\_t} data structure allows arbitrary
user-defined data to be stored in \fort{Descriptor\_t} and
\fort{DataArray\_t} children without the restrictions or implicit
meanings imposed on these node types at other node locations.

\subsection{Boundary Condition Structure Definition: \texttt{BC\_t}}
\label{s:BCdefn}

|BC_t| contains boundary-condition information for a single BC surface
patch of a zone.
A BC patch is the subrange of the face of a zone where a given boundary
condition is applied.

The structure contains a boundary-condition type, as well as one or
more sets of boundary-condition data that are used to define the
boundary-condition equations to be enforced on the BC patch.  For most
boundary conditions, a single data set is all that is needed.  The
structure also contains information describing the normal vector to the
BC surface patch.

\begin{alltt}
  BC\_t< int CellDimension, int IndexDimension, int PhysicalDimension > :=
    \{
    List( Descriptor\_t Descriptor1 ... DescriptorN ) ;                      (o)

    BCType\_t BCType ;                                                       (r)

    GridLocation\_t GridLocation ;                                           (o/d)

    IndexRange\_t<IndexDimension> PointRange ;                               (r:o)
    IndexArray\_t<IndexDimension, ListLength[], int> PointList ;             (o:r)

    int[IndexDimension] InwardNormalIndex ;                                 (o)

    IndexArray\_t<PhysicalDimension, ListLength[], real> InwardNormalList ;  (o)

    List( BCDataSet\_t<CellDimension, IndexDimension, ListLength[],
          GridLocation> BCDataSet1 ... BCDataSetN ) ;                       (o)

    BCProperty\_t BCProperty ;                                               (o)

    FamilyName\_t FamilyName ;                                               (o)

    List( AdditionalFamilyName\_t AddFamilyName1 ... AddFamilyNameN ) ;      (o)

    ReferenceState\_t ReferenceState ;                                       (o)

    DataClass\_t DataClass ;                                                 (o)
                
    DimensionalUnits\_t DimensionalUnits ;                                   (o)

    List( UserDefinedData\_t UserDefinedData1 ... UserDefinedDataN ) ;       (o)

    int Ordinal ;                                                           (o)
    \} ;
\end{alltt}

\begin{notes}
\item Default names for the \fort{Descriptor\_t}, \fort{BCDataSet\_t}, and
      \fort{UserDefinedData\_t}
      lists are as shown; users may choose other legitimate names.
      Legitimate names must be unique within a given instance
      of \fort{BC\_t} and shall not include the names
      \fort{BCProperty}, \fort{BCType}, \fort{DataClass},
      \fort{DimensionalUnits}, 
      \fort{FamilyName}, \fort{GridLocation}, \fort{InwardNormalIndex},
      \fort{InwardNormalList}, \fort{Ordinal}, \fort{PointList},
      \fort{PointRange} or \fort{ReferenceState}.
\item \fort{GridLocation} is optional; if absent its default value is \fort{Vertex}.
      For other allowble values, see the table below. 
\item One of \fort{PointRange} or \fort{PointList} must be specified but not
     both. They must define a face subrange of the zone.
\item \fort{InwardNormalIndex} is only an option for structured grids.
      For unstructured grid boundaries, it should not be used.
      \fort{InwardNormalIndex} may have only one nonzero element, whose sign
      indicates the computational-coordinate direction of the BC patch normal;
      this normal points into the interior of the zone.
\item \fort{InwardNormalList} contains a list of vectors normal to the BC patch
      pointing into the interior of the zone.
      It is a function of \fort{PhysicalDimension} and \fort{ListLength[]}.
      The vectors are located at the vertices of the BC patch when
      \fort{GridLocation} is set to \fort{Vertex}. Otherwise, they are
      located at edge/face midpoints.
      The vectors are not required to have unit magnitude.
\item If \fort{PointRange} and \fort{InwardNormalList} are specified,
      an ordering convention is needed for indices on the BC patch.
      An ordering convention is also needed if a range is specified and
      local data is present in the |BCDataSet_t| substructures.
      FORTRAN multidimensional array ordering is used.
\end{notes}

|BCType| specifies the boundary-condition type, which gives general
information on the boundary-condition equations to be enforced.

The BC patch may be specified by \fort{PointRange}
if it constitutes a logically rectangular region.
In all other cases, \fort{PointList} should be
used to list the vertices or cell edges/faces making up the BC patch.  When
\fort{GridLocation} is set to \fort{Vertex}, then \fort{PointList} or
\fort{PointRange} refer to vertex indices, for both structured and unstructured
grids.  When \fort{GridLocation} is set to \fort{EdgeCenter}, then \fort{PointRange/List}
refer to edge elements.  For 3--D grids, when \fort{GridLocation} is set to
\fort{FaceCenter}, \fort{IFaceCenter}, etc., then \fort{PointRange/List} refer
to face elements.  The interpretation of \fort{PointRange/List} is summarized
in the table below:

\begin{center}
\begin{tabular}{||c|c|c|c|c||}
 \hline
\texttt{CellDimension} & \multicolumn{4}{c||}{\texttt{GridLocation}} \\
& \texttt{Vertex} & \texttt{EdgeCenter} & \texttt{*FaceCenter} & \texttt{CellCenter} \\
 \hline
1 & vertices & $-$ & $-$     & cells (line elements) \\
2 & vertices & edges & $-$   & cells (area elements) \\
3 & vertices & edges & faces & cells (volume elements) \\
 \hline
\end{tabular}
\end{center}

Note: In the table, \fort{*FaceCenter} stands for the possible types: \fort{FaceCenter},
\fort{IFaceCenter}, \fort{JFaceCenter} or \fort{KFaceCenter}.

For structured grids, face centers are indexed using the minimum of the
connecting vertex indices, as described in \autoref{s:structgrid}.  For
unstructured grids, edge and face elements are indexed using their element
numbering as defined in the \fort{Elements\_t} data structures.

The BC patch defined by \fort{PointRange/List} is a surface region over which
the particular set of boundary conditions is applied.  However, in the current
standard there is no mechanism to specify whether boundary conditions are
enforced in the weak or strong form.  If boundary conditions are imposed using
collocation (i.e., strong form), there is also no requirement that they be
imposed at the same locations used to define the BC patch (via
\fort{PointRange/List}).  In the case when BC patches are defined in terms of
vertices (or edges in 3--D), then the bounding vertices will be located on
multiple BC patches.  If boundary conditions are imposed using collocation at
vertices, then for this case there is no mechanism to determine which BC patch
takes precedence for any of these bounding vertices.

Some boundary conditions require a normal direction to be specified in
order to be properly imposed.
For structured zones a computational-coordinate normal can be derived
from the BC patch specification by examining redundant index components.
Alternatively, for structured zones this information can be provided
directly by |InwardNormalIndex|.
From Note 4, this vector points into the zone and can have only one
non-zero element.
For  exterior faces of a zone in 3-D, |InwardNormalIndex| should take the
following values:
\begin{center}
\begin{tabular}{c c @{\quad\quad\quad\quad}c c}
\hline\hline \\*[-2ex]
\bold{Face} & \fort{InwardNormalIndex} & \bold{Face} & \fort{InwardNormalIndex}
\\*[1ex] \hline\hline \\*[-2ex]
$i$-min & $[+1,0,0]$ & $i$-max & $[-1,0,0]$ \\
$j$-min & $[0,+1,0]$ & $j$-max & $[0,-1,0]$ \\
$k$-min & $[0,0,+1]$ & $k$-max & $[0,0,-1]$
\\*[1ex] \hline\hline
\end{tabular}
\end{center}

The physical-space normal vectors of the BC patch may be described by
\fort{InwardNormalList}; these are located at vertices or cell faces,
consistent with the BC patch specification.
\fort{InwardNormalList} is listed as an optional field because it is not
always needed to enforce boundary conditions, and the physical-space
normals of a BC patch can usually be constructed from the grid.
However, there are some situations, such as grid-coordinate singularity
lines, where \fort{InwardNormalList} becomes a required field, because
it cannot be generated from other information.

The |BC_t| structure provides for a list of boundary-condition
data sets, described in the next section.  In general, the proper
|BCDataSet_t| instance to impose on the BC patch is determined
by the |BCType| association table (\autoref{t:BCType_assoc} on
p.~\pageref*{t:BCType_assoc}).  The mechanics of determining the proper
data set to impose is described in \autoref{s:BCType_assoc}.

For a few boundary conditions, such as a symmetry plane or polar
singularity, the value of |BCType| completely describes the equations to
impose, and no instances of |BCDataSet_t| are needed.
For ``simple'' boundary conditions, where a single set of Dirichlet and/or
Neumann data is applied, a single |BCDataSet_t| will likely appear
(although this is not a requirement).
For ``compound'' boundary conditions, where the equations to impose are
dependent on local flow conditions, several instances of |BCDataSet_t|
will likely appear; the procedure for choosing the proper data set is
more complex as described in \autoref{s:BCType_assoc}.

A \fort{BCProperty\_t} data structure, described in
\autoref{s:BCProperty}, may be used to record special properties
associated with particular boundary condition patches, such as wall
functions or bleed regions.

\fort{FamilyName} identifies the family to which the boundary belongs.
Family names link the mesh boundaries to the CAD surfaces.
(See \autoref{s:Family}.)
Boundary conditions may also be defined directly on families.
In this case, the \fort{BCType} must be \fort{FamilySpecified}.
If, under a \fort{BC\_t} structure, both \fort{FamilyName\_t} and
\fort{BCType\_t} are present, and the \fort{BCType} is \emph{not}
\fort{FamilySpecified}, then the \fort{BCType} which \emph{is} specified
takes precedence over any \fort{BCType} which might be stored in a
\fort{FamilyBC\_t} structure under the specified \fort{Family\_t}.

Reference data applicable to the boundary conditions of a BC patch is
contained in the \fort{ReferenceState} structure.
|DataClass| defines the default for the class of data contained in the
boundary conditions.
If the boundary conditions contain dimensional data, |DimensionalUnits|
may be used to describe the system of dimensional units employed.
If present, these three entities take precedence of all corresponding
entities at higher levels of the hierarchy.
These precedence rules are further discussed in \autoref{s:precedence}.

The \fort{UserDefinedData\_t} data structure allows arbitrary
user-defined data to be stored in \fort{Descriptor\_t} and
\fort{DataArray\_t} children without the restrictions or implicit
meanings imposed on these node types at other node locations.

|Ordinal| is user-defined and has no restrictions on the values that
it can contain.
It is included for backward compatibility to assist implementation of
the CGNS system into applications whose I/O depends heavily on the
numbering of BC patches.
Since there are no restrictions on the values contained in |Ordinal|
(or that |Ordinal| is even provided), there is no guarantee that the
BC patches for a given zone in an existing CGNS database will have
sequential values from 1 to $N$ without holes or repetitions.
Use of |Ordinal| is discouraged and is on a user-beware basis.

\subsubsection*{FUNCTION \texttt{ListLength[]}:}

\noindent return value: |int| \\
\noindent dependencies: |PointRange|, |PointList|

\fort{BC\_t} requires the structure function \fort{ListLength}, which
is used to specify the number of vertices or edge/face elements
making up the BC patch.  If
\fort{PointRange} is specified, then \fort{ListLength} is obtained
from the number of points (inclusive) between the beginning and ending
indices of \fort{PointRange}.
If \fort{PointList} is specified, then \fort{ListLength} is the number
of indices in the list of points.
In this situation, \fort{ListLength} becomes a user input along with the
indices of the list \fort{PointList}.
By ``user'' we mean the application code that is generating the CGNS
database.
If \fort{ElementRange} or \fort{ElementList} is specified, then
\fort{ListLength} is defined by the cell faces making up the BC patch.

\fort{ListLength} is also the number of elements in the list
\fort{InwardNormalList}.
Note that syntactically \texttt{PointList} and
\fort{InwardNormalList} must have the same number of elements.

If neither \texttt{PointRange} or \texttt{PointList} is specified in a
particular \texttt{BCDataSet\_t} substructure, \texttt{List\-Length} must
be passed into it to determine the length of BC data arrays.

\subsection{Boundary Condition Data Set Structure Definition: \texttt{BCDataSet\_t}}
\label{s:BCDataSet}

|BCDataSet_t| contains Dirichlet and Neumann data for a single set of
boundary-condition equations.
Its intended use is for simple boundary-condition types, where the
equations imposed do not depend on local flow conditions.

\begin{alltt}
  BCDataSet\_t< int CellDimension, int IndexDimension, int ListLengthParameter,
                GridLocation\_t GridLocationParameter > :=
    \{
    List( Descriptor\_t Descriptor1 ... DescriptorN ) ;                      (o)

    BCTypeSimple\_t BCTypeSimple ;                                           (r)

    BCData\_t<ListLengthBCData[]> DirichletData ;                            (o)
    BCData\_t<ListLengthBCData[]> NeumannData ;                              (o)

    GridLocation\_t GridLocation ;                                           (o/d)

    IndexRange\_t<IndexDimension> PointRange ;                               (o)
    IndexArray\_t<IndexDimension, ListLength, int> PointList ;               (o)

    ReferenceState\_t ReferenceState ;                                       (o)

    DataClass\_t DataClass ;                                                 (o)
                
    DimensionalUnits\_t DimensionalUnits ;                                   (o)

    List( UserDefinedData\_t UserDefinedData1 ... UserDefinedDataN ) ;       (o)
    \} ;
\end{alltt}

\begin{notes}
\item Default names for the \fort{Descriptor\_t} and
      \fort{UserDefinedData\_t} lists are as shown; users may choose other
      legitimate names.
      Legitimate names must be unique within a given instance
      of \fort{BCDataSet\_t} and shall not include the names
      \fort{BCTypeSimple},
      \fort{DataClass}, \fort{DimensionalUnits}, \fort{DirichletData},
      \fort{GridLocation}, \fort{NeumannData}, \fort{PointList},
      \fort{PointRange}, or \fort{ReferenceState}.
\item \fort{BCTypeSimple} is the only required field.
      All other fields are optional and the \fort{Descriptor\_t} list
      may be empty.
\item \fort{GridLocation} is optional; if absent its default value is
      \fort{GridLocationParameter}.  For 2--D grids (\fort{CellDimension} = 2),
      \fort{GridLocation} may take the values of \fort{Vertex} or \fort{EdgeCenter}.
      For 3--D grids (\fort{CellDimension} = 3), \fort{GridLocation} may take the
      values of \fort{Vertex}, \fort{EdgeCenter}, \fort{FaceCenter}, \fort{IFaceCenter},
      \fort{JFaceCenter} or \fort{KFaceCenter}.
\item \fort{PointRange} and \fort{PointList} are both optional;
      only one of them may be specified.  They must define a face subrange of the zone.
\end{notes}

\fort{BCDataSet\_t} requires the structure parameters \fort{CellDimension},
\fort{IndexDimension}, \fort{ListLengthParameter} and
\fort{GridLocationParameter}.  These are all used to control the grid location
and length of data arrays in the \fort{Dirichlet} and \fort{Neumann}
substructures.  They are inputs for the structure functions \fort{ListLength[]}
and \fort{ListLengthBCData[]} defined below.

|BCTypeSimple| specifies the boundary-condition type, which gives general
information on the bound\-ary-con\-di\-tion equations to be enforced.
|BCTypeSimple_t| is defined in \autoref{s:BCType} along with the meanings
of all the |BCTypeSimple| values.  
|BCTypeSimple| is also used for matching boundary condition data sets as 
discussed in \autoref{s:BCType_assoc}.

Boundary-condition data is separated by equation type into Dirichlet
and Neumann conditions.  Dirichlet boundary conditions impose the
value of the given variables, whereas Neumann boundary conditions
impose the normal derivative of the given variables.  The mechanics of
specifying Dirichlet and Neumann data for boundary conditions is covered
in \autoref{s:BC_specdata}.

The substructures \fort{DirichletData} and \fort{NeumannData} contain
boundary-condition data which may be constant over the BC patch or
defined locally at each vertex or edge/face of the patch.
Locally defined data can be specified in one of two ways.
If \fort{GridLocation}, \fort{PointRange} and \fort{PointList} are all absent,
then the data is defined consistent with the BC patch specification of the
parent \fort{BC\_t} structure.  In this case, the location of the locally defined
data is given by \fort{GridLocationParameter} and the length of the data arrays
are given by \fort{ListLengthParameter}.  If \fort{GridLocation} and one of
\fort{PointRange} or \fort{PointList} is present, then the same rules provided
in Section~\autoref{s:BCdefn} apply.  In this case, the length of the data arrays
is given by \fort{ListLength[]}.

Reference quantities applicable to the set of boundary-condition data are
contained in the \fort{ReferenceState} structure.
|DataClass| defines the default for the class of data contained in the
boundary-condition data.
If the boundary conditions contain dimensional data, |DimensionalUnits|
may be used to describe the system of dimensional units employed.
If present, these three entities take precedence of all corresponding
entities at higher levels of the hierarchy.
These precedence rules are further discussed in \autoref{s:precedence}.

The \fort{UserDefinedData\_t} data structure allows arbitrary
user-defined data to be stored in \fort{Descriptor\_t} and
\fort{DataArray\_t} children without the restrictions or implicit
meanings imposed on these node types at other node locations.

\subsubsection*{FUNCTION \texttt{ListLength[]}:}

\noindent return value: |int| \\
\noindent dependencies: |PointRange|, |PointList|

\fort{BCDataSet\_t} requires the structure function \fort{ListLength}, which is
used to specify the length of locally defined Dirichlet and Neumann data arrays
when the grid location of these quatities differs from that of the BC patch
definition. The definition of \fort{ListLength} is identical to that provided in
\fort{BC\_t} (\autoref{s:BCdefn}).

\subsubsection*{FUNCTION \texttt{ListLengthBCData[]}:}

\noindent return value: |int| \\
\noindent dependencies: |ListLengthParameter|, |PointRange|, |PointList|

\fort{BCDataSet\_t} also requires the structure function
\fort{ListLengthBCData}. If \fort{PointRange} or \fort{PointList} is present,
then \fort{ListLengthBCData} takes the value of \fort{ListLength}.  If both are
absent, then it takes the value \fort{ListLengthParameter}.

\subsection{Boundary Condition Data Structure Definition: \texttt{BCData\_t}}
\label{s:BCData}

|BCData_t| contains a list of variables and associated data for
boundary-condition specification.
Each variable may be given as global data (i.e., a scalar) or local data
defined at each grid point or cell face of the BC patch.
By convention all data specified in a given instance of |BCData_t| is to
be used in the same \emph{type} of boundary-condition equation.
For example, the use of separate |BCData_t| substructures for Dirichlet
and Neumann equations in the |BCDataSet_t| structure of the previous
section.

\begin{alltt}
  BCData\_t< int ListLength > :=
    \{
    List( Descriptor\_t  Descriptor1 ... DescriptorN ) ;                     (o)

    List( DataArray\_t<DataType, 1, 1>     
          DataGlobal1 ... DataGlobalN ) ;                                   (o)

    List( DataArray\_t<DataType, 1, ListLength>       
          DataLocal1 ... DataLocalN ) ;                                     (o)

    DataClass\_t DataClass ;                                                 (o)
                
    DimensionalUnits\_t DimensionalUnits ;                                   (o)

    List( UserDefinedData\_t UserDefinedData1 ... UserDefinedDataN ) ;       (o)
    \} ;
\end{alltt}

\begin{notes}
\item
 Default names for the \fort{Descriptor\_t}, \fort{DataArray\_t}, and
 \fort{UserDefinedData\_t}
 lists are as shown; users may choose other legitimate names.
 Legitimate names must be unique within a given instance of
 \fort{BCData\_t} and shall not include the names \fort{DataClass} or
 \fort{DimensionalUnits}.
\item
 There are no required elements; all three lists may be empty.
\end{notes}

This structure definition shows separate lists for global verses local
data.  The global data is essentially scalars, while the local data
variables have size determined by the structure parameter |ListLength|.
For |DataArray_t| entities with standardized data-name identifiers
listed in \hyperref[s:dataname]{Appendix~\ref*{s:dataname}}, |DataType|
is determined by convention.
For user-defined variables, |DataType| is a user input.

Two important points need to be mentioned regarding this structure
definition.  First, this definition allows a given instance of
|BCData_t| to have a mixture of global and local data.  For example, if
a user specifies Dirichlet data that has a uniform stagnation pressure
but has a non-uniform velocity profile, this structure allows the user
to describe the stagnation pressure by a scalar in the |DataGlobal| list
and the velocity by an array in the |DataLocal| list.  Second, the only
distinction between the lists (aside from default names, which will be
seldom used) is the parameters passed into the |DataArray_t| structure.
Therefore, in actual implementation of this |BCData_t| structure it
may not be possible to distinguish between members of the global and
local lists without querying inside the |DataArray_t| substructures.
Straightforward mapping onto the ADF or HDF database will not provide any
distinctions between the members of the two lists.  This hopefully will
not cause any problems.

|DataClass| defines the default for the class of data contained in the
boundary-condition data.
If the boundary-condition data is dimensional, |DimensionalUnits| may be
used to describe the system of dimensional units employed.
If present, these two entities take precedence of all corresponding
entities at higher levels of the hierarchy.
These precedence rules are further discussed in \autoref{s:precedence}.

The \fort{UserDefinedData\_t} data structure allows arbitrary
user-defined data to be stored in \fort{Descriptor\_t} and
\fort{DataArray\_t} children without the restrictions or implicit
meanings imposed on these node types at other node locations.

\subsection{Boundary Condition Property Structure Definition: \texttt{BCProperty\_t}}
\label{s:BCProperty}

\fort{BCProperty\_t} allows the recording of special properties associated
with particular boundary condition patches.
At the current time, only two properties (\fort{WallFunction\_t}
and \fort{Area\_t}) are included, but extensions involving
boundary conditions may be implemented as additional nodes under
\fort{BCProperty\_t} in the future.

\begin{alltt}
  BCProperty\_t :=
    \{
    List( Descriptor\_t  Descriptor1 ... DescriptorN ) ;                     (o)

    WallFunction\_t WallFunction ;                                           (o)

    Area\_t Area ;                                                           (o)

    List( UserDefinedData\_t UserDefinedData1 ... UserDefinedDataN ) ;       (o)
    \} ;
\end{alltt}

\begin{notes}
\item
 Default names for the \fort{Descriptor\_t}, \fort{DataArray\_t}, and
 \fort{UserDefinedData\_t}
 lists are as shown; users may choose other legitimate names.
 Legitimate names must be unique within a given instance of
 \fort{BCProperty\_t} and shall not include the names \fort{WallFunction} or
 \fort{Area}.
\end{notes}

The \fort{WallFunction\_t} and \fort{Area\_t} data structures may be
used to record properties associated with the use of wall functions, or
area-related boundary conditions such as bleed, respectively.

The \fort{UserDefinedData\_t} data structure allows arbitrary
user-defined data to be stored in \fort{Descriptor\_t} and
\fort{DataArray\_t} children without the restrictions or implicit
meanings imposed on these node types at other node locations.

\subsubsection{Wall Function Structure Definition: \texttt{WallFunction\_t}}
\label{s:WallFunction}

The \fort{WallFunction\_t} data structure allows data associated with
the use of wall function boundary conditions to be recorded.

\begin{alltt}
  WallFunction\_t :=
    \{
    List( Descriptor\_t  Descriptor1 ... DescriptorN ) ;                     (o)

    WallFunctionType\_t WallFunctionType ;                                   (r)

    List( UserDefinedData\_t UserDefinedData1 ... UserDefinedDataN ) ;       (o)
    \} ;
\end{alltt}

\begin{notes}
\item
 Default names for the \fort{Descriptor\_t} and \fort{UserDefinedData\_t}
 lists are as shown; users may choose other legitimate names.
 Legitimate names must be unique within a given instance of
 \fort{WallFunction\_t} and shall not include the name
 \fort{WallFunctionType}.
\end{notes}

\fort{WallFunctionType\_t} is a required enumeration data structure
that is used to define the type of wall functions being used.
\begin{alltt}
  WallFunctionType\_t := Enumeration(
    WallFunctionTypeNull,
    WallFunctionTypeUserDefined,
    Generic ) ;
\end{alltt}
Because there is such a wide array of methods for employing wall
functions (few of which are well-documented), the type \fort{Generic}
is used to simply indicate that a wall function is employed, without
specifying details.

\subsubsection{Area Structure Definition: \texttt{Area\_t}}
\label{s:Area}

The \fort{Area\_t} data structure allows data associated with
area-related boundary conditions such as bleed to be recorded.

\begin{alltt}
  Area\_t :=
    \{
    List( Descriptor\_t  Descriptor1 ... DescriptorN ) ;                     (o)

    AreaType\_t AreaType ;                                                   (r)
    DataArray\_t<real, 1, 1>  SurfaceArea ;                                  (r)
    DataArray\_t<real, 1, 32> RegionName ;                                   (r)

    List( UserDefinedData\_t UserDefinedData1 ... UserDefinedDataN ) ;       (o)
    \} ;
\end{alltt}

\begin{notes}
\item
 Default names for the \fort{Descriptor\_t} and \fort{UserDefinedData\_t}
 lists are as shown; users may choose other legitimate names.
 Legitimate names must be unique within a given instance of
 \fort{Area\_t} and shall not include the names \fort{AreaType},
 \fort{RegionName}, or \fort{SurfaceArea}.
\end{notes}

\fort{AreaType\_t} is a required enumeration data structure
that is used to define the type of area being defined.
\begin{alltt}
  AreaType\_t := Enumeration(
    AreaTypeNull,
    AreaTypeUserDefined,
    BleedArea,
    CaptureArea ) ;
\end{alltt}

If \fort{AreaType} is set to \fort{BleedArea}, the value of
\fort{SurfaceArea} is the size of the current bleed surface.
Note that bleed is commonly used with wall boundary conditions.
The bleed area is the surface area of the boundary condition patch.

If \fort{AreaType} is set to \fort{CaptureArea}, then
\fort{SurfaceArea} represents the size of the current capture surface.
For inlet flows, for example, the capture area is the area of a
fictitious surface in front of the inlet in which mass is pulled into
the inlet.
This is used to calculate the mass flow for the boundary condition patch
based on the formula:
$$
{\it mass\:flow} = {\it MFR} \, \rho_{\infty} U_{\infty} A_{\it cap}
$$
where \textit{MFR} is the desired mass flow ratio and
$A_{\it cap}$ is the capture area.
Another interpretation is the far-upstream cross-sectional area of the
stream tube which feeds the inlet.
Note that the capture area is usually defined with an outflow boundary
condition, which is commonly used at an engine face.

The \fort{RegionName} is character identifier, and is needed so that a
specific region can span multiple surfaces over multiple zones.

\subsection{Boundary Condition Type Structure Definition: \texttt{BCType\_t}} 
\label{s:BCType}

\texttt{BCType\_t} is an enumeration type that identifies the boundary-condition
equations to be enforced at a given boundary location.
\begin{alltt}
  BCType\_t := Enumeration(
    BCTypeNull, BCTypeUserDefined, BCAxisymmetricWedge, BCDegenerateLine.
    BCDegeneratePoint, BCDirichlet, BCExtrapolate, BCFarfield, BCGeneral,
    BCInflow, BCInflowSubsonic, BCInflowSupersonic, BCNeumann,
    BCOutflow, BCOutflowSubsonic, BCOutflowSupersonic, BCSymmetryPlane,
    BCSymmetryPolar, BCTunnelInflow, BCTunnelOutflow, BCWall,
    BCWallInviscid, BCWallViscous, BCWallViscousHeatFlux,
    BCWallViscousIsothermal, FamilySpecified ) ;
\end{alltt}
The boundary-condition type is further defined as simple,
\texttt{BCTypeSimple\_t}, or compound, \texttt{BCTypeCompound\_t},
which are subsets of the enumeration type \texttt{BCType\_t}.
\begin{alltt}
  BCTypeSimple\_t := Enumeration(
    BCTypeNull, BCTypeUserDefined, BCAxisymmetricWedge, BCDegenerateLine.
    BCDegeneratePoint, BCDirichlet, BCExtrapolate, BCGeneral,
    BCInflowSubsonic, BCInflowSupersonic, BCNeumann,
    BCOutflowSubsonic, BCOutflowSupersonic, BCSymmetryPlane,
    BCSymmetryPolar, BCTunnelInflow, BCTunnelOutflow, BCWall,
    BCWallInviscid, BCWallViscous, BCWallViscousHeatFlux,
    BCWallViscousIsothermal, FamilySpecified ) ;
\end{alltt}

\begin{alltt}
  BCTypeCompound\_t := Enumeration(
    BCTypeNull, BCTypeUserDefined, BCInflow, BCOutflow,
    BCFarfield ) ;
\end{alltt}

The members of \texttt{BCTypeSimple\_t} completely identify the equations to
impose, while those of \texttt{BCTypeCompound\_t} give a general description of the class of
boundary-condition equations to impose.
The specific boundary-condition equations to enforce for each value
of \texttt{BCType\_t} are listed in \autoref{t:BCTypeSimple} and
\autoref{t:BCTypeCompound}.

The subdivision of \texttt{BCType\_t} is based on function.  For simple
boundary conditions, the equations and data imposed are fixed; whereas,
for compound boundary conditions different sets of equations are
imposed depending on local flow conditions at the boundary.  This
distinction requires additional rules for dealing with simple and
compound boundary-condition types.  These rules are discussed in
\autoref{s:BCType_assoc}.

For the inflow/outflow boundary-condition descriptions, 3-D inviscid
compressible flow is assumed; the 2-D equivalent should be obvious.
These same boundary conditions are typically used for viscous cases
also.  This ``3-D Euler'' assumption will be noted wherever used.

In the following tables, $Q$ is the solution vector, $\v{q}$ is the
velocity vector whose magnitude is $q$, the unit normal to the boundary
is $\h{n}$, and $\pd{()}{n} = \h{n}\dot\nabla$ is differentiation normal
to the boundary.

\newenvironment{mylist}
{\begin{list}{$\bullet$}{%
\topsep    0pt%
\parsep    0pt%
\partopsep 0pt%
\itemsep   0pt}}
{\end{list}}

\settowidth{\tmplengtha}{\fort{BCTypeSimple\_t }\bold{Identifier}\quad}
\setlength{\LTleft}{0pt}
\setlength{\LTright}{0pt}
\setlength{\Pwidth}{\linewidth-4\tabcolsep-\tmplengtha}
\begin{longtable}{>{\ttfamily}p{\tmplengtha} >{\raggedright\arraybackslash}p{\Pwidth}}
\caption[Simple Boundary Condition Types]{\textbf{Simple Boundary Condition Types}}
\label{t:BCTypeSimple}
\\ \hline\hline \\*[-2ex]
BCType\_t \bold{or} \\
BCTypeSimple\_t \bold{Identifier} & \spantwo{\bold{Boundary Condition Description}}
\\*[1ex] \hline\hline \\*[-2ex]
\endfirsthead

\multicolumn{2}{l}{{\bfseries \autoref{t:BCTypeSimple}: Simple Boundary Condition Types} (\emph{Continued})}
\\*[1ex] \hline\hline \\*[-2ex]
BCType\_t \bold{or} \\
BCTypeSimple\_t \bold{Identifier} & \spantwo{\bold{Boundary Condition Description}}
\\*[1ex] \hline\hline \\*[-2ex]
\endhead

\\*[-2ex]\hline
\multicolumn{2}{r}{\emph{Continued on next page}} \\
\endfoot
\\*[-2ex] \hline\hline
\endlastfoot

BCGeneral               & Arbitrary conditions on $Q$ or $\pd{Q}{n}$ \\[0.5\baselineskip]

BCDirichlet             & Dirichlet condition on $Q$ vector \\[0.5\baselineskip]

BCNeumann               & Neumann condition on $\pd{Q}{n}$ \\[0.5\baselineskip]

BCExtrapolate           & Extrapolate $Q$ from interior \\[0.5\baselineskip]

BCWallInviscid          & Inviscid (slip) wall
                          \begin{mylist}
                          \item normal velocity specified (default:
                                $\v{q}\cdot\hat{n} = 0$)
                          \end{mylist}\mbox{}\vspace{-0.5\baselineskip} \\

BCWallViscousHeatFlux   & Viscous no-slip wall with heat flux
                          \begin{mylist}
                          \item velocity Dirichlet (default: $q = 0$)
                          \item temperature Neumann (default: adiabatic,
                                $\pd{T}{n} = 0$)
                          \end{mylist}\mbox{}\vspace{-0.5\baselineskip} \\

BCWallViscousIsothermal & Viscous no-slip, isothermal wall
                          \begin{mylist}
                          \item velocity Dirichlet (default: $q = 0$)
                          \item temperature Dirichlet
                          \end{mylist}\mbox{}\vspace{-0.5\baselineskip} \\

BCWallViscous           & Viscous no-slip wall; special cases are
                          |BCWallViscousHeatFlux| and
                          |BCWallViscousIsothermal|
                          \begin{mylist}
                          \item velocity Dirichlet (default: $q = 0$)
                          \item Dirichlet or Neumann on temperature
                          \end{mylist}\mbox{}\vspace{-0.5\baselineskip} \\

BCWall                  & General wall condition; special cases are
                          |BCWallInviscid|, |BCWallViscous|,
                          |BCWallViscousHeatFlux| and
                          |BCWallViscousIsothermal| \\[0.5\baselineskip]

BCInflowSubsonic        & Inflow with subsonic normal velocity
                          \begin{mylist}
                          \item specify 4; extrapolate 1 (3-D Euler)
                          \end{mylist}\mbox{}\vspace{-0.5\baselineskip} \\

BCInflowSupersonic      & Inflow with supersonic normal velocity
                          \begin{mylist}
                          \item specify 5; extrapolate 0 (3-D Euler)
                          \end{mylist}
                          Same as |BCDirichlet| \\[0.5\baselineskip]

BCOutflowSubsonic       & Outflow with subsonic normal velocity
                          \begin{mylist}
                          \item specify 1; extrapolate 4 (3-D Euler)
                          \end{mylist}\mbox{}\vspace{-0.5\baselineskip} \\

BCOutflowSupersonic     & Outflow with supersonic normal velocity
                          \begin{mylist}
                          \item specify 0; extrapolate 5 (3-D Euler)
                          \end{mylist}
                          Same as |BCExtrapolate| \\[0.5\baselineskip]

BCTunnelInflow          & Tunnel inlet (subsonic normal velocity)
                          \begin{mylist}
                          \item specify cross-flow velocity, stagnation
                                enthalpy, entropy 
                          \item extrapolate 1 (3-D Euler)
                          \end{mylist}\mbox{}\vspace{-0.5\baselineskip} \\

BCTunnelOutflow         & Tunnel exit (subsonic normal velocity)
                          \begin{mylist}
                          \item specify static pressure
                          \item extrapolate 4 (3-D Euler)
                          \end{mylist}\mbox{}\vspace{-0.5\baselineskip} \\

BCDegenerateLine        & Face degenerated to a line \newline
                          \begin{picture}(1.3,0.8)(-2.2,0.0)
                             \drawline(0.0,0.0)(1.0,0.2)(1.0,0.5)(0.0,0.3)(0.0,0.0)
                             \drawline(0.0,0.0)(-0.2,0.8)(0.2,0.9)(1.0,0.5)
                             \drawline(0.0,0.3)(-0.2,0.8)
                             \dashline{0.05}(1.0,0.2)(0.2,0.9)
                          \end{picture} \\

BCDegeneratePoint       & Face degenerated to a point \newline
                          \begin{picture}(1.3,0.8)(-2.2,0.0)
                             \drawline(0.0,0.0)(1.0,0.2)(1.0,0.5)(0.0,0.3)(0.0,0.0)
                             \drawline(0.0,0.0)(-0.2,0.8)(1.0,0.5)
                             \drawline(0.0,0.3)(-0.2,0.8)
                             \dashline{0.05}(1.0,0.2)(-0.2,0.8)
                          \end{picture} \\

BCSymmetryPlane         & Symmetry plane; face should be coplanar
                          \begin{mylist}
                          \item density, pressure: $\pd{()}{n} = 0$
                          \item tangential velocity:
                                $\pd{(\v{q}\!\times\!\hat{n})}{n} = 0$
                          \item normal velocity: $\v{q}\cdot\hat{n} = 0$
                          \end{mylist}\mbox{}\vspace{-0.5\baselineskip} \\

BCSymmetryPolar         & Polar-coordinate singularity line; special
                          case of |BCDegenerateLine| where degenerate
                          face is a straight line and flowfield has
                          polar symmetry; $\hat{s}$ is singularity line
                          tangential unit vector
                          \begin{mylist}
                          \item normal velocity: $\v{q}\!\times\!\hat{s} = 0$
                          \item all others: $\pd{()}{n} = 0$, \ $n$ normal
                                to $\hat{s}$
                          \end{mylist}\mbox{}\vspace{-0.5\baselineskip} \\

BCAxisymmetricWedge     & Axisymmetric wedge; special
                          case of |BCDegenerateLine| where 
                          degenerate face is a straight line \\

FamilySpecified         & A boundary condition type is being specified for
                          the family to which the current boundary belongs.
                          A \fort{FamilyName\_t} specification must exist
                          under \fort{BC\_t}, corresponding to a
                          \fort{Family\_t} structure under \fort{CGNSBase\_t}.
                          Under the \fort{Family\_t} structure there must
                          be a \fort{FamilyBC\_t} structure specifying a
                          valid \fort{BCType} (other than
                          \fort{FamilySpecified}!).
                          If any of these are absent, the boundary
                          condition type is undefined. \\
\end{longtable}

\settowidth{\tmplengtha}{\fort{BCTypeCompound\_t }\bold{Identifier}\quad}
\setlength{\LTleft}{0pt}
\setlength{\LTright}{0pt}
\setlength{\Pwidth}{\linewidth-4\tabcolsep-\tmplengtha}
\begin{longtable}{>{\ttfamily}p{\tmplengtha} >{\raggedright\arraybackslash}p{\Pwidth}}
\caption[Compound Boundary Condition Types]{\textbf{Compound Boundary Condition Types}}
\label{t:BCTypeCompound}
\\ \hline\hline \\*[-2ex]
BCType\_t \bold{or} \\
BCTypeCompound\_t \bold{Identifier} & \spantwo{\bold{Boundary Condition Description}}
\\*[1ex] \hline\hline \\*[-2ex]
\endfirsthead

\multicolumn{2}{l}{{\bfseries \autoref{t:BCTypeCompound}: Compound Boundary Condition Types} (\emph{Continued})}
\\*[1ex] \hline\hline \\*[-2ex]
BCType\_t \bold{or} \\
BCTypeCompound\_t \bold{Identifier} & \spantwo{\bold{Boundary Condition Description}}
\\*[1ex] \hline\hline \\*[-2ex]
\endhead

\\*[-2ex]\hline
\multicolumn{2}{r}{\emph{Continued on next page}} \\
\endfoot
\\*[-2ex] \hline\hline
\endlastfoot

BCInflow   & Inflow, arbitrary normal Mach;
             test on normal Mach, then perform one of:
             |BCInflowSubsonic|, |BCInflowSupersonic| \\[0.5\baselineskip]

BCOutflow  & Outflow, arbitrary normal Mach;
             test on normal Mach, then perform one of:
             |BCOutflowSubsonic|, |BCOutflowSupersonic| \\[0.5\baselineskip]

BCFarfield & Farfield inflow/outflow, arbitrary normal Mach;
             test on normal velocity and normal Mach, 
             then perform one of:
             |BCInflowSubsonic|, |BCInflowSupersonic|,
             |BCOutflowSubsonic|, |BCOutflowSupersonic| \\
\end{longtable}

\subsection{Matching Boundary Condition Data Sets} 
\label{s:BCType_assoc}

\enlargethispage{\baselineskip}
The |BC_t| structure allows for a arbitrary list of boundary-condition
data sets, described by the |BCDataSet_t| structure.  For simple
boundary conditions, a single data set must be chosen from a list
that may contain more than one element.  Likewise, for a compound
boundary condition, a limited number of data sets must be chosen
and applied appropriately.  The mechanism for proper choice of data
sets is controlled by the |BCType| field of the |BC_t| structure,
the |BCTypeSimple| field of the |BCDataSet_t| structure, and the
boundary-condition type association table (\autoref{t:BCType_assoc}).
In the following discussion, we will use the ``|/|'' notation for fields
or elements of a structure type.

|BC_t| is used for both simple and compound boundary conditions; hence,
the field |BC_t/BCType| is of type |BCType_t|.
Conversely, the substructure |BCDataSet_t| is intended to enforce a
single set of boundary-condition equations independent of local flow
conditions (i.e., it is appropriate only for simple boundary conditions).
This is why the field |BCDataSet_t/BCTypeSimple| is of type
|BCTypeSimple_t| and not |BCType_t|.
The appropriate choice of data sets is determined by matching the field
|BC_t/BCType| with the field |BCDataSet_t/BCTypeSimple| as specified in 
\autoref{t:BCType_assoc}.

For simple boundary conditions, a single match from the list of
\fort{BCDataSet\_t} instances is required.
For all \fort{BCTypeSimple\_t} identifiers, except
\fort{BCInflowSupersonic} and \fort{BCOutflowSupersonic}, an exact match
is necessary.
\fort{BCInflowSupersonic} will match itself or \fort{BCDirichlet};
\fort{BCOutflowSupersonic} will match itself or \fort{BCExtrapolate}.

For compound boundary conditions, the association table specifies which
simple boundary-condition types are appropriate.
Since compound boundary conditions enforce different boundary-condition
equation sets depending on local flow conditions, several instances of
|BCDataSet_t| will be matched for each |BCTypeCompound_t| identifier.
The accompanying rule determines which of the matching data sets to
apply at a given location on the BC patch.

This provides a general procedure applicable to both \fort{BCTypeSimple\_t}
and \fort{BCTypeCompound\_t} situations.
For a given \fort{BC\_t/BCType} use those instances of \fort{BCDataSet\_t}
whose field \fort{BC\-Data\-Set\_t/BC\-Type\-Sim\-ple} matches according to
\autoref{t:BCType_assoc}.
Apply the matching data set or sets as prescribed by the appropriate
usage rule.

\newpage
\settowidth{\tmplengtha}{\fort{BCOutflowSupersonic}\quad}
\setlength{\LTleft}{0pt}
\setlength{\LTright}{0pt}
\setlength{\LTcapwidth}{\linewidth}
\setlength{\Pwidth}{\linewidth-4\tabcolsep-\tmplengtha}
\begin{longtable}{>{\ttfamily}p{\tmplengtha} >{\raggedright\arraybackslash}p{\Pwidth}}
\caption[Associated Boundary Condition Types and Usage Rules]{\textbf{Associated Boundary Condition Types and Usage Rules}}
\label{t:BCType_assoc}
\\ \hline\hline \\*[-2ex]
BCType\_t \bold{Identifier} & \bold{Associated} \texttt{BCTypeSimple\_t} \bold{Identifiers and Usage Rules}
\\*[1ex] \hline\hline \\*[-2ex]
\endfirsthead

\multicolumn{2}{l}{{\bfseries \autoref{t:BCType_assoc}: Associated Boundary Condition Types and Usage Rules} (\emph{Continued})}
\\*[1ex] \hline\hline \\*[-2ex]
BCType\_t \bold{Identifier} & \bold{Associated} \texttt{BCTypeSimple\_t} \bold{Identifiers and Usage Rules}
\\*[1ex] \hline\hline \\*[-2ex]
\endhead

\\*[-2ex]\hline
\multicolumn{2}{r}{\emph{Continued on next page}} \\
\endfoot
\\*[-2ex] \hline\hline 
\endlastfoot

BCInflow            & |BCInflowSupersonic| \\
                    & |BCInflowSubsonic| \\[0.25\baselineskip]
                    & {\it Usage Rule:}
                      \begin{mylist}
                      \item if supersonic normal Mach, choose
                            |BCInflowSupersonic|;
                      \item else, choose |BCInflowSubsonic|
                      \end{mylist}\mbox{}\vspace{-0.5\baselineskip} \\

BCOutflow           & |BCOutflowSupersonic| \\
                    & |BCOutflowSubsonic| \\[0.25\baselineskip]
                    & {\it Usage Rule:}
                      \begin{mylist}
                      \item if supersonic normal Mach, choose
                            |BCOutflowSupersonic|;
                      \item else, choose |BCOutflowSubsonic|
                      \end{mylist}\mbox{}\vspace{-0.5\baselineskip} \\

BCFarfield          & |BCInflowSupersonic| \\
                    & |BCInflowSubsonic| \\
                    & |BCOutflowSupersonic| \\
                    & |BCOutflowSubsonic| \\[0.25\baselineskip]
                    & {\it Usage Rule:}
                      \begin{mylist}
                      \item if inflow and supersonic normal Mach, choose
                            |BCInflowSupersonic|;
                      \item else if inflow, choose |BCInflowSubsonic|;
                      \item else if outflow and supersonic normal Mach,
                            choose |BCOutflowSupersonic|;
                      \item else, choose |BCOutflowSubsonic|
                      \end{mylist}\mbox{}\vspace{-0.5\baselineskip} \\

BCInflowSupersonic  & |BCInflowSupersonic| \\
                    & |BCDirichlet| \\[0.25\baselineskip]
                    & {\it Usage Rule:}
                      \begin{mylist}
                      \item choose either; |BCInflowSupersonic| takes
                            precedence
                      \end{mylist}\mbox{}\vspace{-0.5\baselineskip} \\

BCOutflowSupersonic & |BCOutflowSupersonic| \\
                    & |BCExtrapolate| \\[0.25\baselineskip]
                    & {\it Usage Rule:}
                      \begin{mylist}
                      \item choose either; |BCOutflowSupersonic| takes
                            precedence
                      \end{mylist}\mbox{}\vspace{-0.5\baselineskip} \\

All others          & Self-matching \\
\end{longtable}

Although we present a strict division between the two categories of
boundary-condition types, we realize that some overlap may exist.
For example, some of the more general simple boundary-condition types,
such as |BCWall|, may include a situation of inflow/outflow (say if the
wall is porous).  These complications require further guidelines on
appropriate definition and use of boundary-condition types.  The real
distinctions between |BCTypeSimple_t| and |BCTypeCompound_t| are as
follows:
\begin{itemize}
\item |BCTypeSimple_t| identifiers always match themselves;
      |BCTypeCompound_t| identifiers never match themselves.
\item |BCTypeSimple_t| identifiers always produce a single match; 
      |BCTypeCompound_t| will produce multiple matches.
\item The usage rule for |BCTypeSimple_t| identifiers is always
      trivial---apply the single matching data set regardless of local
      flow conditions.
\end{itemize}
Therefore, any boundary condition that involves application of different
data sets depending on local flow conditions should be classified
|BCTypeCompound_t|.  If a type that we have classified |BCTypeSimple_t|
is used as a compound type (|BCWall| for a porous wall is an example),
then it should somehow be reclassified.  One option is to define a new
|BCTypeCompound_t| identifier and provide associated |BCTypeSimple_t|
types and a usage rule.  Another option may be to allow some identifiers
to be both |BCTypeSimple_t| and |BCTypeCompound_t| and let their
appropriate use be based on context.  This is still undetermined.

%\clearpage
\subsection{Boundary Condition Specification Data} 
\label{s:BC_specdata}

For a given simple boundary condition (i.e., one that is not
dependent on local flow conditions), the database provides a set of
boundary-condition equations to be enforced through the structure
definitions for |BCDataSet_t| and |BCData_t| (\autoref{s:BCDataSet}
and \autoref{s:BCData}).  Apart from the boundary-condition type, the
precise equations to be enforced are described by boundary-condition
solution data.  These specified solution data are arranged by ``equation
type'':

\begin{Ventryi}{Neumann:}
   \item [Dirichlet:]
         $Q         = (Q)_{\rm specified}$
   \item [Neumann:]
         $\pd{Q}{n} = ( \pd{Q}{n} )_{\rm specified}$
\end{Ventryi}

The |DirichletData| and |NeumannData| entities of
|BCData_t| list both the solution variables involved in the
equations (through the data-name identifier conventions of
\hyperref[s:dataname]{Appendix~\ref*{s:dataname}}) and the specified
solution data.

Two issues need to be addressed for specifying Dirichlet or Neumann 
boundary-condition data.  The first is whether the data is global or local: 

\begin{Ventryi}{global BC data:}
   \item [Global BC data:]
         Data applied globally to the BC patch; for example, specifying
         a uniform total pressure at an inflow boundary
   \item [Local BC data:]
         Data applied locally at each vertex or cell face
         of the BC patch; an example of this is varying total pressure
         specified at each grid point at an inflow boundary
\end{Ventryi}
The second issue is describing the actual solution quantities that are
to be specified.  Both of these issues are addressed by use of the
|DataArray_t| structure.

For some types of boundary conditions, many different combinations of
solution quantities could be specified.  For example, |BCInflowSubsonic|
requires 4 solution quantities to be specified in 3-D, but what those 4
quantities are varies with applications (e.g., internal verses external
flows) and codes.  We propose the convention that the actual data
being specified for any |BCType| is given by the list of |DataArray_t|
entities included in |DirichletData| and |NeumannData| structures
(actually by the identifier attached to each instance of |DataArray_t|).
This frees us from having to define \emph{many} versions of a given
|BCType| (e.g., |BCInflowSubsonic1|, |BCInflowSubsonic2|, etc.), where
each has a precisely defined set of Dirichlet data.  We are left with
the easier task of defining \emph{how many} Dirichlet or Neumann
quantities must be provided for each |BCType|.

An example of using |DataArray_t|-identifier conventions to describe
BC specification data is the following: subsonic inflow with uniform
stagnation pressure, mass flow and cross-flow angle specified; the
Dirichlet data are stagnation pressure = 2.56, mass flow = 1.34, and
cross-flow angle has a $y$-component of 0.043 and a $z$-component of
0.02 (ignore dimensional-units or normalization for the present).  The
specified solution variables and associated data are described as shown:
\begin{alltt}
  BCData\_t<ListLength=?> DirichletData = 
    \{\{
    DataArray\_t<real, 1, 1> PressureStagnation = \{\{ Data(real, 1, 1) = 2.56  \}\} ;
    DataArray\_t<real, 1, 1> MassFlow           = \{\{ Data(real, 1, 1) = 1.34  \}\} ;
    DataArray\_t<real, 1, 1> VelocityAngleY     = \{\{ Data(real, 1, 1) = 0.043 \}\} ;
    DataArray\_t<real, 1, 1> VelocityAngleZ     = \{\{ Data(real, 1, 1) = 0.02  \}\} ;
    \}\} ;
\end{alltt}
Basically, this states that \fort{DirichletData} contains four
instances of \fort{DataArray\_t} with identifiers or names
\fort{PressureStagnation}, \fort{MassFlow}, \fort{VelocityAngleY} and
\fort{VelocityAngleZ}.
Each \fort{DataArray\_t} structure entity contains a single
floating-point value; these are the Dirichlet data for the BC.
Note that \fort{Data(real, 1, 1)} means a single floating-point value.

The global verses local data issue can be easily handled by storing
either a scalar, as shown above, for the global BC data case; or storing
an array for the local BC data case.  Storing an array of local BC data
allows the capability for specifying non-constant solution profiles,
such as ``analytic'' boundary-layer profiles or profiles derived from
experimental data.  For the above example, if the stagnation pressure is
instead specified at every vertex of the boundary-condition patch the
following results:
\begin{alltt}
  BCData\_t<ListLength=99> DirichletData = 
    \{\{
    DataArray\_t<real, 1, 99> PressureStagnation = 
      \{\{ Data(real, 1, 99) = (PTOT(n), n=1,99) \}\} ;
    DataArray\_t<real, 1, 1> MassFlow           = \{\{ Data(real, 1, 1) = 1.34  \}\} ;
    DataArray\_t<real, 1, 1> VelocityAngleY     = \{\{ Data(real, 1, 1) = 0.043 \}\} ;
    DataArray\_t<real, 1, 1> VelocityAngleZ     = \{\{ Data(real, 1, 1) = 0.02  \}\} ;
    \}\} ;
\end{alltt}
where, say, the boundary face is logically rectangular and contains
$11\cross 9$ vertices and the stagnation pressure at the vertices is
given by the array |PTOT()|.

To facilitate implementation of boundary conditions into existing flow
solvers, we adopt the convention that if no boundary-condition data is
specified, then flow solvers are free to enforce any appropriate
boundary-condition equations.
This includes situations where entities of |BCDataSet_t|, |BCData_t|
or |DataArray_t| are absent within the boundary-condition hierarchy.
By convention, if no \fort{BCDataSet} entities are present, then
application codes are free to enforce appropriate BCs for the given
value of \fort{BCType}.
Furthermore, if the entities \fort{DirichletData} and \fort{NeumannData}
are not present in an instance of \fort{BCDataSet\_t}, or if
insufficient data is present in |DirichletData| or |NeumannData|
(e.g., if only one Dirichlet variable is present for a subsonic
inflow condition), then application codes are free to fill out
the boundary-condition data as appropriate for the |BCTypeSimple|
identifier.

The various levels of BC implementation allowed are shown in
\autoref{f:BCimplementations}, from the lowest level in which
the application codes interpret the \fort{BCType}, to the fully
SIDS-compliant BC implementation which completely defines the BC within
the CGNS file.

\setlength{\tmplength}{\unitlength}	% save existing unit length
\setlength{\unitlength}{\baselineskip}	% unit length for example hierarchy figure
\begin{figure}[!htb]
   \centering
   \begin{minipage}[b]{0.295\linewidth}
      \subfigure[Lowest-level allowed (application code interprets meaning of \fort{BCType})]
          {\resizebox{\linewidth}{!}{\input{bc.figs/bcimpl_low_cet.pic}}}
   \end{minipage}%
   \hfill%
   \begin{minipage}[b]{0.614\linewidth}
      \subfigure[Fully SIDS-compliant]
          {\resizebox{\linewidth}{!}{\input{bc.figs/bcimpl_full_cet.pic}}}
   \end{minipage}
   \caption{Boundary Condition Implementation Levels}
   \label{f:BCimplementations}
\end{figure}
\setlength{\unitlength}{\tmplength}	% reset unit length to original vlue

An alternative approach to the present design could be to list all the
solution variables and data (as |DataArray_t|-like structures) for the
boundary condition, and contain descriptive tags in each one to indicate
if they are Dirichlet or Neumann data.  We have not taken this approach.
We think grouping boundary-condition data by ``equation type'' as we have
done better allows for future extension to other types of boundary
conditions (e.g., 2nd-order non-reflecting BC's that result in P.D.E.'s
to be solved at the boundary).

\subsection{Boundary Condition Examples}  
\label{s:BCexample}

This section contains boundary-condition examples with increasing
complexity.  Included is the most simple |BC_t| entity and one of
the most complex.  The examples show situations of local and global
boundary-condition data, simple and compound boundary-condition types,
and multiple boundary-condition data sets that must be matched with the
appropriate boundary-condition type.

\begin{example}{Symmetry Plane}
\label{ex:bc1}

Symmetry plane for a patch on the $i$-min face of a 3-D structured zone.  
\begin{alltt}
  !  CellDimension = 3, IndexDimension = 3
  BC\_t<3,3,3> BC1 =
    \{\{
    BCType\_t BCType = BCSymmetryPlane ;

    IndexRange\_t<3> PointRange =
      \{\{
      int[3] Begin = [1,1,1 ] ;
      int[3] End   = [1,9,17] ;
      \}\} ;
    \}\} ;
\end{alltt}
Since the boundary-condition equations to be enforced are completely
defined by the boundary-condition type |BCSymmetryPlane|, no other
information needs to be provided, except for the extent of the BC patch.
The BC patch is specified by |PointRange| with a beginning index of
(1,1,1) and an ending index of (1,9,17).
By default, these refer to vertices.
\end{example}

\begin{example}{Viscous Solid Wall I}
\label{ex:bc2}

A viscous solid wall for a 3-D structured zone, where a Dirichlet
condition is enforced for temperature; the wall temperature for the
entire wall is specified to be 273 K.
The BC patch is on the $j$-min face and is bounded by the indices
(1,1,1) and (33,1,9).
\begin{alltt}
  !  CellDimension = 3, IndexDimension = 3
  BC\_t<3,3,3> BC2 =
    \{\{
    BCType\_t BCType = BCWallViscousIsothermal ;

    IndexRange\_t<3> PointRange =
      \{\{
      int[3] Begin = [1 ,1,1] ;
      int[3] End   = [33,1,9] ;
      \}\} ;

    !  ListLength = 33*9 = 297
    BCDataSet\_t<297> BCDataSet1 =
      \{\{
      BCTypeSimple\_t BCTypeSimple = BCWallViscousIsothermal ;

      !  Data array length = ListLength = 297
      BCData\_t<297> DirichletData =
        \{\{
        DataArray\_t<real, 1, 1> Temperature =
          \{\{
          Data(real, 1, 1) = 273. ;
          
          DataClass\_t DataClass = Dimensional ;

          DimensionalUnits\_t DimensionalUnits = 
            \{\{
            MassUnits        = MassUnitsNull ;
            LengthUnits      = LengthUnitsNull ;
            TimeUnits        = TimeUnitsNull ;
            TemperatureUnits = Kelvin ;
            AngleUnits       = AngleUnitsNull ;
            \}\} ;
          \}\} ;
        \}\} ;
      \}\} ;
    \}\} ;
\end{alltt}

This is an example of a simple boundary-condition type,
|BCWallViscousIsothermal|.
By default there is a zero Dirichlet condition on the velocity, and
|BCDataSet1| states there is a Dirichlet condition on temperature with a
global value of 273 K.
The data set contains a single |BCData_t| entity, called
|DirichletData|, meaning a (possibly empty) collection of Dirichlet
conditions should be enforced.
Within |DirichletData|, there is a single |DataArray_t| entity; this
narrows the specification to a single Dirichlet condition.
This lone entity has the identifier |Temperature|, which by conventions
defined in \hyperref[s:dataname]{Appendix~\ref*{s:dataname}} is the
identifier for static temperature.
The data contained in |Temperature| is a floating-point scalar with a
value of 273.
The qualifiers |DataClass| and |DimensionalUnits| specifies that the
temperature is dimensional with units of |Kelvin|.

Since |BCWallViscousIsothermal| is a simple boundary-condition type,
the appropriate data set contains a |BCTypeSimple| entity whose value
is |BCWallViscousIsothermal|.  For this example, only a single data set
is provided, and this data set has the correct boundary-condition type.
This is an example of a trivial data-set match.

Apart from velocity and temperature, additional ``numerical'' boundary
conditions are typically required by Navier-Stokes flow solvers, but
none are given here; therefore, a code is free to implement other
additional boundary conditions as desired.

Although the boundary-condition data is global, we include in this
example structure parameters that are the lengths of potential
local-data arrays.
Comments are added to the example with the ``|!|'' notation to document
the structure parameters.
The \fort{BC\_t} structure function |ListLength| is evaluated based on
|PointRange|.
Since |GridLocation| is not specified in \fort{BC2}, any local data is
at vertices by default.
The entity |Temperature| contains global data, so the value of
|ListLength| is unused in |DirichletData|.

This example raises the question of whether unused structure parameters
are required in structure entities.  The answer is no.  We included them
here for completeness.  The purpose of structure parameters is to mimic
the need to define elements of a entity based on information contained
elsewhere (at a higher level) in the CGNS database.  When this need is
not present in a given instance of a structure entity, the structure
parameters are superfluous.
In some of the following examples, structure parameters that are superfluous
or otherwise not needed are denoted by ``|?|''.
\end{example}

\begin{example}{Subsonic Inflow}
\label{ex:bc3}

Subsonic inflow for a 2-D structured zone: The BC patch is on the
$i$-min face and  includes $j \in [2, 7]$.
As prescribed by the boundary-condition type, three quantities must be
specified.  Uniform entropy and stagnation enthalpy are specified with
values of 0.94 and 2.85, respectively.  A velocity profile is specified
at face midpoints, given by the array |v_inflow(j)|.  No dimensional or
nondimensional information is provided.
\begin{alltt}
  !  CellDimension = 2, IndexDimension = 2
  BC\_t<2,2,?> BC3 =
    \{\{
    BCType\_t BCType = BCInflowSubsonic ;

    GridLocation\_t GridLocation = FaceCenter ;

    IndexRange\_t<2> PointRange = 
      \{\{ 
      int[2] Begin = [1,2] ;
      int[2] End   = [1,6] ;
      \}\} ;

    !  ListLength = 5
    BCDataSet\_t<5> BCDataSet1 =
      \{\{
      BCTypeSimple\_t BCTypeSimple = BCInflowSubsonic ;

      !  Data array length = ListLength = 5
      BCData\_t<5> DirichletData =
        \{\{
        DataArray\_t<real, 1, 1> EntropyApprox = 
          \{\{ 
          Data(real, 1, 1) = 0.94 ;
          \}\} ;

        DataArray\_t<real, 1, 1> EnthalpyStagnation = 
          \{\{ 
          Data(real, 1, 1) = 2.85 ;
          \}\} ;

        DataArray\_t<real, 1, 5> VelocityY =
          \{\{
          Data(real, 1, 5) = (v\_inflow(j), j=3,7) ;
          \}\} ;
        \}\} ;
      \}\} ;
    \}\} ;
\end{alltt}
This is another example of a simple boundary-condition type.
The primary additional complexity included in this example is multiple
Dirichlet conditions with one containing local data.
\texttt{DirichletData} contains three \texttt{DataArray\_t} entities named
\texttt{EntropyApprox}, \texttt{EnthalpyStagnation} and \texttt{VelocityY}.
This specifies three Dirichlet boundary conditions to be enforced, and
the names identify the solution quantities to set.
Since both \texttt{EntropyApprox} and \texttt{EnthalpyStagnation} have an
array-length structure parameter of one, they identify global data, and
the values are provided.
\texttt{VelocityY} is an array of data values and contains the values in
\texttt{v\_inflow()}.
The length of the array is given by \texttt{ListLength}, which represents
the number of cell faces because \texttt{BC3} is specified using
the value of \fort{FaceCenter} for \fort{GridLocation}.
Note that the beginning and ending indices on the array
\texttt{v\_inflow()} are unimportant (they are user inputs); there just
needs to be five values provided.
\end{example}

\begin{example}{Outflow}
\label{ex:bc4}

Outflow boundary condition with unspecified normal Mach number for an
$i$-max face of a 3-D structured zone: for subsonic outflow, a uniform
pressure is specified; for supersonic outflow, no boundary-condition
equations are
specified.
\begin{alltt}
  !  CellDimension = 3, IndexDimension = 3
  BC\_t<3,3,3> BC4 =
    \{\{
    BCType\_t BCType = BCOutflow ;

    IndexRange\_t<3> PointRange = \{\{ \}\} ;

    BCDataSet\_t<?> BCDataSetSubsonic =
      \{\{
      BCTypeSimple\_t BCTypeSimple = BCOutflowSubsonic ;

      BCData\_t<?> DirichletData =
        \{\{
        DataArray\_t<real, 1, 1> Pressure = \{\{ \}\} ;
        \}\} ;
      \}\} ;

    BCDataSet\_t<?> BCDataSetSupersonic =
      \{\{
      BCTypeSimple\_t BCTypeSimple = BCOutflowSupersonic ;
      \}\} ;
    \}\} ;
\end{alltt}
This is an example of a complex boundary-condition type; the equation
set to be enforced depends on the local flow conditions, namely the Mach
number normal to the boundary.
Two data sets are provided, |BCDataSetSubsonic| and |BCDataSetSupersonic|;
recall the names are unimportant and are user defined.
The first data set has a boundary-condition type of |BCOutflowSubsonic|
and prescribes a global Dirichlet condition on static pressure.
Any additional boundary conditions needed may be applied by a flow solver.
The second data set has a boundary-condition type of |BCOutflowSupersonic|
with no additional boundary-condition equation specification.
Typically, all solution quantities are extrapolated from the interior for
supersonic outflow.
From the boundary-condition type association table
(\autoref{t:BCType_assoc}), |BCOutflow| requires two data sets with
boundary-condition types |BCOutflowSubsonic| and |BCOutflowSupersonic|.
The accompanying usage rule states that the data set for
|BCOutflowSubsonic| should be used for a subsonic normal Mach number;
otherwise, the data set for |BCOutflowSupersonic| should be enforced.

Any additional data sets with boundary-condition types other than
\fort{BCOutflowSubsonic} or \fort{BCOutflowSupersonic} could be
provided (the definition of \fort{BC\_t} allows an arbitrary list of
\fort{BCDataSet\_t} entities); however, they should be ignored by any
code processing the boundary-condition information.
Another caveat is that providing two data sets with the same simple
boundary-condition type would cause indeterminate results --- which one is
the correct data set to apply?

The actual global data value for static pressure is not provided; an
abbreviated form of the |Pressure| entity is shown.
This example also uses the ``|?|'' notation for unused data-array-length  
structure parameters.                                                   
\end{example}

\begin{example}{Farfield}
\label{ex:bc5}

Farfield boundary condition with arbitrary flow conditions for a $j$-max
face of a 2-D structured zone: If subsonic inflow, specify entropy,
vorticity and incoming acoustic characteristics; if supersonic inflow
specify entire flow state; if subsonic outflow, specify incoming
acoustic characteristic; and if supersonic outflow, extrapolate all
flow quantities.
None of the extrapolated quantities for the different boundary condition
possibilities need be stated.
\begin{alltt}
  !  CellDimension = 2, IndexDimension = 2
  BC\_t<2,2,2> BC5 =
    \{\{
    BCType\_t BCType = BCFarfield ;

    IndexRange\_t<2> PointRange = \{\{ \}\} ;

    int[2] InwardNormalIndex = [0,-1] ;

    BCDataSet\_t<?> BCDataSetInflowSupersonic =
      \{\{
      BCTypeSimple\_t BCTypeSimple = BCInflowSupersonic ;
      \}\} ;

    BCDataSet\_t<?> BCDataSetInflowSubsonic =
      \{\{
      BCTypeSimple\_t BCTypeSimple = BCInflowSubsonic ;

      BCData<?> DirichletData = 
        \{\{
        DataArray\_t<real, 1, 1> CharacteristicEntropy      = \{\{ \}\} ;
        DataArray\_t<real, 1, 1> CharacteristicVorticity1   = \{\{ \}\} ;
        DataArray\_t<real, 1, 1> CharacteristicAcousticPlus = \{\{ \}\} ;
        \}\} ;
      \}\} ;

    BCDataSet\_t<?> BCDataSetOutflowSupersonic =
      \{\{
      BCTypeSimple\_t BCTypeSimple = BCOutflowSupersonic ;
      \}\} ;

    BCDataSet\_t<?> BCDataSetOutflowSubsonic =
      \{\{
      BCTypeSimple\_t BCTypeSimple = BCOutflowSubsonic ;

      BCData<?> DirichletData = 
        \{\{
        DataArray\_t<real, 1, 1> CharacteristicAcousticMinus = \{\{ \}\} ;
        \}\} ;
      \}\} ;
    \}\} ;
\end{alltt}
The farfield boundary-condition type is the most complex of the compound
boundary-condition types.
\texttt{BCFarfield} requires four data sets; these data sets must
contain the simple boundary-condition types \texttt{BCInflowSupersonic},
\texttt{BCInflowSubsonic}, \texttt{BCOutflowSupersonic} and
\texttt{BCOutflowSubsonic}.
This example provides four appropriate data sets.
The usage rule given for \texttt{BCFarfield} in \autoref{t:BCType_assoc}
states which set of boundary-condition equations to be enforced based on
the normal velocity and normal Mach number.

The data set for supersonic-inflow provides no information
other than the boundary-condition type.  A flow solver is
free to apply any conditions that are appropriate; typically
all solution quantities are set to freestream reference state
values.  The data set for subsonic-inflow states that three
Dirichlet conditions should be enforced; the three data
identifiers provided are among the list of conventions given in
\hyperref[s:dataname_char]{Appendix~\ref*{s:dataname_char}}.
The data set for supersonic-outflow only provides the boundary-condition
type, and the data set for subsonic-outflow provides one
Dirichlet condition on the incoming acoustic characteristic,
|CharacteristicAcousticMinus|.

Also provided in the example is the inward-pointing
computational-coordinate normal; the normal points in the $-j$
direction, meaning the BC patch is a $j$-max face.  This information
could also be obtained from the BC patch description given in
|IndexRange|.

Note that this example shows only the overall layout of the
boundary-condition entity.
\fort{IndexRange} and all \fort{DataArray\_t} entities are abbreviated,
and all unused structure functions are not evaluated.
\end{example}

\newpage
\begin{example}{Viscous Solid Wall II}
\label{ex:bc6}

There are circumstances when a user may wish to define a BC patch using
vertices (under \fort{BC\_t}), but store the BC data at face centers
(under \fort{BCDataSet\_t}).
The following example is similar to \hyperref[ex:bc2]{Example~\ref*{ex:bc2}},
with the exception that the Dirichlet data for temperature is stored at
face centers rather than at vertices.

As before, the example is a viscous solid wall in a 3-D structured
zone, where a Dirichlet condition is enforced for temperature; the wall
temperature for the entire wall is specified to be 273 K.
The BC patch is on the $j$-min face and is bounded by the indices
(1,1,1) and (33,1,9).
\begin{alltt}
  !  CellDimension = 3, IndexDimension = 3
  BC\_t<3,3,3> BC2 =
    \{\{
    BCType\_t BCType = BCWallViscousIsothermal ;

    !  Grid location is Vertex by default
    IndexRange\_t<3> PointRange =
      \{\{
      int[3] Begin = [1 ,1,1] ;
      int[3] End   = [33,1,9] ;
      \}\} ;

    !  ListLength = 33*9 = 297
    BCDataSet\_t<297> BCDataSet1 =
      \{\{
      BCTypeSimple\_t BCTypeSimple = BCWallViscousIsothermal ;

      GridLocation\_t GridLocation = FaceCenter ;
      IndexRange\_t<3> PointRange =
        {{
        int[3] Begin = [1 ,1,1] ;
        int[3] End   = [32,1,8] ;
        }} ;

      !  ListLength = 32*8 = 256
      BCData\_t<256> DirichletData =
        \{\{
        DataArray\_t<real, 1, 1> Temperature =
          \{\{
          Data(real, 1, 1) = 273. ;
          
          DataClass\_t DataClass = Dimensional ;

          DimensionalUnits\_t DimensionalUnits = 
            \{\{
            MassUnits        = MassUnitsNull ;
            LengthUnits      = LengthUnitsNull ;
            TimeUnits        = TimeUnitsNull ;
            TemperatureUnits = Kelvin ;
            AngleUnits       = AngleUnitsNull ;
            \}\} ;
          \}\} ;
        \}\} ;
      \}\} ;
    \}\} ;
\end{alltt}

As in \hyperref[ex:bc2]{Example~\ref*{ex:bc2}},
although the boundary-condition data is global, we include in this
example structure parameters that are the lengths of potential
local-data arrays.
In \fort{BC\_t}, \fort{GridLocation} is not specified, and thus is
\fort{Vertex} by default.
The structure function \fort{ListLength} is 297, based on the
specification of \fort{PointRange}, and that value is passed to
\fort{BCDataSet\_t}.

In this example \fort{PointRange} is specified in \fort{BCDataSet\_t},
so the \fort{ListLength} passed into it from \fort{BC\_t} is not used.
In \fort{BCDataSet\_t}, \fort{GridLocation} is specified as
\fort{FaceCenter}, and \fort{PointRange} is set accordingly.
The corresponding value of \fort{ListLength} is 256, which is passed
into \fort{BCData\_t}.

As before, in \fort{BCData\_t} the entity \fort{Temperature} contains
global data, so the value of \fort{ListLength} is unused.
\end{example}
