\section{Miscellaneous Data Structures}
\label{s:misc}
\thispagestyle{plain}

This section contains miscellaneous structure types for describing
reference states, convergence history, discrete field data, integral or
global data, families, and user-defined data.

\subsection{Reference State Structure Definition: \texttt{ReferenceState\_t}}
\label{s:ReferenceState}

|ReferenceState_t| describes a reference state, which is a list of
geometric or flow-state quantities defined at a common location or
condition.
Examples of typical reference states associated with CFD calculations
are freestream, plenum, stagnation, inlet and exit.
Note that providing a \fort{ReferenceState} description is particularly
important if items elsewhere in the CGNS database are
\fort{NormalizedByUnknownDimensional}.

\begin{alltt}
  ReferenceState\_t :=
    \{
    Descriptor\_t ReferenceStateDescription ;                                (o)
    List( Descriptor\_t Descriptor1 ... DescriptorN ) ;                      (o)

    List( DataArray\_t<DataType, 1, 1> DataArray1 ... DataArrayN ) ;         (o)

    DataClass\_t DataClass ;                                                 (o)
                
    DimensionalUnits\_t DimensionalUnits ;                                   (o)

    List( UserDefinedData\_t UserDefinedData1 ... UserDefinedDataN ) ;       (o)
    \} ;
\end{alltt}

\begin{notes}
\item
 Default names for the \fort{Descriptor\_t}, \fort{DataArray\_t}, and
 \fort{UserDefinedData\_t}
 lists are as shown; users may choose other legitimate names.
 Legitimate names must be unique within a given instance of
 \fort{ReferenceState\_t} and shall not include the names \fort{DataClass},
 \fort{DimensionalUnits}, or \fort{ReferenceStateDescription}.
\end{notes}

Data-name identifiers associated with \fort{ReferenceState} are shown in
\autoref{t:id_reference}.

\begin{table}[htbp]
\centering
\caption[Data-name Identifiers for Reference State]{\textbf{Data-name Identifiers for Reference State}}
\label{t:id_reference}
\begin{tabular}{>{\ttfamily}l >{\quad}l >{\quad\bfseries}l}
\\ \hline\hline \\*[-2ex]
\bold{Data-Name Identifier} & \bold{Description} & \bold{Units}
\\*[1ex] \hline\hline \\*[-2ex]
Mach			     & Mach number, $M = q/c$		   & - \\
Mach\_Velocity  	     & Velocity scale, $q$		   & L/T \\
Mach\_VelocitySound	     & Speed of sound scale, $c$	   & L/T \\
Reynolds		     & Reynolds number, $Re = V L_R / \nu$ & - \\
Reynolds\_Velocity	     & Velocity scale, $V$		   & L/T \\
Reynolds\_Length	     & Length scale, $L_R$		   & L \\
Reynolds\_ViscosityKinematic & Kinematic viscosity scale, $\nu$    & L\tsup{2}/T \\
LengthReference 	     & Reference length, $L$		   & L
\\*[1ex] \hline\hline
\end{tabular}
\end{table}

In addition, any flowfield quantities (such as \fort{Density},
\fort{Pressure}, etc.) can be included in the \fort{ReferenceState}.

The reference length $L$ (\fort{LengthReference}) may be necessary for
\fort{NormalizedByUnknownDimensional} databases, to define the length
scale used for nondimensionalizations.
It may be the same or different from the \fort{Reynolds\_Length} used to
define the Reynolds number.

Because of different definitions for angle of attack and angle of yaw,
these quantities are not explicitly defined in the SIDS.
Instead, the user can unambigouosly denote the freestream velocity      
vector direction by giving \fort{VelocityX}, \fort{VelocityY}, and      
\fort{VelocityZ} in \fort{ReferenceState}, (with the reference state    
denoting the freestream).

Care should be taken when defining the reference state quantities to
ensure consistency.
(See the discussion in \autoref{s:data_normbyunkdim}.)
For example, if velocity, length, and time are all defined, then the
velocity stored should be length/time.
If consistency is not followed, different applications could interpret
the resulting data in different ways.

|DataClass| defines the default for the class of data contained in the
reference state.
If any reference state quantities are dimensional, |DimensionalUnits|
may be used to describe the system of dimensional units employed.
If present, these two entities take precedence of all corresponding
entities at higher levels of the hierarchy.
These precedence rules are further discussed in \autoref{s:precedence}.

The \fort{UserDefinedData\_t} data structure allows arbitrary
user-defined data to be stored in \fort{Descriptor\_t} and
\fort{DataArray\_t} children without the restrictions or implicit
meanings imposed on these node types at other node locations.

We recommend using the |ReferenceStateDescription| entity to document
the flow conditions.  The format of the documentation is currently
unregulated.

\subsection{Reference State Example}

An example is presented in this section of a reference state entity that
contains dimensional data.
An additional example of a nondimensional reference state is provided in
\hyperref[s:twozone]{Appendix~\ref*{s:twozone}}.

\begin{example}{Reference State with Dimensional Data}

A freestream reference state where all data quantities are dimensional.
Standard atmospheric conditions at sea level are assumed for static
quantities, and all stagnation variables are obtained using the
isentropic relations.
The flow velocity is 200 m/s aligned with the $x$-axis.
Dimensional units of kilograms, meters, and seconds are used.
The data class and system of units are specified at the
|ReferenceState_t| level rather than attaching this information directly
to the |DataArray_t| entities for each reference quantity.
Data-name identifiers are provided in
\hyperref[s:dataname]{Appendix~\ref*{s:dataname}}.
\begin{alltt}
  ReferenceState\_t ReferenceState = 
    \{\{
    Descriptor\_t ReferenceStateDescription = 
      \{\{
      Data(char, 1, 45) = "Freestream at standard atmospheric conditions" ;
      \}\} ;
    
    DataClass\_t DataClass = Dimensional ;

    DimensionalUnits\_t DimensionalUnits =
      \{\{
      MassUnits        = Kilogram ;
      LengthUnits      = Meter ;
      TimeUnits        = Second ;
      TemperatureUnits = Kelvin ;
      AngleUnits       = Radian ;
      \}\} ;

    DataArray\_t<real, 1, 1> VelocityX = 
      \{\{
      Data(real, 1, 1) = 200. ;
      \}\} ;
    DataArray\_t<real, 1, 1> VelocityY               = \{\{ 0. \}\} ;
    DataArray\_t<real, 1, 1> VelocityZ               = \{\{ 0. \}\} ;

    DataArray\_t<real, 1, 1> Pressure                = \{\{ 1.0132E+05 \}\} ;
    DataArray\_t<real, 1, 1> Density                 = \{\{ 1.226 \}\} ;
    DataArray\_t<real, 1, 1> Temperature             = \{\{ 288.15 \}\} ;
    DataArray\_t<real, 1, 1> VelocitySound           = \{\{ 340. \}\} ;
    DataArray\_t<real, 1, 1> ViscosityMolecular      = \{\{ 1.780E-05 \}\} ;

    DataArray\_t<real, 1, 1> PressureStagnation      = \{\{ 1.2806E+05 \}\} ;
    DataArray\_t<real, 1, 1> DensityStagnation       = \{\{ 1.449 \}\} ;
    DataArray\_t<real, 1, 1> TemperatureStagnation   = \{\{ 308.09 \}\} ;
    DataArray\_t<real, 1, 1> VelocitySoundStagnation = \{\{ 351.6 \}\} ;

    DataArray\_t<real, 1, 1> PressureDynamic         = \{\{ 0.2542E+05 \}\} ;
    \}\} ;                        
\end{alltt}
Note that all |DataArray_t| entities except |VelocityX| have been
abbreviated.
\end{example}

\subsection{Convergence History Structure Definition: \texttt{ConvergenceHistory\_t}}
\label{s:ConvergenceHistory}

Flow solver convergence history information is described by the 
|ConvergenceHistory_t| structure.
This structure contains the number of iterations and a list of data
arrays containing convergence information at each iteration.

\begin{alltt}
  ConvergenceHistory\_t :=
    \{
    Descriptor\_t NormDefinitions ;                                          (o)
    List( Descriptor\_t Descriptor1 ... DescriptorN ) ;                      (o)

    int Iterations ;                                                        (r)

    List( DataArray\_t<DataType, 1, Iterations> 
      DataArray1 ... DataArrayN ) ;                                         (o)

    DataClass\_t DataClass ;                                                 (o)
                
    DimensionalUnits\_t DimensionalUnits ;                                   (o)

    List( UserDefinedData\_t UserDefinedData1 ... UserDefinedDataN ) ;       (o)
    \} ;
\end{alltt}

\begin{notes}
\item
 Default names for the \fort{Descriptor\_t}, \fort{DataArray\_t}, and
 \fort{UserDefinedData\_t}
 lists are as shown; users may choose other legitimate names.
 Legitimate names must be unique within a given instance of
 \fort{ConvergenceHistory\_t} and shall not include the names
 \fort{DataClass}, \fort{DimensionalUnits}, or \fort{NormDefinitions}. 
\item
 |Iterations| is the only required field for |ConvergenceHistory_t|.
\end{notes}

|Iterations| identifies the number of iterations for which convergence
information is recorded.  This value is also passed into each of the
|DataArray_t| entities, defining the length of the data arrays.

|DataClass| defines the default for the class of data contained in the
convergence history.
If any convergence-history data is dimensional, |DimensionalUnits| may
be used to describe the system of dimensional units employed.
If present, these two entities take precedence over all corresponding
entities at higher levels of the hierarchy.
These precedence rules are further discussed in \autoref{s:precedence}.

The \fort{UserDefinedData\_t} data structure allows arbitrary
user-defined data to be stored in \fort{Descriptor\_t} and
\fort{DataArray\_t} children without the restrictions or implicit
meanings imposed on these node types at other node locations.

Measures used to record convergence vary greatly among current
flow-solver implementations.
Convergence information typically includes global forces, norms of
equation residuals, and norms of solution changes.
Attempts to systematically define a set of convergence measures within
the CGNS project have been futile.
For global parameters, such as forces and moments,
\hyperref[s:dataname]{Appendix~\ref*{s:dataname}} lists a set of
standardized data-array identifiers.
For equations residuals and solution changes, no such standard list
exists.
It is suggested that data-array identifiers for norms of equations
residuals begin with |RSD|, and those for solution changes begin with
|CHG|.
For example, |RSDMassRMS| could be used for the $L_2$-norm (RMS) of mass
conservation residuals.
It is also strongly recommended that |NormDefinitions| be utilized to
describe the convergence information recorded in the data arrays.
The format used to describe the convergence norms in |NormDefinitions|
is currently unregulated.

\subsection{Discrete Data Structure Definition: \texttt{DiscreteData\_t}} 
\label{s:DiscreteData}

|DiscreteData_t| provides a description of generic discrete data (i.e.,
data defined on a computational grid); it is identical to
|FlowSolution_t| except for its type name.
This structure can be used to store field data, such as fluxes or
equation residuals, that is not typically considered part of the flow
solution.
|DiscreteData_t| contains a list for data arrays, identification of
grid location, and a mechanism for identifying rind-point data included
in the data arrays.
All data contained within this structure must be defined at the same
grid location and have the same amount of rind-point data.
\begin{alltt}
  DiscreteData\_t< int CellDimension, int IndexDimension, 
                  int VertexSize[IndexDimension],
                  int CellSize[IndexDimension] > :=
    \{
    List( Descriptor\_t Descriptor1 ... DescriptorN ) ;                      (o)

    GridLocation\_t GridLocation ;                                           (o/d)

    Rind\_t<IndexDimension> Rind ;                                           (o/d)

    IndexRange\_t<IndexDimension> PointRange ;                               (o)
    IndexArray\_t<IndexDimension, ListLength[], int> PointList ;             (o)

    List( DataArray\_t<DataType, IndexDimension, DataSize[]> 
          DataArray1 ... DataArrayN ) ;                                     (o)

    DataClass\_t DataClass ;                                                 (o)
    
    DimensionalUnits\_t DimensionalUnits ;                                   (o)

    List( UserDefinedData\_t UserDefinedData1 ... UserDefinedDataN ) ;       (o)
    \} ;
\end{alltt}

\begin{notes}
\item Default names for the \texttt{Descriptor\_t},
      \texttt{DataArray\_t}, and \texttt{UserDefinedData\_t} lists are
      as shown; users may choose other legitimate names.
      Legitimate names must be unique within a given instance
      of \texttt{DiscreteData\_t} and shall not include the
      names \texttt{DataClass}, \texttt{DimensionalUnits},
      \texttt{GridLocation}, \fort{PointRange}, \fort{PointList}, or \texttt{Rind}.
\item There are no required fields for \texttt{DiscreteData\_t}.
      \texttt{GridLocation} has a default of \texttt{Vertex} if absent.
      \texttt{Rind} also has a default if absent; the default
      is equivalent to having an instance of \texttt{Rind}
      whose \texttt{RindPlanes} array contains all zeros (see
      \autoref{s:Rind}).
\item Both of the fields \texttt{PointRange} and
      \texttt{PointList} are optional.  Only one of these
      two fields may be specified.
\item The structure parameter \texttt{DataType} must be consistent
      with the data stored in the \texttt{DataArray\_t} entities (see
      \autoref{s:DataArray}).
\item For unstructured zones \texttt{GridLocation} options are limited
      to \texttt{Vertex} or \texttt{CellCenter}, unless
      one of \texttt{PointRange} or \texttt{PointList} is present.
\item Indexing of data within the \texttt{DataArray\_t} structures, must be
      consistent with the associated numbering of vertices or elements.
\end{notes}

|DiscreteData_t| requires four structure parameters:
|CellDimension| identifies the dimensionality of cells or
elements, |IndexDimension| identifies the dimensionality of the grid
size arrays, and |VertexSize| and |CellSize| are the number of core vertices and
cells, respectively, in each index direction. For unstructured
zones, \fort{IndexDimension} is always 1.

The arrays of discrete data are stored in the list of |DataArray_t|
entities.
The field |GridLocation| specifies the location of the data with respect
to the grid; if absent, the data is assumed to coincide with grid
vertices (i.e., |GridLocation = Vertex|).
All data within a given instance of |DiscreteData_t| must reside at the
same grid location.

For structured grids, the value of |GridLocation| alone specifies the location
and indexing of the discrete data.  Vertices are explicity indexed.  Cell
centers and face centers are indexed using the minimum of the connecting vertex
indices, as described in the section Structured Grid Notation and Indexing
Conventions (\autoref{s:structgrid}).

For unstructured grids, the value of |GridLocation| alone specifies location and
indexing of discrete data only for vertex and cell-centered data.  The
reason for this is that element-based grid connectivity provided in the
\texttt{Elements\_t} data structures explicitly indexes only vertices and cells.
For data stored at alternate grid locations (e.g. edges), additional
connectivity information is needed.  This is provided by the optional fields
\texttt{PointRange} and \texttt{PointList}; these refer to
vertices, edges, faces or cell centers, depending on the values of
\texttt{CellDimension} and \texttt{GridLocation}.  The following table shows
these relations.

\begin{center}
\begin{tabular}{||c|c|c|c|c||}
 \hline
\texttt{CellDimension} & \multicolumn{4}{c||}{\texttt{GridLocation}} \\
& \texttt{Vertex} & \texttt{EdgeCenter} & \texttt{*FaceCenter} & \texttt{CellCenter} \\
 \hline
1 & vertices & $-$ & $-$ & cells (line elements) \\
2 & vertices & edges & $-$ & cells (area elements) \\
3 & vertices & edges & faces & cells (volume elements) \\
 \hline
\end{tabular}
\end{center}

In the table, \fort{*FaceCenter} stands for the possible types: \fort{FaceCenter},
\fort{IFaceCenter}, \fort{JFaceCenter} or \fort{KFaceCenter}.

Although intended for edge or face-based discrete data for unstructured grids,
the fields \texttt{PointRange/List} may also be used to (redundantly) index
vertex and cell-centered data.  In all cases, indexing of discrete data
corresponds to the element numbering as defined in the \texttt{Elements\_t} data
structures.

\texttt{Rind} is an optional field that indicates
the number of rind planes (for structured grids) or rind points or
elements (for unstructured grids) included in the data.
Its purpose and function are identical to those described in
\autoref{s:Grid}.
Note, however, that the \texttt{Rind} in this structure is independent
of the \texttt{Rind} contained in \texttt{GridCoordinates\_t}.
They are not required to contain the same number of rind planes or
elements.
Also, the location of any flow-solution rind points is assumed to be
consistent with the location of the core flow solution points (e.g.,
if \texttt{GridLocation = CellCenter}, rind points are assumed to be
located at fictitious cell centers).

|DataClass| defines the default class for data contained in the
|DataArray_t| entities.
For dimensional data, |DimensionalUnits| may be used to describe the
system of units employed.
If present these two entities take precedence over the corresponding
entities at higher levels of the CGNS hierarchy.
The rules for determining precedence of entities of this type are
discussed in \autoref{s:precedence}.

The \fort{UserDefinedData\_t} data structure allows arbitrary
user-defined data to be stored in \fort{Descriptor\_t} and
\fort{DataArray\_t} children without the restrictions or implicit
meanings imposed on these node types at other node locations.

\subsubsection*{FUNCTION \texttt{ListLength[]}:}

\noindent return value: |int| \\
\noindent dependencies: |PointRange|, |PointList|

\fort{DiscreteData\_t} requires the structure function \fort{ListLength}, which
is used to specify the number of entities (e.g. vertices) corresponding to a
given \fort{PointRange} or \fort{PointList}. If
\fort{PointRange} is specified, then \fort{ListLength} is obtained from
the number of points (inclusive) between the beginning and ending indices of
\fort{PointRange}. If \fort{PointList} is specified, then
\fort{ListLength} is the number of indices in the list of points. In this
situation, \fort{ListLength} becomes a user input along with the indices of the
list \fort{PointList}. By ``user'' we mean the application code that is
generating the CGNS database.

%\noindent {\bf FUNCTION} |DataSize[]|:
\subsubsection*{FUNCTION \texttt{DataSize[]}:}

\noindent return value: one-dimensional |int| array of length
          |IndexDimension| \\
\noindent dependencies: |IndexDimension|, |VertexSize[]|, |CellSize[]|,
          |GridLocation|, |Rind|, |ListLength|

The function \fort{DataSize[]} is the size of discrete-data arrays.
It is identical to the function \fort{DataSize[]} defined for
\fort{FlowSolution\_t} (see \autoref{s:FlowSolution}).

\subsection{Integral Data Structure Definition: \texttt{IntegralData\_t}} 
\label{s:IntegralData}

|IntegralData_t| provides a description of generic global or integral data
that may be associated with a particular zone or an entire database.
In contrast to |DiscreteData_t|, integral data is not associated with
any specific field location.
\begin{alltt}
  IntegralData\_t :=
    \{
    List( Descriptor\_t Descriptor1 ... DescriptorN ) ;                      (o)

    List( DataArray\_t<DataType, 1, 1> DataArray1 ... DataArrayN ) ;         (o)

    DataClass\_t DataClass ;                                                 (o)
    
    DimensionalUnits\_t DimensionalUnits ;                                   (o)

    List( UserDefinedData\_t UserDefinedData1 ... UserDefinedDataN ) ;       (o)
    \} ;
\end{alltt}

\begin{notes}
\item
 Default names for the \fort{Descriptor\_t}, \fort{DataArray\_t}, and
 \fort{UserDefinedData\_t}
 lists are as shown; users may choose other legitimate names.
 Legitimate names must be unique within a given instance of
 \fort{DiscreteData\_t} and shall not include the names \fort{DataClass}
 or \fort{DimensionalUnits}.
\item
 There are no required fields for |IntegralData_t|.  
\item
 The structure parameter \fort{DataType} must be consistent with the
 data stored in the \fort{DataArray\_t} entities (see \autoref{s:DataArray}).
\end{notes}

|DataClass| defines the default class for data contained in the
|DataArray_t| entities.
For dimensional data, |DimensionalUnits| may be used to describe the
system of units employed.
If present these two entities take precedence over the corresponding
entities at higher levels of the CGNS hierarchy.
The rules for determining precedence of entities of this type are
discussed in \autoref{s:precedence}.

The \fort{UserDefinedData\_t} data structure allows arbitrary
user-defined data to be stored in \fort{Descriptor\_t} and
\fort{DataArray\_t} children without the restrictions or implicit
meanings imposed on these node types at other node locations.

\subsection{Family Data Structure Definition: \texttt{Family\_t}}
\label{s:Family}

Geometric associations need to be set through one layer of indirection.
That is, rather than setting the geometry data for each mesh entity
(nodes, edges, and faces), they are associated to intermediate objects.
The intermediate objects are in turn linked to nodal regions of the
computational mesh.
We define a CFD \emph{family} as this intermediate object.
This layer of indirection is necessary since there is rarely a 1-to-1
connection between mesh regions and geometric entities.

The \fort{Family\_t} data structure holds the CFD family data.
Each mesh surface is linked to the geometric entities of the CAD databases
by a name attribute.
This attribute corresponds to a family of CAD geometric entities on which
the mesh face is projected.
Each one of these geometric entities is described in a CAD file and is not
redefined within the CGNS file.
A \fort{Family\_t} data structure may be included in the \fort{CGNSBase\_t}
structure for each CFD family of the model.

The \fort{Family\_t} structure contains all information pertinent to a
CFD family.
This information includes the name attribute or family name, the
boundary conditions applicable to these mesh regions, and the referencing
to the CAD databases.

\begin{alltt}
  Family\_t :=
    \{
    List( Descriptor\_t Descriptor1 ... DescriptorN ) ;                      (o)

    FamilyBC\_t FamilyBC ;                                                   (o)

    List( GeometryReference\_t GeometryReference1 ... GeometryReferenceN ) ; (o)

    RotatingCoordinates\_t RotatingCoordinates ;                             (o)

    List( FamilyName\_t FamilyName1 ... FamilyNameN ) ;                      (o)

    List( UserDefinedData\_t UserDefinedData1 ... UserDefinedDataN ) ;       (o)

    int Ordinal ;                                                           (o)
    \} ;
\end{alltt}

\begin{notes}
\item All data structures contained in \fort{Family\_t} are optional.
\item Default names for the \fort{Descriptor\_t},
      \fort{GeometryReference\_t}, and \fort{UserDefinedData\_t} lists
      are as shown; users may choose other legitimate names.
      Legitimate names must be unique at this level and must
      not include the names \fort{FamilyBC}, \fort{Ordinal}, or
      \fort{RotatingCoordinates}.
\item The CAD referencing data are written in the
      \fort{GeometryReference\_t} data structures.
      They identify the CAD systems and databases where the geometric
      definition of the family is stored.
\item The boundary condition type pertaining to a family is contained in
      the data structure \fort{FamilyBC\_t}.
      If this boundary condition type is to be used, the \fort{BCType}
      specified under \fort{BC\_t} must be \fort{FamilySpecified}.
\item For the purpose of defining zone properties, families are extended
      to a volume of cells.
      In such case, the \fort{GeometryReference\_t} structures are not
      used.
\item The mesh is linked to the family by attributing a family name
      to a BC patch or a zone in the data structure \fort{BC\_t} or
      \fort{Zone\_t}, respectively.
\item A hierarchy of families is possible through the list of <tt>FamilyName\_t</tt>
      nodes. These nodes contain both a user defined node name and a
      family name. The node name <tt>FamilyParent</tt> may be used to specify
      the family name for the parent of the current <tt>Family\_t</tt> node.
\item \fort{Ordinal} is defined in the SIDS as a user-defined integer
      with no restrictions on the values that it can contain.
      It may be used here to attribute a number to the family.
\end{notes}

Rotation of the CFD family may be defined using the
\fort{RotatingCoordinates\_t} data structure.

The \fort{UserDefinedData\_t} data structure allows arbitrary
user-defined data to be stored in \fort{Descriptor\_t} and
\fort{DataArray\_t} children without the restrictions or implicit
meanings imposed on these node types at other node locations.

\subsection{Geometry Reference Structure Definition: \texttt{GeometryReference\_t}}
\label{s:GeometryReference}

The standard interface data structure identifies the CAD systems used
to generate the geometry, the CAD files where the geometry is stored,
and the geometric entities corresponding to the family.
The \fort{GeometryReference\_t} structures contain all the information
necessary to associate a CFD family to the CAD databases.
For each \fort{GeometryReference\_t} structure, the CAD format
is recorded in \fort{GeometryFormat}, and the CAD file in
\fort{GeometryFile}.
The geometry entity or entities within this CAD file that correspond to
the family are recorded under the \fort{GeometryEntity\_t} nodes.

\begin{alltt}
  GeometryReference\_t :=
    \{
    List( Descriptor\_t Descriptor1 ... DescriptorN ) ;                      (o)

    GeometryFormat\_t GeometryFormat ;                                       (r)

    GeometryFile\_t GeometryFile ;                                           (r)

    List (GeometryEntity\_t GeometryEntity1 ... GeometryEntityN) ;           (o/d)

    List( UserDefinedData\_t UserDefinedData1 ... UserDefinedDataN ) ;       (o)
    \} ;
\end{alltt}

The \fort{GeometryFormat} is an enumeration type that identifies the CAD
system used to generate the geometry.

\begin{alltt}
  GeometryFormat_t := Enumeration(
    GeometryFormatNull,
    GeometryFormatUserDefined,
    NASA-IGES,
    SDRC,
    Unigraphics,
    ProEngineer,
    ICEM-CFD ) ;
\end{alltt}

\begin{notes}
\item
Default names for the \fort{Descriptor\_t}, \fort{GeometryEntity\_t}, and
\fort{UserDefinedData\_t}
lists are as shown; users may choose other legitimate names.
Legitimate names must be unique at this level and must not include the
names \fort{GeometryFile} or \fort{GeometryFormat}.
\item
By default, there is only one \fort{GeometryEntity} and its name is
the family name.
\item
There is no limit to the number of CAD files or CAD systems referenced
in a CGNS file.
Different parts of the same model may be described with different CAD
files of different CAD systems.
\item
Other CAD geometry formats may be added to this list as needed.
\end{notes}

\subsection{Family Boundary Condition Structure Definition: \texttt{FamilyBC\_t}}
\label{s:FamilyBC}

One of the main advantages of the concept of a layer of indirection
(called a family here) is that the mesh density and the geometric
entities may be modified without altering the association between
nodes and intermediate objects, or between intermediate objects and
geometric entities.
This is very beneficial when handling boundary conditions and properties.
Instead of setting boundary conditions directly on mesh entities,
or on CAD entities, they may be associated to the intermediate objects.
Since these intermediate objects are stable in the sense that they are
not subject to mesh or geometric variations, the boundary conditions
do not need to be redefined each time the model is modified.
Using the concept of indirection, the boundary conditions and property
settings are made independent of operations such as geometric changes,
modification of mesh topology (i.e., splitting into zones), mesh
refinement and coarsening, etc.

The \fort{FamilyBC\_t} data structure contains the boundary condition type.
It is envisioned that it will be extended to hold both material and
volume properties as well.

\begin{alltt}
  FamilyBC\_t :=
    \{
    BCType\_t BCType;                                                        (r)

    List( FamilyBCDataSet\_t<ListLength>; BCDataSet1 ... BCDataSetN ) ;      (o)
    \} ;
\end{alltt}

\begin{notes}
\item Default names for the \fort{FamilyBCDataSet\_t} list are as shown; users
      may choose other legitimate names.
      Legitimate names must be unique within a given instance of
      \fort{FamilyBC\_t} and shall not include the name \fort{BCType}.
\end{notes}

\fort{BCType} specifies the boundary-condition type, which gives general
information on the boundary-condition equations to be enforced.
Boundary conditions are to be applied at the locations specified by the
\fort{BC\_t} structure(s) associated with the CFD family.

The \fort{FamilyBC\_t} structure provides for a list of
boundary-condition data sets.
In general, the proper \fort{FamilyBCDataSet\_t} instance to impose on
the CFD family is determined by the \fort{BCType} association table
(\autoref{t:BCType_assoc} on p.~\pageref*{t:BCType_assoc}).
The mechanics of determining the proper data set to impose is described
in \autoref{s:BCType_assoc}.

For a few boundary conditions, such as a symmetry plane or polar
singularity, the value of \fort{BCType} completely describes the
equations to impose, and no instances of \fort{FamilyBCDataSet\_t} are needed.
For ``simple'' boundary conditions, where a single set of Dirichlet
and/or Neumann data is applied, a single \fort{FamilyBCDataSet\_t} will likely
appear (although this is not a requirement).
For ``compound'' boundary conditions, where the equations to impose
are dependent on local flow conditions, several instances of
\fort{FamilyBCDataSet\_t} will likely appear; the procedure for choosing the
proper data set is more complex as described in \autoref{s:BCType_assoc}.

\subsection{Family Boundary Condition Data Set Structure Definition: \texttt{FamilyBCDataSet\_t}}
\label{s:FamilyBCDataSet}

|FamilyBCDataSet_t| contains Dirichlet and Neumann data for a single set of
boundary-condition equations. Its intended use is for simple boundary-condition
types, where the equations imposed do not depend on local flow conditions.

\begin{alltt}
  FamilyBCDataSet\_t :=
    \{
    List( Descriptor\_t Descriptor1 ... DescriptorN ) ;                      (o)

    BCTypeSimple\_t BCTypeSimple ;                                           (r)

    BCData\_t<1> DirichletData ;                                             (o)
    BCData\_t<1> NeumannData ;                                               (o)

    ReferenceState\_t ReferenceState ;                                       (o)

    DataClass\_t DataClass ;                                                 (o)

    DimensionalUnits\_t DimensionalUnits ;                                   (o)

    List( UserDefinedData\_t UserDefinedData1 ... UserDefinedDataN ) ;       (o)
    \} ;
\end{alltt}

\begin{notes}
\item Default names for the \fort{Descriptor\_t} and
      \fort{UserDefinedData\_t} lists are as shown; users may choose other
      legitimate names.
      Legitimate names must be unique within a given instance
      of \fort{FamilyBCDataSet\_t} and shall not include the names
      \fort{BCTypeSimple},
      \fort{DataClass}, \fort{DimensionalUnits}, \fort{DirichletData},
      \fort{NeumannData} or \fort{ReferenceState}.
\item \fort{BCTypeSimple} is the only required field.
      All other fields are optional and the \fort{Descriptor\_t} list
      may be empty.
\end{notes}

|BCTypeSimple| specifies the boundary-condition type, which gives general
information on the bound\-ary-con\-di\-tion equations to be enforced.
|BCTypeSimple_t| is defined in \autoref{s:BCType} along with the meanings
of all the |BCTypeSimple| values.
|BCTypeSimple| is also used for matching boundary condition data sets as
discussed in \autoref{s:BCType_assoc}.

Boundary-condition data is separated by equation type into Dirichlet
and Neumann conditions.  Dirichlet boundary conditions impose the
value of the given variables, whereas Neumann boundary conditions
impose the normal derivative of the given variables.  The mechanics of
specifying Dirichlet and Neumann data for boundary conditions is covered
in \autoref{s:BC_specdata}.

The substructures \fort{DirichletData} and \fort{NeumannData} contain
boundary-condition data defined as globally constant over the family.

Reference quantities applicable to the set of boundary-condition data are
contained in the \fort{ReferenceState} structure.
|DataClass| defines the default for the class of data contained in the
boundary-condition data.
If the boundary conditions contain dimensional data, |DimensionalUnits|
may be used to describe the system of dimensional units employed.
If present, these three entities take precedence of all corresponding
entities at higher levels of the hierarchy.
These precedence rules are further discussed in \autoref{s:precedence}.

The \fort{UserDefinedData\_t} data structure allows arbitrary
user-defined data to be stored in \fort{Descriptor\_t} and
\fort{DataArray\_t} children without the restrictions or implicit
meanings imposed on these node types at other node locations.

Note that \fort{FamilyBCDataSet\_t} is similar to the data structure
\fort{BCDataSet\_t} (\autoref{s:BCDataSet}).  The primary difference is that
\fort{FamilyBCDataSet\_t} only allows for globally constant Dirichlet and
Neumann data.

\subsection{User-Defined Data Structure Definition: \texttt{UserDefinedData\_t}}
\label{s:UserDefinedData}

Since the needs of all CGNS users cannot be anticipated,
\fort{UserDefinedData\_t} provides a means of storing arbitrary
user-defined data in \fort{Descriptor\_t} and \fort{DataArray\_t}
children without the restrictions or implicit meanings imposed on these
node types at other node locations.

\begin{alltt}
  UserDefinedData\_t :=
    \{
    List( Descriptor\_t Descriptor1 ... DescriptorN ) ;                      (o)

    GridLocation\_t GridLocation ;                                           (o/d)

    IndexRange\_t<IndexDimension> PointRange ;                               (o)
    IndexArray\_t<IndexDimension, ListLength, int> PointList ;               (o)

    List( DataArray\_t<> DataArray1 ... DataArrayN ) ;                       (o)

    DataClass\_t DataClass ;                                                 (o)

    DimensionalUnits\_t DimensionalUnits ;                                   (o)

    FamilyName\_t FamilyName ;                                               (o)

    List( AdditionalFamilyName\_t AddFamilyName1 ... AddFamilyNameN ) ;      (o)

    List( UserDefinedData\_t UserDefinedData1 ... UserDefinedDataN ) ;       (o)

    int Ordinal ;                                                           (o)
    \} ;
\end{alltt}

\begin{notes}
\item Default names for the \fort{Descriptor\_t}, \fort{DataArray\_t},
      and \fort{UserDefinedData\_t} lists are as shown; users may choose
      other legitimate names.
      Legitimate names must be unique within a given instance of
      \fort{UserDefinedData\_t} and shall not include the names
      \fort{DataClass}, \fort{DimensionalUnits}, \fort{FamilyName},
      \fort{GridLocation}, \fort{Ordinal}, \fort{PointList}, or
      \fort{PointRange}.
\item \fort{GridLocation} may be set to \fort{Vertex},
      \fort{IFaceCenter}, \fort{JFaceCenter}, \fort{KFaceCenter},
      \fort{FaceCenter}, \fort{CellCenter}, or \fort{EdgeCenter}.
      If \fort{GridLocation} is absent, then its default value is
      \fort{Vertex}.
      When \fort{GridLocation} is set to \fort{Vertex}, then
      \fort{PointList} or \fort{PointRange} refer to node indices, for
      both structured and unstructured grids.
      When \fort{GridLocation} is set to \fort{FaceCenter}, then
      \fort{PointList} or \fort{PointRange} refer to face elements.
\item \fort{GridLocation}, \fort{PointRange}, and \fort{PointList}
      may only be used when \fort{UserDefinedData\_t} is located below a
      \fort{Zone\_t} structure in the database hierarchy.
\item Only one of \fort{PointRange} and \fort{PointList} may be
      specified.
\end{notes}

\subsection{Gravity Data Structure Definition: \texttt{Gravity\_t}}
\label{s:Gravity}

The \fort{Gravity\_t} data structure may be used to define the
gravitational vector.

\begin{alltt}
  Gravity\_t :=
    \{
    List( Descriptor\_t Descriptor1 ... DescriptorN ) ;                      (o)

    DataArray\_t<real, 1, PhysicalDimension> GravityVector ;                 (r)

    DataClass\_t DataClass ;                                                 (o)

    DimensionalUnits\_t DimensionalUnits ;                                   (o)

    List( UserDefinedData\_t UserDefinedData1 ... UserDefinedDataN ) ;       (o)
    \} ;
\end{alltt}

\begin{notes}
\item
Default names for the \fort{Descriptor\_t} and \fort{UserDefinedData\_t}
lists are as shown; users may choose other legitimate names.
Legitimate names must be unique within a given instance of
\fort{Gravity\_t} and shall not include the names \fort{DataClass},
\fort{DimensionalUnits}, or \fort{GravityVector}.
\end{notes}

The only required field under the \fort{Gravity\_t} data structure is
\fort{GravityVector}, which contains the components of the gravity
vector in the coordinate system being used.

\fort{DataClass} defines the default class for data contained in the
\fort{DataArray\_t} entity.
For dimensional data, \fort{DimensionalUnits} may be used to describe
the system of units employed.
If present, these two entities take precedence over the corresponding
entities at higher levels of the CGNS hierarchy, following the standard
precedence rules.

The \fort{UserDefinedData\_t} data structure allows arbitrary
user-defined data to be stored in \fort{Descriptor\_t} and
\fort{DataArray\_t} children without the restrictions or implicit
meanings imposed on these node types at other node locations.
