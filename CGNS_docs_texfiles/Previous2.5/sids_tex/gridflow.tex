\section{Grid Coordinates, Elements, and Flow Solutions}
\label{s:gridflow}
\thispagestyle{plain}

This section defines structure types for describing the grid
coordinates, element data, and flow solution data pertaining to a zone.
Entities of each of the structure types defined in this section are
contained in the |Zone_t| structure (see \autoref{s:Zone}).

\subsection{Grid Coordinates Structure Definition: \texttt{GridCoordinates\_t}}
\label{s:Grid}

The physical coordinates of the grid vertices are described by the 
|GridCoordinates_t| structure.  This structure contains a list for the 
data arrays of the individual components of the position vector.  It also
provides a mechanism for identifying rind-point data included within the
position-vector arrays.

\begin{alltt}
  GridCoordinates\_t< int IndexDimension, int VertexSize[IndexDimension] > :=
    \{
    List( Descriptor\_t Descriptor1 ... DescriptorN ) ;                      (o)
 
    Rind\_t<IndexDimension> Rind ;                                           (o/d)

    List( DataArray\_t<DataType, IndexDimension, DataSize[]> 
          DataArray1 ... DataArrayN ) ;                                     (o)

    DataClass\_t DataClass ;                                                 (o)
    
    DimensionalUnits\_t DimensionalUnits ;                                   (o)

    List( UserDefinedData\_t UserDefinedData1 ... UserDefinedDataN ) ;       (o)
    \} ;
\end{alltt}

\begin{notes}
\item Default names for the \texttt{Descriptor\_t},
      \texttt{DataArray\_t}, and \texttt{UserDefinedData\_t} lists are
      as shown; users may choose other legitimate names.
      Legitimate names must be unique within a given instance of
      \texttt{GridCoordinates\_t} and shall not include the names
      \texttt{DataClass}, \texttt{DimensionalUnits}, or \texttt{Rind}.
\item There are no required fields for \texttt{GridCoordinates\_t}.
      \texttt{Rind} has a default if absent; the default is equivalent
      to having a \texttt{Rind} structure whose \texttt{RindPlanes}
      array contains all zeros (see \autoref{s:Rind}).
\item The structure parameter \texttt{DataType} must be consistent with
      the data stored in the \texttt{DataArray\_t} substructures (see
      \autoref{s:DataArray}).
\end{notes}

\texttt{GridCoordinates\_t} requires two structure parameters:
\texttt{IndexDimension} identifies the dimensionality of the grid-size
arrays, and \texttt{VertexSize} is the number of vertices in each index
direction excluding rind points.
For unstructured zones, \texttt{IndexDimension} is always 1 and
\texttt{VertexSize} is the total number of vertices, excluding rind
points.

The grid-coordinates data is stored in the list of |DataArray_t| entities;
each |DataArray_t| structure entity may contain a single component of
the position vector (e.g., three separate |DataArray_t| entities are used
for $x$, $y$, and $z$).  
Standardized data-name identifiers for the grid coordinates are described
in \hyperref[s:dataname]{Appendix~\ref*{s:dataname}}.

\texttt{Rind} is an optional field that indicates the number of
rind planes (for structured grids) or rind points or elements (for
unstructured grids) included in the grid-coordinates data.
If \texttt{Rind} is absent, then the \texttt{DataArray\_t} structure
entities contain only ``core'' vertices of a zone; core refers to all
interior and bounding vertices of a zone --- \texttt{VertexSize} is the
number of core vertices.
Core vertices in a zone are assumed to begin at |[1,1,1]|
(for a structured zone in 3-D) and end at \texttt{VertexSize}.
If \texttt{Rind} is present, it will provide information on the number
of ``rind'' points in addition to the core points that are contained in
the \texttt{DataArray\_t} structures.

|DataClass| defines the default class for data contained in the
|DataArray_t| entities.
For dimensional grid coordinates, |DimensionalUnits| may be used to
describe the system of units employed.
If present, these two entities take precedence over the corresponding
entities at higher levels of the CGNS hierarchy.
The rules for determining precedence of entities of this type are
discussed in \autoref{s:precedence}.
An example that uses these grid-coordinate defaults is shown in
\autoref{s:grid_example}.

The \fort{UserDefinedData\_t} data structure allows arbitrary
user-defined data to be stored in \fort{Descriptor\_t} and
\fort{DataArray\_t} children without the restrictions or implicit
meanings imposed on these node types at other node locations.

\enlargethispage{\baselineskip}
%\noindent {\bf FUNCTION} |DataSize[]|:
\subsubsection*{FUNCTION \texttt{DataSize[]}:}

\noindent return value: one-dimensional |int| array of length
                        |IndexDimension| \\
\noindent dependencies: |IndexDimension|, |VertexSize[]|, |Rind|

|GridCoordinates_t| requires a single structure function, named |DataSize|,
to identify the array sizes of the grid-coordinates data.  A function is 
required for the following reasons:
\begin{myitemize}
\item
 the entire grid, including both core and rind points, is stored in 
 the |DataArray_t| entities;
\item
 the |DataArray_t| structure is simple in that it doesn't know anything
 about core versus rind data; it just knows that it contains data 
 of some given size;
\item
 to make all the |DataArray_t| entities syntactically consistent in
 their size (i.e., by syntax entities containing $x$, $y$ and $z$ must
 have the same dimensionality and dimension sizes), the size of the
 array is passed onto the |DataArray_t| structure as a parameter.
\end{myitemize}

\begin{alltt}
  if (Rind is absent) then
    \{
    DataSize[] = VertexSize[] ;
    \}
  else if (Rind is present) then
    \{ 
    DataSize[] = VertexSize[] + [a + b,...] ;
    \}
\end{alltt}
where |RindPlanes = [a,b,...]| (see \autoref{s:Rind} 
for the definition of |RindPlanes|). 

\subsection{Grid Coordinates Examples}
\label{s:grid_example}

This section contains examples of grid coordinates.  These examples
show the storage of the grid-coordinate data arrays, as well as different
mechanisms for describing the class of data and the system of units or
normalization.

\begin{example}{Cartesian Coordinates for a 2-D Structured Grid}
\label{ex:grid1}

Cartesian coordinates for a 2-D grid of size $17\times33$; the data
arrays include only core vertices, and the coordinates are in units of feet.
\begin{alltt}
  !  IndexDimension = 2
  !  VertexSize = [17,33]
  GridCoordinates\_t<2, [17,33]> GridCoordinates =
    \{\{
    DataArray\_t<real, 2, [17,33]> CoordinateX =
      \{\{
      Data(real, 2, [17,33]) = ((x(i,j), i=1,17), j=1,33) ;

      DataClass\_t DataClass = Dimensional ;
      
      DimensionalUnits\_t DimensionalUnits = 
        \{\{ 
        MassUnits        = Null ;
        LengthUnits      = Foot ;
        TimeUnits        = Null ;
        TemperatureUnits = Null ;
        AngleUnits       = Null ;
        \}\} ;
      \}\} ;

    DataArray\_t<real, 2, [17,33]> CoordinateY =
      \{\{
      Data(real, 2, [17,33]) = ((y(i,j), i=1,17), j=1,33) ;

      DataClass\_t DataClass = Dimensional ;
      
      DimensionalUnits\_t DimensionalUnits = 
        \{\{ 
        MassUnits        = Null ;
        LengthUnits      = Foot ;
        TimeUnits        = Null ;
        TemperatureUnits = Null ;
        AngleUnits       = Null ;
        \}\} ;
      \}\} ;
    \}\} ;
\end{alltt}
From \hyperref[s:dataname]{Appendix~\ref*{s:dataname}}, the identifiers
for $x$ and $y$ are |CoordinateX| and |CoordinateY|, respectively, and
both have a data type of |real|.
The value of |DataClass| in |CoordinateX| and |CoordinateY| indicate
the data is dimensional, and |DimensionalUnits| specifies the appropriate
units are feet.
The |DimensionalExponents| entity is absent from both |CoordinateX| and
|CoordinateY|; the information that $x$ and $y$ are lengths can be
inferred from the data-name identifier conventions in
\autoref{s:dataname_grid}.

Note that FORTRAN multidimensional array indexing is used to store
the data; this is reflected in the FORTRAN-like implied do-loops for
|x(i,j)| and |y(i,j)|.

Since the dimensional units for both $x$ and $y$ are the same, an
alternate approach is to set the data class and system of units using
|DataClass| and |DimensionalUnits| at the |GridCoordinates_t| level, and
eliminate this information from each instance of |DataArray_t|.
\begin{alltt}
  GridCoordinates\_t<2, [17,33]> GridCoordinates =
    \{\{
    DataClass\_t DataClass = Dimensional ;
    
    DimensionalUnits\_t DimensionalUnits = 
      \{\{ 
      MassUnits        = Null ;
      LengthUnits      = Foot ;
      TimeUnits        = Null ;
      TemperatureUnits = Null ;
      AngleUnits       = Null ;
      \}\} ;

    DataArray\_t<real, 2, [17,33]> CoordinateX =
      \{\{
      Data(real, 2, [17,33]) = ((x(i,j), i=1,17), j=1,33) ;
      \}\} ;

    DataArray\_t<real, 2, [17,33]> CoordinateY =
      \{\{
      Data(real, 2, [17,33]) = ((y(i,j), i=1,17), j=1,33) ;
      \}\} ;
    \}\} ;
\end{alltt}
Since the \fort{DataClass} and \fort{DimensionalUnits} entities are not
present in \fort{CoordinateX} and \fort{CoordinateY}, the rules
established in \autoref{s:data_dim} dictate that \fort{DataClass}
and \fort{DimensionalUnits} specified at the \fort{GridCoordinates\_t}
level be used to retrieve the information.
\end{example}

\newpage
\begin{example}{Cylindrical Coordinates for a 3-D Structured Grid}
\label{ex:grid2}

Cylindrical coordinates for a 3-D grid whose core size is
$17\times33\times9$.
The grid contains a single plane of rind on the minimum and maximum $k$
faces.
The coordinates are nondimensional.
\begin{alltt}
  !  IndexDimension = 3
  !  VertexSize = [17,33,9]
  GridCoordinates\_t<3, [17,33,9]> GridCoordinates =
    \{\{
    Rind\_t<3> Rind =
      \{\{
      int[6] RindPlanes = [0,0,0,0,1,1] ;
      \}\} ;

    ! DataType = real
    ! IndexDimension = 3
    ! DataSize = VertexSize + [0,0,2] = [17,33,11]
    DataArray\_t<real, 3, [17,33,11]> CoordinateRadius =
      \{\{
      Data(real, 3, [17,33,11]) = (((r(i,j,k), i=1,17), j=1,33), k=0,10) ;

      DataClass\_t DataClass = NormalizedByUnknownDimensional ;
      \}\} ;

    DataArray\_t<real, 3, [17,33,11]> CoordinateZ     = \{\{ \}\} ;
    DataArray\_t<real, 3, [17,33,11]> CoordinateTheta = \{\{ \}\} ;
    \}\} ;
\end{alltt}
The value of |RindPlanes| specifies two rind planes on the minimum and
maximum $k$ faces.
These rind planes are reflected in the structure function |DataSize|
which is equal to the number of core vertices plus two in the  $k$ dimension.
The value of |DataSize| is passed to the |DataArray_t| entities.
The value of |DataClass| indicates the data is nondimensional.
Note that if \fort{DataClass} is set as \fort{NormalizedByUnknownDimensional}
at a  higher level (\fort{CGNSBase\_t} or \fort{Zone\_t}), then it is not
needed here.

Note that the entities |CoordinateZ| and |CoordinateTheta| are abbreviated.
\end{example}

\begin{example}{Cartesian Coordinates for a 3-D Unstructured Grid}
\label{ex:grid3}

Cartesian grid coordinates for a 3-D unstructured zone where
\fort{VertexSize} is 15.
\begin{alltt}
  GridCoordinates\_t<1, 15> GridCoordinates =
    \{\{

    ! DataType = real
    ! IndexDimension = 1
    ! DataSize = VertexSize = 15
    DataArray\_t<real, 1, 15> CoordinateX =
      \{\{
      Data(real, 1, 15) = (x(i), i=1,15) ;
      \}\} ;

    DataArray\_t<real, 1, 15> CoordinateY =
      \{\{
      Data(real, 1, 15) = (y(i), i=1,15) ;
      \}\} ;

    DataArray\_t<real, 1, 15> CoordinateZ =
      \{\{
      Data(real, 1, 15) = (z(i), i=1,15) ;
      \}\} ;
    \}\} ;
\end{alltt}
\end{example}

\subsection{Elements Structure Definition: \texttt{Elements\_t}}
\label{s:Elements}

The \fort{Elements\_t} data structure is required for unstructured
zones, and contains the element connectivity data, the element type,
the element range, the parent elements data, and the number of boundary
elements.

\begin{alltt}
  Elements\_t :=
    \{
    List( Descriptor\_t Descriptor1 ... DescriptorN ) ;                      (o)
 
    Rind\_t<IndexDimension> Rind ;                                           (o/d)

    IndexRange\_t ElementRange ;                                             (r)

    int ElementSizeBoundary ;                                               (o/d)

    ElementType\_t ElementType ;                                             (r)

    DataArray\_t<int, 1, ElementDataSize> ElementConnectivity ;              (r)

    DataArray\_t<int, 2, [ElementSize, 4]> ParentData;                       (o)

    List( UserDefinedData\_t UserDefinedData1 ... UserDefinedDataN ) ;       (o)
    \} ;
\end{alltt}

\begin{notes}
\item Default names for the \texttt{Descriptor\_t} and
      \texttt{UserDefinedData\_t} lists are as shown; users may choose
      other legitimate names.
      Legitimate names must be unique within a given instance
      of \texttt{Elements\_t} and shall not include the names
      \texttt{ElementConnectivity}, \texttt{ElementRange},
      \texttt{ParentData}, or \texttt{Rind}.
\item \texttt{IndexRange\_t}, \texttt{ElementType\_t}, and
      \texttt{ElementConnectivity\_t} are the required fields within the
      \texttt{Elements\_t} structure.
      \texttt{Rind} has a default if absent; the default is equivalent
      to having a \texttt{Rind} structure whose \texttt{RindPlanes}
      array contains all zeros (see \autoref{s:Rind}).
\end{notes}

\texttt{Rind} is an optional field that indicates the number of rind
elements included in the elements data.
If \texttt{Rind} is absent, then the \texttt{DataArray\_t} structure
entities contain only core elements of a zone.
If \texttt{Rind} is present, it will provide information on the number
of rind elements, in addition to the core elements, that are contained
in the \texttt{DataArray\_t} structures.

Note that the usage of rind data with respect to the size of the
\texttt{DataArray\_t} structures is different under \texttt{Elements\_t}
than elsewhere.
For example, when rind coordinate data is stored under
\texttt{GridCoordinates\_t}, the parameter \texttt{VertexSize} accounts
for the core data only.
The size of the \texttt{DataArray\_t} structures containing the grid
coordinates is determined by the \texttt{DataSize} function, which adds
the number of rind planes or points to \texttt{VertexSize}.
But for the element connectivity, the size of the \texttt{DataArray\_t}
structures containing the connectivity data is just
\texttt{ElementDataSize}, which depends on \texttt{ElementSize}, and
includes both the core and rind elements.

\fort{ElementRange} contains the index of the first and last elements defined
in \fort{ElementConnectivity}.
The elements are indexed with a global numbering system, starting at 1,
for all element sections under a given \fort{Zone\_t} data structure.
The global numbering insures that each element, whether it's a cell,
a face, or an edge, is uniquely identified by its number.
They are also listed as a continuous list of element numbers within
any single element section.
Therefore the number of elements in a section is:

\begin{alltt}
  ElementSize = ElementRange.end - ElementRange.start + 1
\end{alltt}

The element indices are used for the boundary condition and zone
connectivity definition.

\fort{ElementSizeBoundary} indicates if the elements are sorted, and
how many boundary elements are recorded.
By default, \fort{ElementSizeBoundary} is set to zero, indicating that
the elements are not sorted.
If the elements are sorted, \fort{ElementSizeBoundary} is set to the
number of elements at the boundary.
Consequently:

\begin{alltt}
  ElementSizeInterior = ElementSize - ElementSizeBoundary
\end{alltt}

\fort{ElementType\_t} is an enumeration of the supported element types:

\begin{alltt}
  ElementType\_t := Enumeration(
     Null, NODE, BAR\_2, BAR\_3,
     TRI\_3, TRI\_6, QUAD\_4, QUAD\_8, QUAD\_9,
     TETRA\_4, TETRA\_10, PYRA\_5, PYRA\_14,
     PENTA\_6, PENTA\_15, PENTA\_18, 
     HEXA\_8, HEXA\_20, HEXA\_27, MIXED, NGON\_n, UserDefined );
\end{alltt}
\autoref{s:unstructgrid} illustrates the convention for element numbering.

For all element types except type \fort{MIXED}, \fort{ElementConnectivity}
contains the list of nodes for each element.
If the elements are sorted, then it must first list the connectivity of
the boundary elements, then that of the interior elements.

\begin{alltt}
  ElementConnectivity = Node1\tsub{1}, Node2\tsub{1}, ... NodeN\tsub{1},
                        Node1\tsub{2}, Node2\tsub{2}, ... NodeN\tsub{2},
                        ...
                        Node1\tsub{M}, Node2\tsub{M}, ... NodeN\tsub{M}
\end{alltt}

When the section \fort{ElementType} is \fort{MIXED}, the data array
\fort{ElementConnectivity} contains one extra integer per element,
to hold each individual element type:

\begin{alltt}
  ElementConnectivity = Etype\tsub{1}, Node1\tsub{1}, Node2\tsub{1}, ... NodeN\tsub{1},
                        Etype\tsub{2}, Node1\tsub{2}, Node2\tsub{2}, ... NodeN\tsub{2},
                        ...
                        Etype\tsub{M}, Node1\tsub{M}, Node2\tsub{M}, ... NodeN\tsub{M}
\end{alltt}

\fort{ElementDataSize} indicates the size (number of integers) of the array
\fort{ElementConnectivity}.
For all element types except type \fort{MIXED}, the \fort{ElementDataSize}
is given by:

\begin{alltt}
  ElementDataSize = ElementSize * NPE[ElementType]
\end{alltt}

In the case of \fort{MIXED} element section, \fort{ElementDataSize} is
given by:

\begin{alltt}
  ElementDataSize = \(\displaystyle\sum\sb{n=\mbox{\scriptsize\ital{start}}}\sp{\mbox{\scriptsize\ital{end}}} (\mbox{NPE[ElementType}\sb{n}\mbox{] + 1})\)
\end{alltt}

\fort{NPE[ElementType]} is a function returning the number of nodes for the
given \fort{ElementType}.
For example, \fort{NPE[HEXA\_8]=8}.

For face elements in 3D, or bar element in 2D, four more data may be
saved for each element --- the corresponding parents' element number,
and the face position within these parent elements.
At the boundaries, the second parent is set to zero.

\fort{NGON\_n} is used to express a polygon of $n$ nodes.
In order to record the number of nodes of any ngons, the \fort{ElementType}
must be set to \fort{NGON\_n + Nnodes}.
For example, for an element type \fort{NGON\_n} composed of 25
nodes, one would set the \fort{ElementType} to \fort{NGON\_n + 25}.

The \fort{UserDefinedData\_t} data structure allows arbitrary
user-defined data to be stored in \fort{Descriptor\_t} and
\fort{DataArray\_t} children without the restrictions or implicit
meanings imposed on these node types at other node locations.

\subsection{Elements Examples}
\label{s:element_example}

This section contains two examples of elements definition in CGNS.
In both cases, the unstructured zone contains 15 tetrahedral and 10
hexahedral elements.

\begin{example}{Unstructured Elements, Separate Element Types}
\label{ex:elements1}

In this first example, the elements are written in two separate sections,
one for the tetrahedral elements and one for the hexahedral elements.
\begin{alltt}
  Zone\_t UnstructuredZone =
    \{\{
    Elements\_t TetraElements =
      \{\{
      IndexRange\_t ElementRange = [1,15] ;

      int ElementSizeBoundary = 10 ;

      ElementType\_t ElementType = TETRA\_4 ;

      DataArray\_t<int, 1, NPE[TETRA\_4]\(\times\)15> ElementConnectivity =
        \{\{
        Data(int, 1, NPE[TETRA\_4]\(\times\)15) = (node(i,j), i=1,NPE[TETRA\_4], j=1,15) ;
        \}\} ;
      \}\} ;
    Elements\_t HexaElements =
      \{\{
      IndexRange\_t ElementRange = [16,25] ;

      int ElementSizeBoundary = 0 ;

      ElementType\_t ElementType = HEXA\_8 ;

      DataArray\_t<int, 1, NPE[HEXA\_8]\(\times\)10> ElementConnectivity =
        \{\{
        Data(int, 1, NPE[HEXA\_8]\(\times\)10) = (node(i,j), i=1,NPE[HEXA\_8], j=1,10) ;
        \}\} ;
      \}\} ;
    \}\} ;
\end{alltt}
\end{example}

\begin{example}{Unstructured Elements, Element Type \texttt{MIXED}}
\label{ex:elements2}

In this second example, the same unstructured zone described in 
\hyperref[ex:elements1]{Example~\ref*{ex:elements1}} is written in a
single element section of type \fort{MIXED} (i.e., an unstructured grid
composed of mixed elements).
\begin{alltt}
  Zone\_t UnstructuredZone =
    \{\{
    Elements\_t MixedElementsSection =
      \{\{
      IndexRange\_t ElementRange = [1,25] ;

      ElementType\_t ElementType = MIXED ;

      DataArray\_t<int, 1, ElementDataSize> ElementConnectivity =
        \{\{
        Data(int, 1, ElementDataSize) = (etype(j),(node(i,j),
             i=1,NPE[etype(j)]), j=1,25) ;
        \}\} ;
      \}\} ;
    \}\} ;
\end{alltt}
\end{example}

\subsection{Axisymmetry Structure Definition: \texttt{Axisymmetry\_t}}
\label{s:Axisymmetry}

The \fort{Axisymmetry\_t} data structure allows recording the axis of
rotation and the angle of rotation around this axis for a two-dimensional
dataset that represents an axisymmetric database.

\begin{alltt}
  Axisymmetry\_t :=
    \{
    List( Descriptor\_t Descriptor1 ... DescriptorN ) ;                      (o)

    DataArray\_t<real,1,2> AxisymmetryReferencePoint ;                       (r)
    DataArray\_t<real,1,2> AxisymmetryAxisVector ;                           (r)
    DataArray\_t<real,1,1> AxisymmetryAngle ;                                (o)
    DataArray\_t<char,2,[32,2]> CoordinateNames ;                            (o)

    DataClass\_t DataClass ;                                                 (o)

    DimensionalUnits\_t DimensionalUnits ;                                   (o)

    List( UserDefinedData\_t UserDefinedData1 ... UserDefinedDataN ) ;       (o)
    \} ;
\end{alltt}

\begin{notes}
\item Default names for the \fort{Descriptor\_t} and
      \fort{UserDefinedData\_t} lists are as shown; users may choose
      other legitimate names.
      Legitimate names must be unique within a given instance
      of \fort{Axisymmetry\_t} and shall not include the names
      \fort{AxisymmetryAngle}, \fort{AxisymmetryAxisVector},
      \fort{AxisymmetryReferencePoint}, \fort{CoordinateNames},
      \fort{DataClass}, or \fort{DimensionalUnits}.
\item \fort{AxisymmetryReferencePoint} and \fort{AxisymmetryAxisVector}
      are the required fields within the \fort{Axisymmetry\_t} structure.
\end{notes}

\fort{AxisymmetryReferencePoint} specifies the origin used for
defining the axis of rotation.

\fort{AxisymmetryAxisVector} contains the direction cosines of the
axis of rotation, through the \fort{AxisymmetryReferencePoint}.
For example, for a 2-D dataset defined in the $(x,y)$ plane,
if \fort{AxisymmetryReferencePoint} contains $(0,0)$ and
\fort{AxisymmetryAxisVector} contains $(1,0)$, the $x$-axis is the
axis of rotation.

\fort{AxisymmetryAngle} allows specification of the circumferential
extent about the axis of rotation.
If this angle is undefined, it is assumed to be 360\degree.

\fort{CoordinateNames} may be used to specify the first and second
coordinates used in the definition of \fort{AxisymmetryReferencePoint}
and \fort{AxisymmetryAxisVector}.
If not found, it is assumed that the first coordinate is
\fort{CoordinateX} and the second is \fort{CoordinateY}.
The coordinates given under \fort{CoordinateNames}, or implied by
using the default, must correspond to those found under
\fort{GridCoordinates\_t}.

\fort{DataClass} defines the default class for numerical data contained
in the \fort{DataArray\_t} entities.
For dimensional data, \fort{DimensionalUnits} may be used to describe
the system of units employed.
If present, these two entities take precedence over the corresponding
entities at higher levels of the CGNS hierarchy, following the standard
precedence rules.

The \fort{UserDefinedData\_t} data structure allows arbitrary
user-defined data to be stored in \fort{Descriptor\_t} and
\fort{DataArray\_t} children without the restrictions or implicit
meanings imposed on these node types at other node locations.

\subsection{Rotating Coordinates Structure Definition: \texttt{RotatingCoordinates\_t}}
\label{s:RotatingCoordinates}

The \fort{RotatingCoordinates\_t} data structure is used to record
the rotation center and rotation rate vector of a rotating coordinate
system.

\begin{alltt}
  RotatingCoordinates\_t :=
    \{
    List( Descriptor\_t Descriptor1 ... DescriptorN ) ;                      (o)

    DataArray\_t<real,1,PhysicalDimension> RotationCenter ;                  (r)
    DataArray\_t<real,1,PhysicalDimension> RotationRateVector ;              (r)

    DataClass\_t DataClass ;                                                 (o)

    DimensionalUnits\_t DimensionalUnits ;                                   (o)

    List( UserDefinedData\_t UserDefinedData1 ... UserDefinedDataN ) ;       (o)
    \} ;
\end{alltt}

\begin{notes}
\item Default names for the \fort{Descriptor\_t} and
      \fort{UserDefinedData\_t} lists are as shown; users may choose
      other legitimate names.
      Legitimate names must be unique within a given instance
      of \fort{RotatingCoordinates\_t} and shall not include the names
      \fort{DataClass}, \fort{DimensionalUnits}, \fort{RotationCenter},
      or \fort{RotationRateVector}.
\item \fort{RotationCenter} and \fort{RotationRateVector}
      are the required fields within the \fort{RotatingCoordinates\_t}
      structure.
\end{notes}

\texttt{RotationCenter} specifies the coordinates of the center of
rotation, and \texttt{RotationRateVector} specifies the components of
the angular velocity of the grid about the center of rotation.
Together, they define the angular velocity vector.
The direction of the angular velocity vector specifies the axis of
rotation, and its magnitude specifies the rate of rotation.

For example, for the common situation of rotation about the $x$-axis,
\texttt{RotationCenter} would be specified as any point on the $x$-axis,
like $(0,0,0)$.
\texttt{RotationRateVector} would then be specified as ($\omega$,0,0),
where $\omega$ is the rotation rate.
Using the right-hand rule, $\omega$ would be positive for clockwise
rotation (looking in the $+x$ direction), and negative for
counter-clockwise rotation.

Note that for a rotating coordinate system, the axis of rotation is
defined in the inertial frame of reference, while the grid coordinates
stored using the \fort{GridCoordinates\_t} data structure are
relative to the rotating frame of reference.

\fort{DataClass} defines the default class for data contained in the
\fort{DataArray\_t} entities.
For dimensional data, \fort{DimensionalUnits} may be used to describe
the system of units employed.
If present, these two entities take precedence over the corresponding
entities at higher levels of the CGNS hierarchy, following the standard
precedence rules.

The \fort{UserDefinedData\_t} data structure allows arbitrary
user-defined data to be stored in \fort{Descriptor\_t} and
\fort{DataArray\_t} children without the restrictions or implicit
meanings imposed on these node types at other node locations.

If rotating coordinates are used, it is useful to store variables
relative to the rotating frame.
Standardized data-name identifiers should be used for these variables,
as defined for flow-solution quantities in
\hyperref[s:dataname]{Appendix~\ref*{s:dataname}}.

\subsection{Flow Solution Structure Definition: \texttt{FlowSolution\_t}} 
\label{s:FlowSolution}

The flow solution within a given zone is described by the |FlowSolution_t|
structure.  This structure contains a list for the data arrays of the
individual flow-solution variables, as well as identifying the grid location
of the solution.  It also provides a mechanism for identifying rind-point
data included within the data arrays.

\begin{alltt}
  FlowSolution\_t< int IndexDimension, int VertexSize[IndexDimension], 
                  int CellSize[IndexDimension] > :=
    \{
    List( Descriptor\_t Descriptor1 ... DescriptorN ) ;                      (o)

    GridLocation\_t GridLocation ;                                           (o/d)

    Rind\_t<IndexDimension> Rind ;                                           (o/d)

    List( DataArray\_t<DataType, IndexDimension, DataSize[]> 
          DataArray1 ... DataArrayN ) ;                                     (o)

    DataClass\_t DataClass ;                                                 (o)
    
    DimensionalUnits\_t DimensionalUnits ;                                   (o)

    List( UserDefinedData\_t UserDefinedData1 ... UserDefinedDataN ) ;       (o)
    \} ;
\end{alltt}

\begin{notes}
\item Default names for the \texttt{Descriptor\_t},
      \texttt{DataArray\_t}, and \texttt{UserDefinedData\_t} lists are
      as shown; users may choose other legitimate names.
      Legitimate names must be unique within a given instance
      of \texttt{FlowSolution\_t} and shall not include the
      names \texttt{DataClass}, \texttt{DimensionalUnits},
      \texttt{GridLocation}, or \texttt{Rind}.
\item There are no required fields for \texttt{FlowSolution\_t}.
      \texttt{GridLocation} has a default of \texttt{Vertex} if
      absent. \texttt{Rind} also has a default if absent; the
      default is equivalent to having an instance of \texttt{Rind}
      whose \texttt{RindPlanes} array contains all zeros (see
      \autoref{s:Rind}).
\item The structure parameter \texttt{DataType} must be consistent
      with the data stored in the \texttt{DataArray\_t} structure
      entities (see \autoref{s:DataArray}); \texttt{DataType} is
      \texttt{real} for all flow-solution identifiers defined in
      \hyperref[s:dataname]{Appendix~\ref*{s:dataname}}.
\item For unstructured zones \texttt{GridLocation} options are limited
      to \texttt{Vertex} or \texttt{CellCenter}, meaning that solution
      data may only be expressed at these locations.
      The data arrays must follow the node ordering
      if \texttt{GridLocation = Vertex}, and the element ordering if
      \texttt{GridLocation = CellCenter}.
\end{notes}

|FlowSolution_t| requires three structure parameters; |IndexDimension|
identifies the dimensionality of the grid-size arrays, and |VertexSize|
and |CellSize| are the number of core vertices and cells, respectively,
in each index direction.
For unstructured zones, \fort{IndexDimension} is always 1.

The flow solution data is stored in the list of |DataArray_t| entities;
each |DataArray_t| structure entity may contain a single component of
the solution vector.
Standardized data-name identifiers for the flow-solution quantities are
described in \hyperref[s:dataname]{Appendix~\ref*{s:dataname}}.
The field |GridLocation| specifies the location of the solution data
with respect to the grid; if absent, the data is assumed to coincide
with grid vertices (i.e., |GridLocation| |=| |Vertex|).
All data within a given instance of |FlowSolution_t| must reside at the
same grid location.

\texttt{Rind} is an optional field that indicates
the number of rind planes (for structured grids) or rind points or
elements (for unstructured grids) included in the data.
Its purpose and function are identical to those described in
\autoref{s:Grid}.
Note, however, that the \texttt{Rind} in this structure is independent
of the \texttt{Rind} contained in \texttt{GridCoordinates\_t}.
They are not required to contain the same number of rind planes or
elements.
Also, the location of any flow-solution rind points is assumed to be
consistent with the location of the core flow solution points (e.g.,
if \texttt{GridLocation = CellCenter}, rind points are assumed to be
located at fictitious cell centers).

|DataClass| defines the default class for data contained in the
|DataArray_t| entities.  For dimensional flow solution data,
|DimensionalUnits| may be used to describe the system of units employed.
If present these two entities take precedence over the corresponding entities
at higher levels of the CGNS hierarchy.  The rules for determining precedence
of entities of this type are discussed in \autoref{s:precedence}.

The \fort{UserDefinedData\_t} data structure allows arbitrary
user-defined data to be stored in \fort{Descriptor\_t} and
\fort{DataArray\_t} children without the restrictions or implicit
meanings imposed on these node types at other node locations.

%\noindent {\bf FUNCTION} |DataSize[]|:
\subsubsection*{FUNCTION \texttt{DataSize[]}:}

\noindent return value: one-dimensional |int| array of length
                        |IndexDimension| \\
\noindent dependencies: |IndexDimension|, |VertexSize[]|, |CellSize[]|,
                        |GridLocation|, |Rind|

The function |DataSize[]| is the size of flow solution data arrays.
If |Rind| is absent then |DataSize| represents only the core points;
it will be the same as |VertexSize| or |CellSize| depending on
|GridLocation|.
The definition of the function |DataSize[]| is as follows:

\begin{alltt}
  if (Rind is absent) then
    \{
    if (GridLocation = Vertex) or (GridLocation is absent)
      \{
      DataSize[] = VertexSize[] ;
      \}
    else if (GridLocation = CellCenter) then
      \{
      DataSize[] = CellSize[] ;
      \}
    \}
  else if (Rind is present) then
    \{
    if (GridLocation = Vertex) or (GridLocation is absent) then
      \{
      DataSize[] = VertexSize[] + [a + b,...] ;
      \}
    else if (GridLocation = CellCenter)
      \{
      DataSize[] = CellSize[] + [a + b,...] ;
      \}
    \}
\end{alltt}
where |RindPlanes = [a,b,...]| (see \autoref{s:Rind} 
for the definition of |RindPlanes|). 

\subsection{Flow Solution Example}
\label{s:flow_example}

This section contains an example of the flow solution entity, including
the designation of grid location and rind planes and data-normalization
mechanisms.

\begin{example}{Flow Solution}
\label{ex:flow}

Conservation-equation variables ($\rho$, $\rho u$, $\rho v$ and
$\rho e_0$) for a 2-D grid of size $11\times5$.
The flowfield is cell-centered with two planes of rind data.
The density, momentum and stagnation energy ($\rho e_0$) data is
nondimensionalized with respect to a freestream reference state whose
quantities are dimensional.
The freestream density and pressure are used for normalization; these
values are 1.226 kg/m\tsup{3} and $1.0132 \!\times\! 10^{5}$ N/m\tsup{2}
(standard atmosphere conditions).
The data-name identifier conventions for the conservation-equation
variables are \texttt{Density}, \texttt{MomentumX}, \texttt{MomentumY} and
\texttt{EnergyStagnationDensity}.
\begin{alltt}
  !  IndexDimension = 2
  !  VertexSize = [11,5]
  !  CellSize = [10,4]
  FlowSolution\_t<2, [11,5], [10,4]> FlowExample =
    \{\{
    GridLocation\_t GridLocation = CellCenter ;

    Rind\_t<2> Rind =
      \{\{
      int[4] RindPlanes = [2,2,2,2] ;
      \}\} ;

    DataClass\_t DataClass = NormalizedByDimensional ;
    
    DimensionalUnits\_t DimensionalUnits = 
      \{\{ 
      MassUnits        = Kilogram ;
      LengthUnits      = Meter ;
      TimeUnits        = Second ;
      TemperatureUnits = Null ;
      AngleUnits       = Null ;
      \}\} ;

    !  DataType = real
    !  Dimension = 2
    !  DataSize = CellSize + [4,4] = [14,8]
    DataArray\_t<real, 2, [14,8]> Density =
      \{\{
      Data(real, 2, [14,8]) = ((rho(i,j), i=-1,12), j=-1,6) ;

      DataConversion\_t DataConversion =
        \{\{
        ConversionScale  = 1.226 ;
        ConversionOffset = 0 ;
        \}\} ;

      DimensionalExponents\_t DimensionalExponents =
        \{\{
        MassExponent        = +1 ;
        LengthExponent      = -3 ;
        TimeExponent        =  0 ;
        TemperatureExponent =  0 ;
        AngleExponent       =  0 ;
        \}\} ;
      \}\} ;

    DataArray\_t<real, 2, [14,8]> MomentumX =
      \{\{
      Data(real, 2, [14,8]) = ((rho\_u(i,j), i=-1,12), j=-1,6) ;

      DataConversion\_t DataConversion =
        \{\{
        ConversionScale  = 352.446 ;
        ConversionOffset = 0 ;
        \}\} ;
      \}\} ;

    DataArray\_t<real, 2, [14,8]> MomentumY =
      \{\{
      Data(real, 2, [14,8]) = ((rho\_v(i,j), i=-1,12), j=-1,6) ;

      DataConversion\_t DataConversion =
        \{\{
        ConversionScale  = 352.446 ;
        ConversionOffset = 0 ;
        \}\} ;
      \}\} ;

    DataArray\_t<real, 2, [14,8]> EnergyStagnationDensity =
      \{\{
      Data(real, 2, [14,8]) = ((rho\_e0(i,j), i=-1,12), j=-1,6) ;

      DataConversion\_t DataConversion =
        \{\{
        ConversionScale  = 1.0132e+05 ;
        ConversionOffset = 0 ;
        \}\} ;
      \}\} ;
    \}\} ;
\end{alltt}
The value of |GridLocation| indicates the data is at cell centers, and the
value of |RindPlanes| specifies two rind planes on each face of the zone.
The resulting value of the structure function |DataSize| is the number of
cells plus four in each coordinate direction; this value is passed to each
of the |DataArray_t| entities.

Since the data are all nondimensional and normalized by dimensional
reference quantities, this information is stated in |DataClass| and 
|DimensionalUnits| at the |FlowSolution_t| level rather than attaching
the appropriate |DataClass| and |DimensionalUnits| to each |DataArray_t|
entity.
It could possibly be at even higher levels in the heirarchy.
The contents of |DataConversion| are in each case the denominator of the
normalization; this is $\rho_\infty$ for density,
$\sqrt{p_\infty \rho_\infty}$ for momentum, and $p_\infty$ for
stagnation energy.
The dimensional exponents are specified for density.
For all the other data, the dimensional exponents are to be inferred from
the data-name identifiers.

Note that no information is provided to identify the actual reference
state or indicate that it is freestream.  This information is not needed
for data manipulations involving renormalization or changing the units
of the converted raw data.
\end{example}


