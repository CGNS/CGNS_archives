%
% last modif: 2004-12-08
%
% NEW CONCEPT: the ADF node is no more a real node, we say now CGNS node,
% because ADF itself exist at the same time with HDF ! 
%
% ****************************************************************************
% ** WE SHOULD NOT DOCUMENT THE MAPPING ASSUMING THE INTERFACE LIBRARY IS ADFH
% ** ANY HDF TOOL WOULD BE ABLE TO PRODUCE AN HDF5 STRUCTURE UNDERSTOOD AS A
% ** CGNS COMPLIANT STORAGE STRUCTURE 
% ****************************************************************************
%
% I don't know if we have to add some implementation details that would have
% an actual impact on interface. For example, the ways we create datasets 
% and dataspaces, always with default properties. Certainly we should locate
% the places where the properties could/would/should be set.
%
% See ADFH.c comments, such as:
%%   /*
%%    * The ADF documentation allows the dimension values to be changed
%%    * without affecting the data, so long as the data type and number
%%    * of dimensions are the same. With HDF5, we could emulate that by
%%    * using extendable data spaces (with chunking). However this only
%%    * allows the data size to increase, not decrease, and coming up
%%    * with a good value for chunking is difficult. Since changing the
%%    * dimension values without rewiting the data is not a common
%%    * operation, I decided to use fixed sizes, then buffer the data
%%    * in these rare cases.
%%    */
%
% Clearly, this is an implementation choice, but it has an impact on
% interface. The question is : ``Can somebody else implement an interface
% AND an HDF5 file compliant without the emulation described above ?''
%

\section{SLL Mapping to \HDF}
\label{s:hdfsummary}
\thispagestyle{plain}

% Cheating: almost replace ADF with HDF ;)

The purpose of the current document is to describe the way in which
CFD data is to be represented in an \HDF data tree. To do this, it is necessary
to first describe the \HDF data structure itself in some
detail. Therefore, a conceptual summary of \HDF is given here in order
to make the current document relatively independent, and to allow
the reader to focus on those aspects of \HDF which are essential to
understanding the file mapping.

Any \HDF file with a conformant mapping is \emph{CGNS/\HDF} compliant. The
mapping has been made using a \emph{per-node} basis. Instead of having a new
mapping dedicated to \HDF, we have made a mapping from the ADF nodes
to a set of \HDF groups. While this is not an optimal mapping, the use of 
such an \HDF node allows us to re-use the ADF API without change.

The \HDF documentation should be used as the authoritative
references to resolve any issues not covered by this summary.

\subsection{General Description of \HDF}

% Here, LOW level is a bit scornfull, we may change this term
The \HDF library provides CGNS with a low level storage system.  
This library is developed and maintained by the National Center
for Supercomputing Applications at the University of Illinois at
Urbana-Champaign.
It offers a
free storage system, with support for a very large number of platforms
used in the scientific world.  This system is in charge of all
physical features such as storage, compression, virtual access to data
and many other services also useful for the CGNS users.

\HDF allows a fine tuning of the physical devices, however, the \SLL
use of \HDF forces all \emph{property lists} to \texttt{DEFAULT}
values.

Refer to the HDF5 web site at \url{http://hdf.ncsa.uiuc.edu/HDF5} for
further information about \HDF.

\subsection{Dedicated \HDF Structures}
\label{s:adfnodestructure}

The \emph{CGNS node} is an \HDF group. It contains attributes, an optional
dataset and optional child groups. This structure is required and can be
extended by other implementors as long as the following requirements are
fullfilled.

In a CGNS tree, most of the nodes are \emph{normal} nodes. These nodes
can optionally contain data and have other nodes as children. 

There are special nodes, such as the root node and
the nodes managing the \emph{links}. These special link structures use
the \HDF mount system. It is not necessary for either the MLL 
or \SLL users to understand the \HDF mount system, but it should be 
understood by \SLL implementors.

\subsubsection{The CGNS Node Mapping}

The CGNS node uses \emph{groups} and \emph{attributes} elements of \HDF. 
A basic CGNS node is composed of elements listed in \autoref{t:groupnode}.
The attribute names have been enclosed in square brackets, because the
SLL uses private attributes with a leading \textit{space} character.
These brackets are not part of the names.

\begin{table}[htbp]
\centering
\caption[Basic Node]{\textbf{Basic Node}}
\label{t:groupnode}
\begin{tabular}{l >{\quad}l >{\quad}l}
\\ \hline\hline \\*[-2ex]
\bold{GROUP} &  &  
\\*[1ex] \hline\hline \\*[-2ex]
% is there a reason why types are not native ints/chars ?
REQUIRED ATTRIBUTE  & \fort{[name]}   & \fort{U8LE[33]} \\
REQUIRED ATTRIBUTE  & \fort{[label]}  & \fort{U8LE[33]} \\
REQUIRED ATTRIBUTE  & \fort{[type]}   & \fort{U8LE[3]} \\
REQUIRED ATTRIBUTE  & \fort{[ order]} & \fort{H5T\_NATIVE\_INT} \\
OPTIONAL DATASET    & \fort{[ data]}  & 
\\*[1ex] \hline\hline
\end{tabular}
\end{table}

All the children groups of a node are understood as \SLL children nodes of this
node, unless the group name starts with a \emph{space} character.
The \emph{dataset} contains the actual data of a node. In the case of
a node without data (i.e. \texttt{MT} type), the dataset does not exist.

In addition to the attributes of a \emph{normal} node, special attribute 
names are reserved for the root node (see 
\autoref{t:rootnode}). 
The name of the root node is \texttt{[HDF5 MotherNode]}, its label is
\texttt{[Root Node of HDF5 File]}, type is \texttt{MT}.

\begin{table}[htbp]
\centering
\caption[Root Node]{\textbf{Root Node}}
\label{t:rootnode}
\begin{tabular}{l >{\quad}l >{\quad}l}
\\ \hline\hline \\*[-2ex]
\bold{GROUP} & \fort{[/]} &  
\\*[1ex] \hline\hline \\*[-2ex]
% is there a reason why types are not native ints/chars ?
REQUIRED ATTRIBUTE  & \fort{[ version]} & \fort{U8LE[33]} \\
REQUIRED ATTRIBUTE  & \fort{[ format]}  & \fort{U8LE[33]}
\\*[1ex] \hline\hline
\end{tabular}
\end{table}

The node hosting the link entries is a special node named \texttt{[ mount]}
without attributes nor data. This group should be a child of the root node,
its absolute path is \texttt{[/ mount]}. The group \texttt{[ mount]} contains
one group for each mounted file refered to by a link. 
Each entry has the attributes described in \autoref{t:linknode}. 
The link node itself is a group with the special name \texttt{[ link]}. 
This implementation is discussed further in \autoref{s:linkmechs}.
% clear enough ?

\begin{table}[htbp]
\centering
\caption[Link Node]{\textbf{Link Node}}
\label{t:linknode}
\begin{tabular}{l >{\quad}l >{\quad}l}
\\ \hline\hline \\*[-2ex]
\bold{GROUP} & \fort{[/ mount/ *]} &  
\\*[1ex] \hline\hline \\*[-2ex]
REQUIRED ATTRIBUTE  & \fort{[ refcnt]} & \fort{H5T\_NATIVE\_INT} \\
REQUIRED ATTRIBUTE  & \fort{[ file]}   & \fort{H5T\_NATIVE\_CHAR}
\\*[1ex] \hline\hline
\end{tabular}
\end{table}

\subsubsection{The Node ID}
The node ID is a unique identifier assigned to each existing node by
\HDF when the file containing it is opened, and to new nodes as they are
created. \ADFH inquiries generally return node IDs as a result and
accept node IDs as input. By building a table of IDs, calling software
can subsequently access specific nodes without further search. The Node
ID is real and is not under user control, its lifetime is ended by
the closing of the \HDF file.

\subsubsection{The Node Name}
The node Name is a 32-byte character field which is user
controllable. Its general use is to distinguish among the children of a
given node; consequently, no two children of the same parent may have
the same Name. Constraints related to the node name are detailed in
\autoref{s:nodename}.

\subsubsection{The Label}
The Label is a 32-byte character field which is user controllable. \ADFH
assigns no formal role to the Label, but the intent was to identify
the structure of the included data. It is common for the various
children of a single parent to store different instances of the same
structure. Therefore, there is no prohibition against more than one
child of the same parent having the same Label.

\subsubsection{The Data Type}
The Data Type is a 32-byte character field which specifies the type and
precision of any data which is stored in the data field. Types provided
by \HDF are listed in \autoref{t:datatype}.

\begin{table}[htbp]
\centering
\caption[Data Types]{\textbf{Data Types}}
\label{t:datatype}
\begin{tabular}{l >{\quad}l >{\quad}l}
\\ \hline\hline \\*[-2ex]
\bold{Data Type} & \bold{Notation} & \bold{\HDF Type}
\\*[1ex] \hline\hline \\*[-2ex]
No Data             & \fort{MT} & \fort{-} \\
Integer 32          & \fort{I4} & \fort{H5T\_NATIVE\_INT32} \\
Integer 64          & \fort{I8} & \fort{H5T\_NATIVE\_INT64} \\
Unsigned Integer 32 & \fort{U4} & \fort{H5T\_NATIVE\_UINT32} \\
Unsigned Integer 64 & \fort{U8} & \fort{H5T\_NATIVE\_UINT64} \\
Real 32             & \fort{R4} & \fort{H5T\_NATIVE\_FLOAT} \\
Real 64             & \fort{R8} & \fort{H5T\_NATIVE\_DOUBLE} \\
Complex 64          & \fort{X4} & \fort{-} \\
Complex 128         & \fort{X8} & \fort{-} \\
Character           & \fort{C1} & \fort{H5T\_NATIVE\_CHAR} \\
Byte                & \fort{B1} & \fort{H5T\_NATIVE\_UCHAR} \\
Link                & \fort{LK} & \fort{-} 
\\*[1ex] \hline\hline
\end{tabular}
\end{table}

\noindent
%If the data type is \fort{MT} or \fort{LK}, the node attributes which
%described the data may be left undefined. 
There is no mapping to \HDF
\fort{MT} and \fort{LK} types, because there is no actual
\emph{data space} associated with the nodes. The type information itself
is stored in the node as the strings ``\fort{MT}'' and ``\fort{LK}''.
The types \fort{X4} and \fort{X8}
are also not mapped.

The data storage format is translated as described in \autoref{t:nativeformat}.
\begin{table}[htbp]
\centering
\caption[Native Formats]{\textbf{Native Formats}}
\label{t:nativeformat}
\begin{tabular}{l >{\quad}l}
\\ \hline\hline \\*[-2ex]
\bold{Native format} & \bold{\HDF Type}
\\*[1ex] \hline\hline \\*[-2ex]
\fort{H5T\_IEEE\_F32BE} & \fort{IEEE\_BIG\_32} \\
\fort{H5T\_IEEE\_F32LE} & \fort{IEEE\_LITTLE\_32} \\
\fort{H5T\_IEEE\_F64BE} & \fort{IEEE\_BIG\_64} \\
\fort{H5T\_IEEE\_F64LE} & \fort{IEEE\_LITTLE\_64}
\\*[1ex] \hline\hline
\end{tabular}
\end{table}

\subsubsection{The Number of Dimensions}

The Data portion of a node is designed to store multi-dimensional
arrays of data, each element of which is presumed to be of the Data
Type specified. The Number of Dimensions specifies the number of
integers required to reference a single datum within the array.

\subsubsection{The Dimension Values}

The Dimension Values are a list of integers expressing the actual
sizes of the stored array in each of the dimensions specified. These
dimensions are stored by the \emph{dataspace} associated
with the \emph{dataset}; no other attributes contains these values.

\subsubsection{The Data}

In an \HDF node, the portion of the node holding the actual stored data array
is a \textit{dataset}.

\subsubsection{The Child Table}

All groups of the current group are said to be the children of this group,
except the groups with a name starting with a space character. Each of
these children groups is a \emph{normal} node (i.e. group).

Children may be identified by their names and labels, and, thence,
by their node IDs once these have been determined. \HDF provides no
notion of order among children, but the \SLL layer adds a creation order
stored in the \texttt{[ order]} attribute. This order is guaranteed to
be the same from call to call, even after the file has been closed and
re-opened.

%In particular, the order of a list of
%children returned by ADF has nothing to do with the order in which they
%were inserted in the file. However, the order returned is consistent
%from call to call provided the file has not been closed and the node
%structure has not been modifed.

Note that there is no \emph{parent} table; that is, a node has no direct
knowledge of its parent. Since calling software must open the file
from the root, it presumably cannot access a child without having
first accessed the parent.  It is the responsibility of the calling
software to record the node ID of the parent if this information will be
required.

\subsubsection{The link mechanism}
\label{s:linkmechs}

A \texttt{LK} typed node is a \emph{link}. Such a node refers to
another node elsewhere. In other words, the \emph{link} has no child
or contents, but is a name of a node somewhere in the current file or
in another file.  The \texttt{LK} typed node is said to be the
\emph{emph} node, while the node elsewhere is said to be the
\emph{destination} node.

Is is the role of the \SLL layer to insure consistency\footnote{In
particular in order to avoid an acyclic graph.} and
transparency of the link mechanism, so that any \emph{normal} node
request to the \emph{link} node is performed as if it is performed
on the destination node.

A \emph{link} destination can be in the same file as the \emph{link} source
or in another file. In both cases, the \emph{link} is made using an \HDF
soft link from source to destination. In the case of a destination in another
file, the destination file is mounted in the \texttt{[/ mount]}\footnote{Absolute path.} group and
the soft link is made from the source to the destination now present in
this mounted file. 
% Ok. Is there still somebody in the room ?
% No ? Now, going to explain how to cook the pancakes.

Each time a file is mounted, the \texttt{[/ refcnt]}\footnote{Absolute path.}
attribute is incremented.
The file is unmounted if there is no reference to itself.


