\section{CGNS Documentation}
\label{s:documentation}
\thispagestyle{plain}

Documentation of CGNS is found in a number of related
publications. These are maintained separately for several
reasons. First, they describe logically independent aspects of the CGNS
system. Second, many users will find that a subset of the documentation
is sufficient for their needs. And last, some portions of the system can
be viewed as independent entities useful outside the context of CGNS.

The main documents currently available are:

\begin{itemize*}
\item CGNS Overview and Entry-Level Document
\item A User's Guide to CGNS
\item ADF User's Guide
\item HDF Documentation
\item Standard Interface Data Structures
\item SIDS-to-ADF File Mapping Manual
\item SIDS-to-HDF File Mapping Manual (this document)
\item CGNS Mid-Level Library
\end{itemize*}

\subsection{Description of the Documents}

In order to make the current document as self-contained as possible,
basic information regarding all components of the CGNS standard has
been included. However, the reader should be aware of the documents
describing these other components. These should be regarded as the
authoritative and complete descriptions of their respective components
of CGNS. Each of these documents is described briefly in this section.

\subsubsection{CGNS Overview and Entry-Level Document}

This is an introductory document which provides an overall view of the
purpose and components of CGNS. The Overview is intended as an entry
point into CGNS. Prospective users completely unfamiliar with CGNS
should consult the Overview to determine whether CGNS provides the
capabilities they seek.

\subsubsection{A User's Guide to CGNS}

The \textit{User's Guide to CGNS} has been written to aid users in the
implementation of CGNS.
It is intended as a tutorial: light in content, but heavy in examples,
advice, and guidelines.
The guide provides a concise overview of many of the most commonly-used
features of the SIDS, and gives coding examples using the CGNS Mid-Level
Library to write and read simple CGNS files.

\subsubsection{ADF User's Guide}

The \textit{ADF User's Guide} describes
one of the underlying database managers, ADF (Advanced Data Format),
which may be used to create CGNS files.
ADF is a binary format, based on a simple tree structure.
In principal, nearly any kind of data could be stored in ADF format.
ADF was, however, especially designed for the storage of large
quantities of scientific data in a platform-independent and
randomly-accessible manner.
The ``ADF Core'' is a set of portable routines which store and retrieve
data in ADF format.
These
routines are written in C, but Fortran versions are also provided.
The \textit{ADF User's Guide} describes both the ADF format and the ADF Core
routines in detail.

It should be emphasized that ADF, with its Core routines, constitutes a
very general stand-alone database manager which is not directly related
to CFD. It can therefore be used to store any kind of data \emph{once it has
been specified where to place that data within the ADF format}.
The \textit{SIDS-to-ADF File Mapping Manual} describes the CGNS
conventions governing that placement for CFD-specific data.

\subsubsection{HDF Documentation}

HDF5 (Hierarchical Data Format) is another underlying database manager
that may be used to create CGNS files, and is developed and maintained
by the National Center for Supercomputing Applications at the University
of Illinois at Urbana-Champaign.
Documentation for HDF5 is available at the HDF5 Home Page, at
\url{http://hdf.ncsa.uiuc.edu/HDF5/}.

Like ADF, HDF5 is a binary format, based on a tree structure,
designed for the storage of large quantities of scientific data in a
platform-independent and randomly-accessible manner.
Also like ADF, HDF5 can be used to store any kind of data.
This File Mapping document describes the CGNS conventions governing the
placement of CFD-specific data within the HDF5 format.

\subsubsection{Standard Interface Data Structures}

The \textit{Standard Interface Data Structures}, usually abbreviated as
``SIDS,'' define the ``intellectual content'' of CFD data.
They detail the data which must be stored to completely characterize
each CFD entity. They describe, for example, what exactly is meant by a
``grid'' or a ``boundary condition''.
They also establish a system of nomenclature which gives standard
meaning to certain names, such as ``Density'' and ``SubsonicInflow''.

The SIDS description of the CFD data is hierarchical in nature, in that
complex entities are built up out of simpler ones. In the SIDS document,
this is reflected in a syntax which uses C-like structures to define
the various entities. The result is a tree-like structure which maps
naturally onto the ADF format. Consequently, the File Mapping described
by the current document exactly parallels the SIDS. Thus in terms of
basic structure, the Mapping itself summarizes the SIDS. However, there
are many conventions regarding the nomenclature and meaning of data
which are not summarized in the current document, and for these the SIDS
is the authoritative document.

It is worth emphasizing that the SIDS may be regarded as a stand-alone
definition of the data associated with CFD, and that these data could
be stored in any sufficiently general format, given a mapping onto that
format.

\subsubsection{The SIDS-to-ADF File Mapping Manual}

The \textit{SIDS-to-ADF File Mapping Manual} specifies the exact manner
in which, under CGNS conventions, CFD data structures (the SIDS) are
to be stored in, i.e., mapped onto, the file structure provided by the
ADF database manager.
Adherence to the mapping conventions guarantees uniform meaning
and location of CFD data within ADF files, and thereby allows the
construction of universal software to read and write the data.

\subsubsection{The SIDS-to-HDF File Mapping Manual}

The current document is similar to the
\textit{SIDS-to-ADF File Mapping Manual},
and specifies the mapping from the
SIDS to the file structure provided by the \HDF database manager.

The document describes the \emph{node level} system, whereas the
Mid-Level Library can be understood as the \emph{tree level} system.
A description
is given for every Mid-Level Library type, or sub-tree, in terms of sets
of atomic nodes. Moreover, it specifies the exact manner in which, under
CGNS conventions, CFD data structures (the SIDS) are to be 
mapped onto the data structures provided by the low level layer
(\HDF).

Some system-specific mechanisms, such as \emph{link} management are
detailed.\footnote{Please note that only the \emph{final}
representation of these \emph{links} are relevant, in other words
the sequence of \HDF calls required to obtain such a representation
is not important.}
% What ? You cannot understand this sentence... me neither :(
% By the way, is it true (bruce ?)

Adherence to the mapping conventions guarantees uniform meaning
and location of CFD data within \HDF files, and thereby allows the
construction of universal software to read and write the data.

\subsubsection{The CGNS Mid-Level Library}

The \textit{CGNS Mid-Level Library} document describes a set of routines
which store and retrieve the CFD data objects defined in the SIDS.
Their purpose is to provide CGNS compliant I/O without the need for
detailed programming in the ADF or HDF Core.
These ``mid-level routines'' are designed to be inserted directly into
applications codes, such as flow solvers and grid generators.

\subsection{Which Documents Do You Need?}

Ideally, all users of the CGNS system will want to have all the
documents available for reference. However, many will find it possible
to begin to use the system effectively without reading all the documents
beforehand.  In fact, since the CGNS system is intended to minimize
interaction with underlying data structures, some users will find
they need very little knowledge of the system's internal workings. We
distinguish four classes of users who may wish to consult the CGNS
Documentation.

\subsubsection{Prospective Users}

Prospective users are presumably unfamiliar with CGNS. They will
probably wish to begin with the Overview, or, if they require more
detailed information, one or more of the various papers that have been
written describing CGNS.
Beyond that, most will find a quick read of this file mapping document
(or the SIDS-to-ADF file mapping document) enlightening as to the
logical form of the contents of CGNS files.
Browsing the figures in
\hyperref[s:figures]{Appendix~\ref*{s:figures}}
of this document, as well as the SIDS itself, will provide some feel
for the scope of the system.
The \textit{User's Guide to CGNS}, and the \textit{CGNS Mid-Level
Library} document should give an indication of what might be required to
implement CGNS in a given application.
Prospective users should probably not concern themselves with the
details of ADF or HDF.

\subsubsection{End Users}

The end user is the practitioner of CFD who generates the grids,
runs the flow codes and/or analyzes the results. For this user, CGNS
provides a mechanism for accumulating the output of the various
processes related to CFD, e.g., grid generation and flow solution,
and for making this output available to subsequent processes or for
archiving final results. For this user, a scan of the Overview will
sufficiently explain the overall workings of the system. This includes
end user responsibilities for matters not governed by CGNS, such as the
maintenance of files and directories. The end user will also find useful
those portions of the SIDS which deal with standard nomenclature. AIAA
98-3007 may also be useful if more details about the capabilities of
CGNS are desired.

The end user is, by definition, not involved in the building of
CGNS-compliant applications code.

\subsubsection{Applications Code Developers}

The applications code developer builds or maintains code to support the
various sub-processes encountered in CFD, e.g., grid generation, flow
solution, post-processing, or flow visualization. The code developer
must be able to install CGNS compliant I/O. The most convenient method
for doing so is to utilize the CGNS Mid-Level Library.
The \textit{User's Guide to CGNS} is the starting point for learning to
use the Mid-Level Library to create and use CGNS files.
The \textit{CGNS Mid-Level
Library} document itself should also be considered essential.
This library of routines will perform the most common I/O operations in
a CGNS-compliant manner.
However, even when the CGNS Library suffices to implement all necessary
I/O, an understanding of the SIDS and the file mapping (either
SIDS-to-ADF or SIDS-to-HDF) will be useful.
It will likely be necessary to consult the SIDS to determine the
precise meaning of the nomenclature.

% Remark from Chris ``is it still true?'', I think we should not put
% this paragraph, now I have example of ONERA customers using the
% ADF level without care, it leads to non-standard CGNS trees :(
%
%At present, the File mapping implements the full SIDS, but the CGNS
%Library does not. Applications coders wishing to encompass options not
%provided by the Library will need to understand the File Mapping in
%detail and how to implement it using the \HDF Core routines, and will
%need the related documents. In addition, it will be necessary to refer
%to the SIDS for any conventions not fully explained by the File Mapping
%document. AIAA 98-3007 provides an excellent description of all the data
%storage options.

\subsubsection{CGNS System Developers}

CGNS System development can be kept somewhat compartmentalized.
Developers responsible for the maintenance or building of supplements
to the ADF or HDF Core, need not concern themselves with documentation
other than the \textit{ADF User Guide} or the HDF5 documentation.
System developers wishing to add to the CGNS Mid-Level Library will need
all the documents.
Theoretical developments, such as extensions to the SIDS, may possibly
be undertaken with a knowledge of the SIDS alone, but such contributions
must also be added to the SIDS-to-ADF and SIDS-to-HDF file mappings
before they can be implemented.

\subsection{How to Use This Document}

Those wishing to do more than simply browse this document will
find that the detailed information begins with the summary of \HDF
in \autoref{s:hdfsummary}, and continues with the general file
mapping concepts in \autoref{s:general}. The detailed textual node
descriptions in \autoref{s:nodes} are more useful as reference than
as sequential literature. The best overall technical view of the
layout of CGNS files can be acquired by reference to the figures in
\hyperref[s:figures]{Appendix~\ref*{s:figures}}.
