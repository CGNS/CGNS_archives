\subsection{Database-Level Routines}
\label{s:subs_database}

\fcthdra{Database\_Open}{Open a Database}
\label{sub:Database_Open}

\begin{fctbox}
   {ADF\_Database\_Open}
   {ADFDOPN}
   {(filename,status,format,root\_ID,error\_return)}
\hline
Input  & const char *filename & character*(*) filename \\
       & const char *status   & character*(*) status \\
       & const char *format   & character*(*) format \\
\hline
Output & double *root\_ID     & real*8 root\_ID \\
       & int *error\_return   & integer error\_return \\
\hline
\end{fctbox}

\begin{Ventryi}{\textit{error\_return}}
\item[\textit{filename}]
     A legal file name that may include a relative or absolute path
     where it is directly usable by the C \fort{fopen()} system
     routine (no environment expansion is done).
\item[\textit{status}]
     Similar to a Fortran \fort{OPEN()} status.
     Input is required: there is no default.
     Allowable values are:
     \begin{Ventryic}{\fort{READ\_ONLY}}
     \item[\fort{READ\_ONLY}]
         File must exist; writing is not allowed.
     \item[\fort{OLD}]
          File must exist; reading and writing are allowed.
     \item[\fort{NEW}]
          File must not exist.
     \item[\fort{SCRATCH}]
          Temporary new file is created with a system name, and
          \textit{filename} is ignored.
          The temporary file is deleted when the program exits or
          the file is closed.
     \item[\fort{UNKNOWN}]
          \fort{OLD} if file exists or else \fort{NEW} is used.
     \end{Ventryic}
\item[\textit{format}]
     Specifies the numeric format for the file.
     This field is used only when a file is created and is ignored
     when \textit{status} = \fort{OLD}.
     Allowable values are:
     \begin{Ventryic}{\fort{IEEE\_LITTLE}}
     \item[\fort{NATIVE}]
          Use the native numeric format of the computer that creates
          the file.
          This is the default for a new file if the input string for
          format is null.
          Note that if the native numeric format is not one of the
          supported formats listed here, then the file cannot be
          read on machines using any other format.
     \item[\fort{IEEE\_BIG}]
          Use the IEEE big endian format.
     \item[\fort{IEEE\_LITTLE}]
          Use the IEEE little endian format.
     \item[\fort{CRAY}]
          Use the native CRAY format.
     \end{Ventryic}
\item[\textit{root\_ID}]
     The root identity of the database.
\item[\textit{error\_return}]
     Error return code.
     (See \hyperref[s:errors]{Appendix~\ref*{s:errors}}.)
\end{Ventryi}

This routine, \fort{ADF\_Database\_Open}, opens a new or existing database.
If links to other ADF files exist in the current file, they will be
opened only as required.
Using this routine is similar to opening a file in Fortran with the
corresponding clarifiers, such as whether it is \fort{READ\_ONLY},
\fort{OLD}, \fort{NEW}, or named as \fort{SCRATCH} file.

The format of the file, which is ignored when the status of the file is
\fort{OLD}, is used when the file is first created.
Big endian is a binary format in which the most significant byte or bit
comes first, whereas in little endian, the most significant byte or bit
comes last.
To specify the format more explicitly, you can use the following formats:
\begin{alltt}
   IEEE\_BIG\_32
   IEEE\_BIG\_64
   IEEE\_LITTLE\_32
   IEEE\_LITTLE\_64
\end{alltt}
for IEEE big or little endian formats.

\Example

This example opens a new database using the native format of the host
computer.
Note that the default format is specified by using the empty string.
In the C programming language, a null string could have been used.

\begin{alltt}
   PROGRAM TEST
      CHARACTER*(80) MSG
      REAL*8 RID
      INTEGER IERR
      CALL ADFDOPN('db.adf','NEW',' ',RID,IERR)
      IF (IERR .GT. 0) THEN
         CALL ADFERR(IERR,MSG)
         PRINT *,MSG
         STOP
      ENDIF
      STOP
      END
\end{alltt}

\fcthdr{Database\_Close}{Close a Database}
\label{sub:Database_Close}

\begin{fctbox}
   {ADF\_Database\_Close}
   {ADFDCLO}
   {(root\_ID,error\_return))}
\hline
Input  & const double *root\_ID & real*8 root\_ID \\
\hline
Output & int *error\_return     & integer error\_return \\
\hline
\end{fctbox}

\begin{Ventryi}{\textit{error\_return}}
\item[\textit{root\_ID}]
     The root identification of the database.
     This can be a valid node ID for this database.
\item[\textit{error\_return}]
     Error return code.
     (See \hyperref[s:errors]{Appendix~\ref*{s:errors}}.)
\end{Ventryi}

This routine, \fort{ADF\_Database\_Close}, closes an existing database,
as well as the other ADF files that may be attached through links.
For example, if there is another ADF file that is opened and linked to
this database, only the file stream associated with this database will
be closed.
This routine is similar to the close of a file in Fortran.

\Example

This example closes a database.
Note that while the root ID is used in the call to \fort{ADFDCLO}, any
valid node ID for this file will work.
Also, in general, it is not necessary to close open ADF files when the
program exits normally.
The standard shutdown procedures will flush all buffers and bring files
up to date.
The primary use of \fort{ADFDCLO} is to clean up file tables or to
release unused files.

\begin{alltt}
   PROGRAM TEST
      CHARACTER*(80) MSG
      REAL*8 RID
      INTEGER IERR
      CALL ADFDOPN('db.adf','NEW',' ',RID,IERR)
      IF (IERR .GT. 0) THEN
         CALL ADFERR(IERR,MSG)
         PRINT *,MSG
         STOP
      ENDIF
      .
      ...\ital{do useful stuff (hopefully)}
      .
      CALL ADFDCLO(RID,IERR)
      STOP
      END
\end{alltt}

\fcthdr{Database\_Delete}{Delete a File}
\label{sub:Database_Delete}

\begin{fctbox}
   {ADF\_Database\_Delete}
   {ADFDDEL}
   {(filename,error\_return))}
\hline
Input  & const char *filename & character*(*) filename \\
\hline
Output & int *error\_return   & integer error\_return \\
\hline
\end{fctbox}

\begin{Ventryi}{\textit{error\_return}}
\item[\textit{filename}]
     A legal file name of an existing ADF database.
     The filename may include a relative or absolute path where it
     is directly usable by the C \fort{fopen()} system routine (no
     environment expansion is done).
\item[\textit{error\_return}]
     Error return code.
     (See \hyperref[s:errors]{Appendix~\ref*{s:errors}}.)
\end{Ventryi}

This routine, \fort{ADF\_Database\_Delete}, deletes an existing database
file.
It does not delete links referenced in the database.
This routine is similar to the deletion of a file in Fortran.

\noindent
\emph{Note: This routine will be implemented in a future release.}

\fcthdr{Database\_Get\_Format}{Get the Data Format}
\label{sub:Database_Get_Format}

\begin{fctbox}
   {ADF\_Database\_Get\_Format}
   {ADFDGF}
   {(root\_ID,format,error\_return))}
\hline
Input  & const double *root\_ID & real*8 root\_ID \\
\hline
Output & char *format           & character*(*) format \\
       & int *error\_return     & integer error\_return \\
\hline
\end{fctbox}

\begin{Ventryi}{\textit{error\_return}}
\item[\textit{root\_ID}]
     Any valid node ID for the given ADF file.
\item[\textit{format}]
     The format for the file.
\item[\textit{error\_return}]
     Error return code.
     (See \hyperref[s:errors]{Appendix~\ref*{s:errors}}.)
\end{Ventryi}

This routine, \fort{ADF\_Database\_Get\_Format}, gets the data format for
an existing database.

\Example

This example opens an existing ADF database, creates a new node, and
then uses the node ID for the new node to ask what the file type is.
Note that the file \textit{format} is ignored because the
database already exists.

\begin{alltt}
   PROGRAM TEST
      CHARACTER*(80) MSG
      CHARACTER*(32) FORM
      REAL*8 RID,CID
      INTEGER IERR
      CALL ADFDOPN('db.adf','OLD',' ',RID,IERR)
      IF (IERR .GT. 0) THEN
         CALL ADFERR(IERR,MSG)
         PRINT *,MSG
         STOP
      ENDIF
      CALL ADFCRE(RID,'junk_node',CID,IERR)
      CALL ADFDGF(CID,FORM,IERR)
      PRINT *,'FILE FORMAT = ',FORM
      STOP
      END
\end{alltt}

\fcthdr{Database\_Set\_Format}{Set the Data Format}
\label{sub:Database_Set_Format}

\begin{fctbox}
   {ADF\_Database\_Set\_Format}
   {ADFDSF}
   {(root\_ID,format,error\_return))}
\hline
Input  & const double *root\_ID & real*8 root\_ID \\
\hline
Output & char *format           & character*(*) format \\
       & int *error\_return     & integer error\_return \\
\hline
\end{fctbox}

\begin{Ventryi}{\textit{error\_return}}
\item[\textit{root\_ID}]
     The root identity of the database.
\item[\textit{format}]
     The numeric format for the file.
\item[\textit{error\_return}]
     Error return code.
     (See \hyperref[s:errors]{Appendix~\ref*{s:errors}}.)
\end{Ventryi}

This routine, \fort{ADF\_Database\_Set\_Format}, sets the data format in
an existing database.

\noindent
\emph{Note: Use with extreme caution. This routine is needed only for the
data conversion utilities and not intended for the general user.}
