\subsection{Data Query Routines}
\label{s:subs_query}

\fcthdra{Get\_Label}{Get the String in a Node's Label Field}
\label{sub:Get_Label}

\begin{fctbox}
   {ADF\_Get\_Label}
   {ADFGLB}
   {(ID,label,error\_return)}
\hline
Input  & const double ID    & real*8 ID \\
\hline
Output & char *label        & character*(*) label \\
       & int *error\_return & integer error\_return \\
\hline
\end{fctbox}

\begin{Ventryi}{\textit{error\_return}}
\item[\textit{ID}]
     The ID of the node to use.
\item[\textit{label}]
     The 32-character label of the node.
\item[\textit{error\_return}]
     Error return code.
     (See \hyperref[s:errors]{Appendix~\ref*{s:errors}}.)
\end{Ventryi}

This routine returns the 32-character string stored in the node's label
field.
In C, the label will be null terminated after the last nonblank
character.
Therefore, in general, 33 characters should be allocated (32 for the
label, plus 1 for the null).
In Fortran, the label is left justified and blank filled to the right.
The null character is not returned in Fortran, therefore, the variable
declaration can be for 32 characters (e.g., \fort{CHARACTER*(32)}).

\hypertarget{ex:Get\_Label}{}
\Example
\begin{alltt}
   PROGRAM TEST
   C
         PARAMETER (MAXCHR=32)
   C
         CHARACTER*(MAXCHR) NODNAM,LABL
   C
   C *** NODE IDS
   C
         REAL*8 RID,CID
   C
   C *** OPEN DATABASE
   C
         CALL ADFDOPN('db.adf','NEW',' ',RID,IERR)
         CALL ADFCRE(RID,'NODE 1',CID,IERR)
         CALL ADFSLB(CID,'THIS IS A NODE LABEL',IERR)
   C
         CALL ADFGNAM(CID,NODNAM,IERR)
         CALL ADFGLB(CID,LABL,IERR)
   C
         PRINT *,' NODE NAME = ',NODNAM
         PRINT *,' LABEL     = ',LABL
   C
         STOP
         END
\end{alltt}

\noindent
The resulting output is:

\begin{alltt}
   NODE NAME = NODE 1
   LABEL     = THIS IS A NODE LABEL
\end{alltt}

\fcthdr{Set\_Label}{Set the String in a Node's Label Field}
\label{sub:Set_Label}

\begin{fctbox}
   {ADF\_Set\_Label}
   {ADFSLB}
   {(ID,label,error\_return)}
\hline
Input  & const double ID    & real*8 ID \\
       & const char *label  & character*(*) label \\
\hline
Output & int *error\_return & integer error\_return \\
\hline
\end{fctbox}

\begin{Ventryi}{\textit{error\_return}}
\item[\textit{ID}]
     The ID of the node to use.
\item[\textit{label}]
     The 32-character label for the node.
\item[\textit{error\_return}]
     Error return code.
     (See \hyperref[s:errors]{Appendix~\ref*{s:errors}}.)
\end{Ventryi}

This routine, \fort{ADF\_Set\_Label}, sets the 32-character string in
the node's label field.

\Example

See the \hyperlink{ex:Get\_Label}{example for \fort{ADF\_Get\_Label}}.

\fcthdr{Get\_Data\_Type}{Get the String in a Node's Data-Type Field}
\label{sub:Get_Data_Type}

\begin{fctbox}
   {ADF\_Get\_Data\_Type}
   {ADFGDT}
   {(ID,data\_type,error\_return)}
\hline
Input  & const double ID    & real*8 ID \\
\hline
Output & char *data\_type   & character*(*) data\_type \\
       & int *error\_return & integer error\_return \\
\hline
\end{fctbox}

\begin{Ventryi}{\textit{error\_return}}
\item[\textit{ID}]
     The ID of the node to use.
\item[\textit{data\_type}]
     The 32-character data type field stored in the node information
     header.
\item[\textit{error\_return}]
     Error return code.
     (See \hyperref[s:errors]{Appendix~\ref*{s:errors}}.)
\end{Ventryi}

This routine, \fort{ADF\_Get\_Data\_Type}, returns the 32-character
string in the node's data-type field.
In C, the label will be null terminated after the last nonblank;
therefore, at least 33 characters (32 for the label, plus 1 for the
null) should be allocated for the \textit{data\_type} string.
In Fortran, the null character is not returned; therefore, the
declaration for the string \textit{data\_type} can be for 32
characters.

\hypertarget{ex:Get\_Data\_Type}{}
\Example

This example illustrates the process of creating a node, storing data in
it, querying the node for the information in the header, and reading the
data back out.

\begin{alltt}
   PROGRAM TEST
   C
         PARAMETER (MAXCHR=32)
         PARAMETER (MAXROW=2)
         PARAMETER (MAXCOL=10)
   C
         CHARACTER*(MAXCHR) NODNAM,LABL
         CHARACTER*(MAXCHR) DTYPE
         REAL R4DATI(MAXROW,MAXCOL),R4DATO(MAXROW,MAXCOL)
         INTEGER IDIMI(2),IDIMO(2)
   C
   C *** NODE IDS
   C
         REAL*8 RID,CID
   C
   C *** OPEN DATABASE
   C
         CALL ADFDOPN('db.adf','NEW',' ',RID,IERR)
   C
   C *** GENERATE SOME DATA
   C
         IDIMI(1) = MAXROW
         IDIMI(2) = MAXCOL
         DO 200 ICOL = 1,MAXCOL
            DO 100 IROW = 1,MAXROW
               R4DATI(IROW,ICOL) = 2.0*ICOL*IROW
     100    CONTINUE
     200 CONTINUE
   C
   C *** GENERATE A NODE AND PUT DATA IN IT
   C
         CALL ADFCRE(RID,'NODE 1',CID,IERR)
         CALL ADFSLB(CID,'LABEL FOR NODE 1',IERR)
         CALL ADFPDIM(CID,'R4',2,IDIMI,IERR)
         CALL ADFWALL(CID,R4DATI,IERR)
   C
   C *** GET INFORMATION FROM NODE
   C
         CALL ADFGNAM(CID,NODNAM,IERR)
         CALL ADFGLB(CID,LABL,IERR)
         CALL ADFGDT(CID,DTYPE,IERR)
         CALL ADFGND(CID,NDIM,IERR)
         CALL ADFGDV(CID,IDIMO,IERR)
         CALL ADFRALL(CID,R4DATO,IERR)
   C
         PRINT *,' NODE NAME            = ',NODNAM
         PRINT *,' LABEL                = ',LABL
         PRINT *,' DATA TYPE            = ',DTYPE
         PRINT *,' NUMBER OF DIMENSIONS = ',NDIM
         PRINT *,' DIMENSIONS           = ',IDIMO
         PRINT *,' DATA:'
         WRITE(*,300)((R4DATO(I,J),I=1,MAXROW),J=1,MAXCOL)
     300 FORMAT(2(5X,F10.2))
   C
         STOP
         END
\end{alltt}

\noindent
The resulting output is:

\begin{alltt}
   NODE NAME            = NODE 1
   LABEL                = LABEL FOR NODE 1
   DATA TYPE            = R4
   NUMBER OF DIMENSIONS =            2
   DIMENSIONS           =            2           10
   DATA:
            2.00           4.00
            4.00           8.00
            6.00          12.00
            8.00          16.00
           10.00          20.00
           12.00          24.00
           14.00          28.00
           16.00          32.00
           18.00          36.00
           20.00          40.00
\end{alltt}

\fcthdr{Get\_Number\_of\_Dimensions}{Get the Number of Node Dimensions}
\label{sub:Get_Number_of_Dimensions}

\begin{fctbox}
   {ADF\_Get\_Number\_of\_Dimensions}
   {ADFGND}
   {(ID,num\_dims,error\_return)}
\hline
Input  & const double ID    & real*8 ID \\
\hline
Output & int *num\_dims     & integer num\_dims \\
       & int *error\_return & integer error\_return \\
\hline
\end{fctbox}

\begin{Ventryi}{\textit{error\_return}}
\item[\textit{ID}]
     The ID of the node to use.
\item[\textit{num\_dims}]
     The integer dimension value.
\item[\textit{error\_return}]
     Error return code.
     (See \hyperref[s:errors]{Appendix~\ref*{s:errors}}.)
\end{Ventryi}

This routine, \fort{ADF\_Get\_Number\_of\_Dimensions}, returns the number
of data dimensions used in the node.
Values will be returned only for the number of dimensions defined in the
node.
If the number of dimensions for the node is zero, an error is returned.

\Example

See the \hyperlink{ex:Get\_Data\_Type}{example for \fort{ADF\_Get\_Data\_Type}}.

\fcthdr{Get\_Dimension\_Values}{Get the Values of the Node Dimensions}
\label{sub:Get_Dimension_Values}

\begin{fctbox}
   {ADF\_Get\_Dimension\_Values}
   {ADFGDV}
   {(ID,dim\_vals[],error\_return)}
\hline
Input  & const double ID    & real*8 ID \\
\hline
Output & int *dim\_vals[]   & integer dim\_vals() \\
       & int *error\_return & integer error\_return \\
\hline
\end{fctbox}

\begin{Ventryi}{\textit{error\_return}}
\item[\textit{ID}]
     The ID of the node to use.
\item[\textit{dim\_vals}]
     The array (list) of dimension values.
\item[\textit{error\_return}]
     Error return code.
     (See \hyperref[s:errors]{Appendix~\ref*{s:errors}}.)
\end{Ventryi}

This routine, \fort{ADF\_Get\_Dimension\_Values}, returns the array
(list) of dimension values for a node.
Values will be returned only for the number of dimensions defined in the
node.
If the number of dimensions for the node is zero, an error is returned.

\Example

See the \hyperlink{ex:Get\_Data\_Type}{example for \fort{ADF\_Get\_Data\_Type}}.

\fcthdr{Put\_Dimension\_Information}{Set or Change the Data Type and Dimensions of a Node}
\label{sub:Put_Dimension_Information}

\begin{fctbox}
   {ADF\_Put\_Dimension\_Information}
   {ADFPDIM}
   {(ID,data\_type,dims,dim\_vals[],error\_return)}
\hline
Input  & const double ID        & real*8 ID \\
       & const char *data\_type & character*(*) data\_type \\
       & const int dims         & integer dims \\
       & int dim\_vals[]        & integer dim\_vals() \\
\hline
Output & int *error\_return     & integer error\_return \\
\hline
\end{fctbox}

\begin{Ventryi}{\textit{error\_return}}
\item[\textit{ID}]
     The ID of the node to use.
\item[\textit{data\_type}]
     The 32-character data type of the node.
     The valid user-definable data types are:

     \noindent\smallskip
     \begin{tabular}{@{}l c}
        \uline{\textbf{Data Type}} & \uline{\textbf{Notation}} \\
        No Data                    & \fort{MT} \\
        Integer 32                 & \fort{I4} \\
        Integer 64                 & \fort{I8} \\
        Unsigned Integer 32        & \fort{U4} \\
        Unsigned Integer 64        & \fort{U8} \\
        Real 32                    & \fort{R4} \\
        Real 64                    & \fort{R8} \\
        Complex 64                 & \fort{X4} \\
        Complex 128                & \fort{X8} \\
        Character (unsigned byte)  & \fort{C1} \\
        Byte (unsigned byte)       & \fort{B1}
     \end{tabular}\smallskip

     Compound data types can be defined as a combination of types
     (``\fort{I4,I4,R8}''), an array (``\fort{I4[25]}''), or a
     combination of types and arrays (``\fort{I4,C1[20],R8[3]}'').
     They can contain up to 32 characters.
     This style of data type definition is not very portable across
     platforms; therefore, it is not recommended.
\item[\textit{dims}]
     The number of dimensions of this node.
     \textit{dims} can be a number from 0 to 12.
     ``0'' means no data.
     The dimension of an array can range from 1 (vector) to 12.
\item[\textit{dim\_vals}]
     The array (list) of dimension values for this node.
     \textit{dim\_vals} is a vector of integers that define
     the size of the array in each dimension as defined by
     \textit{dims}.
     If the dims is zero, the \textit{dims\_vals} are not used.
     The valid range of \textit{dim\_vals} is from 1 to
     2,147,483,648.
     The total data size in bytes, calculated by the
     \textit{data\_type} size times the dimension values, cannot
     exceed 2,147,483,648 for any one node.
\item[\textit{error\_return}]
     Error return code.
     (See \hyperref[s:errors]{Appendix~\ref*{s:errors}}.)
\end{Ventryi}

This routine, \fort{ADF\_Put\_Dimension\_Information}, sets or changes
the data type and dimension information for a node.

\noindent
\emph{Note:}
When this routine is used to change the data type or number of
dimensions of an existing node, any data currently associated with the
node are lost.
The dimension values can be changed and the data space will be extended
as needed.
Be very careful changing the dimension values.
The layout of the data is assumed to be Fortran-like.
If the left-most dimension values are changed, the data are not shifted
around on disk to account for this; only the amount of data is changed.
Therefore, the indexing into the data will be changed.
In general, it is safe to change the right-most dimension value.
For example, if the array of dimension values was (10,11,20,50), then
changing the 10, 11, or 20 is very risky, whereas changing the 50 should
be safe.

\noindent
\emph{Note:}
See the discussion of Fortran character array portability in
\hyperref[s:design:fortran]{Appendix~\ref*{s:design:fortran}}.

\Example

See the \hyperlink{ex:Get\_Data\_Type}{example for \fort{ADF\_Get\_Data\_Type}}.
