\section{ADF Glossary of Terms}
\label{s:glossary}
\thispagestyle{plain}

\begin{Ventryi}{Node Name}
\item [ADF]
      The initialism (acronym) for Advanced Data Format.
\item [Child]
      One of the subnodes of a Parent.
      A child node does not have knowledge of its parent node.
      The user must keep track of this relationship.
\item [Database]
      A hierarchy of ADF nodes.
      By use of links, it may physically span multiple files.
\item [File]
      An ADF file, which a single root node and its underlying structure.
\item [ID]
      A unique identifier to access a given node within a file.
      This 8-byte field contains sufficient information for ADF to
      locate the node within a file.
      For any given node, the ID is generated only after the file it
      resides in has been opened by a program and the user requests
      information about the node.
      The ID is valid only within the program that opened the file and
      while that file is open.
      If the file is closed and reopened, the ID for any given node
      will be different.
      Within different programs, the node-ID for the same node will be
      different.
      The ID is never actually written into a file.
\item [Link-Node]
      A special type of node.
      Links are created using the \hyperlink{sub:Link}{\fort{ADF\_Link}}
      subroutine.
      The data type of this node is \fort{LK}, and its data is a
      one-dimensional array containing the name of the file (if other
      than the current file) containing the node to be linked and the
      full path name in that file from the root node to the desired
      node.

      Links provide a mechanism for referring to a node that physically
      resides in a different part of the hierarchy.
      The node pointed to by a link may or may not reside in the same
      file as the link itself.
      A link within ADF is very similar to a ``soft'' link in the
      UNIX operating system in that it does not guarantee that the
      referenced node exists.
      ADF will ``resolve'' the link only when information is requested
      about the node.
      If the ID of a link-node is used in an ADF call, the effect of
      the call is the same as if the ID of the linked-to node was used.
      Note that a link node does not have children itself.
      In \autoref{f:example} on p.~\pageref*{f:example}, the children
      seen for \fort{L3} are \fort{F4} and \fort{F5}.
      If a child is ``added'' to \fort{L3}, then in reality, the child
      is added to \fort{F3}.
      There are specialized subroutines provided to create link nodes
      and extract the link details.
\item [Node]
      The single component used to construct an ADF database.
\item [Node name]
      A node has a 32-character name.
      Every child node directly under a given parent must have a unique
      name.
      Legal characteristics in a name are \fort{A-Z}, \fort{a-z},
      \fort{0-9}, and special characters (ASCII values from 32 to
      126, omitting the forward slash ``\fort{/}'', ASCII number 47).
      Names will be blank filled to 32 bytes; they are case sensitive.
      Leading blanks are discarded and trailing blanks are ignored,
      whereas internal blanks are significant.
\item [Parent]
      A node that has subnodes directly associated with it.
\item [Pathname]
      Within a database, nodes can be referenced using the name of a node
      along with its parent ID, or by using a ``pathname'' whose syntax is
      roughly the same as a path name in the UNIX environment.
      A pathname that begins with a leading slash ``\fort{/}'' is
      assumed to begin at the root node of the file.
      If no leading slash is given, the name is assumed to begin at the
      node specified by the parent ID.
      Although there is a 32-character limitation on the node Name,
      there is no restriction on the length of the pathname.
      For example, equivalent ways to refer to node \fort{N8} in
      \autoref{f:example} are:
      \begin{itemize}
      \item Node-ID for \fort{N6} and name = ``\fort{N8}''
      \item Node-ID for \fort{N4} and name = ``\fort{N6/N8}''
      \item Node-ID for \fort{N1} and name = ``\fort{N4/N6/N8}''
      \item Node-ID for the \fort{Root\_Node} and name =
            ``\fort{/N1/N4/N6/N8}''
      \end{itemize}
\end{Ventryi}
