\section{ADF Conventions and Implementations}
\label{s:conventions}
\thispagestyle{plain}

\begin{Ventryi}{\fort{error\_return}}
\item [C]
      All input strings are to be null terminated.
      All returned strings will have the trailing blanks removed and
      will be null terminated.
      Variables declared to hold Names, Labels, and Data-Types should
      be at least 33 characters long.
      \textit{ADF.h} has a number of variables defined.
      An example declaration would be:
\begin{indlefttt}
char name[ADF\_NAME\_LENGTH+1];
\end{indlefttt}
\item [Fortran]
      Strings will be determined using inherited length.
      Returned strings will be blank filled to the specified length.
      All returned names will be left justified and blank filled on the
      right.
      There will be no null character.
      An example declaration would be:
\begin{indlefttt}
PARAMETER ADF\_NAME\_LENGTH=32
CHARACTER*(ADF\_NAME\_LENGTH) NAME
\end{indlefttt}
\item [ID]
      A unique identifier to access a given node within a file.
      This 8-byte field contains sufficient information for ADF to
      locate the node within a file.
      For any given node, the ID is generated only after the file it
      resides in has been opened by a program and the user requests
      information about the node.
      The ID is valid only within the program that opened the file and
      while that file is open.
      If the file is closed and reopened, the ID for any given node
      will be different.
      Within different programs, the node ID for the same node will be
      different.
      The ID is not ever actually written into a file.

      The declaration for variables that will hold node IDs should be
      for an 8-byte real number.

      The ID is actually a 64-bit combination of a system-generated
      file index along with the block and offset of the location of the
      node on the disk.
      In general, users do not need to know the internal coding of this
      information.
\item [\fort{error\_return}]
      The error code for the ADF routines is the following:
      \begin{Ventryi}{$n$ ($> 0$)}
      \item [$-1$]
            No error.
      \item [$n$ ($> 0$)]
            ADF error code.
            The routine
            \hyperlink{sub:Error_Message}{\fort{ADF\_Error\_Message}}
            is used to get a textual description of the error.
      \item [0]
            ADF should never return the value zero.
      \end{Ventryi}
\item [Indexing]
      All indexing is Fortran-like in that the starting index is 1 and
      the last is \fort{N} for $N$ items in an index or array
      dimension.
      The array structure is assumed to be the same as in Fortran
      with the first array dimension varying the fastest and the last
      dimension varying the slowest.
 
      The index starting at one is used in
      \hyperlink{sub:Read_Data}{\fort{ADF\_Read\_Data}},
      \hyperlink{sub:Write_Data}{\fort{ADF\_Write\_Data}}, and
      \hyperlink{sub:Children_Names}{\fort{ADF\_Children\_Names}}.
 
      The user should be aware of the differences in array indexing
      between Fortran and C.
      The subroutines
      \hyperlink{sub:Read_All_Data}{\fort{ADF\_Read\_All\_Data}} and
      \hyperlink{sub:Write_All_Data}{\fort{ADF\_Write\_All\_Data}}
      merely take a pointer to the beginning of the data, compute how
      much data is to be read/written, and process as many bytes as
      have been requested.
      Thus, these routines effectively make a copy of memory onto disk
      or vice versa.
      Given this convention, it is possible for a C program to
      use standard C conventions for array indexing and use
      \fort{ADF\_Write\_All\_Data} to store the array on disk.
      Then a Fortran program might use \fort{ADF\_Read\_All\_Data} to
      read the data set.
      Unless the user is aware of the structure of the data, it is
      possible for the array to be transposed relative to what is
      expected.
 
      The implications of the assumed array structure convention can be
      quite subtle.
      The subroutines \fort{ADF\_Write\_Data} and
      \fort{ADF\_Read\_Data} assume the Fortran array structure in
      order to index the data.
      Again, unless the user is aware of the implications of this, it
      is possible to write an array on disk and later try to change a
      portion of the data and not change the correct numbers.
 
      As long as users are aware of how their data structure maps onto
      ADF, there will not be any problems.
\end{Ventryi}
