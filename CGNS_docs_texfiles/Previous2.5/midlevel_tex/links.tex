\section{Links}
\label{s:links}
\thispagestyle{plain}

\begin{fctbox}
\textcolor{output}{\textit{ier}} = cg\_link\_write(\textcolor{input}{char *nodename}, \textcolor{input}{char *filename}, & - w m \\
~~~~~~\textcolor{input}{char *name\_in\_file}); & \\
\textcolor{output}{\textit{ier}} = cg\_is\_link(\textcolor{output}{\textit{int *path\_length}}); & r - m \\
\textcolor{output}{\textit{ier}} = cg\_link\_read(\textcolor{output}{\textit{char **filename}}, \textcolor{output}{\textit{char **link\_path}}); & r - m \\
\hline
call cg\_link\_write\_f(\textcolor{input}{nodename}, \textcolor{input}{filename}, \textcolor{input}{name\_in\_file}, \textcolor{output}{\textit{ier}}) & - w m \\
call cg\_is\_link\_f(\textcolor{output}{\textit{path\_length}}, \textcolor{output}{\textit{ier}}) & r - m \\
call cg\_link\_read\_f(\textcolor{output}{\textit{filename}}, \textcolor{output}{\textit{link\_path}}, \textcolor{output}{\textit{ier}}) & r - m \\
\end{fctbox}

\noindent
\textbf{\textcolor{input}{Input}/\textcolor{output}{\textit{Output}}}

\begin{Ventryi}{\texttt{name\_in\_file}}\raggedright
\item [\texttt{nodename}]
      Name of the link node to create, e.g., \texttt{GridCoordinates}.
      (\textcolor{input}{Input})
\item [\texttt{filename}]
      Name of the linked file, or empty string if the link is within the
      same file.
      (\textcolor{input}{Input} for \texttt{cg\_link\_write};
      \textcolor{output}{\textit{output}} for \texttt{cg\_link\_read})
\item [\texttt{name\_in\_file}]
      Path name of the node which the link points to.
      This can be a simple or a compound name, e.g.,
      \texttt{Base/Zone 1/GridCoordinates}.
      (\textcolor{input}{Input})
\item [\texttt{path\_length}]
      Length of the path name of the linked node.
      The value 0 is returned if the node is not a link.
      (\textcolor{output}{\textit{Output}})
\item [\texttt{link\_path}]
      Path name of the node which the link points to.
      (\textcolor{output}{\textit{Output}})
\item [\texttt{ier}]
      Error status.
      (\textcolor{output}{\textit{Output}})
\end{Ventryi}

Use \texttt{cg\_goto(\_f)}, described in \autoref{s:navigating}, to
position to a location in the file prior to calling these routines.

When using \texttt{cg\_link\_write}, the node being linked to does not have 
to exist when the link is created.
However, when the link is used, an error will occur if the linked-to
node does not exist.

Only nodes that support child nodes will support links.

It is assumed that the CGNS version for the file containing the link,
as determined by the \texttt{CGNSLibraryVersion\_t} node, is also
applicable to \texttt{filename}, the file containing the linked node.

Memory is allocated by the library for the return values of the C
function \texttt{cg\_link\_read}.
This memory should be freed by the user when no longer needed by calling
\texttt{cg\_free(filename)} and \texttt{cg\_free(link\_path)}, described
in \autoref{s:free}.
