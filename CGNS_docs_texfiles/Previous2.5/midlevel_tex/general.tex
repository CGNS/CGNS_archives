\section{General Remarks}
\label{s:general}
\thispagestyle{plain}

\subsection{Language}

The CGNS Mid-Level Library is written in C, but each function has a
Fortran counterpart.
All function names start with ``\texttt{cg\_}''.
The Fortran functions have the same name as their C counterpart with the
addition of the suffix ``\texttt{\_f}''.

\subsection{Character Strings}

All data structure names and labels in CGNS are limited to 32
characters.
When reading a file, it is advised to pre-allocate the
character string variables to 32 characters in Fortran, and 33
in C (to include the string terminator).
Other character strings, such as the CGNS file name or
descriptor text, are unlimited in length.
The space for unlimited length character strings will be created by the
Mid-Level Library; it is then the responsibility of the application to
release this space by a call to \texttt{cg\_free}, described in
\autoref{s:free}.

\subsection{Error Status}

All C functions return an integer value representing the error
status.
All Fortran functions have an additional parameter, \texttt{ier},
which contains the value of the error status.
An error status different from zero implies that an error occured.
The error message can be printed using the error
handling functions of the CGNS library, described in \autoref{s:error}.
The error codes are coded in the C and Fortran include files
\textit{cgnslib.h} and \textit{cgnslib\_f.h}.

\subsection{Typedefs}
\label{s:typedefs}

Several types of variables are defined using typedefs in the
\textit{cgnslib.h} file.
These are intended to facilitate the implementation of
CGNS in C.
These variable types are defined as an enumeration of key
words admissible for any variable of these types.
The file \textit{cgnslib.h} must be included in any C application
programs which use these data types.

In Fortran, the same key words are defined as integer parameters in the
include file \textit{cgnslib\_f.h}.
Such variables should be declared as \texttt{integer} in Fortran
applications.
The file \textit{cgnslib\_f.h} must be included in any
Fortran application using these key words.

The list of supported values (key words) for each of these variable
types (typedef) are:
        
{\raggedright
\begin{Ventryi}{\texttt{AverageInterfaceType\_t}}
   \item [\texttt{ZoneType\_t}]
         \texttt{Structured, Unstructured}
   \item [\texttt{ElementType\_t}]
         \texttt{NODE, BAR\_2, BAR\_3, TRI\_3, TRI\_6, QUAD\_4,
         QUAD\_8, QUAD\_9, TETRA\_4, TETRA\_10, PYRA\_5,
         PYRA\_14, PENTA\_6, PENTA\_15, PENTA\_18, HEXA\_8,
         HEXA\_20, HEXA\_27, MIXED, NGON\_n}
   \item [\texttt{DataType\_t}]
         \texttt{Integer, RealSingle, RealDouble, Character}
   \item [\texttt{DataClass\_t}]
         \texttt{Dimensional, NormalizedByDimensional,
         NormalizedByUnknownDimensional, NondimensionalParameter,
         DimensionlessConstant}
   \item [\texttt{MassUnits\_t}]
         \texttt{Null, UserDefined, Kilogram, Gram, Slug, PoundMass}
   \item [\texttt{LengthUnits\_t}]
         \texttt{Null, UserDefined, Meter, Centimeter, Millimeter, Foot,
         Inch}
   \item [\texttt{TimeUnits\_t}]
         \texttt{Null, UserDefined, Second}
   \item [\texttt{TemperatureUnits\_t}]
         \texttt{Null, UserDefined, Kelvin, Celsius, Rankine,
         Fahrenheit}
   \item [\texttt{AngleUnits\_t}]
         \texttt{Null, UserDefined, Degree, Radian}
   \item [\texttt{ElectricCurrentUnits\_t}]
         \texttt{Null, UserDefined, Ampere, Abampere, Statampere, Edison,
         auCurrent}
   \item [\texttt{SubstanceAmountUnits\_t}]
         \texttt{Null, UserDefined, Mole, Entities, StandardCubicFoot,
         StandardCubicMeter}
   \item [\texttt{LuminousIntensityUnits\_t}]
         \texttt{Null, UserDefined, Candela, Candle, Carcel, Hefner, Violle}
   \item [\texttt{GoverningEquationsType\_t}]
         \texttt{Null, UserDefined, FullPotential, Euler, NSLaminar,
         NSTurbulent, NSLaminarIncompressible,
         NSTurbulentIncompressible}
   \item [\texttt{ModelType\_t}]
         \texttt{Null, UserDefined, Ideal, VanderWaals, Constant,
         PowerLaw, SutherlandLaw, ConstantPrandtl, EddyViscosity,
         ReynoldsStress, ReynoldsStressAlgebraic,
         Algebraic\_BaldwinLomax, Algebraic\_CebeciSmith,
         HalfEquation\_JohnsonKing, OneEquation\_BaldwinBarth,
         OneEquation\_SpalartAllmaras, TwoEquation\_JonesLaunder,
         TwoEquation\_MenterSST, TwoEquation\_Wilcox,
	 CaloricallyPerfect, ThermallyPerfect,
	 ConstantDensity, RedlichKwong,
	 Frozen, ThermalEquilib, ThermalNonequilib,
	 ChemicalEquilibCurveFit, ChemicalEquilibMinimization,
	 ChemicalNonequilib, EMElectricField, EMMagneticField,
         EMConductivity, Voltage, Interpolated,
         Equilibrium\_LinRessler, Chemistry\_LinRessler}
   \item [\texttt{GridLocation\_t}]
         \texttt{Vertex, IFaceCenter, CellCenter, JFaceCenter,
         FaceCenter, KFaceCenter, EdgeCenter}
   \item [\texttt{GridConnectivityType\_t}]
         \texttt{Overset, Abutting, Abutting1to1}
   \item [\texttt{PointSetType\_t}]
         \texttt{PointList, PointRange, PointListDonor, PointRangeDonor,
         ElementList, ElementRange, CellListDonor}
   \item [\texttt{BCType\_t}]
         \texttt{Null, UserDefined, BCAxisymmetricWedge,
         BCDegenerateLine, BCExtrapolate,
         BCDegeneratePoint, BCDirichlet, BCFarfield, BCNeumann,
         BCGeneral, BCInflow, BCOutflow, BCInflowSubsonic,
         BCOutflowSubsonic, BCInflowSupersonic,
         BCOutflowSupersonic, BCSymmetryPlane, BCTunnelInflow,
         BCSymmetryPolar, BCTunnelOutflow, BCWallViscous,
         BCWall, BCWallViscousHeatFlux, BCWallInviscid,
         BCWallViscousIsothermal, FamilySpecified}
   \item [\texttt{BCDataType\_t}]
         \texttt{Dirichlet, Neumann}
   \item [\texttt{RigidGridMotionType\_t}]
         \texttt{Null, UserDefined, ConstantRate, VariableRate}
   \item [\texttt{ArbitraryGridMotionType\_t}]
         \texttt{Null, UserDefined, NonDeformingGrid, DeformingGrid}
   \item [\texttt{SimulationType\_t}]
         \texttt{TimeAccurate, NonTimeAccurate}
   \item [\texttt{WallFunctionType\_t}]
         \texttt{Generic}
   \item [\texttt{AreaType\_t}]
         \texttt{BleedArea, CaptureArea}
   \item [\texttt{AverageInterfaceType\_t}]
         \texttt{AverageAll, AverageCircumferential, AverageRadial, AverageI,
	 AverageJ, AverageK}
\end{Ventryi}
}

Note that these key words need to be written exactly as they
appear here, including the use of upper and lower case,
to be recognized by the library.

\subsection{Character Names For Typedefs}

The CGNS library defines character arrays which map the typedefs above
to character strings.
These are global arrays dimensioned to the size of each list of
typedefs.
To retrieve a character string representation of a typedef, use the
typedef value as an index to the appropiate character array.
For example, to retrieve the string ``\texttt{Meter}''
for the \texttt{LengthUnits\_t Meter} typedef, use
\texttt{LengthUnitsName[Meter]}.
Functions are available to retrieve these names without the need for
direct global data access.
These functions also do bounds checking on the input, and if out of
range, will return the string ``\texttt{<invalid>}''.
An additional benefit is that these will work from within a Windows DLL,
and are thus the recommended access technique.
The routines have the same name as the global data arrays, but with a
``\texttt{cg\_}'' prepended.
For the example above, use ``\texttt{cg\_LengthUnitsName(Meter)}''.

\subsection{Acquiring the Software and Documentation}

The CGNS Mid-Level Library may be downloaded from SourceForge, at
{\itshape\url{http://sourceforge.net/projects/cgns}}.
This manual, as well as the other CGNS documentation, is available in
both HTML and PDF format from the CGNS documentation web site, at
{\itshape\url{http://www.grc.nasa.gov/www/cgns/}}.

\subsection{Organization of This Manual}

The sections that follow describe the Mid-Level Library functions
in detail.
The first three sections cover some basic file operations (i.e.,
opening and closing a CGNS file, and some configuration options)
(\autoref{s:fileops}), accessing a specific node in a CGNS database
(\autoref{s:navigating}), and error handling (\autoref{s:error}).
The remaining sections describe the functions used to read, write, and
modify nodes and data in a CGNS database.
These sections basically follow the organization used in
the ``Detailed CGNS Node Descriptions'' section of both the
\href{../filemap/filemap.pdf}{SIDS-to-ADF} and
\href{../filemap_hdf/filemap.pdf}{SIDS-to-HDF} file mapping manuals.

At the start of each sub-section is a \textit{Node} line, listing the
the applicable CGNS node label.

Next is a table illustrating the syntax for the Mid-Level Library
functions.
The C functions are shown in the top half of the table, followed by the
corresponding Fortran routines in the bottom half of the table.
Input variables are shown in an \textcolor{input}{\texttt{upright blue}}
font, and output variables are shown in a
\textcolor{output}{\texttt{\textit{slanted red}}} font.
For each function, the right-hand column lists the modes (read, write,
and/or modify) applicable to that function.

The input and output variables are then listed and defined.
