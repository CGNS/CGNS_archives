\section{Error Handling}
\label{s:error}
\thispagestyle{plain}

\begin{fctbox}
\textcolor{output}{\textit{error\_message}} = char *cg\_get\_error(); & r w m \\
void cg\_error\_exit(); & r w m \\
void cg\_error\_print(); & r w m \\
\hline
call cg\_get\_error\_f(\textcolor{output}{\textit{error\_message}}) & r w m \\
call cg\_error\_exit\_f() & r w m \\
call cg\_error\_print\_f() & r w m \\
\end{fctbox}

If an error occurs during the execution of a CGNS library function,
signified by a non-zero value of the error status variable
\texttt{ier}, an error message may be retrieved using the function
\texttt{cg\_get\_error}.
The function \texttt{cg\_error\_exit} may then be used to print the
error message and stop the execution of the program.
Alternatively, \texttt{cg\_error\_print} may be used to print the
error message and continue execution of the program.

In C, you may define a function to be called automatically in the case
of a warning or error using the \texttt{cg\_configure} routine.
The function is of the form \texttt{void err\_func(int is\_error, char
*errmsg)}, and will be called whenever an error or warning occurs.
The first argument, \texttt{is\_error}, will be 0 for warning messages,
1 for error messages, and $-1$ if the program is going to terminate
(i.e., a call to \texttt{cg\_error\_exit}).
The second argument is the error or warning message.
