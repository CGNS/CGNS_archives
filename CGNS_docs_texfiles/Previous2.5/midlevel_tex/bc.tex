\section{Boundary Conditions}
\label{s:bc}
\thispagestyle{plain}

\subsection{Boundary Condition Type and Location}
\label{s:bctype}

\noindent
\textit{Node}: \texttt{BC\_t}

\begin{fctbox}
\textcolor{output}{\textit{ier}} = cg\_boco\_write(\textcolor{input}{int fn}, \textcolor{input}{int B}, \textcolor{input}{int Z}, \textcolor{input}{char *boconame}, & - w m \\
~~~~~~\textcolor{input}{BCType\_t bocotype}, \textcolor{input}{PointSetType\_t ptset\_type}, \textcolor{input}{int npnts}, & \\
~~~~~~\textcolor{input}{int *pnts}, \textcolor{output}{\textit{int *BC}}); & \\
\textcolor{output}{\textit{ier}} = cg\_boco\_normal\_write(\textcolor{input}{int fn}, \textcolor{input}{int B}, \textcolor{input}{int Z}, \textcolor{input}{int BC}, & - w m \\
~~~~~~\textcolor{input}{int *NormalIndex}, \textcolor{input}{int NormalListFlag}, & \\
~~~~~~\textcolor{input}{DataType\_t NormalDataType}, \textcolor{input}{void *NormalList}); & \\
\textcolor{output}{\textit{ier}} = cg\_nbocos(\textcolor{input}{int fn}, \textcolor{input}{int B}, \textcolor{input}{int Z}, \textcolor{output}{\textit{int *nbocos}}); & r - m \\
\textcolor{output}{\textit{ier}} = cg\_boco\_info(\textcolor{input}{int fn}, \textcolor{input}{int B}, \textcolor{input}{int Z}, \textcolor{input}{int BC}, \textcolor{output}{\textit{char *boconame}}, & r - m \\
~~~~~~\textcolor{output}{\textit{BCType\_t *bocotype}}, \textcolor{output}{\textit{PointSetType\_t *ptset\_type}}, \textcolor{output}{\textit{int *npnts}}, & \\
~~~~~~\textcolor{output}{\textit{int *NormalIndex}}, \textcolor{output}{\textit{int *NormalListFlag}}, & \\
~~~~~~\textcolor{output}{\textit{DataType\_t *NormalDataType}}, \textcolor{output}{\textit{int *ndataset}}); & \\
\textcolor{output}{\textit{ier}} = cg\_boco\_read(\textcolor{input}{int fn}, \textcolor{input}{int B}, \textcolor{input}{int Z}, \textcolor{input}{int BC}, \textcolor{output}{\textit{int *pnts}}, & r - m \\
~~~~~~\textcolor{output}{\textit{void *NormalList}}); & \\
\hline
call cg\_boco\_write\_f(\textcolor{input}{fn}, \textcolor{input}{B}, \textcolor{input}{Z}, \textcolor{input}{boconame}, \textcolor{input}{bocotype}, \textcolor{input}{ptset\_type}, \textcolor{input}{npnts}, & - w m \\
~~~~~\textcolor{input}{pnts}, \textcolor{output}{\textit{BC}}, \textcolor{output}{\textit{ier}}) & \\
call cg\_boco\_normal\_write\_f(\textcolor{input}{fn}, \textcolor{input}{B}, \textcolor{input}{Z}, \textcolor{input}{BC}, \textcolor{input}{NormalIndex}, \textcolor{input}{NormalListFlag}, & - w m \\
~~~~~\textcolor{input}{NormalDataType}, \textcolor{input}{NormalList}, \textcolor{output}{\textit{ier}}) & \\
call cg\_nbocos\_f(\textcolor{input}{fn}, \textcolor{input}{B}, \textcolor{input}{Z}, \textcolor{output}{\textit{nbocos}}, \textcolor{output}{\textit{ier}}) & r - m \\
call cg\_boco\_info\_f(\textcolor{input}{fn}, \textcolor{input}{B}, \textcolor{input}{Z}, \textcolor{input}{BC}, \textcolor{output}{\textit{boconame}}, \textcolor{output}{\textit{bocotype}}, \textcolor{output}{\textit{ptset\_type}}, & r - m \\
~~~~~\textcolor{output}{\textit{npnts}}, \textcolor{output}{\textit{NormalIndex}}, \textcolor{output}{\textit{NormalListFlag}}, \textcolor{output}{\textit{NormalDataType}}, \textcolor{output}{\textit{ndataset}}, & \\
~~~~~\textcolor{output}{\textit{ier}}) & \\
call cg\_boco\_read\_f(\textcolor{input}{fn}, \textcolor{input}{B}, \textcolor{input}{Z}, \textcolor{input}{BC}, \textcolor{output}{\textit{pnts}}, \textcolor{output}{\textit{NormalList}}, \textcolor{output}{\textit{ier}}) & r - m \\
\end{fctbox}

\noindent
\textbf{\textcolor{input}{Input}/\textcolor{output}{\textit{Output}}}

\begin{Ventryi}{\texttt{NormalListFlag}}\raggedright
\item [\texttt{fn}]
      CGNS file index number.
      (\textcolor{input}{Input})
\item [\texttt{B}]
      Base index number, where $1 \leq \text{\texttt{B}} \leq \text{\texttt{nbases}}$.
      (\textcolor{input}{Input})
\item [\texttt{Z}]
      Zone index number, where $1 \leq \text{\texttt{Z}} \leq \text{\texttt{nzones}}$.
      (\textcolor{input}{Input})
\item [\texttt{BC}]
      Boundary condition index number, where $1 \leq \text{\texttt{BC}} \leq \text{\texttt{nbocos}}$.
      (\textcolor{input}{Input} for \texttt{cg\_boco\_normal\_write},
      \texttt{cg\_boco\_info}, \texttt{cg\_boco\_read};
      \textcolor{output}{\textit{output}} for \texttt{cg\_boco\_write})
\item [\texttt{nbocos}]
      Number of boundary conditions in zone \texttt{Z}.
      (\textcolor{output}{\textit{Output}})
\item [\texttt{boconame}]
      Name of the boundary condition.
      (\textcolor{input}{Input} for \texttt{cg\_boco\_write};
      \textcolor{output}{\textit{output}} for \texttt{cg\_boco\_info})
\item [\texttt{bocotype}]
      Type of boundary condition defined.
      See the eligible types for \texttt{BCType\_t} in \autoref{s:typedefs}.
      Note that if \texttt{bocotype} is \texttt{FamilySpecified}
      the boundary condition type is being specified for the family
      to which the boundary belongs.
      The boundary condition type for the family may be read and written
      using \texttt{cg\_fambc\_read} and \texttt{cg\_fambc\_write},
      as described in \autoref{s:familybc}.
      (\textcolor{input}{Input} for \texttt{cg\_boco\_write};
      \textcolor{output}{\textit{output}} for \texttt{cg\_boco\_info})
\item [\texttt{ptset\_type}]
      The extent of the boundary condition may be defined using a range
      of points or elements, or using a discrete list of all points or
      elements at which the boundary condition is applied.
      Depending on the method used, the possible values for
      \texttt{ptset\_type} are:

      \setlength{\tmplength}{\extrarowheight}	% temporarily decrease row heights
      \setlength{\extrarowheight}{0pt}
      \begin{tabular}{@{}l >{\ttfamily}l}
         \uline{\texttt{ptset\_type}} & \uline{\textit{Method}} \\
         \texttt{PointRange}   & Range of points or elements \\
         \texttt{PointList}    & List of points or elements \\
         \texttt{ElementRange} & Range of elements \\
         \texttt{ElementRList} & List of elements
      \end{tabular}
      \setlength{\extrarowheight}{\tmplength}

      Note that when \texttt{ptset\_type} is \texttt{ElementRange} or
      \texttt{ElementList}, the \texttt{pnts} values (see below) are assumed
      to be element indices.
      When \texttt{ptset\_type} is \texttt{PointRange} or \texttt{PointList},
      the choice of point or element indices is determined by
      \texttt{GridLocation\_t} under the \texttt{BC\_t} node.
      The value of \texttt{GridLocation\_t} must be read or
      written by first using \texttt{cg\_goto} to access the
      \texttt{BC\_t} node, then using \texttt{cg\_gridlocation\_read} or
      \texttt{cg\_gridlocation\_write}.
      (\textcolor{input}{Input} for \texttt{cg\_boco\_write};
      \textcolor{output}{\textit{output}} for \texttt{cg\_boco\_info})
\item [\texttt{npnts}]
      Number of points or elements defining the boundary
      condition region.
      For a \texttt{ptset\_type} of \texttt{PointRange} or \texttt{ElementRange},
      \texttt{npnts} is always two.
      For a \texttt{ptset\_type} of \texttt{PointList} or \texttt{ElementList},
      \texttt{npnts} is the number of points or elements in the list.
      (\textcolor{input}{Input} for \texttt{cg\_boco\_write};
      \textcolor{output}{\textit{output}} for \texttt{cg\_boco\_info})
\item [\texttt{pnts}]
      Array of point or element indices defining the boundary condition
      region.
      There should be \texttt{npnts} values, each of dimension
      \texttt{IndexDimension} (i.e., 1 for unstructured grids,
      and 2 or 3 for structured grids with 2-D or 3-D elements,
      respectively).
      (\textcolor{input}{Input} for \texttt{cg\_boco\_write};
      \textcolor{output}{\textit{output}} for \texttt{cg\_boco\_read})
\item [\texttt{NormalIndex}]
      Index vector indicating the computational coordinate direction
      of the boundary condition patch normal.
      (\textcolor{input}{Input} for \texttt{cg\_boco\_normal\_write};
      \textcolor{output}{\textit{output}} for \texttt{cg\_boco\_info})
\item [\texttt{NormalListFlag}]
      For \texttt{cg\_boco\_normal\_write}, \texttt{NormalListFlag} is a
      flag indicating if the normals are defined in \texttt{NormalList};
      1 if they are defined, 0 if they're not.

      For \texttt{cg\_boco\_info}, if the normals are defined in
      \texttt{NormalList}, \texttt{NormalListFlag} is the number of points
      in the patch times \texttt{phys\_dim}, the number of coordinates
      required to define a vector in the field.
      If the normals are not defined in \texttt{NormalList},
      \texttt{NormalListFlag} is 0.
      (\textcolor{input}{Input} for \texttt{cg\_boco\_normal\_write};
      \textcolor{output}{\textit{output}} for \texttt{cg\_boco\_info})
\item [\texttt{NormalDataType}]
      Data type used in the definition of the normals.
      Admissible data types for the normals are \texttt{RealSingle} and
      \texttt{RealDouble}.
      (\textcolor{input}{Input} for \texttt{cg\_boco\_normal\_write};
      \textcolor{output}{\textit{output}} for \texttt{cg\_boco\_info})
\item [\texttt{NormalList}]
      List of vectors normal to the boundary condition patch pointing
      into the interior of the zone.
      (\textcolor{input}{Input} for \texttt{cg\_boco\_normal\_write};
      \textcolor{output}{\textit{output}} for \texttt{cg\_boco\_read})
\item [\texttt{ndataset}]
      Number of boundary condition datasets for the current boundary
      condition.
      (\textcolor{output}{\textit{Output}})
\item [\texttt{ier}]
      Error status.
      (\textcolor{output}{\textit{Output}})
\end{Ventryi}

\subsection{Boundary Condition Datasets}
\label{s:bcdataset}

\noindent
\textit{Node}: \texttt{BCDataSet\_t}

\begin{fctbox}
\textcolor{output}{\textit{ier}} = cg\_dataset\_write(\textcolor{input}{int fn}, \textcolor{input}{int B}, \textcolor{input}{int Z}, \textcolor{input}{int BC}, \textcolor{input}{char *DatasetName}, & - w m \\
~~~~~~\textcolor{input}{BCType\_t BCType}, \textcolor{output}{\textit{int *Dset}}); & \\
\textcolor{output}{\textit{ier}} = cg\_dataset\_read(\textcolor{input}{int fn}, \textcolor{input}{int B}, \textcolor{input}{int Z}, \textcolor{input}{int BC}, \textcolor{input}{int Dset}, & r - m \\
~~~~~~\textcolor{output}{\textit{char *DatasetName}}, \textcolor{output}{\textit{BCType\_t *BCType}}, \textcolor{output}{\textit{int *DirichletFlag}}, & \\
~~~~~~\textcolor{output}{\textit{int *NeumannFlag}}); & \\
\hline
call cg\_dataset\_write\_f(\textcolor{input}{fn}, \textcolor{input}{B}, \textcolor{input}{Z}, \textcolor{input}{BC}, \textcolor{input}{DatasetName}, \textcolor{input}{BCType}, \textcolor{output}{\textit{Dset}}, \textcolor{output}{\textit{ier}}) & - w m \\
call cg\_dataset\_read\_f(\textcolor{input}{fn}, \textcolor{input}{B}, \textcolor{input}{Z}, \textcolor{input}{BC}, \textcolor{input}{Dset}, \textcolor{output}{\textit{DatasetName}}, \textcolor{output}{\textit{BCType}}, & r - m \\
~~~~~~\textcolor{output}{\textit{DirichletFlag}}, \textcolor{output}{\textit{NeumannFlag}}, \textcolor{output}{\textit{ier}}) & \\
\end{fctbox}

\noindent
\textbf{\textcolor{input}{Input}/\textcolor{output}{\textit{Output}}}

\begin{Ventryi}{\texttt{DirichletFlag}}\raggedright
\item [\texttt{fn}]
      CGNS file index number.
      (\textcolor{input}{Input})
\item [\texttt{B}]
      Base index number, where $1 \leq \text{\texttt{B}} \leq \text{\texttt{nbases}}$.
      (\textcolor{input}{Input})
\item [\texttt{Z}]
      Zone index number, where $1 \leq \text{\texttt{Z}} \leq \text{\texttt{nzones}}$.
      (\textcolor{input}{Input})
\item [\texttt{BC}]
      Boundary condition index number, where $1 \leq \text{\texttt{BC}} \leq \text{\texttt{nbocos}}$.
      (\textcolor{input}{Input})
\item [\texttt{Dset}]
      Dataset index number, where $1 \leq \text{\texttt{Dset}} \leq \text{\texttt{ndataset}}$.
      (\textcolor{input}{Input} for \texttt{cg\_dataset\_read};
      \textcolor{output}{\textit{output}} for \texttt{cg\_dataset\_write})
\item [\texttt{DatasetName}]
      Name of dataset.
      (\textcolor{input}{Input} for \texttt{cg\_dataset\_write};
      \textcolor{output}{\textit{output}} for \texttt{cg\_dataset\_read})
\item [\texttt{BCType}]
      Simple boundary condition type for the dataset.
      The supported types are listed in the table of ``Simple
      Boundary Condition Types'' in the SIDS manual, but note that
      \texttt{FamilySpecified} does not apply here.
      (\textcolor{input}{Input} for \texttt{cg\_dataset\_write};
      \textcolor{output}{\textit{output}} for \texttt{cg\_dataset\_read})
\item [\texttt{DirichletFlag}]
      Flag indicating if the dataset contains Dirichlet data.
      (\textcolor{output}{\textit{Output}})
\item [\texttt{NeumannFlag}]
      Flag indicating if the dataset contains Neumann data.
      (\textcolor{output}{\textit{Output}})
\item [\texttt{ier}]
      Error status.
      (\textcolor{output}{\textit{Output}})
\end{Ventryi}

The above functions are applicable to \texttt{BCDataSet\_t} nodes
that are children of \texttt{BC\_t} nodes.

For \texttt{BCDataSet\_t} nodes that are children of a \texttt{BC\_t}
node, after accessing a particular \texttt{BCDataSet\_t} node using
\texttt{cg\_goto}, the Point Set functions described in \autoref{s:ptset}
may be used to read or write the locations at which the boundary
conditions are to be applied.
This is only applicable when the boundary conditions are to be applied
at locations different from those used with \texttt{cg\_boco\_write} to
define the boundary condition region (e.g., when the region is being
defined by specification of vertices, but the boundary conditions are to
be applied at face centers).

When writing point set data to a \texttt{BCDataSet\_t} node, in addition
to the specification of the indices using \texttt{cg\_ptset\_write},
the function \texttt{cg\_gridlocation\_write} must also be used to
specify the location of the data with respect to the grid (e.g.,
\texttt{Vertex} or \texttt{FaceCenter}).

\begin{fctbox}
\textcolor{output}{\textit{ier}} = cg\_bcdataset\_write(\textcolor{input}{char *DatasetName}, \textcolor{input}{BCType\_t BCType}, & - w m \\
~~~~~~\textcolor{input}{BCDataType\_t BCDataType}); & \\
\textcolor{output}{\textit{ier}} = cg\_bcdataset\_info(\textcolor{output}{\textit{int *ndataset}}); & \\
\textcolor{output}{\textit{ier}} = cg\_bcdataset\_read(\textcolor{input}{int Dset}, \textcolor{output}{\textit{char *DatasetName}}, \textcolor{output}{\textit{BCType\_t *BCType}}, & r - m \\
~~~~~~\textcolor{output}{\textit{int *DirichletFlag}}, \textcolor{output}{\textit{int *NeumannFlag}}); & \\
\hline
call cg\_bcdataset\_write\_f(\textcolor{input}{DatasetName}, \textcolor{input}{BCType}, \textcolor{input}{BCDataType\_t BCDataType}, \textcolor{output}{\textit{ier}}) & - w m \\
call cg\_bcdataset\_info\_f(\textcolor{output}{\textit{int *ndataset}}, \textcolor{output}{\textit{ier}}) & \\
call cg\_bcdataset\_read\_f(\textcolor{input}{Dset}, \textcolor{output}{\textit{DatasetName}}, \textcolor{output}{\textit{BCType}}, \textcolor{output}{\textit{DirichletFlag}}, & r - m \\
~~~~~~\textcolor{output}{\textit{NeumannFlag}}, \textcolor{output}{\textit{ier}}) & \\
\end{fctbox}

\noindent
\textbf{\textcolor{input}{Input}/\textcolor{output}{\textit{Output}}}

\begin{Ventryi}{\texttt{DirichletFlag}}\raggedright
\item [\texttt{Dset}]
      Dataset index number, where $1 \leq \text{\texttt{Dset}} \leq \text{\texttt{ndataset}}$.
      (\textcolor{input}{Input})
\item [\texttt{DatasetName}]
      Name of dataset.
      (\textcolor{input}{Input} for \texttt{cg\_bcdataset\_write};
      \textcolor{output}{\textit{output}} for \texttt{cg\_bcdataset\_read})
\item [\texttt{BCType}]
      Simple boundary condition type for the dataset.
      The supported types are listed in the table of ``Simple
      Boundary Condition Types'' in the SIDS manual, but note that
      \texttt{FamilySpecified} does not apply here.
      (\textcolor{input}{Input} for \texttt{cg\_bcdataset\_write};
      \textcolor{output}{\textit{output}} for \texttt{cg\_bcdataset\_read})
\item [\texttt{BCDataType}]
      Type of boundary condition in the dataset (i.e., for a
      \texttt{BCData\_t} child node).
      Admissible types are \texttt{Dirichlet} and \texttt{Neumann}.
      (\textcolor{input}{Input})
\item [\texttt{ndataset}]
      Number of \texttt{BCDataSet} nodes under the current
      \texttt{FamilyBC\_t} node.
      (\textcolor{output}{\textit{Output}})
\item [\texttt{DirichletFlag}]
      Flag indicating if the dataset contains Dirichlet data.
      (\textcolor{output}{\textit{Output}})
\item [\texttt{NeumannFlag}]
      Flag indicating if the dataset contains Neumann data.
      (\textcolor{output}{\textit{Output}})
\item [\texttt{ier}]
      Error status.
      (\textcolor{output}{\textit{Output}})
\end{Ventryi}

The above functions are applicable to \texttt{BCDataSet\_t} nodes
that are used to define boundary conditions for a CFD family, and thus
are children of a \texttt{FamilyBC\_t} node.
The \texttt{FamilyBC\_t} node must first be accessed using \texttt{cg\_goto}.

The first time \texttt{cg\_bcdataset\_write} is called with a particular
\texttt{DatasetName}, \texttt{BCType}, and \texttt{BCDataType}, a new
\texttt{BCDataSet\_t} node is created, with a child \texttt{BCData\_t} node.
Subsequent calls with the same \texttt{DatasetName} and \texttt{BCType}
may be made to add additional \texttt{BCData\_t} nodes, of type
\texttt{BCDataType}, to the existing \texttt{BCDataSet\_t} node.

\subsection{Boundary Condition Data}
\label{s:bcdata}

\noindent
\textit{Node}: \texttt{BCData\_t}

\begin{fctbox}
\textcolor{output}{\textit{ier}} = cg\_bcdata\_write(\textcolor{input}{int fn}, \textcolor{input}{int B}, \textcolor{input}{int Z}, \textcolor{input}{int BC}, \textcolor{input}{int Dset}, & - w m \\
~~~~~~\textcolor{input}{BCDataType\_t BCDataType}); & \\
\hline
call cg\_bcdata\_write\_f(\textcolor{input}{fn}, \textcolor{input}{B}, \textcolor{input}{Z}, \textcolor{input}{BC}, \textcolor{input}{Dset}, \textcolor{input}{BCDataType}, \textcolor{output}{\textit{ier}}) & - w m \\
\end{fctbox}

\noindent
\textbf{\textcolor{input}{Input}/\textcolor{output}{\textit{Output}}}

\begin{Ventryi}{\texttt{ptset\_type}}\raggedright
\item [\texttt{fn}]
      CGNS file index number.
      (\textcolor{input}{Input})
\item [\texttt{B}]
      Base index number, where $1 \leq \text{\texttt{B}} \leq \text{\texttt{nbases}}$.
      (\textcolor{input}{Input})
\item [\texttt{Z}]
      Zone index number, where $1 \leq \text{\texttt{Z}} \leq \text{\texttt{nzones}}$.
      (\textcolor{input}{Input})
\item [\texttt{BC}]
      Boundary condition index number, where $1 \leq \text{\texttt{BC}} \leq \text{\texttt{nbocos}}$.
      (\textcolor{input}{Input})
\item [\texttt{Dset}]
      Dataset index number, where $1 \leq \text{\texttt{Dset}} \leq \text{\texttt{ndataset}}$.
      (\textcolor{input}{Input})
\item [\texttt{BCDataType}]
      Type of boundary condition in the dataset.
      Admissible boundary condition types are \texttt{Dirichlet} and
      \texttt{Neumann}.
      (\textcolor{input}{Input})
\item [\texttt{ier}]
      Error status.
      (\textcolor{output}{\textit{Output}})
\end{Ventryi}

To write the boundary condition data itself, after creating the
\texttt{BCData\_t} node using the function \texttt{cg\_bcdata\_write}, use
\texttt{cg\_goto} to access the node, then \texttt{cg\_array\_write} to
write the data.
Note that when using \texttt{cg\_goto} to access a \texttt{BCData\_t}
node, the node index should be specified as either \texttt{Dirichlet} or
\texttt{Neumann}, depending on the type of boundary condition.
See the description of \texttt{cg\_goto} in \autoref{s:navigating} for
details.

\newpage
\subsection{Special Boundary Condition Properties}
\label{s:bcproperty}

\noindent
\textit{Node}: \texttt{BCProperty\_t}

\begin{fctbox}
\textcolor{output}{\textit{ier}} = cg\_bc\_wallfunction\_write(\textcolor{input}{int fn}, \textcolor{input}{int B}, \textcolor{input}{int Z}, \textcolor{input}{int BC}, & - w m \\
~~~~~~\textcolor{input}{WallFunctionType\_t WallFunctionType}); & \\
\textcolor{output}{\textit{ier}} = cg\_bc\_area\_write(\textcolor{input}{int fn}, \textcolor{input}{int B}, \textcolor{input}{int Z}, \textcolor{input}{int BC}, & - w m \\
~~~~~~\textcolor{input}{AreaType\_t AreaType}, \textcolor{input}{float SurfaceArea}, \textcolor{input}{\textit{char *RegionName}}); & \\
\textcolor{output}{\textit{ier}} = cg\_bc\_wallfunction\_read(\textcolor{input}{int fn}, \textcolor{input}{int B}, \textcolor{input}{int Z}, \textcolor{input}{int BC}, & r - m \\
~~~~~~\textcolor{output}{\textit{WallFunctionType\_t *WallFunctionType}}); & \\
\textcolor{output}{\textit{ier}} = cg\_bc\_area\_read(\textcolor{input}{int fn}, \textcolor{input}{int B}, \textcolor{input}{int Z}, \textcolor{input}{int BC}, & r - m \\
~~~~~~\textcolor{output}{\textit{AreaType\_t *AreaType}}, \textcolor{output}{\textit{float *SurfaceArea}}, \textcolor{output}{\textit{char *RegionName}}); & \\
\hline
call cg\_bc\_wallfunction\_write\_f(\textcolor{input}{fn}, \textcolor{input}{B}, \textcolor{input}{Z}, \textcolor{input}{BC}, \textcolor{input}{WallFunctionType}, \textcolor{output}{\textit{ier}}) & - w m \\
call cg\_bc\_area\_write\_f(\textcolor{input}{fn}, \textcolor{input}{B}, \textcolor{input}{Z}, \textcolor{input}{BC}, \textcolor{input}{AreaType}, \textcolor{input}{SurfaceArea}, & - w m \\
~~~~~\textcolor{input}{RegionName}, \textcolor{output}{\textit{ier}}) & \\
call cg\_bc\_wallfunction\_read\_f(\textcolor{input}{fn}, \textcolor{input}{B}, \textcolor{input}{Z}, \textcolor{input}{BC}, \textcolor{output}{\textit{WallFunctionType}}, \textcolor{output}{\textit{ier}}) & r - m \\
call cg\_bc\_area\_read\_f(\textcolor{input}{fn}, \textcolor{input}{B}, \textcolor{input}{Z}, \textcolor{input}{BC}, \textcolor{output}{\textit{AreaType}}, \textcolor{output}{\textit{SurfaceArea}}, & r - m \\
~~~~~\textcolor{output}{\textit{RegionName}}, \textcolor{output}{\textit{ier}}) & \\
\end{fctbox}

\noindent
\textbf{\textcolor{input}{Input}/\textcolor{output}{\textit{Output}}}

\begin{Ventryi}{\texttt{WallFunctionType}}\raggedright
\item [\texttt{fn}]
      CGNS file index number.
      (\textcolor{input}{Input})
\item [\texttt{B}]
      Base index number, where $1 \leq \text{\texttt{B}} \leq \text{\texttt{nbases}}$.
      (\textcolor{input}{Input})
\item [\texttt{Z}]
      Zone index number, where $1 \leq \text{\texttt{Z}} \leq \text{\texttt{nzones}}$.
      (\textcolor{input}{Input})
\item [\texttt{BC}]
      Boundary condition index number, where $1 \leq \text{\texttt{BC}} \leq \text{\texttt{nbocos}}$.
      (\textcolor{input}{Input})
\item [\texttt{WallFunctionType}]
      The wall function type.
      Valid types are \texttt{Null}, \texttt{UserDefined}, and \texttt{Generic}.
      (\textcolor{input}{Input} for \texttt{cg\_bc\_wallfunction\_write};
      \textcolor{output}{\textit{output}} for
      \texttt{cg\_bc\_wallfunction\_read})
\item [\texttt{AreaType}]
      The type of area.
      Valid types are \texttt{Null}, \texttt{UserDefined}, \texttt{BleedArea},
      and \texttt{CaptureArea}.
      (\textcolor{input}{Input} for \texttt{cg\_bc\_area\_write};
      \textcolor{output}{\textit{output}} for \texttt{cg\_bc\_area\_read})
\item [\texttt{SurfaceArea}]
      The size of the area.
      (\textcolor{input}{Input} for \texttt{cg\_bc\_area\_write};
      \textcolor{output}{\textit{output}} for \texttt{cg\_bc\_area\_read})
\item [\texttt{RegionName}]
      The name of the region, 32 characters max.
      (\textcolor{input}{Input} for \texttt{cg\_bc\_area\_write};
      \textcolor{output}{\textit{output}} for \texttt{cg\_bc\_area\_read})
\item [\texttt{ier}]
      Error status.
      (\textcolor{output}{\textit{Output}})
\end{Ventryi}

The ``\texttt{write}'' functions will create the \texttt{BCProperty\_t}
node if it doesn't already exist, then add the appropriate boundary
condition property.
Multiple boundary condition properties may be recorded under the same
\texttt{BCProperty\_t} node.

The ``\texttt{read}'' functions will return with $\text{\texttt{ier}} = 2 =
\text{\texttt{CG\_NODE\_NOT\_FOUND}}$ if the requested boundary condition
property, or the \texttt{BCProperty\_t} node itself, doesn't exist.
