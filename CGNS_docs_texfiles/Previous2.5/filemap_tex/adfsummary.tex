\section{Summary Description of ADF (Advanced Data Format)}
\label{s:adfsummary}
\thispagestyle{plain}

The purpose of the current document is to describe the way in which
CFD data is to be stored in an ADF file.  To do this, it is necessary
to first describe the structure of the ADF file itself in some
detail. Therefore, a conceptual summary of ADF is given here in order
to make the current document relatively independent, and to allow
the reader to focus on those aspects of ADF which are essential to
understanding the file mapping.
The \textit{ADF User's Guide} should be used as the authoritative
reference to resolve any issues not covered by this summary.

\subsection{General Description of ADF}

The ADF, or Advanced Data Format, together with its access software
known as the ADF Core, constitutes a general database manager
particularly suited to the storage of numerical data. Its use is not
restricted to data connected with CFD, and ADF contains no built-in
references to concepts from CFD.

\subsubsection{ADF Files and the ADF Core}

Files created by the ADF Core are referred to as ADF files. These
are binary files whose precise physical form on the external storage
medium is completely controlled by the Core routines. ADF files are not
intelligible or accessible except through the ADF Core routines, and
their physical form is of interest only to ADF Core programmers.

The ADF Core routines perform typical operations on ADF files: open,
close, create, delete, read, write, and so on. They are written in ANSI
C and are thus themselves portable to any platform supporting an ANSI
C compiler. Because the Core completely determines the physical form
of the ADF files, the files themselves can be read on those platforms
as well.\footnote{There are necessarily some issues relating to the
retention of precision on platforms of varying word length.}  In
addition to portability, this arrangement provides integrity of data
across both space and time. In particular, it is never necessary to know
more about an ADF file (other than that it is one) in order to open it
and find out what it contains.

The ADF Core implements the minimal set of procedures required to fully
manipulate the database. The Core itself is written in C, but each Core
call is also provided in Fortran. This enables the user to access ADF
data from either of these languages.

\subsubsection{The Conceptual Structure of ADF Files}

Although the physical structure of an ADF file in storage is (or should
be) of little concern to users of ADF, an understanding of its logical
or conceptual structure is essential. This structure determines its
suitability for the type of data at hand and is reflected in all of the
ADF Core calling sequences.

An ADF file consists entirely of a collection of elements called
nodes. These nodes are arranged in a tree structure which is logically
similar to a UNIX file system. The nodes are said to be connected in a
``child-parent'' relationship according to the following simple rules:

\begin{enumerate}
\item Each node may have any number of child nodes.
\item Except for one node, called the root, each node is the child of
      exactly one other node, called its parent.
\item The root node has no parent.
\end{enumerate}

Every node in an ADF file has exactly the same internal structure. Each
node contains identifying information, pointers to any children, and,
optionally, data.

When an ADF file is opened (by the appropriate Core routine),
information is returned to the calling program which is sufficient to
access the root node. It is then the responsibility of the program to
search the tree for whatever information is required, or to add to the
tree any information it wishes to store.

There is a special kind of node called a link, which serves as a pointer
to a node in another ADF file, or in another part of the same ADF
file. The tree structure at and below the node to which the link points
is available as if that node were present instead of the link. This
allows an ADF file to span multiple physical files, and also allows a
portion of one ADF file to be referenced by several other ADF files.

\subsubsection{The ADF Mid-Level Library (projected)}

The ADF Core routines access the data at a very fundamental level. Since
by definition the Core implements a minimum number of basic functions,
it necessarily deals with the data at a very fundamental level. While
skilled programmers may find this acceptable, most programs define
higher level routines which coalesce oft-repeated sequences of Core
calls. We envision that these routines will eventually be gathered into
an ADF ``Mid-Level'' Library.

At this time, there are approximately four such routines. However, there
has been no coordinated effort to gather, organize, or distribute such a
Library.

(These remarks apply only to routines designed to access general ADF
data. There does exist a set of higher level routines used to access CFD
related data, namely, the CGNS Mid-Level Library.)

\subsection{The Structure of an ADF Node}
\label{s:adfnodestructure}

The File Mapping specifies not only the location of the node at which
a particular kind of data is to be stored, but also how the internal
structure of the node is to be used. Each node contains a number of
fields into which data may be entered directly via ADF Core calls. They
are:

\begin{itemize*}
\item The Node Name
\item The Label
\item The Data Type
\item The Number of Dimensions
\item The Dimension Values
\item The Data
\end{itemize*}
In addition, two other entities associated with the nodes are managed by
ADF itself. These are:
\begin{itemize*}
\item The Node ID
\item The Child Table
\end{itemize*}

\subsubsection{The Node ID}

The node ID is a unique identifier assigned to each existing node by
ADF when the file containing it is opened, and to new nodes as they are
created. ADF Core inquiries generally return node IDs as a result and
accept node IDs as input. By building a table of IDs, calling software
can subsequently access specific nodes without further search. The Node
ID is real and is not under user control.

\subsubsection{The Node Name}

The node Name is a 32-byte character field which is user
controllable. Its general use is to distinguish among the children of a
given node; consequently, no two children of the same parent may have
the same Name.

\subsubsection{The Label}

The Label is a 32-byte character field which is user controllable. ADF
assigns no formal role to the Label, but the intent was to identify
the structure of the included data. It is common for the various
children of a single parent to store different instances of the same
structure. Therefore, there is no prohibition against more than one
child of the same parent having the same Label.

\subsubsection{The Data Type}

The Data Type is a 32-byte character field which specifies the type and
precision of any data which is stored in the data field. Types provided
by ADF are:

\begin{table}[htbp]
\centering
\caption[Data Types]{\textbf{Data Types}}
\label{t:datatype}
\begin{tabular}{l >{\quad}l}
\\ \hline\hline \\*[-2ex]
\bold{Data Type} & \bold{Notation}
\\*[1ex] \hline\hline \\*[-2ex]
No Data             & \fort{MT} \\
Integer 32          & \fort{I4} \\
Integer 64          & \fort{I8} \\
Unsigned Integer 32 & \fort{U4} \\
Unsigned Integer 64 & \fort{U8} \\
Real 32             & \fort{R4} \\
Real 64             & \fort{R8} \\
Complex 64          & \fort{X4} \\
Complex 128         & \fort{X8} \\
Character           & \fort{C1} \\
Byte                & \fort{B1} \\
Link                & \fort{LK}
\\*[1ex] \hline\hline
\end{tabular}
\end{table}

\noindent
If the data type is \fort{MT} or \fort{LK}, the node attributes which
described the data may be left undefined.

\subsubsection{The Number of Dimensions}

The Data portion of a node is designed to store multi-dimensional arrays
of data, each element of which is presumed to be of the Data Type
specified. The Number of Dimensions specifies the number of integers
required to reference a single datum within the array.

\subsubsection{The Dimension Values}

The Dimension Values are a list of integers expressing the actual sizes
of the stored array in each of the dimensions specified.

\subsubsection{The Data}

The portion of the node holding the actual stored data array.

\subsubsection{The Child Table}

ADF maintains a table recording the number of, and pointers to, the
children of each node. The table is adjusted when children are added or
deleted by ADF Core calls.

Children may be identified by their names and labels, and, thence,
by their node IDs once these have been determined. ADF provides no
notion of order among children. In particular, the order of a list of
children returned by ADF has nothing to do with the order in which they
were inserted in the file. However, the order returned is consistent
from call to call provided the file has not been closed and the node
structure has not been modifed.

Note that there is no \emph{parent} table; that is, a node has no direct
knowledge of its parent. Since calling software must open the file
from the root, it presumably cannot access a child without having
first accessed the parent.  It is the responsibility of the calling
software to record the node ID of the parent if this information will be
required.
