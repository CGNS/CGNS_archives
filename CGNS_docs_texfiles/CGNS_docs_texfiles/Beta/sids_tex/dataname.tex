\section{Conventions for Data-Name Identifiers} 
\label{s:dataname}
\thispagestyle{plain}

Identifiers or names can be attached to |DataArray_t| entities to
identify and describe the quantity being stored.  To facilitate
communication between different application codes, we propose to
establish a set of standardized data-name identifiers with fairly
precise definitions.  For any identifier in this set, the associated
data should be unambiguously understood.  In essence, this section
proposes standardized terminology for labeling CFD-related data,
including grid coordinates, flow solution, turbulence model quantities,
nondimensional governing parameters, boundary condition quantities, and
forces and moments.

We use the convention that all standardized identifiers denote
scalar quantities; this is consistent with the intended use of the
|DataArray_t| structure type to describe an array of scalars.  For
quantities that are vectors, such as velocity, their components are
listed.

Included with the lists of standard data-name identifiers, the
fundamental units of dimensions associated with that quantity are
provided.
The following notation is used for the fundamental units: $\M$ is mass,
$\L$ is length, $\T$ is time, $\TH$ is temperature, $\A$ is angle, and
$\I$ is electric current.
These fundamental units are directly associated with the elements of the
\fort{DimensionalExponents\_t} structure.
For example, a quantity that has dimensions $\M\L/\T$ corresponds to
\texttt{MassExponent = +1},
\texttt{LengthExponent = +1}, and
\texttt{TimeExponent = -1}.

Unless otherwise noted, all
quantities in the following sections denote floating-point data
types, and the appropriate |DataType| structure parameter for
|DataArray_t| is |real|.

\subsection{Coordinate Systems}
\label{s:dataname_grid}

Coordinate systems for identifying physical location are as follows:
\begin{center}
\begin{tabular}{l @{\qquad}c @{\qquad}c}
\hline\hline \\*[-2ex]
\bold{System} & \bold{3-D} & \bold{2-D}
\\*[1ex] \hline\hline \\*[-2ex]
Cartesian   & $(x,y,z)$          & $(x,y)$ or $(x,z)$ or $(y,z)$    \\
Cylindrical & $(r,\theta,z)$     & $(r,\theta)$                     \\
Spherical   & $(r,\theta,\phi)$                                     \\
Auxiliary   & $(\xi,\eta,\zeta)$ & $(\xi,\eta)$ or $(\xi,\zeta)$ or
                                   $(\eta,\zeta)$
\\*[1ex] \hline\hline
\end{tabular}
\end{center}
Associated with these coordinate systems are the following unit vector 
conventions:
\begin{center}
\begin{tabular}{l c @{\qquad\qquad}l c @{\qquad\qquad}l c}
$x$-direction & $\i$ & $r$-direction      & $\er$   & $\xi$-direction   & $\exi$ \\
$y$-direction & $\j$ & $\theta$-direction & $\eth$  & $\eta$-direction  & $\eeta$ \\
$z$-direction & $\k$ & $\phi$-direction   & $\ephi$ & $\zeta$-direction & $\ezeta$ \\
\end{tabular}
\end{center}
Note that $\er$, $\eth$ and $\ephi$ are functions of position.

We envision that one of the ``standard'' coordinate systems (cartesian,
cylindrical or spherical) will be used within a zone (or perhaps
the entire database) to define grid coordinates and other related
data.  The auxiliary coordinates will be used for special quantities,
including forces and moments, which may not be defined in the same
coordinate system as the rest of the data.  When auxiliary coordinates
are used, a transformation must also be provided to uniquely define
them.  For example, the transform from cartesian to orthogonal auxiliary
coordinates is,
$$
 \pmatrix{ \exi \cr \eeta \cr \ezeta } =
 {\bf T} \pmatrix{ \i \cr \j \cr \noalign{\smallskip} \k },
$$
where ${\bf T}$ is an orthonormal matrix ($2\cross2$ in 2-D and $3\cross3$
in 3-D).

In addition, normal and tangential coordinates are often used to define
boundary conditions and data related to surfaces.  The normal coordinate
is identified as $n$ with the unit vector $\en$.

The data-name identifiers defined for coordinate systems are listed in
\autoref{t:id_coords}.
All represent real \fort{DataType}s, except for \fort{ElementConnectivity}
and \fort{ParentData}, which are integer.

\begin{table}[htbp]
\centering
\caption[Data-Name Identifiers for Coordinate Systems]{\textbf{Data-Name Identifiers for Coordinate Systems}}
\label{t:id_coords}
\begin{tabular}{>{\ttfamily}l >{\quad}l >{\quad}c}
\\ \hline\hline \\*[-2ex]
\bold{Data-Name Identifier} & \bold{Description} & \bold{Units}
\\*[1ex] \hline\hline \\*[-2ex]
CoordinateX          & $x$                               & $\L$ \\
CoordinateY          & $y$                               & $\L$ \\
CoordinateZ          & $z$                               & $\L$ \\
CoordinateR          & $r$                               & $\L$ \\
CoordinateTheta      & $\theta$                          & $\A$ \\
CoordinatePhi        & $\phi$                            & $\A$ \\
\\
CoordinateNormal     & Coordinate in direction of $\en$  & $\L$ \\
CoordinateTangential & Tangential coordinate (2-D only)  & $\L$ \\ 
\\
CoordinateXi         & $\xi$                             & $\L$ \\
CoordinateEta        & $\eta$                            & $\L$ \\
CoordinateZeta       & $\zeta$                           & $\L$ \\
\\
CoordinateTransform  & Transformation matrix (${\bf T}$) & -    \\
\\
InterpolantsDonor    & Interpolation factors             & -    \\
\\
ElementConnectivity  & Nodes making up an element        & -    \\
ParentData           & Element parent identification     & -
\\*[1ex] \hline\hline
\end{tabular}
\end{table}

\subsection{Flowfield Solution}
\label{s:dataname_flow}

This section describes data-name identifiers for typical Navier-Stokes
solution variables.  The list is obviously incomplete, but should
suffice for initial implementation of the CGNS system.  The variables
listed in this section are dimensional or raw quantities; nondimensional
parameters and coefficients based on these variables are discussed in
\autoref{s:dataname_nondim}.

We use fairly universal notation for state variables.  Static quantities
are measured with the fluid at speed: static density ($\rho$), static
pressure ($p$), static temperature ($T$), static internal energy per
unit mass ($e$), static enthalpy per unit mass ($h$), entropy ($s$),
and static speed of sound ($c$).  We also approximate the true entropy
by the function $\tilde{s} = p/\rho^\gamma$ (this assumes an ideal
gas).  The velocity is $\v{q} = u \i + v \j + w \k$, with magnitude
$q = \sqrt{ \v{q}\dot\v{q} }$. Stagnation quantities are obtained
by bringing the fluid isentropically to rest; these are identified
by a subscript ``${}_0$''.  The term ``total'' is also used to refer to
stagnation quantities.

Conservation variables are density, momentum
($\rho \v{q} = \rho u \i + \rho v \j + \rho w \k$), and stagnation
energy per unit volume ($\rho e_0$).

For rotating coordinate systems, $u$, $v$, and $w$ are the $x$, $y$,
and $z$ components of the velocity vector in the inertial frame;
$\vec{\omega}$ is the rotation rate vector; $\vec{R}$ is a vector from
the center of rotation to the point of interest; and $\vec{w}_r =
\vec{\omega} \times \vec{R}$ is the rotational velocity vector of the
rotating frame of reference, with components $w_{rx}$, $w_{ry}$, and
$w_{rz}$.

Molecular diffusion and heat transfer introduce the molecular viscosity
($\mu$), kinematic viscosity ($\nu$) and thermal conductivity coefficient
($k$).  These are obtained from the state variables through auxiliary
correlations.  For a perfect gas, $\mu$ and $k$ are functions of static
temperature only.

The Navier-Stokes equations involve the strain tensor ($\bar{\bar{S}}$) and
the shear-stress tensor ($\bar{\bar{\tau}}$).  Using indicial notation, the
3-D cartesian components of the strain tensor are,
$$
 \bar{\bar{S}}_{i,j} = \left( \pdf{u_i}{x_j} + \pdf{u_j}{x_i} \right),
$$
and the stress tensor is, 
$$
 \bar{\bar{\tau}}_{i,j} = \mu \left( \pdf{u_i}{x_j} + \pdf{u_j}{x_i} \right)
 + \lambda \pdf{u_k}{x_k},
$$
where $(x_1,x_2,x_3) = (x,y,z)$ and $(u_1,u_2,u_3) = (u,v,w)$.  The bulk
viscosity is usually approximated as $\lambda = -2/3 \mu$.  

Reynolds averaging of the Navier-Stokes equations introduce Reynolds
stresses ($- \rho \overline{u' v'}$, etc.) and turbulent heat flux terms
($- \rho \overline{u' e'}$, etc.), where primed quantities are
instantaneous fluctuations and the bar is an averaging operator.
These quantities are obtained from auxiliary turbulence closure models.
Reynolds-stress models formulate transport equations for the Reynolds
stresses directly; whereas, eddy-viscosity models correlate the Reynolds
stresses with the mean strain rate,
$$
 - \overline{u' v'} = \nu_t \left( \pdf{u}{y} + \pdf{v}{x} \right),
$$
where $\nu_t$ is the kinematic eddy viscosity.
The eddy viscosity is either correlated to mean flow quantities by
algebraic models or by auxiliary transport models.
An example two-equation turbulence transport model is the $k$-$\epsilon$
model, where transport equations are formulated for the turbulent kinetic
energy ($k = {1\over2}(\overline{u'u'} + \overline{v'v'} + \overline{w'w'})$)
and turbulent dissipation ($\epsilon$).

Skin friction evaluated at a surface is the dot product of the shear
stress tensor with the surface normal:
$$
 \v{\tau} = \bar{\bar{\tau}} \dot \h{n},
$$
Note that skin friction is a vector.

The data-name identifiers defined for flow solution quantities are listed
in \autoref{t:id_flow}.

Note that for some vector quantities, like momentum, the table only
explicitly lists data-name identifiers for the $x$, $y$, and $z$
components, and for the magnitude.
It should be understood, however, that for any vector quantity with a
standardized data name ``\fort{Vector}'', the following standardized data
names are also defined:

\begin{Ventryic}{\fort{VectorTangential}}
   \item [\fort{VectorX}]
         $x$-component of vector
   \item [\fort{VectorY}]
         $y$-component of vector
   \item [\fort{VectorZ}]
         $z$-component of vector
   \item [\fort{VectorR}]
         Radial component of vector
   \item [\fort{VectorTheta}]
         $\theta$-component of vector
   \item [\fort{VectorPhi}]
         $\phi$-component of vector
   \item [\fort{VectorMagnitude}]
         Magnitude of vector
   \item [\fort{VectorNormal}]
         Normal component of vector
   \item [\fort{VectorTangential}]
         Tangential component of vector (2-D only)
\end{Ventryic}

Also note that some data-name identifiers used with multi-species flows
include the variable string \textit{Symbol}, which represents either the
chemical symbol for a species, or a defined name for a mixture.
See \autoref{s:ChemicalKineticsModel} for examples, and
\autoref{t:id_chemicalkineticssymbols} on
p.~\pageref*{t:id_chemicalkineticssymbols} for a list of defined names.

\settowidth{\tmplengtha}{\fort{RotatingEnergyStagnationDensity}}
\settowidth{\tmplengthb}{$\L^{3\gamma - 1}/(\M^{\gamma - 1} \T^2)$}
\setlength{\LTleft}{0pt}
\setlength{\LTright}{0pt}
\setlength{\Pwidth}{\linewidth-6\tabcolsep-\tmplengtha-\tmplengthb}
\begin{longtable}{>{\ttfamily}l >{\raggedright\arraybackslash}p{\Pwidth} c}
\caption[Data-Name Identifiers for Flow Solution Quantities]{\textbf{Data-Name Identifiers for Flow Solution Quantities}}
\label{t:id_flow}
\\ \hline\hline \\*[-2ex]
\bold{Data-Name Identifier} & \bold{Description} & \bold{Units}
\\*[1ex] \hline\hline \\*[-2ex]
\endfirsthead

\multicolumn{3}{l}{{\bfseries \autoref{t:id_flow}: Data-Name Identifiers for Flow Solution Quantities} (\emph{Continued})}
\\*[1ex] \hline\hline \\*[-2ex]
\bold{Data-Name Identifier} & \bold{Description} & \bold{Units}
\\*[1ex] \hline\hline \\*[-2ex]
\endhead

\\*[-2ex]\hline
\multicolumn{3}{r}{\emph{Continued on next page}} \\
\endfoot
\\*[-2ex] \hline\hline
\endlastfoot
Potential               & Potential: $\nabla\phi = \v{q}$ &
   $\L^2/\T$ \\
StreamFunction          & Stream function (2-D): $\nabla\cross\psi = \v{q}$ &
   $\L^2/\T$ \\
\\
Density                 & Static density ($\rho$)         &
   $\M/\L^3$ \\
Pressure                & Static pressure ($p$)           &
   $\M/(\L \T^2)$ \\
Temperature             & Static temperature ($T$)        &
   $\TH$ \\
EnergyInternal          & Static internal energy per unit mass ($e$) &
   $\L^2/\T^2$ \\
Enthalpy                & Static enthalpy per unit mass ($h$) &
   $\L^2/\T^2$ \\
Entropy                 & Entropy ($s$) &
   $\M \L^2/(\T^2 \TH)$ \\
EntropyApprox           & Approximate entropy ($\tilde{s} = p/\rho^\gamma$) &
   $\L^{3\gamma - 1}/(\M^{\gamma - 1} \T^2)$ \\
\\
DensityStagnation       & Stagnation density ($\rho_0$) &
   $\M/\L^3$ \\
PressureStagnation      & Stagnation pressure ($p_0$) &
   $\M/(\L \T^2)$ \\
TemperatureStagnation   & Stagnation temperature ($T_0$) &
   $\TH$ \\
EnergyStagnation        & Stagnation energy per unit mass ($e_0$) &
   $\L^2/\T^2$ \\
EnthalpyStagnation      & Stagnation enthalpy per unit mass ($h_0$) &
   $\L^2/\T^2$ \\
EnergyStagnationDensity & Stagnation energy per unit volume ($\rho e_0$) &
   $\M/(\L \T^2)$ \\
\\
VelocityX               & $x$-component of velocity ($u = \v{q}\dot\i$) &
   $\L/\T$ \\
VelocityY               & $y$-component of velocity ($v = \v{q}\dot\j$) &
   $\L/\T$ \\
VelocityZ               & $z$-component of velocity ($w = \v{q}\dot\k$) &
   $\L/\T$ \\
VelocityR               & Radial velocity component ($\v{q}\dot\er$) &
   $\L/\T$ \\
VelocityTheta           & Velocity component in $\theta$ direction ($\v{q}\dot\eth$) &
   $\L/\T$ \\
VelocityPhi             & Velocity component in $\phi$ direction ($\v{q}\dot\ephi$) &
   $\L/\T$ \\
VelocityMagnitude       & Velocity magnitude ($q= \sqrt{ \v{q}\dot\v{q} }$) &
   $\L/\T$ \\
VelocityNormal          & Normal velocity component ($\v{q}\dot\h{n}$) &
   $\L/\T$ \\
VelocityTangential      & Tangential velocity component (2-D) &
   $\L/\T$ \\
VelocitySound           & Static speed of sound &
   $\L/\T$ \\      
VelocitySoundStagnation & Stagnation speed of sound &
   $\L/\T$ \\
\\
MomentumX               & $x$-component of momentum ($\rho u$) &
   $\M/(\L^2\T)$ \\
MomentumY               & $y$-component of momentum ($\rho v$) &
   $\M/(\L^2\T)$ \\
MomentumZ               & $z$-component of momentum ($\rho w$) &
   $\M/(\L^2\T)$ \\
MomentumMagnitude       & Magnitude of momentum ($\rho q$) &
   $\M/(\L^2\T)$ \\
\\
RotatingVelocityX       & $x$-component of velocity, relative to rotating frame ($u_{rx} = u - w_{rx}$) &
   $\L/\T$ \\
RotatingVelocityY       & $y$-component of velocity, relative to rotating frame ($u_{ry} = v - w_{ry}$) &
   $\L/\T$ \\
RotatingVelocityZ       & $z$-component of velocity, relative to rotating frame ($u_{rz} = w - w_{rz}$) &
   $\L/\T$ \\
RotatingMomentumX       & $x$-component of momentum, relative to rotating frame ($\rho u_{rx}$) &
   $\M/(\L^2\T)$ \\
RotatingMomentumY       & $y$-component of momentum, relative to rotating frame ($\rho u_{ry}$) &
   $\M/(\L^2\T)$ \\
RotatingMomentumZ       & $z$-component of momentum, relative to rotating frame ($\rho u_{rz}$) &
   $\M/(\L^2\T)$ \\
RotatingVelocityMagnitude & Velocity magnitude in rotating frame ($q_r = \sqrt{ u_{rx}^2 + u_{ry}^2 + u_{rz}^2 }$) &
   $\L/\T$ \\
RotatingPressureStagnation      & Stagnation pressure in rotating frame &
   $\M/(\L \T^2)$ \\
RotatingEnergyStagnation        & Stagnation energy per unit mass in rotating frame ($(e_0)_r$) &
   $\L^2/\T^2$ \\
RotatingEnergyStagnationDensity & Stagnation energy per unit volume in rotating frame ($\rho (e_0)_r$) &
   $\M/(\L \T^2)$ \\
RotatingEnthalpyStagnation      & Stagnation enthalpy per unit mass in rotating frame, rothalpy &
   $\L^2/\T^2$ \\
\\
EnergyKinetic           & $(u^2 + v^2 + w^2) / 2 = q^2 / 2$ &
   $\L^2/\T^2$ \\
PressureDynamic         & $\rho q^2 / 2$ &
   $\M/(\L \T^2)$ \\
\\
SoundIntensityDB        & Sound intensity level in decibels,
                          $10 \log_{10} (I/I_{\rm ref}) =
                          20 \log_{10} (p/p_{\rm ref})$, where
                          $I$ is the sound power per unit area,
                          $I_{\rm ref} = 10^{-12}$ watts/m\tsup{2}
                          is the reference sound power per unit area,
                          $p$ is the pressure wave amplitude, and
                          $p_{\rm ref} = 2 \times 10^{-5}$ N/m\tsup{2}
                          is the reference pressure. &
   - \\
SoundIntensity          & Sound intensity (i.e., sound power per unit area, $I$) &
   $\M/\T^3$ \\
\\
VorticityX              & $\omega_x = \pd{w}{y} - \pd{v}{z} = \v{\omega}\dot\i$ &
   $\T^{-1}$ \\
VorticityY              & $\omega_y = \pd{u}{z} - \pd{w}{x} = \v{\omega}\dot\j$ &
   $\T^{-1}$ \\
VorticityZ              & $\omega_z = \pd{v}{x} - \pd{u}{y} = \v{\omega}\dot\k$ &
   $\T^{-1}$ \\
VorticityMagnitude      & $\omega = \sqrt{ \v{\omega}\dot\v{\omega} }$ &
   $\T^{-1}$ \\
\\
SkinFrictionX           & $x$-component of skin friction ($\v{\tau} \cdot \i$) &
   $\M/(\L \T^2)$ \\
SkinFrictionY           & $y$-component of skin friction ($\v{\tau} \cdot \j$) &
   $\M/(\L \T^2)$ \\
SkinFrictionZ           & $z$-component of skin friction ($\v{\tau} \cdot \k$) &
   $\M/(\L \T^2)$ \\
SkinFrictionMagnitude   & Skin friction magnitude ($\sqrt{ \v{\tau}\dot\v{\tau} }$) &
   $\M/(\L \T^2)$ \\
\\
VelocityAngleX          & Velocity angle ($\arccos(u/q) \in [0,\,180^\circ)$) &
   $\A$ \\
VelocityAngleY          & $\arccos(v/q)$ &
   $\A$ \\
VelocityAngleZ          & $\arccos(w/q)$ &
   $\A$ \\
\\
VelocityUnitVectorX     & $x$-component of velocity unit vector ($\v{q}\dot\i / q$) &
   - \\
VelocityUnitVectorY     & $y$-component of velocity unit vector ($\v{q}\dot\j / q$) &
   - \\
VelocityUnitVectorZ     & $z$-component of velocity unit vector ($\v{q}\dot\k / q$) &
   - \\
\\
MassFlow                & Mass flow normal to a plane ($\rho \v{q}\dot\h{n}$) &
   $\M/(\L^2 \T)$ \\
\\
ViscosityKinematic      & Kinematic viscosity ($\nu = \mu / \rho$) &
   $\L^2/\T$ \\
ViscosityMolecular      & Molecular viscosity ($\mu$) &
   $\M/(\L \T)$ \\
ViscosityEddyKinematic  & Kinematic eddy viscosity ($\nu_t$) &
   $\L^2/\T$ \\
ViscosityEddy           & Eddy viscosity ($\mu_t$) &
   $\M/(\L \T)$ \\
ThermalConductivity     & Thermal conductivity coefficient ($k$) &
   $\M \L/(\T^3 \TH)$ \\
\\
PowerLawExponent        & Power-law exponent ($n$) in molecular viscosity
                          or thermal conductivity model &
   - \\
SutherlandLawConstant   & Sutherland's Law constant ($T_s$) in molecular
                          viscosity or thermal conductivity model &
   $\TH$ \\
TemperatureReference    & Reference temperature ($T_{\rm ref}$) in molecular
                          viscosity or thermal conductivity model &
   $\TH$ \\
ViscosityMolecularReference  & Reference viscosity ($\mu_{\rm ref}$) in
                               molecular viscosity model &
   $\M/(\L \T)$ \\
ThermalConductivityReference & Reference thermal conductivity ($k_{\rm ref}$)
                               in thermal conductivity model &
   $\M \L/(\T^3 \TH)$ \\
\\
IdealGasConstant        & Ideal gas constant ($R = c_p - c_v$) &
   $\L^2/(\T^2 \TH)$ \\
SpecificHeatPressure    & Specific heat at constant pressure ($c_p$) &
   $\L^2/(\T^2 \TH)$ \\
SpecificHeatVolume      & Specific heat at constant volume ($c_v$) &
   $\L^2/(\T^2 \TH)$ \\
\\
ReynoldsStressXX        & Reynolds stress $-\rho \overline{u' u'}$ &
   $\M/(\L \T^2)$ \\
ReynoldsStressXY        & Reynolds stress $-\rho \overline{u' v'}$ &
   $\M/(\L \T^2)$ \\
ReynoldsStressXZ        & Reynolds stress $-\rho \overline{u' w'}$ &
   $\M/(\L \T^2)$ \\
ReynoldsStressYY        & Reynolds stress $-\rho \overline{v' v'}$ &
   $\M/(\L \T^2)$ \\
ReynoldsStressYZ        & Reynolds stress $-\rho \overline{v' w'}$ &
   $\M/(\L \T^2)$ \\
ReynoldsStressZZ        & Reynolds stress $-\rho \overline{w' w'}$ &
   $\M/(\L \T^2)$ \\
\\
MolecularWeight\textit{Symbol}     & Molecular weight for species
                                     \textit{Symbol} &
   - \\
HeatOfFormation\textit{Symbol}     & Heat of formation per unit mass for species
                                     \textit{Symbol} &
   $\L^2/\T^2$ \\
FuelAirRatio                       & Fuel/air mass ratio &
   - \\
ReferenceTemperatureHOF            & Reference temperature for the heat
                                     of formation &
   $\TH$ \\
MassFraction\textit{Symbol}        & Mass of species \textit{Symbol},
                                     divided by total mass &
   - \\
LaminarViscosity\textit{Symbol}    & Laminar viscosity of species
                                     \textit{Symbol} &
   $\M/(\L \T)$ \\
ThermalConductivity\textit{Symbol} & Thermal conductivity of species
                                     \textit{Symbol} &
   $\M \L/(\T^3 \TH)$ \\
EnthalpyEnergyRatio                & The ratio $\beta = h/e =
                                     \int_{T_{\it ref}}^T c_p\,dT / \int_{T_{\it ref}}^T c_v\,dT$ &
   - \\
CompressibilityFactor              & The gas constant of the mixture divided
                                     by the freestream gas constant,
                                     $Z = R / R_\infty$ &
   - \\
VibrationalElectronEnergy          & Vibrational-electronic excitation
                                     energy per unit mass &
   $\L^2/\T^2$ \\
HeatOfFormation                    & Heat of formation per unit mass
                                     for the entire mixture,
                                     $H = \sum_{i=1}^n Y_i H_i$, where
                                     $n$ is the number of species,
                                     $Y_i$ is the mass fraction of
                                     species $i$, and $H_i$ is the
                                     heat of formation for species
                                     $i$ at the reference temperature
                                     \fort{ReferenceTemperatureHOF}.
                                     This requires that
                                     \fort{ReferenceTemperatureHOF}
                                     be specified using the
                                     \fort{ChemicalKineticsModel} data
                                     structure. &
   $\L^2/\T^2$ \\
VibrationalElectronTemperature     & Vibrational electron temperature &
   $\TH$ \\
SpeciesDensity\textit{Symbol}      & Density of species \textit{Symbol} &
   $\M/\L^3$ \\
MoleFraction\textit{Symbol}        & Number of moles of species
                                     \textit{Symbol} divided by the total
                                     number of moles for all species &
   - \\
\\
Voltage              & Voltage                                 &
   $\M \L^2/\T \I$ \\
ElectricFieldX       & $x$-component of electric field vector  &
   $\M \L/\T \I$ \\
ElectricFieldY       & $y$-component of electric field vector  &
   $\M \L/\T \I$ \\
ElectricFieldZ       & $z$-component of electric field vector  &
   $\M \L/\T \I$ \\
MagneticFieldX       & $x$-component of magnetic field vector  &
   $\I/\L$ \\
MagneticFieldY       & $y$-component of magnetic field vector  &
   $\I/\L$ \\
MagneticFieldZ       & $z$-component of magnetic field vector  &
   $\I/\L$ \\
CurrentDensityX      & $x$-component of current density vector &
   $\I/\L^2$ \\
CurrentDensityY      & $y$-component of current density vector &
   $\I/\L^2$ \\
CurrentDensityZ      & $z$-component of current density vector &
   $\I/\L^2$ \\
ElectricConductivity & Electrical conductivity                 &
   $\M \L/\T^3 \I^2$ \\
LorentzForceX        & $x$-component of Lorentz force vector   &
   $\M \L/\T^2$ \\
LorentzForceY        & $y$-component of Lorentz force vector   &
   $\M \L/\T^2$ \\
LorentzForceZ        & $z$-component of Lorentz force vector   &
   $\M \L/\T^2 $ \\
JouleHeating         & Joule heating                           &
   $\M \L^2/\T^2$ \\
\\
LengthReference         & Reference length $L$ &
   $\L$
\end{longtable}

\newpage
\subsection{Turbulence Model Solution}

This section lists data-name identifiers for typical Reynolds-averaged
Navier-Stokes turbulence model variables.
Turbulence model solution quantities and model constants present a
particularly difficult nomenclature problem---to be precise we need to
identify both the variable and the model (and version) that it comes
from.
The list in \autoref{t:id_turbulence} falls short in this respect.

\begin{table}[htbp]
\centering
\caption[Data-Name Identifiers for Typical Turbulence Models]{\textbf{Data-Name Identifiers for Typical Turbulence Models}}
\label{t:id_turbulence}
\begin{tabular}{>{\ttfamily}l >{\quad}l >{\quad}c}
\\ \hline\hline \\*[-2ex]
\bold{Data-Name Identifier} & \bold{Description} & \bold{Units}
\\*[1ex] \hline\hline \\*[-2ex]
TurbulentDistance        & Distance to nearest wall &
   $\L$ \\
\\
TurbulentEnergyKinetic   & $k = {1\over2}(\overline{u'u'} + \overline{v'v'} + \overline{w'w'})$ &
   $\L^2/\T^2$ \\
TurbulentDissipation     & $\epsilon$ &
   $\L^2/\T^3$ \\
TurbulentDissipationRate & $\epsilon / k = \omega$ &
   $\T^{-1}$ \\
\\
TurbulentBBReynolds      & Baldwin-Barth one-equation model $R_T$ &
   - \\
TurbulentSANuTilde       & Spalart-Allmaras one-equation model $\tilde{\nu}$ &
   $\L^2/\T$
\\*[1ex] \hline\hline
\end{tabular}
\end{table}

\subsection{Nondimensional Parameters}
\label{s:dataname_nondim}

CFD codes are rich in nondimensional governing parameters, such as
Mach number and Reynolds number, and nondimensional flowfield coefficients,
such as pressure coefficient.  The problem with these parameters is
that their definitions and conditions that they are evaluated at can
vary from code to code.  Reynolds number is particularly notorious in
this respect.

These parameters have posed us with a difficult dilemma.  Either we
impose a rigid definition for each and force all database users to
abide by it, or we develop some methodology for describing the
particular definition that the user is employing.  The first limits
applicability and flexibility, and the second adds complexity.  We have
opted for the second approach, but we include only enough information
about the definition of each parameter to allow for conversion
operations.  For example, the Reynolds number includes velocity, length,
and kinematic viscosity scales in its definition (i.e., $Re = V L_R / \nu$).
The database description of Reynolds number includes these different
scales.  By providing these ``definition components'', any code that reads
Reynolds number from the database can transform its value to an
appropriate internal definition.  These definition components are
identified by appending a ``|_|'' to the data-name identifier of the parameter.

Definitions for nondimensional flowfield coefficients follow: the
pressure coefficient is defined as,
$$
 c_p = {p - p_\refer \over {1\over2} \rho_\refer q_\refer^2},
$$
where ${1\over2} \rho_\refer q_\refer^2$ is the dynamic pressure evaluated at
some reference condition, and $p_\refer$ is some reference pressure.  The 
skin friction coefficient is,
$$
 \v{c}_f = {\v{\tau} \over {1\over2} \rho_\refer q_\refer^2},
$$
where $\v{\tau}$ is the shear stress or skin friction vector.  Usually, 
$\v{\tau}$ is evaluated at the wall surface.

The data-name identifiers defined for nondimensional governing
parameters and flowfield coefficients are listed in
\autoref{t:id_nondimensional}.

\settowidth{\tmplengtha}{\fort{Prandtl\_SpecificHeatPressure}}
\settowidth{\tmplengthb}{$\M \L/(\T^3 \TH)$}
\setlength{\LTleft}{0pt}
\setlength{\LTright}{0pt}
\setlength{\Pwidth}{\linewidth-6\tabcolsep-\tmplengtha-\tmplengthb}
\begin{longtable}{>{\ttfamily}l >{\raggedright\arraybackslash}p{\Pwidth} c}
\caption[Data-Name Identifiers for Nondimensional Parameters]{\textbf{Data-Name Identifiers for Nondimensional Parameters}}
\label{t:id_nondimensional}
\\ \hline\hline \\*[-2ex]
\bold{Data-Name Identifier} & \bold{Description} & \bold{Units}
\\*[1ex] \hline\hline \\*[-2ex]
\endfirsthead

\multicolumn{3}{l}{{\bfseries \autoref{t:id_nondimensional}: Data-Name Identifiers for Nondimensional Parameters} (\emph{Continued})}
\\*[1ex] \hline\hline \\*[-2ex]
\bold{Data-Name Identifier} & \bold{Description} & \bold{Units}
\\*[1ex] \hline\hline \\*[-2ex]
\endhead

\\*[-2ex]\hline
\multicolumn{3}{r}{\emph{Continued on next page}} \\
\endfoot
\\*[-2ex] \hline\hline
\endlastfoot
Mach                          & Mach number: $M = q/c$ &
   - \\
Mach\_Velocity                & Velocity scale ($q$) &
   $\L/\T$ \\
Mach\_VelocitySound           & Speed of sound scale ($c$) &
   $\L/\T$ \\
RotatingMach                  & Mach number relative to rotating frame: $M_r = q_r / c$ &
   - \\
\\
Reynolds                      & Reynolds number: $Re = V L_R / \nu$ &
   - \\
Reynolds\_Velocity            & Velocity scale ($V$) &
   $\L/\T$ \\
Reynolds\_Length              & Length scale ($L_R$) &
   $\L$ \\
Reynolds\_ViscosityKinematic  & Kinematic viscosity scale ($\nu$) &
   $\L^2/\T$ \\
\\
Prandtl                       & Prandtl number: $Pr = \mu c_p / k$ &
   - \\
Prandtl\_ThermalConductivity  & Thermal conductivity scale ($k$) &
   $\M \L/(\T^3 \TH)$ \\
Prandtl\_ViscosityMolecular   & Molecular viscosity scale ($\mu$) &
   $\M/(\L \T)$ \\
Prandtl\_SpecificHeatPressure & Specific heat scale ($c_p$) &
   $\L^2/(\T^2 \TH)$ \\
PrandtlTurbulent              & Turbulent Prandtl number, $\rho \nu_t c_p / k_t$ &
   - \\
\\
SpecificHeatRatio             & Specific heat ratio: $\gamma = c_p / c_v$ &
   - \\
SpecificHeatRatio\_Pressure   & Specific heat at constant pressure ($c_p$) &
   $\L^2/(\T^2 \TH)$ \\
SpecificHeatRatio\_Volume     & Specific heat at constant volume ($c_v$) &
   $\L^2/(\T^2 \TH)$ \\
\\
CoefPressure                  & $c_p$ &
   - \\
CoefSkinFrictionX             & $\v{c}_f \dot \i$ &
   - \\
CoefSkinFrictionY             & $\v{c}_f \dot \j$ &
   - \\
CoefSkinFrictionZ             & $\v{c}_f \dot \k$ &
   - \\
\\
Coef\_PressureDynamic         & $\rho_\refer q_\refer^2 / 2$ &
   $\M/(\L \T^2)$ \\
Coef\_PressureReference       & $p_\refer$ &
   $\M/(\L \T^2)$
\end{longtable}

\subsection{Characteristics and Riemann Invariants Based on 1-D Flow}
\label{s:dataname_char}

Boundary condition specification for inflow/outflow or farfield
boundaries often involves Riemann invariants or characteristics of
the linearized inviscid flow equations.  For an ideal compressible
gas, these are typically defined as follows: Riemann invariants for an
isentropic 1-D flow are,
$$
 \left[ \pdf{}{t} + (u \pm c) \pdf{}{x} \right] 
 \left( u \pm {2\over{\gamma - 1}} c \right) = 0.
$$
Characteristic variables for the 3-D Euler equations linearized about a
constant mean flow are,
$$ 
 \left[ \pdf{}{t} + \bar{\Lambda}_n \pdf{}{x} \right] W'_n(x,t) = 0,  
 \qquad n = 1,2,\ldots 5,
$$
where the characteristics and corresponding characteristic variables are
\begin{center}
\begin{tabular}{l >{\quad}c >{\quad}c}
\hline\hline \\*[-2ex]
\bold{Characteristic} & {\boldmath $\bar{\Lambda}_n$} & {\boldmath $W'_n$}
\\*[1ex] \hline\hline \\*[-2ex]
Entropy   & $\bar{u}$             & $p' - \rho'/\bar{c}^2$ \\
Vorticity & $\bar{u}$             & $v'$  \\
Vorticity & $\bar{u}$             & $w'$  \\
Acoustic  & $\bar{u} \pm \bar{c}$ & $p' \pm u'/(\bar{\rho} \bar{c})$
\\*[1ex] \hline\hline
\end{tabular}
\end{center}
Barred quantities are evaluated at the mean flow, and primed quantities
are linearized perturbations.  The only non-zero mean-flow velocity
component is $\bar{u}$.  The data-name identifiers defined for
Riemann invariants and characteristic variables are listed in
\autoref{t:id_chars}.

\begin{table}[htbp]
\centering
\caption[Data-Name Identifiers for Characteristics and Riemann Invariants]{\textbf{Data-Name Identifiers for Characteristics and Riemann Invariants}}
\label{t:id_chars}
\begin{tabular}{>{\ttfamily}l >{\quad}l >{\quad}c}
\\ \hline\hline \\*[-2ex]
\bold{Data-Name Identifier} & \bold{Description} & \bold{Units}
\\*[1ex] \hline\hline \\*[-2ex]
RiemannInvariantPlus        & $u + 2 c/(\gamma - 1)$         & $\L/\T$ \\
RiemannInvariantMinus       & $u - 2 c/(\gamma - 1)$         & $\L/\T$ \\
\\
CharacteristicEntropy       & $p' - \rho'/\bar{c}^2$         & $\M/(\L \T^2)$ \\
CharacteristicVorticity1    & $v'$                           & $\L/\T$ \\
CharacteristicVorticity2    & $w'$                           & $\L/\T$ \\ 
CharacteristicAcousticPlus  & $p' + u'/(\bar{\rho} \bar{c})$ & $\M/(\L \T^2)$ \\
CharacteristicAcousticMinus & $p' - u'/(\bar{\rho} \bar{c})$ & $\M/(\L \T^2)$
\\*[1ex] \hline\hline
\end{tabular}
\end{table}

\subsection{Forces and Moments}

Conventions for data-name identifiers for forces and moments are defined
in this section.
Ideally, forces and moments should be attached to geometric components
or less ideally to surface grids.
Currently, the standard mechanism for storing forces and moments is
generally through the \fort{ConvergenceHistory\_t} node described in
\autoref{s:ConvergenceHistory}, either attached to the entire
configuration (under \fort{CGNSBase\_t}, \autoref{s:CGNSBase}) or
attached to a zone (under \fort{Zone\_t}, \autoref{s:Zone}).

Given a differential force $\v{f}$ (i.e., a force per unit area), the force 
integrated over a surface is,
$$
 \v{F} = F_x \i + F_y \j + F_z \k = \int \v{f} \,dA,
$$
where $\i$, $\j$ and $\k$ are the unit vectors in the $x$, $y$ and $z$ 
directions, respectively.  The moment about a point $\v{r}_0$ integrated 
over a surface is,
$$
 \v{M} = M_x \i + M_y \j + M_z \k 
       = \int (\v{r} - \v{r}_0) \times \v{f} \,dA.
$$
Lift and drag components of the integrated force are,
$$
 L = \v{F} \cdot \h{L}  \qquad  D = \v{F} \cdot \h{D} 
$$
where $\h{L}$ and $\h{D}$ are the unit vectors in the positive lift and 
drag directions, respectively.  

Lift, drag and moment are often computed in auxiliary coordinate frames
(e.g., wind axes or stability axes).  We introduce the convention that
lift, drag and moment are computed in the $(\xi,\eta,\zeta)$ coordinate
system.  Positive drag is assumed parallel to the $\xi$-direction (i.e.,
$\h{D} = \exi$); and positive lift is assumed parallel to the
$\eta$-direction (i.e., $\h{L} = \eeta$).  Thus, forces and
moments defined in this auxiliary coordinate system are,
$$
 L = \v{F} \cdot \eeta  \qquad  D = \v{F} \cdot \exi 
$$
$$
 \v{M} = M_\xi \exi + M_\eta \eeta + M_\zeta \ezeta 
       = \int (\v{r} - \v{r}_0) \times \v{f} \,dA.
$$

Lift, drag and moment coefficients in 3-D are defined as,
$$
 C_L = {L \over {1\over2} \rho_\refer q_\refer^2 S_\refer}  \qquad
 C_D = {D \over {1\over2} \rho_\refer q_\refer^2 S_\refer}  \qquad
 \v{C}_M = {\v{M} \over {1\over2} \rho_\refer q_\refer^2 c_\refer S_\refer}, 
$$
where ${1\over2} \rho_\refer q_\refer^2$ is a reference dynamic pressure,
$S_\refer$ is a reference area, and $c_\refer$ is a reference length.
For a wing, $S_\refer$ is typically the wing area and $c_\refer$ is the
mean aerodynamic chord.
In 2-D, the sectional force coefficients are,
$$
 c_l = {L' \over {1\over2} \rho_\refer q_\refer^2 c_\refer}  \qquad
 c_d = {D' \over {1\over2} \rho_\refer q_\refer^2 c_\refer}  \qquad
 \v{c}_m = {\v{M}' \over {1\over2} \rho_\refer q_\refer^2 c_\refer^2}, 
$$
where the forces are integrated along a contour (e.g., an airfoil
cross-section) rather than a surface.

The data-name identifiers and definitions provided for forces
and moments and their associated coefficients are listed in
\autoref{t:id_forces}.
For coefficients, the dynamic pressure and length scales used in the
normalization are provided.

\setlength{\LTleft}{\fill}
\setlength{\LTright}{\fill}
\begin{longtable}{>{\ttfamily}l >{\quad}l >{\quad}c}
\caption[Data-Name Identifiers for Forces and Moments]{\textbf{Data-Name Identifiers for Forces and Moments}}
\label{t:id_forces}
\\ \hline\hline \\*[-2ex]
\bold{Data-Name Identifier} & \bold{Description} & \bold{Units}
\\*[1ex] \hline\hline \\*[-2ex]
\endfirsthead

\multicolumn{3}{l}{{\bfseries \autoref{t:id_forces}: Data-Name Identifiers for Forces and Moments} (\emph{Continued})}
\\*[1ex] \hline\hline \\*[-2ex]
\bold{Data-Name Identifier} & \bold{Description} & \bold{Units}
\\*[1ex] \hline\hline \\*[-2ex]
\endhead

\\*[-2ex]\hline
\multicolumn{3}{r}{\emph{Continued on next page}} \\
\endfoot
\\*[-2ex] \hline\hline
\endlastfoot
ForceX                & $F_x = \v{F} \dot \i$        & $\M \L/\T^2$ \\
ForceY                & $F_y = \v{F} \dot \j$        & $\M \L/\T^2$ \\
ForceZ                & $F_z = \v{F} \dot \k$        & $\M \L/\T^2$ \\
ForceR                & $F_r = \v{F} \dot \er$       & $\M \L/\T^2$ \\
ForceTheta            & $F_\theta = \v{F} \dot \eth$ & $\M \L/\T^2$ \\
ForcePhi              & $F_\phi = \v{F} \dot \ephi$  & $\M \L/\T^2$ \\
\\
Lift                  & $L$ or $L'$                  & $\M \L/\T^2$ \\
Drag                  & $D$ or $D'$                  & $\M \L/\T^2$ \\
MomentX               & $M_x = \v{M} \dot \i$         & $\M \L^2/\T^2$ \\
MomentY               & $M_y = \v{M} \dot \j$         & $\M \L^2/\T^2$ \\
MomentZ               & $M_z = \v{M} \dot \k$         & $\M \L^2/\T^2$ \\
MomentR               & $M_r = \v{M} \dot \er$        & $\M \L^2/\T^2$ \\
MomentTheta           & $M_\theta = \v{M} \dot \eth$  & $\M \L^2/\T^2$ \\
MomentPhi             & $M_\phi = \v{M} \dot \ephi$   & $\M \L^2/\T^2$ \\
MomentXi              & $M_\xi = \v{M} \dot \exi$     & $\M \L^2/\T^2$ \\
MomentEta             & $M_\eta = \v{M} \dot \eeta$   & $\M \L^2/\T^2$ \\
MomentZeta            & $M_\zeta = \v{M} \dot \ezeta$ & $\M \L^2/\T^2$ \\
\\
Moment\_CenterX       & $x_0 = \v{r}_0 \dot \i$       & $\L$ \\
Moment\_CenterY       & $y_0 = \v{r}_0 \dot \j$       & $\L$ \\
Moment\_CenterZ       & $z_0 = \v{r}_0 \dot \k$       & $\L$ \\
\\
CoefLift              & $C_L$ or $c_l$                                 & - \\
CoefDrag              & $C_D$ or $c_d$                                 & - \\
CoefMomentX           & $\v{C}_M \dot \i$ or $\v{c}_m \dot \i$         & - \\
CoefMomentY           & $\v{C}_M \dot \j$ or $\v{c}_m \dot \j$         & - \\
CoefMomentZ           & $\v{C}_M \dot \k$ or $\v{c}_m \dot \k$         & - \\
CoefMomentR           & $\v{C}_M \dot \er$ or $\v{c}_m \dot \er$       & - \\
CoefMomentTheta       & $\v{C}_M \dot \eth$ or $\v{c}_m \dot \eth$     & - \\
CoefMomentPhi         & $\v{C}_M \dot \ephi$ or $\v{c}_m \dot \ephi$   & - \\
CoefMomentXi          & $\v{C}_M \dot \exi$ or $\v{c}_m \dot \exi$     & - \\
CoefMomentEta         & $\v{C}_M \dot \eeta$ or $\v{c}_m \dot \eeta$   & - \\
CoefMomentZeta        & $\v{C}_M \dot \ezeta$ or $\v{c}_m \dot \ezeta$ & - \\
\\
Coef\_PressureDynamic & ${1/2} \rho_\refer q_\refer^2$ & $\M/(\L \T^2)$ \\
Coef\_Area            & $S_\refer$                     & $\L^2$ \\
Coef\_Length          & $c_\refer$                     & $\L$
\end{longtable}

\newpage
\subsection{Time-Dependent Flow}

Data-name identifiers related to time-dependent flow include those
associated with the storage of grid coordinates and flow solutions
as a function of time level or iteration.
Also included are identifiers for storing information defining
both rigid and arbitrary (i.e., deforming) grid motion.

%\setlength{\LTleft}{\fill}
%\setlength{\LTright}{\fill}
\settowidth{\tmplengtha}{\fort{ArbitraryGridMotionPointers}}
\settowidth{\tmplengthb}{\bold{Type}}
\settowidth{\tmplengthc}{\bold{Units}}
\setlength{\Pwidth}{\linewidth-8\tabcolsep-\tmplengtha-\tmplengthb-\tmplengthc}
\begin{longtable}{>{\ttfamily}l >{\ttfamily}l >{\raggedright\arraybackslash}p{\Pwidth} c}
\caption[Data-Name Identifiers for Time-Dependent Flow]{\textbf{Data-Name Identifiers for Time-Dependent Flow}}
\label{t:id_timedep}
\\ \hline\hline \\*[-2ex]
                                      & \bold{Data} & & \\
\spantwo{\bold{Data-Name Identifier}} & \bold{Type} & \spantwo{\bold{Description}} & \spantwo{\bold{Units}}
\\*[1ex] \hline\hline \\*[-2ex]
\endfirsthead

\multicolumn{4}{l}{{\bfseries \autoref{t:id_timedep}: Data-Name Identifiers for Time-Dependent Flow} (\emph{Continued})}
\\*[1ex] \hline\hline \\*[-2ex]
                                      & \bold{Data} & & \\
\spantwo{\bold{Data-Name Identifier}} & \bold{Type} & \spantwo{\bold{Description}} & \spantwo{\bold{Units}}
\\*[1ex] \hline\hline \\*[-2ex]
\endhead

\\*[-2ex]\hline
\multicolumn{4}{r}{\emph{Continued on next page}} \\
\endfoot
\\*[-2ex] \hline\hline
\endlastfoot
TimeValues       & real & Time values & $\T$ \\
IterationValues  & int  & Iteration values & - \\
NumberOfZones    & int  & Number of zones used for each recorded step & - \\
NumberOfFamilies & int  & Number of families used for each recorded step & - \\
ZonePointers     & char & Names of zones used for each recorded step & - \\
FamilyPointers   & char & Names of families used for each recorded step & - \\
\\
RigidGridMotionPointers     & char &
   Names of \fort{RigidGridMotion} structures used for each recorded
      step for a zone       & - \\
ArbitraryGridMotionPointers & char &
   Names of \fort{ArbitraryGridMotion} structures used for each recorded
      step for a zone       & - \\
GridCoordinatesPointers     & char &
   Names of \fort{GridCoordinates} structures used for each recorded step
      for a zone            & - \\
FlowSolutionPointers        & char &
   Names of \fort{FlowSolution} structures used for each recorded step
      for a zone            & - \\
\\
OriginLocation              & real &
   Physical coordinates of the origin before and after a rigid grid
      motion                & $\L$ \\
RigidRotationAngle & real   &
   Rotation angles about each axis of the translated coordinate system
      for rigid grid motion & $\A$ \\
RigidVelocity               & real &
   Grid velocity vector of the origin translation for rigid grid
      motion                & $\L/\T$ \\
RigidRotationRate           & real &
   Rotation rate vector about the axis of the translated coordinate system
      for rigid grid motion & $\A/\T$ \\
\\
GridVelocityX     & real & $x$-component of grid velocity      &
   $\L/\T$ \\
GridVelocityY     & real & $y$-component of grid velocity      &
   $\L/\T$ \\
GridVelocityZ     & real & $z$-component of grid velocity      &
   $\L/\T$ \\
GridVelocityR     & real & $r$-component of grid velocity      &
   $\L/\T$ \\
GridVelocityTheta & real & $\theta$-component of grid velocity &
   $\A/\T$ \\
GridVelocityPhi   & real & $\phi$-component of grid velocity   &
   $\A/\T$ \\
GridVelocityXi    & real & $\xi$-component of grid velocity    &
   $\L/\T$ \\
GridVelocityEta   & real & $\eta$-component of grid velocity   &
   $\L/\T$ \\
GridVelocityZeta  & real & $\zeta$-component of grid velocity  &
   $\L/\T$ \\
\end{longtable}
