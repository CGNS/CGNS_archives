\section{Design Philosophy of Standard Interface Data Structures}
\label{s:design}
\thispagestyle{plain}

The major design goal of the SIDS is a comprehensive and unambiguous
description of the ``intellectual content'' of information that must be
passed from code to code in a multizone Navier-Stokes analysis system.
This information includes grids, flow solutions, multizone interface
connectivity, boundary conditions, reference states and dimensional
units or normalization associated with data.

\subsubsection*{Implications of CFD Data Sets}

The goal is description of the data sets typical of CFD analysis, which
tend to contain a small number of extremely large data arrays.  This
has a number of implications for both the design of the SIDS and the
ultimate physical files where the data resides.  The first is that any
I/O system built for CFD analysis must be designed to efficiently store
and process large data arrays.  This is reflected in the SIDS, which
includes provisions for describing large data arrays.

The second implication is that the nature of the data sets allows
for thorough description of the data with relatively little storage
overhead and performance penalty.  For example, the flow solution of a
CFD analysis may contain several millions of quantities.  Therefore,
with little penalty it is possible to include information describing
the flow variables stored, their location in the grid, and dimensional
units or nondimensionalization associated with the data.  The SIDS take
advantage of this situation and includes an extensive description of the
information stored in the database.

The third implication of CFD data sets is that files containing a CFD
database are almost always required to be binary -- ASCII storage of
CFD data involves excessive storage and performance penalties.  This
means the files are not readable by humans and the information contained
in them is not directly modifiable by text editors and such.  This is
reflected in the syntax of the SIDS, which tends to be verbose and
thorough; whereas, directly modifiable ASCII file formats would tend to
foster a more brief syntax.

It is important to note that the description of information by the
SIDS is independent of physical file formats.  However, it is targeted
towards implementation using the ADF Core library.  Some of the language
components used to define the SIDS are meant to directly map into
elements of an ADF node.

\subsubsection*{Topologically Based Hierarchical Database}

An early decision in the CGNS project was that any new CFD I/O standard 
should include a hierarchical database, such as a tree or directed graph.
The SIDS describe a hierarchical database, precisely defining both the
data and their hierarchical relationships.

There are two major alternatives to organizing a CFD hierarchy:
topologically based and data-type based.  In a topologically based
graph, overall organization is by zones; information pertaining to a
particular zone, including its grid coordinates or flow solution, hangs
off the zone.  In a data-type based graph, organization is by related
data.  For example, there would be two nodes at the same level, one for
grid coordinates and another for the flow solution.  Hanging off each of
these nodes would be separate lists of the zones.

\begin{figure}
\begin{picture}(6.5,3.0)(-3.25,-3.0)
% \put(-3.25,-3.0){\framebox(6.5,3.0){}}

 %  CGNS database
 \put(-0.6,-0.3){\framebox(1.2,0.3){CGNS database}}
 \put(0,-0.3){\circle*{0.04}}

 %  reference state
 \put(0,-0.3){\line(-4,-1){2.39}}
 \put(-2.4,-0.9){%
 \begin{picture}(0,0)
  \put(0,0){\circle*{0.04}}
  \put(-0.55,-0.3){\framebox(1.2,0.3){Reference state}}
 \end{picture}}

 %  zone 1
 \put(0,-0.3){\line(-5,-3){0.99}}
 \put(-1,-0.9){%
 \begin{picture}(0,0)
  \put(0, 0  ){\circle*{0.04}}
  \put(0,-0.3){\circle*{0.04}}
  \put(-0.3,-0.3){\framebox(0.6,0.3){Zone 1}}

  %  grid coords for zone 1
  \put(0,-0.3){\line(-5,-2){1.49}}
  \put(-1.5,-0.9){%
  \begin{picture}(0,0)
   \put(0, 0  ){\circle*{0.04}}
   \put(0,-0.3){\circle*{0.04}}
   \put(-0.6,-0.3){\framebox(1.2,0.3){Grid coordinates}}

   \put(-0.75,-1.2){\framebox(0.3,0.3){$x$}} 
   \put(-0.15,-1.2){\framebox(0.3,0.3){$y$}} 
   \put( 0.45,-1.2){\framebox(0.3,0.3){$z$}}

   \put(0,-0.3){\line( 0,-1){0.59}}
   \put(0,-0.3){\line( 1,-1){0.59}}
   \put(0,-0.3){\line(-1,-1){0.59}}

   \put(-0.6,-0.9){\circle*{0.04}}
   \put( 0  ,-0.9){\circle*{0.04}}
   \put( 0.6,-0.9){\circle*{0.04}}
  \end{picture}}

  %  flow solution for zone 1
  \put(0,-0.3){\line(5,-2){1.49}}
  \put(+1.5,-0.9){%
  \begin{picture}(0,0)
   \put(0, 0  ){\circle*{0.04}}
   \put(0,-0.3){\circle*{0.04}}
   \put(-0.5,-0.3){\framebox(1.0,0.3){Flow solution}}

   \put(-1.35,-1.2){\framebox(0.3,0.3){$\rho$}} 
   \put(-0.75,-1.2){\framebox(0.3,0.3){$\rho u$}} 
   \put(-0.15,-1.2){\framebox(0.3,0.3){$\rho v$}} 
   \put( 0.45,-1.2){\framebox(0.3,0.3){$\rho w$}}
   \put( 1.05,-1.2){\framebox(0.3,0.3){$\rho e_0$}}

   \put(0,-0.3){\line( 0,-1){0.59}}
   \put(0,-0.3){\line( 1,-1){0.59}}
   \put(0,-0.3){\line(-1,-1){0.59}}
   \put(0,-0.3){\line( 2,-1){1.19}}
   \put(0,-0.3){\line(-2,-1){1.19}}

   \put(-1.2,-0.9){\circle*{0.04}}
   \put(-0.6,-0.9){\circle*{0.04}}
   \put( 0  ,-0.9){\circle*{0.04}}
   \put( 0.6,-0.9){\circle*{0.04}}
   \put( 1.2,-0.9){\circle*{0.04}}
  \end{picture}}

  %  interface connectivity for zone 1
  \put(0,-0.3){\line(0,-1){0.59}}
  \put(0.0,-0.9){%
  \begin{picture}(0,0)
   \put(0, 0  ){\circle*{0.04}}
   \put(0,-0.5){\circle*{0.04}}
   \put(-0.7,-0.5){\framebox(1.4,0.5){\shortstack{Multizone interface\\connectivity}}}
   \put(0,-0.5){\line(-1,-1){0.15}}
   \put(0,-0.5){\line( 0,-1){0.15}}
   \put(0,-0.5){\line( 1,-1){0.15}}
  \end{picture}}

  %  BC's for zone 1
  \put(0,-0.3){\line(5,-1){2.99}}
  \put(+3.0,-0.9){%
  \begin{picture}(0,0)
   \put(0, 0  ){\circle*{0.04}}
   \put(0,-0.3){\circle*{0.04}}
   \put(-0.75,-0.3){\framebox(1.5,0.3){Boundary conditions}}
   \put(0,-0.3){\line(-1,-1){0.15}}
   \put(0,-0.3){\line( 0,-1){0.15}}
   \put(0,-0.3){\line( 1,-1){0.15}}
  \end{picture}}
 \end{picture}}

 %  zone 2
 \put(0,-0.3){\line(1,-3){0.195}}
 \put(0.2,-0.9){%
 \begin{picture}(0,0)
  \put(0, 0  ){\circle*{0.04}}
  \put(0,-0.3){\circle*{0.04}}
  \put(-0.3,-0.3){\framebox(0.6,0.3){Zone 2}}
  \put(0,-0.3){\line(-5,-2){0.3}}
  \put(0,-0.3){\line( 5,-2){0.3}}
  \put(0,-0.3){\line( 0,-1){0.1}}
  \put(0,-0.3){\line( 5,-1){0.5}}
 \end{picture}}

 %  zone N
 \put(0,-0.3){\line(4,-1){2.39}}
 \put(+2.4,-0.9){%
 \begin{picture}(0,0)
  \put(0, 0  ){\circle*{0.04}}
  \put(0,-0.3){\circle*{0.04}}
  \put(-0.3,-0.3){\framebox(0.6,0.3){Zone $N$}}
  \put(0,-0.3){\line(-5,-2){0.3}}
  \put(0,-0.3){\line( 5,-2){0.3}}
  \put(0,-0.3){\line( 0,-1){0.1}}
  \put(0,-0.3){\line( 5,-1){0.5}}
 \end{picture}}

 %  ...
 \put(+1.3,-0.9){%
 \begin{picture}(0,0)
  \put(0,-0.15){\makebox(0,0){$\cdots$}}
 \end{picture}}
\end{picture}
\caption{Sample Topologically Based CFD Hierarchy}
\label{f:hierarchy}
\end{figure}


The hierarchy described in this document is topologically based;
a simplified illustration of the database hierarchy is shown in
\autoref{f:hierarchy}.  Hanging off the root ``node'' of the database is a
node containing global reference-state information, such as freestream
conditions, and a list of nodes for each zone.  The figure shows the
nodes that hang off the first zone; similar nodes would hang off of each
zone in the database.  Nodes containing the physical-coordinate data
arrays ($x$, $y$ and $z$) for the first zone are shown hanging off the
``grid coordinates'' node.  Likewise, nodes containing the first zone's
flow-solution data arrays hang off the ``flow solution'' node.  The figure
also depicts nodes containing multizone interface connectivity and
boundary condition information for the first zone; subnodes hanging off
each of these are not pictured.

\subsubsection*{Additional Design Objectives}

The data structures comprising the SIDS are the result of several
additional design objectives:

One objective is to minimize duplication of data within the hierarchy.
Many parameters, such as the grid size of a zone, are defined in only
one location.  This avoids implementation problems arising from data
duplication within the physical file containing the database; these
problems include simultaneous update of all copies and error checking
when two copies of a data quantity are found to be different.  One
consequence of minimizing data duplication is that information at lower
levels of the hierarchy may not be completely decipherable without
access to information at higher levels.  For example, the grid size
is defined in the zone structure (see \autoref{s:Zone}), but this
parameter is needed in several substructures to define the size of grid
and flow-solution data arrays.  Therefore, these substructures are not
autonomous and deciphering information within them requires access to
information contained in the zone structure itself.  The SIDS must
reflect this cascade of information within the database.

Another objective is elimination of nonsensical descriptions of
the data.  The SIDS have been carefully developed to avoid data
qualifiers and other optional descriptive information that could be
inconsistent.  This has led to the use of specialized structures for
certain CFD-related information.  One example is a single-purpose
structure for defining physical grid coordinates of a zone.  It is
possible to define the grid coordinates, flow solution and any other
field quantities within a zone by a generic discrete-data structure.
However, this requires the generic structure to include information
defining the grid location of the data (e.g., the data is located at
vertices or cell centers).  Using the generic structure to describe the
grid coordinates leads to a possible inconsistency.  By definition the
physical coordinates that define the grid are located at vertices, and
including an optional qualifier that states otherwise makes no sense.

A final objective is to allow documentation inclusion throughout the
database.  The SIDS contain a uniform documentation mechanism for all
major structures in the hierarchy.  However, this document establishes
no conventions for using the documentation mechanism.
