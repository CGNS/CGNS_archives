\section{Time-Dependent Flow}
\label{s:timedep}
\thispagestyle{plain}

This section describes structure types intended primarily for time-dependent
flows.
Data structures are presented for storing time-dependent or iterative
data, and for recording rigid and arbitary grid motion.
The section concludes with several examples.

\subsection{Iterative Data Structure Definitions}
\label{s:IterativeData}

In order to keep a record of time dependent or iterative data, the data
structures \fort{BaseIterativeData\_t} and \fort{ZoneIterativeData\_t}
are used.

\subsubsection{Base Iterative Data Structure Definition: \texttt{BaseIterativeData\_t}}
\label{s:BaseIterativeData}

The \fort{BaseIterativeData\_t} data structure is located directly under
the \fort{CGNSBase\_t} node.
It contains information about the number of time steps or iterations
being recorded, and the time and/or iteration values at each step.
In addition, it may include the list of zones and families for each step
of the simulation, if these vary throughout the simulation.

The \fort{BaseIterativeData\_t} data structure is defined as follows:

\begin{alltt}
  BaseIterativeData\_t :=
    \{
    int NumberOfSteps                                                       (r)

    DataArray\_t<real, 1, NumberOfSteps> TimeValues ;                        (o/r)
    DataArray\_t<int,  1, NumberOfSteps> IterationValues ;                   (r/o)

    DataArray\_t<int,  1, NumberOfSteps> NumberOfZones ;                     (o)
    DataArray\_t<int,  1, NumberOfSteps> NumberOfFamilies ;                  (o)
    DataArray\_t<char, 3, [32, MaxNumberOfZones, NumberOfSteps]>
       ZonePointers ;                                                       (o)
    DataArray\_t<char, 3, [32, MaxNumberOfFamilies, NumberOfSteps]>
       FamilyPointers ;                                                     (o)

    List( DataArray\_t<> DataArray1 ... DataArrayN ) ;                       (o)

    List( Descriptor\_t Descriptor1 ... DescriptorN ) ;                      (o)

    DataClass\_t DataClass ;                                                 (o)

    DimensionalUnits\_t DimensionalUnits ;                                   (o)

    List( UserDefinedData\_t UserDefinedData1 ... UserDefinedDataN ) ;       (o)
    \}
\end{alltt}

\begin{notes}
\item Default names for the \fort{DataArray\_t}, \fort{Descriptor\_t}, and
      \fort{UserDefinedData\_t}
      lists are as shown; users may choose other legitimate names.
      Legitimate names must be unique within a given instance
      of \fort{BaseIterativeData\_t} and shall not include the names
      \fort{DataClass}, \fort{DimensionalUnits}, \fort{FamilyPointers},
      \fort{IterationValues}, \fort{NumberOfFamilies},
      \fort{NumberOfZones}, \fort{TimeValues}, or \fort{ZonePointers}.
\item \fort{NumberOfSteps} is a required element of the
      \fort{BaseIterativeData\_t} data structure.
      It holds either the number of time steps or the number of iterations
      being recorded.
\item Either \fort{TimeValues} or \fort{IterationValues} must be defined.
      If both are used, there must be a one-to-one correspondence between
      them.
\end{notes}

\fort{TimeValues} and \fort{IterationValues} are data-name identifiers
corresponding to the time and iteration values stored in the file.
When \fort{IterationValues} are used, the iterative data stored in the
database correspond to values at the end of the associated iteration.

The data-name identifiers \fort{NumberOfZones} and \fort{ZonePointers}
are only used if different zone data structures apply to different steps
of the simulation.
(See \hyperref[ex:adaptedunstructuredmesh]{Example~\ref*{ex:adaptedunstructuredmesh}}.)

Similarly, if the geometry varies with time or iteration, then the
data-name identifiers \fort{NumberOfFamilies} and \fort{FamilyPointers}
are used to record which \fort{Family\_t} data structure(s)
correspond(s) to which step.

The \fort{DataArray\_t} nodes for \fort{ZonePointers} and
\fort{FamilyPointers} are defined as three-dimensional arrays.
For each recorded step, the names of all zones and families being
used for the step may be recorded.
Note that the names are limited to 32 characters, as usual in the SIDS.
The variables \fort{MaxNumberOfZones} and \fort{MaxNumberOfFamilies}
represent the maximum number of zones and families that apply to
one step.
So if \fort{NumberOfSteps} = 5 and \fort{NumberOfZones} = \{2,2,3,4,3\},
then \fort{MaxNumberOfZones} equals 4.

When \fort{NumberOfZones} and \fort{NumberOfFamilies} vary for
different stored time steps, the name \fort{Null} is used in
\fort{ZonePointers} and \fort{FamilyPointers} as appropriate for
steps in which the \fort{NumberOfZones} or \fort{NumberOfFamilies}
is less than the \fort{MaxNumberOfZones} or \fort{MaxNumberOfFamilies}.

Any number of extra \fort{DataArray\_t} nodes are allowed.
These should be used to record data not covered by this specification.

\subsubsection{Zone Iterative Data Structure Definition: \texttt{ZoneIterativeData\_t}}
\label{s:ZoneIterativeData}

The \fort{ZoneIterativeData\_t} data structure is located under the
\fort{Zone\_t} node.
It may be used to record pointers to zonal data for each recorded step of
the simulation, and is defined as follows:

\begin{alltt}
  ZoneIterativeData\_t< int NumberOfSteps > :=
    \{
    DataArray\_t<char, 2, [32, NumberOfSteps]> RigidGridMotionPointers ;     (o)
    DataArray\_t<char, 2, [32, NumberOfSteps]> ArbitraryGridMotionPointers ; (o)
    DataArray\_t<char, 2, [32, NumberOfSteps]> GridCoordinatesPointers ;     (o)
    DataArray\_t<char, 2, [32, NumberOfSteps]> FlowSolutionPointers ;        (o)

    List( DataArray\_t<> DataArray1 ... DataArrayN ) ;                       (o)

    List( Descriptor\_t Descriptor1 ... DescriptorN ) ;                      (o)

    DataClass\_t DataClass ;                                                 (o)

    DimensionalUnits\_t DimensionalUnits ;                                   (o)

    List( UserDefinedData\_t UserDefinedData1 ... UserDefinedDataN ) ;       (o)
    \}
\end{alltt}

\begin{notes}
\item Default names for the \fort{DataArray\_t}, \fort{Descriptor\_t}, and
      \fort{UserDefinedData\_t}
      lists are as shown; users may choose other legitimate names.
      Legitimate names must be unique within a given
      instance of \fort{ZoneIterativeData\_t} and shall not
      include the names \fort{ArbitraryGridMotionPointers},
      \fort{DataClass}, \fort{DimensionalUnits},
      \fort{FlowSolutionPointers}, \fort{Grid\-Co\-ord\-i\-nates\-Point\-ers}, or
      \fort{RigidGridMotionPointers}.
\end{notes}

The data arrays with data-name identifiers \fort{\textit{xxx}Pointers}
contain lists of associated data structures for each recorded time value
or iteration.
These data structures contain data at the associated time value, or at
the end of the associated iteration.
There is an implied one-to-one correspondence between each pointer (from
1, 2, \ldots, \fort{NumberOfSteps}) and the associated \fort{TimeValues}
and/or \fort{IterationValues} under \fort{BaseIterativeData\_t}.
They refer by name to data structures within the current zone.
The name \fort{Null} is used when a particular time or iteration
does not have a corresponding data structure to point to.

Any number of extra \fort{DataArray\_t} nodes are allowed.
These should be used to record data not covered by this specification.

The \fort{ZoneIterativeData\_t} data structure may not exist without
the \fort{BaseIterativeData\_t} under the \fort{CGNSBase\_t} node.
However \fort{BaseIterativeData\_t} may exist without
\fort{ZoneIterativeData\_t}.

\subsection{Rigid Grid Motion Structure Definition: \texttt{RigidGridMotion\_t}}
\label{s:RigidGridMotion}

Adding rigid grid motion information to the CGNS file enables an
application code to determine the mesh location without the need to
alter the original mesh definition recorded under \fort{GridCoordinates\_t}.
A data structure named \fort{RigidGridMotion\_t} is used to record the
necessary data defining a rigid translation and/or rotation of the grid
coordinates.

The rigid grid motion is recorded independently for each zone of the
CGNS base.
Therefore the \fort{RigidGridMotion\_t} data structure is located under
the zone data structure (\fort{Zone\_t}).
There may be zero to several \fort{RigidGridMotion\_t} nodes under a
\fort{Zone\_t} node.
The multiple rigid grid motion definitions may be associated with
different iterations or time steps in the computation.
This association is recorded under the \fort{ZoneIterativeData\_t} data
structure, described in \autoref{s:ZoneIterativeData}.

\begin{alltt}
  RigidGridMotion\_t :=
    \{
    List( Descriptor\_t Descriptor1 ... DescriptorN ) ;                      (o)

    RigidGridMotionType\_t RigidGridMotionType ;                             (r)

    DataArray\_t<real, 2, [PhysicalDimension, 2]> OriginLocation ;           (r)
    DataArray\_t<real, 1,  PhysicalDimension>     RigidRotationAngle ;       (o/d)
    DataArray\_t<real, 1,  PhysicalDimension>     RigidVelocity ;            (o)
    DataArray\_t<real, 1,  PhysicalDimension>     RigidRotationRate ;        (o)

    List( DataArray\_t DataArray1 ... DataArrayN ) ;                         (o)

    DataClass\_t DataClass ;                                                 (o)

    DimensionalUnits\_t DimensionalUnits ;                                   (o)

    List( UserDefinedData\_t UserDefinedData1 ... UserDefinedDataN ) ;       (o)
    \} ;
\end{alltt}

\begin{notes}
\item Default names for the \fort{Descriptor\_t}, \fort{DataArray\_t}, and
      \fort{UserDefinedData\_t}
      lists are as shown; users may choose other legitimate names.
      Legitimate names must be unique within a given instance
      of \fort{RigidGridMotion\_t} and shall not include the names
      \fort{DataClass}, \fort{DimensionalUnits}, \fort{OriginLocation},
      \fort{RigidGridMotionType}, \fort{RigidRotationAngle},
      \fort{RigidRotationRate}, or \fort{RigidVelocity}.
\item \fort{RigidGridMotionType} and \fort{OriginLocation} are the
      only required elements under \fort{RigidGridMotion\_t}.
      All other elements are optional.
\end{notes}

\fort{RigidGridMotionType\_t} is an enumeration type that describes
the type of rigid grid motion.

\begin{alltt}
  RigidGridMotionType\_t := Enumeration(
    Null,
    ConstantRate,
    VariableRate,
    UserDefined ) ;
\end{alltt}

The characteristics of the grid motion are defined by the
data-name identifiers in \autoref{t:id_rigidgrid}.

\settowidth{\tmplengtha}{\bold{Data-Name Identifier}}
\settowidth{\tmplengthb}{\bold{Units}}
\setlength{\Pwidth}{\linewidth-6\tabcolsep-\tmplengtha-\tmplengthb}
\begin{table}[htbp]
\centering
\caption[Data-Name Identifiers for Rigid Grid Motion]{\textbf{Data-Name Identifiers for Rigid Grid Motion}}
\label{t:id_rigidgrid}
\begin{tabular}{>{\ttfamily}l >{\raggedright\arraybackslash}p{\Pwidth} c}
\\ \hline\hline \\*[-2ex]
\bold{Data-Name Identifier} & \bold{Description} & \bold{Units}
\\*[1ex] \hline\hline \\*[-2ex]
OriginLocation
   & Physical coordinates of the origin before and after the rigid grid motion
   & $\L$ \\
RigidRotationAngle
   & Rotation angles about each axis of the translated coordinate system.
     If not specified, \fort{RigidRotationAngle} is set to zero.
   & $\A$ \\
RigidVelocity
   & Grid velocity vector of the origin translation
   & $\L/\T$ \\
RigidRotationRate
   & Rotation rate vector about the axis of the translated coordinate system
   & $\A/\T$
\\*[1ex] \hline\hline
\end{tabular}
\end{table}

Any number of additional \fort{DataArray\_t} nodes are allowed.
These may be used to record data not covered by this specification.

``Rigid grid motion'' implies relative motion of grid zones.
However, no attempt is made in the \fort{RigidGridMotion\_t} data
structure to require that the \fort{ZoneGridConnectivity\_t} information
be updated to be consistent with the new grid locations.
Whether the \fort{ZoneGridConnectivity\_t} information refers to
the original connectivity (of \fort{GridCoordinates}) or the latest
connectivity (of the moved or deformed grid) is currently left up to the
user.

\subsection{Arbitrary Grid Motion Structure Definition: \texttt{ArbitraryGridMotion\_t}}
\label{s:ArbitraryGridMotion}

When a grid is in motion, it is often necessary to account for the
position of each grid point as the grid deforms.
When all grid points move at the same velocity, the grid keeps its
original shape.
This particular case of grid motion may be recorded under the
\fort{RigidGridMotion\_t} data structure described in
\autoref{s:RigidGridMotion}.
On the other hand, if the grid points have different velocity, the grid
is deforming.
The \fort{ArbitraryGridMotion\_t} data structure allows the CGNS file to
contain information about arbitrary grid deformations.
If not present, the grid is assumed to be rigid.

Note that multiple \fort{GridCoordinates\_t} nodes may be stored under a
\fort{Zone\_t}.
This allows the storage of the instantaneous grid locations at different
time steps or iterations.

The arbitrary grid motion is recorded independently for each zone of the
CGNS base.
Therefore the \fort{ArbitraryGridMotion\_t} data structure is located
under the zone data structure (\fort{Zone\_t}).
There may be zero to several \fort{ArbitraryGridMotion\_t} nodes under a
single \fort{Zone\_t} node.
The multiple arbitrary grid motion definitions may be associated with
different iterations or time steps in the computation.
This association is recorded under the \fort{ZoneIterativeData\_t} data
structure, described in \autoref{s:ZoneIterativeData}.

\begin{alltt}
  ArbitraryGridMotion\_t< int IndexDimension, int VertexSize[IndexDimension], 
                         int CellSize[IndexDimension] > :=
    \{
    ArbitraryGridMotionType\_t ArbitraryGridMotionType ;                     (r)

    List(DataArray\_t<real, IndexDimension, DataSize[]>
       GridVelocityX GridVelocityY ... ) ;                                  (o)

    List( Descriptor\_t Descriptor1 ... DescriptorN ) ;                      (o)

    GridLocation\_t GridLocation ;                                           (o/d)

    Rind\_t<IndexDimension> Rind ;                                           (o/d)

    DataClass\_t DataClass ;                                                 (o)

    DimensionalUnits\_t DimensionalUnits ;                                   (o)

    List( UserDefinedData\_t UserDefinedData1 ... UserDefinedDataN ) ;       (o)
    \}
\end{alltt}

\begin{notes}
\item Default names for the \fort{Descriptor\_t}, \fort{DataArray\_t}, and
      \fort{UserDefinedData\_t}
      lists are as shown; users may choose other legitimate names.
      Legitimate names must be unique within a given instance
      of \fort{ArbitraryGridMotion\_t} and shall not include the names
      \fort{ArbitraryGridMotionType}, \fort{DataClass},
      \fort{DimensionalUnits}, \fort{GridLocation}, or \fort{Rind}.
\item The only required element of the \fort{ArbitraryGridMotion\_t}
      data structure is the \fort{ArbitraryGridMotionType}.
      Thus, even if a deforming grid application does not require the
      storage of grid velocity data, the \fort{ArbitraryGridMotion\_t}
      node must exist (with
      \fort{ArbitraryGridMotionType} = \fort{DeformingGrid}) to indicate
      that deformed grid points (\fort{GridCoordinates\_t}) exist for this
      zone.
\item \fort{Rind} is an optional field that indicates the number of rind
      planes (for structured grids) or rind points or elements (for
      unstructured grids) included in the grid velocity data.
\item The \fort{GridLocation} specifies the location of the velocity
      data with respect to the grid; if absent, the data is assumed
      to coincide with grid vertices (i.e., \fort{GridLocation} =
      \fort{Vertex}).
\end{notes}

\fort{ArbitraryGridMotion\_t} requires three structure parameters;
\fort{IndexDimension} identifies the dimensionality of the grid-size
arrays, and \fort{VertexSize} and \fort{CellSize} are the number of
core vertices and cells, respectively, in each index direction.
For unstructured zones, \fort{IndexDimension} is always 1.

\fort{ArbitraryGridMotionType\_t} is an enumeration type that describes
the type of arbitrary grid motion.

\begin{alltt}
  ArbitraryGridMotionType\_t := Enumeration(
    Null,
    NonDeformingGrid,
    DeformingGrid,
    UserDefined ) ;
\end{alltt}

The \fort{DataArray\_t} nodes are used to store the components of the grid
velocity vector.
\autoref{t:id_gridvel} lists the data-name identifiers used to record these
vectors in cartesian, cylindrical, and spherical coordinate systems.

\begin{table}[htbp]
\centering
\caption[Data-Name Identifiers for Grid Velocity]{\textbf{Data-Name Identifiers for Grid Velocity}}
\label{t:id_gridvel}
\begin{tabular}{>{\ttfamily}l >{\quad}l >{\quad}c}
\\ \hline\hline \\*[-2ex]
\bold{Data-Name Identifier} & \bold{Description} & \bold{Units}
\\*[1ex] \hline\hline \\*[-2ex]
GridVelocityX     & $x$-component of grid velocity      &
   $\L/\T$ \\
GridVelocityY     & $y$-component of grid velocity      &
   $\L/\T$ \\
GridVelocityZ     & $z$-component of grid velocity      &
   $\L/\T$ \\
\\
GridVelocityR     & $r$-component of grid velocity      &
   $\L/\T$ \\
GridVelocityTheta & $\theta$-component of grid velocity &
   $\alpha/\T$ \\
GridVelocityPhi   & $\phi$-component of grid velocity   &
   $\alpha/\T$ \\
\\
GridVelocityXi    & $\xi$-component of grid velocity    &
   $\L/\T$ \\
GridVelocityEta   & $\eta$-component of grid velocity   &
   $\L/\T$ \\
GridVelocityZeta  & $\zeta$-component of grid velocity  &
   $\L/\T$
\\*[1ex] \hline\hline
\end{tabular}
\end{table}

The field \fort{GridLocation} specifies the location of the grid
velocities with respect to the grid; if absent, the grid velocities
are assumed to coincide with grid vertices (i.e., \fort{GridLocation =
Vertex}).
All grid velocities within a given instance of
\fort{ArbitraryGridMotion\_t} must reside at the same grid location.

\texttt{Rind} is an optional field for structured zones that indicates
the number of rind planes (for structured grids) or rind points or
elements (for unstructured grids) included in the data.
Its purpose and function are identical to those described for the
\texttt{GridCoordinates\_t} structure in \autoref{s:Grid}.
Note, however, that the \texttt{Rind} in this structure is independent
of the \texttt{Rind} contained in \texttt{GridCoordinates\_t} or
\texttt{FlowSolution\_t}.
They are not required to contain the same number of rind planes or
elements.
Also, the location of any rind points is assumed to be consistent with
the location of the core data points (e.g., if \texttt{GridLocation =
CellCenter}, rind points are assumed to be located at fictitious cell
centers).

\fort{DataClass} defines the default class for data contained in the
\fort{DataArray\_t} entities.
For dimensional grid velocities, \fort{DimensionalUnits} may be used to
describe the system of units employed.
If present these two entities take precedence over the corresponding
entities at higher levels of the CGNS hierarchy.
The rules for determining precedence of entities of this type are
discussed in \autoref{s:precedence}.

Point-by-point grid velocity implies a deformation (or potentially only
motion) of the grid points relative to each other.
Because the original grid coordinates definition remains
unchanged with the name \fort{GridCoordinates}, any deformed
coordinates must be written with a different name (e.g.,
\fort{MovedGrid\#1} or another used-defined name) and are pointed
to using \fort{GridCoordinatesPointers} in the data structure
\fort{ZoneIterativeData\_t}, as described in \autoref{s:ZoneIterativeData}.

The \fort{UserDefinedData\_t} data structure allows arbitrary
user-defined data to be stored in \fort{Descriptor\_t} and
\fort{DataArray\_t} children without the restrictions or implicit
meanings imposed on these node types at other node locations.

Point-by-point grid velocity may also lead to relative motion of grid
zones, or movement of grid along abutting interfaces.
However, no attempt is made here to require that the
\fort{ZoneGridConnectivity\_t} information be updated to be consistent
with the new grid locations.
Whether the \fort{ZoneGridConnectivity\_t} information refers to
the original connectivity (of \fort{GridCoordinates}) or the latest
connectivity (of the moved or deformed grid) is currently left up to the
user.

\subsubsection*{FUNCTION \texttt{DataSize[]}:}

\noindent return value: one-dimensional |int| array of length
          |IndexDimension| \\
\noindent dependencies: |IndexDimension|, |VertexSize[]|, |CellSize[]|,
          |GridLocation|, |Rind|

The function \fort{DataSize[]} is the size of the \fort{DataArray}s
containing the grid velocity components.
It is identical to the function \fort{DataSize[]} defined for
\fort{FlowSolution\_t} (see \autoref{s:FlowSolution}).

\subsection{Examples for Time-Dependent Flow}
\label{s:Examplestimedep}

\enlargethispage{\baselineskip}
\begin{example}{Rigid Grid Motion}
\label{ex:rigidmotion}

In this example, the whole mesh moves rigidly, so the only
time-dependant data are the grid coordinates and flow solutions.
However, since the mesh moves rigidly, the grid coordinates need not be
recorded at each time step.
Instead, a \fort{RigidGridMotion\_t} data structure is recorded for
each step of the computation.

The number of steps and time values for each step are recorded under
\fort{BaseIterativeData\_t}.

\begin{alltt}
  CGNSBase\_t \{
    BaseIterativeData\_t \{
      NumberOfSteps = N ;
      TimeValues = time1, time2, ..., timeN ;
    \} ;
  \} ;
\end{alltt}

The multiple rigid grid motion and flow solution data structures are
recorded under the zone.
\fort{RigidGridMotionPointers} and \fort{FlowSolutionPointers} keep the
lists of which \fort{RigidGridMotion\_t} and \fort{FlowSolution\_t} nodes
correspond to each time step.

\begin{alltt}
  Zone\_t Zone \{

    \ital{--- Time independent data}
    GridCoordinates\_t GridCoordinates
    ZoneBC\_t ZoneBC
    ZoneGridConnectivity\_t ZoneGridConnectivity

    \ital{--- Time dependant data}
    RigidGridMotion\_t RigidGridMotion\#1
    RigidGridMotion\_t RigidGridMotion\#2
    ...
    RigidGridMotion\_t RigidGridmotion\#N

    FlowSolution\_t Solution\#0
    FlowSolution\_t Solution\#1
    FlowSolution\_t Solution\#2
    ...
    FlowSolution\_t Solution\#N

    ZoneIterativeData\_t \{
      RigidGridMotionPointers = \{"RigidGridMotion\#1", "RigidGridMotion\#2", ...,
         "RigidGridMotion\#N"\}
      FlowSolutionPointers = \{"Solution\#1", "Solution\#2, ..., "Solution\#N"\}
    \}
  \}
\end{alltt}

Note that there may be more solutions under a zone than those pointed to
by \fort{FlowSolutionPointers}.
In this example, \fort{Solution\#0} could correspond to a restart solution.
\end{example}

\enlargethispage{\baselineskip}
\begin{example}{Deforming Grid Motion}
\label{ex:deformingmotion}

In this example, velocity vectors are node dependant allowing for mesh
deformation.
In such a case, it is difficult or even impossible to recompute the mesh
at each time step.
Therefore the grid coordinates are recorded for each step.

Multiple \fort{GridCoordinates\_t} and \fort{FlowSolution\_t} data
structures are recorded under the zone.
In addition, the data structure \fort{ArbitraryGridMotion\_t} is recorded
for each step.
\fort{GridCoordinatesPointers}, \fort{FlowSolutionPointers}, and
\fort{ArbitraryGridMotionPointers\_t} keep the list of which grid
coordinates definition, flow solution, and arbitrary grid motion
definition correspond to each time step.

\begin{alltt}
  Zone\_t Zone \{

    \ital{--- Time independent data}
    ZoneBC\_t ZoneBC
    ZoneGridConnectivity\_t ZoneGridConnectivity

    \ital{--- Time dependent data}
    List ( GridCoordinates\_t GridCoordinates MovedGrid\#1 MovedGrid\#2 ...
           MovedGrid\#N )
    List ( FlowSolution\_t Solution\#0 Solution\#1 Solution\#2 ... Solution\#N )
    List ( ArbitraryGridMotion\_t ArbitraryGridMotion\#1 
           ArbitraryGridMotion\#2 ... ArbitraryGridMotion\#N )
    ZoneIterativeData\_t \{
      GridCoordinatesPointers = \{"MovedGrid\#1", "MovedGrid\#2", ...,
         "MovedGrid\#N"\}
      FlowSolutionPointers = \{"Solution\#1", "Solution\#2, ..., "Solution\#N"\}
      ArbitraryGridMotionPointers = \{"ArbitraryGridMotion\#1",
         "ArbitraryGridMotion\#2", ..., "ArbitraryGridMotion\#N"\}
    \}
  \}
\end{alltt}
\end{example}

\newpage
\begin{example}{Adapted Unstructured Mesh}
\label{ex:adaptedunstructuredmesh}

In this example, the mesh size varies at each remeshing, therefore new
zones must be created.
\fort{ZonePointers} is used to keep a record of the zone definition
corresponding to each recorded step.
Let's assume that the solution is recorded every 50 iterations, and the
grid is adapted every 100 iterations.

The number of steps, iteration values for each step, number of zones
for each step, and name of these zones are recorded under
\fort{BaseIterativeData\_t}.

\begin{alltt}
  CGNSBase\_t \{
    BaseIterativeData\_t \{
      NumberOfSteps = 4
      IterationValues = \{50, 100, 150, 200\}
      NumberOfZones = \{1, 1, 1, 1\}
      ZonePointers = \{"Zone1", "Zone1", "Zone2", "Zone2"\}
    \}
  \}
\end{alltt}

Each zone holds 2 solutions recorded at 50 iterations apart.
Therefore the \fort{ZoneIterativeData\_t} data structure must be included
to keep track of the \fort{FlowSolutionPointers}.

\begin{alltt}
  Zone\_t Zone1 \{

    \ital{--- Constant data}
    GridCoordinates\_t GridCoordinates
    Elements\_t Elements
    ZoneBC\_t ZoneBC

    \ital{--- Variable data}
    List ( FlowSolution\_t InitialSolution Solution50 Solution100 )
    ZoneIterativeData\_t \{
      FlowSolutionPointers = \{"Solution50", "Solution100", "Null", "Null"\}
    \}
  \}

  Zone\_t Zone2 \{

    \ital{--- Constant data}
    GridCoordinates\_t GridCoordinates
    Elements\_t Elements
    ZoneBC\_t ZoneBC

    \ital{--- Variable data}
    List ( FlowSolution\_t RestartSolution Solution150 Solution200 )
    ZoneIterativeData\_t \{
      FlowSolutionPointers = \{"Null", "Null", "Solution150", "Solution200"\}
    \}
  \}
\end{alltt}
\end{example}

\begin{notes}
\item If the solution was recorded every 100 iterations instead of
      every 50 iterations, then each zone would have only one
      \fort{FlowSolution\_t} node and the data structure
      \fort{ZoneIterativeData\_t} would not be required.
\item Note that \fort{FlowSolutionPointers} is always an array of size
      \fort{NumberOfSteps} even if some of the steps are defined in
      another zone.
\end{notes}

\begin{example}{Combination of Grid Motion and Time-Accuracy}
\label{ex:combination}

The following is an example demonstrating the use of the rigid grid motion,
arbitrary grid motion, and time-accurate data nodes in CGNS.
The example is a 3-zone case.
Zone 1 is rigidly rotating about the $x$-axis at a constant rate, with
no translation.
Zone 2 is a deforming zone.
Zone 3 is a fixed zone.
This is a time-accurate simulation with two solutions saved at times
15.5 and 31.0, corresponding to iteration numbers 1000 and 2000.

No units are given in this example, but a real case would establish them.
Also, a real case would include connectivity, boundary conditions, and
possibly other information as well.
Each indentation represents a level down (a child) from the parent node.

\begin{alltt}
  Base (CGNSBase\_t)
    SimulationType (SimulationType\_t) Data=TimeAccurate
    BaseIterativeData (BaseIterativeData\_t) Data=NumberOfSteps=2
      TimeValues (DataArray\_t) Data=(15.5, 31.0)
      IterationValues (DataArray\_t) Data=(1000, 2000)
    Zone\#1 (Zone\_t)
      GridCoordinates (GridCoordinates\_t)
        CoordinateX (DataArray\_t)
        CoordinateY (DataArray\_t)
      RigidGridMotion\#1(RigidGridMotion\_t) Data=RigidGridMotionType=ConstantRate
        OriginLocation (DataArray\_t) Data=(0,0,0), (0,0,0)
        RigidRotationAngle (DataArray\_t) Data=(5., 0., 0.)
      RigidGridMotion\#2(RigidGridMotion\_t) Data=RigidGridMotionType=ConstantRate
        OriginLocation (DataArray\_t) Data=(0,0,0), (0,0,0)
        RigidRotationAngle (DataArray\_t) Data=(10., 0., 0.)
      ZoneIterativeData (ZoneIterativeData\_t)
        RigidGridMotionPointers (DataArray\_t) Data=(RigidGridMotion\#1,
                                                    RigidGridMotion\#2)
        FlowSolutionPointers (DataArray\_t) Data=(Soln\#1, Soln\#2)
      Soln\#1 (FlowSolution\_t)
        Density (DataArray\_t)
        VelocityX (DataArray\_t)
      Soln\#2 (FlowSolution\_t)
        Density (DataArray\_t)
        VelocityX (DataArray\_t)
    Zone\#2 (Zone\_t)
      GridCoordinates (GridCoordinates\_t)
        CoordinateX (DataArray\_t)
        CoordinateY (DataArray\_t)
      MovedGrid\#1 (GridCoordinates\_t)
        CoordinateX (DataArray\_t)
        CoordinateY (DataArray\_t)
      MovedGrid\#2 (GridCoordinates\_t)
        CoordinateX (DataArray\_t)
        CoordinateY (DataArray\_t)
      ArbitraryGridMotion\#1 (ArbitraryGridMotion\_t)
                             Data=ArbitraryGridMotionType=DeformingGrid
      ArbitraryGridMotion\#2 (ArbitraryGridMotion\_t)
                             Data=ArbitraryGridMotionType=DeformingGrid
        GridVelocityX (DataArray\_t)
        GridVelocityY (DataArray\_t)
      ZoneIterativeData (ZoneIterativeData\_t)
        ArbitraryGridMotionPointers (DataArray\_t) Data=("ArbitraryGridMotion\#1",
                                                        "ArbitraryGridMotion\#2")
        GridCoordinatesPointers (DataArray\_t) Data=("MovedGrid\#1",
                                                    "MovedGrid\#2")
        FlowSolutionPointers (DataArray\_t) Data=("Soln\#1", "Soln\#2")
      Soln\#1 (FlowSolution\_t)
        Density (DataArray\_t)
        VelocityX (DataArray\_t)
      Soln\#2 (FlowSolution\_t)
        Density (DataArray\_t)
        VelocityX (DataArray\_t)
    Zone\#3 (Zone\_t)
      GridCoordinates (GridCoordinates\_t)
        CoordinateX (DataArray\_t)
        CoordinateY (DataArray\_t)
      ZoneIterativeData (ZoneIterativeData\_t)
        FlowSolutionPointers (DataArray\_t) Data=("Soln\#1", "Soln\#2")
      Soln\#1 (FlowSolution\_t)
        Density (DataArray\_t)
        VelocityX (DataArray\_t)
      Soln\#2 (FlowSolution\_t)
        Density (DataArray\_t)
        VelocityX (DataArray\_t)
\end{alltt}

\newpage
\begin{notes}
\item Under \fort{BaseIterativeData\_t}, one can give either
      \fort{TimeValues}, or \fort{IterationValues}, or both.
      In the example, both have been given.
\item The nodes \fort{NumberOfZones} and \fort{ZonePointers} are not
      required under the \fort{BaseIterativeData\_t} node in this example
      because all existing zones are used for each time step.
\item Under \fort{ArbitraryGridMotion}, the \fort{GridVelocity} data
      is optional.
      In the example, it was put under one of the nodes but not under
      the other.
      Hence, \fort{"ArbitraryGridMotion\#1"} in the example has no
      children nodes, while \fort{"ArbitraryGridMotion\#2"} does.
\item The pointers under \fort{ZoneIterativeData\_t} point to names of
      nodes within the same zone.
      Thus, for example, \fort{Soln\#1} refers to the flow solution
      named \fort{Soln\#1} in the same zone, even though there are flow
      solution nodes in other zones with the same name.
\item The name \fort{GridCoordinates} always refers to the
      \emph{original} grid.
      Thus, when a grid is deforming, the deformed values must be put in
      \fort{GridCoordinates\_t} nodes of a different name.
      In the example, the deformed grids (for \fort{Zone\#2}) at the two
      times of interest were put into \fort{"MovedGrid\#1"} and
      \fort{"MovedGrid\#2"}.
\item Because the node \fort{"ArbitraryGridMotion\#1"} doesn't really
      add any information in the current example (since it was decided
      not to store \fort{GridVelocity} data under it), one has the option
      of not including this node in the CGNS file.
      If it is removed, then under \fort{Zone\#2}'s \fort{ZoneIterativeData},
      the \fort{ArbitraryGridMotionPointers} data would be replaced by:
      \begin{alltt}
   Data = (Null, ArbitraryGridMotion\#2)
      \end{alltt}
\end{notes}
\end{example}
