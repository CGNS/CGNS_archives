\section{Grid Specification}
\label{s:grid}
\thispagestyle{plain}

\subsection{Zone Grid Coordinates}
\label{s:gridcoordinates}

\noindent
\textit{Node}: \fort{GridCoordinates\_t}

\fort{GridCoordinates\_t} nodes are used to describe grids associated
with a particular zone.
The original grid must be described by a \fort{GridCoordinates\_t} node
named \fort{GridCoordinates}.
Additional \fort{GridCoordinates\_t} nodes may be used, with user-defined
names, to store grids at multiple time steps or iterations.
In addition to the discussion of the \fort{GridCoordinates\_t} node in
the \href{../sids/sids.pdf}{SIDS} and \href{../filemap/filemap.pdf}{File
Mapping} manuals, see the discussion of the \fort{ZoneIterativeData\_t}
and \fort{ArbitraryGridMotion\_t} nodes in the SIDS manual.

\begin{fctbox}
\textcolor{output}{\textit{ier}} = cg\_grid\_write(\textcolor{input}{int fn}, \textcolor{input}{int B}, \textcolor{input}{int Z}, \textcolor{input}{char *GridCoordName}, & - w m \\
~~~~~~\textcolor{output}{\textit{int *G}}); & \\
\textcolor{output}{\textit{ier}} = cg\_ngrids(\textcolor{input}{int fn}, \textcolor{input}{int B}, \textcolor{input}{int Z}, \textcolor{output}{\textit{int *ngrids}}); & - w m \\
\textcolor{output}{\textit{ier}} = cg\_grid\_read(\textcolor{input}{int fn}, \textcolor{input}{int B}, \textcolor{input}{int Z}, \textcolor{input}{int G}, \textcolor{output}{\textit{char *GridCoordName}}); & r - m \\
\hline
call cg\_grid\_write\_f(\textcolor{input}{fn}, \textcolor{input}{B}, \textcolor{input}{Z}, \textcolor{input}{GridCoordName}, \textcolor{output}{\textit{G}}, \textcolor{output}{\textit{ier}}) & - w m \\
call cg\_ngrids\_f(\textcolor{input}{fn}, \textcolor{input}{B}, \textcolor{input}{Z}, \textcolor{output}{\textit{ngrids}}, \textcolor{output}{\textit{ier}}) & - w m \\
call cg\_grid\_read\_f(\textcolor{input}{fn}, \textcolor{input}{B}, \textcolor{input}{Z}, \textcolor{input}{G}, \textcolor{output}{\textit{GridCoordName}}, \textcolor{output}{\textit{ier}}) & r - m \\
\end{fctbox}

\noindent
\textbf{\textcolor{input}{Input}/\textcolor{output}{Output}}

\begin{Ventryi}{\fort{GridCoordinateName}}\raggedright
\item [\fort{fn}]
      CGNS file index number.
      (\textcolor{input}{Input})
\item [\fort{B}]
      Base index number, where $1 \leq \text{\fort{B}} \leq \text{\fort{nbases}}$.
      (\textcolor{input}{Input})
\item [\fort{Z}]
      Zone index number, where $1 \leq \text{\fort{Z}} \leq \text{\fort{nzones}}$.
      (\textcolor{input}{Input})
\item [\fort{G}]
      Grid index number, where $1 \leq \text{\fort{G}} \leq \text{\fort{ngrids}}$.
      (\textcolor{input}{Input} for \fort{cg\_grid\_read};
      \textcolor{output}{\textit{output}} for \fort{cg\_grid\_write})
\item [\fort{ngrids}]
      Number of \fort{GridCoordinates\_t} nodes for zone \fort{Z}.
      (\textcolor{output}{\textit{Output}})
\item [\fort{GridCoordinateName}]
      Name of the \fort{GridCoordinates\_t} node.
      Note that the name ``\fort{GridCoordinates}'' is reserved for the
      original grid and must be the first \fort{GridCoordinates\_t}
      node to be defined.
      (\textcolor{input}{Input} for \fort{cg\_grid\_write};
      \textcolor{output}{\textit{output}} for \fort{cg\_grid\_read})
\item [\fort{ier}]
      Error status.
      (\textcolor{output}{\textit{Output}})
\end{Ventryi}

The above functions are applicable to any \fort{GridCoordinates\_t} node.

\begin{fctbox}
\textcolor{output}{\textit{ier}} = cg\_coord\_write(\textcolor{input}{int fn}, \textcolor{input}{int B}, \textcolor{input}{int Z}, \textcolor{input}{DataType\_t datatype}, & - w m \\
~~~~~~\textcolor{input}{char *coordname}, \textcolor{input}{void *coord\_array}, \textcolor{output}{\textit{int *C}}); & \\
\textcolor{output}{\textit{ier}} = cg\_coord\_partial\_write(\textcolor{input}{int fn}, \textcolor{input}{int B}, \textcolor{input}{int Z}, & - w m \\
~~~~~~\textcolor{input}{DataType\_t datatype}, \textcolor{input}{char *coordname}, \textcolor{input}{int*range\_min}, & \\
~~~~~~\textcolor{input}{int*range\_max}, \textcolor{input}{void *coord\_array}, \textcolor{output}{\textit{int *C}}); & \\
\textcolor{output}{\textit{ier}} = cg\_ncoords(\textcolor{input}{int fn}, \textcolor{input}{int B}, \textcolor{input}{int Z}, \textcolor{output}{\textit{int *ncoords}}); & r - m \\
\textcolor{output}{\textit{ier}} = cg\_coord\_info(\textcolor{input}{int fn}, \textcolor{input}{int B}, \textcolor{input}{int Z}, \textcolor{input}{int C}, \textcolor{output}{\textit{DataType\_t *datatype}}, & r - m \\
~~~~~~\textcolor{output}{\textit{char *coordname}}); & \\
\textcolor{output}{\textit{ier}} = cg\_coord\_read(\textcolor{input}{int fn}, \textcolor{input}{int B}, \textcolor{input}{int Z}, \textcolor{input}{char *coordname}, & r - m \\
~~~~~~\textcolor{input}{DataType\_t datatype}, \textcolor{input}{int *range\_min}, \textcolor{input}{int *range\_max}, & \\
~~~~~~\textcolor{output}{\textit{void *coord\_array}}); & \\
\hline
call cg\_coord\_write\_f(\textcolor{input}{fn}, \textcolor{input}{B}, \textcolor{input}{Z}, \textcolor{input}{datatype}, \textcolor{input}{coordname}, \textcolor{input}{coord\_array}, \textcolor{output}{\textit{C}}, & - w m \\
~~~~~\textcolor{output}{\textit{ier}}) & \\
call cg\_coord\_partial\_write\_f(\textcolor{input}{fn}, \textcolor{input}{B}, \textcolor{input}{Z}, \textcolor{input}{datatype}, \textcolor{input}{coordname}, \textcolor{input}{range\_min}, & - w m \\
~~~~~\textcolor{input}{range\_max}, \textcolor{input}{coord\_array}, \textcolor{output}{\textit{C}}, \textcolor{output}{\textit{ier}}) & \\
call cg\_ncoords\_f(\textcolor{input}{fn}, \textcolor{input}{B}, \textcolor{input}{Z}, \textcolor{output}{\textit{ncoords}}, \textcolor{output}{\textit{ier}}) & r - m \\
call cg\_coord\_info\_f(\textcolor{input}{fn}, \textcolor{input}{B}, \textcolor{input}{Z}, \textcolor{input}{C}, \textcolor{output}{\textit{datatype}}, \textcolor{output}{\textit{coordname}}, \textcolor{output}{\textit{ier}}) & r - m \\
call cg\_coord\_read\_f(\textcolor{input}{fn}, \textcolor{input}{B}, \textcolor{input}{Z}, \textcolor{input}{coordname}, \textcolor{input}{datatype}, \textcolor{input}{range\_min}, & r - m \\
~~~~~\textcolor{input}{range\_max}, \textcolor{output}{\textit{coord\_array}}, \textcolor{output}{\textit{ier}}) & \\
\end{fctbox}

\noindent
\textbf{\textcolor{input}{Input}/\textcolor{output}{\textit{Output}}}

\begin{Ventryi}{\fort{coord\_array}}\raggedright
\item [\fort{fn}]
      CGNS file index number.
      (\textcolor{input}{Input})
\item [\fort{B}]
      Base index number, where $1 \leq \text{\fort{B}} \leq \text{\fort{nbases}}$.
      (\textcolor{input}{Input})
\item [\fort{Z}]
      Zone index number, where $1 \leq \text{\fort{Z}} \leq \text{\fort{nzones}}$.
      (\textcolor{input}{Input})
\item [\fort{C}]
      Coordinate array index number, where $1 \leq \text{\fort{C}} \leq \text{\fort{ncoords}}$.
      (\textcolor{input}{Input} for \fort{cg\_coord\_info};
      \textcolor{output}{\textit{output}} for \fort{cg\_coord\_write})
\item [\fort{ncoords}]
      Number of coordinate arrays for zone \fort{Z}.
      (\textcolor{output}{\textit{Output}})
\item [\fort{datatype}]
      Data type in which the coordinate array is written.
      Admissible data types for a coordinate array are \fort{RealSingle}
      and \fort{RealDouble}.
      (\textcolor{input}{Input} for \fort{cg\_coord\_write},
      \fort{cg\_coord\_partial\_write}, \fort{cg\_coord\_read};
      \textcolor{output}{\textit{output}} for \fort{cg\_coord\_info})
\item [\fort{coordname}]
      Name of the coordinate array.
      It is strongly advised to use the SIDS nomenclature conventions
      when naming the coordinate arrays to insure file compatibility.
      (\textcolor{input}{Input} for \fort{cg\_coord\_write},
      \fort{cg\_coord\_partial\_write}, \fort{cg\_coord\_read};
      \textcolor{output}{\textit{output}} for \fort{cg\_coord\_info})
\item [\fort{range\_min}]
      Lower range index (eg., \fort{imin, jmin, kmin}).
      (\textcolor{input}{Input})
\item [\fort{range\_max}]
      Upper range index (eg., \fort{imax, jmax, kmax}).
      (\textcolor{input}{Input})
\item [\fort{coord\_array}]
      Array of coordinate values for the range prescribed.
      (\textcolor{input}{Input} for \fort{cg\_coord\_write};
      \fort{cg\_coord\_partial\_write}, \textcolor{output}{\textit{output}} for \fort{cg\_coord\_read})
\item [\fort{ier}]
      Error status.
      (\textcolor{output}{\textit{Output}})
\end{Ventryi}

The above functions are applicable \emph{only} to the 
\fort{GridCoordinates\_t} node named \fort{GridCoordinates}, used
for the original grid in a zone.
Coordinates for additional \fort{GridCoordinates\_t} nodes in a zone
must be read and written using the \fort{cg\_array\_\textit{xxx}} functions
described in \autoref{s:dataarray}.

When writing, the function \fort{cg\_coord\_write} will
automatically write the full range of coordinates (i.e., the entire
\fort{coord\_array}).
The function \fort{cg\_coord\_partial\_write} may be used to write
only a subset of \fort{coord\_array}.

If the file was opened in ``write mode'', using \fort{cg\_coord\_partial\_write}
will overwrite existing coordinates only for the specified range.
In ``modify mode'', the existing coordinates will first be deleted, then
replaced by the new coordinates.

The function \fort{cg\_coord\_read} returns the coordinate array
\fort{coord\_array}, for the range prescribed by \fort{range\_min} and
\fort{range\_max}.
The array is returned to the application in the data type requested in
\fort{datatype}.
This data type does not need to be the same as the one in which the
coordinates are stored in the file.
A coordinate array stored as double precision in the CGNS file can be
returned to the application as single precision, or vice versa.

In Fortran, when using \fort{cg\_coord\_read\_f} to read 2D or 3D
coordinates, the extent of each dimension of \fort{coord\_array} must
be consistent with the requested range.
When reading a 1D solution, the declared size can be larger than the
requested range.
For example, for a 2D zone with $100 \times 50$ vertices, if
\fort{range\_min} and \fort{range\_max} are set to (11,11) and (20,20)
to read a subset of the coordinates, then \fort{coord\_array} must be
dimensioned (10,10).
If \fort{coord\_array} is declared larger (e.g., (100,50)) the
indices for the returned coordinates will be wrong.

\newpage
\subsection{Element Connectivity}
\label{s:elements}

\noindent
\textit{Node}: \fort{Elements\_t}

\begin{fctbox}
\textcolor{output}{\textit{ier}} = cg\_section\_write(\textcolor{input}{int fn}, \textcolor{input}{int B}, \textcolor{input}{int Z}, \textcolor{input}{char *ElementSectionName}, & - w m \\
~~~~~~\textcolor{input}{ElementType\_t type}, \textcolor{input}{int start}, \textcolor{input}{int end}, \textcolor{input}{int nbndry}, & \\
~~~~~~\textcolor{input}{int *Elements}, \textcolor{output}{\textit{int *S}}); & \\
\textcolor{output}{\textit{ier}} = cg\_section\_partial\_write(\textcolor{input}{int fn}, \textcolor{input}{int B}, \textcolor{input}{int Z}, & - w m \\
~~~~~~\textcolor{input}{char *ElementSectionName}, \textcolor{input}{ElementType\_t type}, \textcolor{input}{int rmin\_elems}, & \\
~~~~~~\textcolor{input}{int rmax\_elems}, \textcolor{input}{int nbndry}, \textcolor{input}{int *Elements}, \textcolor{output}{\textit{int *S}}); & \\
\textcolor{output}{\textit{ier}} = cg\_parent\_data\_write(\textcolor{input}{int fn}, \textcolor{input}{int B}, \textcolor{input}{int Z}, \textcolor{input}{int S}, & - w m \\
~~~~~~\textcolor{output}{\textit{int *ParentData}}); & \\
\textcolor{output}{\textit{ier}} = cg\_parent\_data\_partial\_write(\textcolor{input}{int fn}, \textcolor{input}{int B}, \textcolor{input}{int Z}, \textcolor{input}{int S}, & - w m \\
~~~~~~\textcolor{input}{int rmin\_parent}, \textcolor{input}{int rmax\_parent}, \textcolor{output}{\textit{int *ParentData}}); & \\
\textcolor{output}{\textit{ier}} = cg\_nsections(\textcolor{input}{int fn}, \textcolor{input}{int B}, \textcolor{input}{int Z}, \textcolor{output}{\textit{int *nsections}}); & r - m \\
\textcolor{output}{\textit{ier}} = cg\_section\_read(\textcolor{input}{int fn}, \textcolor{input}{int B}, \textcolor{input}{int Z}, \textcolor{input}{int S}, & r - m \\
~~~~~~\textcolor{output}{\textit{char *ElementSectionName}}, \textcolor{output}{\textit{ElementType\_t *type}}, \textcolor{output}{\textit{int *rmin\_elems}}, & \\
~~~~~~\textcolor{output}{\textit{int *rmax\_elems}}, \textcolor{output}{\textit{int *nbndry}}, \textcolor{output}{\textit{int *parent\_flag}}); & \\
\textcolor{output}{\textit{ier}} = cg\_section\_read\_ext(\textcolor{input}{int fn}, \textcolor{input}{int B}, \textcolor{input}{int Z}, \textcolor{input}{int S}, & r - m \\
~~~~~~\textcolor{output}{\textit{char *ElementSectionName}}, \textcolor{output}{\textit{ElementType\_t *type}}, \textcolor{output}{\textit{int *rmin\_elems}}, & \\
~~~~~~\textcolor{output}{\textit{int *rmax\_elems}}, \textcolor{output}{\textit{int *nbndry}}, \textcolor{output}{\textit{int *parent\_flag}}, & \\
~~~~~~\textcolor{output}{\textit{int *rmin\_parent}}, \textcolor{output}{\textit{int *rmax\_parent}}); & \\
\textcolor{output}{\textit{ier}} = cg\_ElementDataSize(\textcolor{input}{int fn}, \textcolor{input}{int B}, \textcolor{input}{int Z}, \textcolor{input}{int S}, & r - m \\
~~~~~~\textcolor{output}{\textit{int *ElementDataSize}}); & \\
\textcolor{output}{\textit{ier}} = cg\_elements\_read(\textcolor{input}{int fn}, \textcolor{input}{int B}, \textcolor{input}{int Z}, \textcolor{input}{int S}, \textcolor{output}{\textit{int *Elements}}, & r - m \\
~~~~~~\textcolor{output}{\textit{int *ParentData}}); & \\
\textcolor{output}{\textit{ier}} = cg\_elements\_read\_ext(\textcolor{input}{int fn}, \textcolor{input}{int B}, \textcolor{input}{int Z}, \textcolor{input}{int S}, \textcolor{output}{\textit{int *Elements}}, & r - m \\
~~~~~~\textcolor{input}{int *rmin\_elems}, \textcolor{input}{int *rmax\_elems}, \textcolor{output}{\textit{int *ParentData}}, & \\
~~~~~~\textcolor{input}{int *rmin\_parent}, \textcolor{input}{int *rmax\_parent}); & \\
\textcolor{output}{\textit{ier}} = cg\_npe(\textcolor{input}{ElementType\_t type}, \textcolor{output}{\textit{int *npe}}); & r w m \\
\end{fctbox}

\begin{fctbox}
call cg\_section\_write\_f(\textcolor{input}{fn}, \textcolor{input}{B}, \textcolor{input}{Z}, \textcolor{input}{ElementSectionName}, \textcolor{input}{type}, \textcolor{input}{start}, \textcolor{input}{end}, & - w m \\
~~~~~\textcolor{input}{nbndry}, \textcolor{input}{Elements}, \textcolor{output}{\textit{S}}, \textcolor{output}{\textit{ier}}) & \\
call cg\_section\_write\_partial\_f(\textcolor{input}{fn}, \textcolor{input}{B}, \textcolor{input}{Z}, \textcolor{input}{ElementSectionName}, \textcolor{input}{type}, & - w m \\
~~~~~\textcolor{input}{rmin\_elems}, \textcolor{input}{rmax\_elems}, \textcolor{input}{nbndry}, \textcolor{input}{Elements}, \textcolor{output}{\textit{S}}, \textcolor{output}{\textit{ier}}) & \\
call cg\_parent\_data\_write\_f(\textcolor{input}{fn}, \textcolor{input}{B}, \textcolor{input}{Z}, \textcolor{input}{S}, \textcolor{output}{\textit{ParentData}}, \textcolor{output}{\textit{ier}}) & - w m \\
call cg\_parent\_data\_partial\_write\_f(\textcolor{input}{fn}, \textcolor{input}{B}, \textcolor{input}{Z}, \textcolor{input}{S}, \textcolor{input}{rmin\_parent}, & - w m \\
~~~~~\textcolor{input}{rmax\_parent}, \textcolor{output}{\textit{ParentData}}, \textcolor{output}{\textit{ier}}) & \\
call cg\_nsections\_f(\textcolor{input}{fn}, \textcolor{input}{B}, \textcolor{input}{Z}, \textcolor{output}{\textit{nsections}}, \textcolor{output}{\textit{ier}}) & r - m \\
call cg\_section\_read\_f(\textcolor{input}{fn}, \textcolor{input}{B}, \textcolor{input}{Z}, \textcolor{input}{S}, \textcolor{output}{\textit{ElementSectionName}}, \textcolor{output}{\textit{type}}, & r - m \\
~~~~~\textcolor{output}{\textit{rmin\_elems}}, \textcolor{output}{\textit{rmax\_elems}}, \textcolor{output}{\textit{nbndry}}, \textcolor{output}{\textit{parent\_flag}}, \textcolor{output}{\textit{ier}}) & \\
call cg\_section\_read\_ext\_f(\textcolor{input}{fn}, \textcolor{input}{B}, \textcolor{input}{Z}, \textcolor{input}{S}, \textcolor{output}{\textit{ElementSectionName}}, \textcolor{output}{\textit{type}}, & r - m \\
~~~~~\textcolor{output}{\textit{rmin\_elems}}, \textcolor{output}{\textit{rmax\_elems}}, \textcolor{output}{\textit{nbndry}}, \textcolor{output}{\textit{parent\_flag}}, \textcolor{output}{\textit{rmin\_parent}}, & \\
~~~~~\textcolor{output}{\textit{rmax\_parent}}, \textcolor{output}{\textit{ier}}) & \\
call cg\_ElementDataSize\_f(\textcolor{input}{fn}, \textcolor{input}{B}, \textcolor{input}{Z}, \textcolor{input}{S}, \textcolor{output}{\textit{ElementDataSize}}, \textcolor{output}{\textit{ier}}) & r - m \\
call cg\_elements\_read\_f(\textcolor{input}{fn}, \textcolor{input}{B}, \textcolor{input}{Z}, \textcolor{input}{S}, \textcolor{output}{\textit{Elements}}, \textcolor{output}{\textit{ParentData}}, \textcolor{output}{\textit{ier}}) & r - m \\
call cg\_elements\_read\_ext\_f(\textcolor{input}{fn}, \textcolor{input}{B}, \textcolor{input}{Z}, \textcolor{input}{S}, \textcolor{output}{\textit{Elements}}, \textcolor{input}{rmin\_elems}, & r - m \\
~~~~~\textcolor{input}{rmax\_elems}, \textcolor{output}{\textit{ParentData}}, \textcolor{input}{rmin\_parent}, \textcolor{input}{rmax\_parent}, \textcolor{output}{\textit{ier}}) & \\
call cg\_npe\_f(\textcolor{input}{type}, \textcolor{output}{\textit{npe}}, \textcolor{output}{\textit{ier}}) & r w m \\
\end{fctbox}

\noindent
\textbf{\textcolor{input}{Input}/\textcolor{output}{\textit{Output}}}

\begin{Ventryi}{\fort{ElementSectionName}}\raggedright
\item [\fort{fn}]
      CGNS file index number.
      (\textcolor{input}{Input})
\item [\fort{B}]
      Base index number, where $1 \leq \text{\fort{B}} \leq \text{\fort{nbases}}$.
      (\textcolor{input}{Input})
\item [\fort{Z}]
      Zone index number, where $1 \leq \text{\fort{Z}} \leq \text{\fort{nzones}}$.
      (\textcolor{input}{Input})
\item [\fort{ElementSectionName}]
      Name of the \fort{Elements\_t} node.
      (\textcolor{input}{Input} for \fort{cg\_section\_write};
      \textcolor{output}{\textit{output}} for \fort{cg\_section\_read})
\item [\fort{type}]
      Type of element.
      See the eligible types for \fort{ElementType\_t} in \autoref{s:typedefs}.
      (\textcolor{input}{Input} for \fort{cg\_section\_write}, \fort{cg\_npe};
      \textcolor{output}{\textit{output}} for \fort{cg\_section\_read})
\item [\fort{start}]
      Index of first element in the section.
      (\textcolor{input}{Input} for \fort{cg\_section\_write};
      \textcolor{output}{\textit{output}} for \fort{cg\_section\_read},
      \fort{cg\_section\_read\_ext})
\item [\fort{end}]
      Index of last element in the section.
      (\textcolor{input}{Input} for \fort{cg\_section\_write};
      \textcolor{output}{\textit{output}} for \fort{cg\_section\_read},
      \fort{cg\_section\_read\_ext})
\item [\fort{rmin\_elems}]
      Lower element index.
      (\textcolor{input}{Input})
\item [\fort{rmax\_elems}]
      Upper element index.
      (\textcolor{input}{Input})
\item [\fort{rmin\_parent}]
      Lower element index for parent data.
      (\textcolor{input}{Input} for \fort{cg\_parent\_data\_partial\_write},
      \fort{cg\_elements\_read\_ext};
      \textcolor{output}{\textit{output}} for \fort{cg\_section\_read\_ext})
\item [\fort{rmax\_parent}]
      Upper element index for parent data.
      (\textcolor{input}{Input} for \fort{cg\_parent\_data\_partial\_write},
      \fort{cg\_elements\_read\_ext};
      \textcolor{output}{\textit{output}} for \fort{cg\_section\_read\_ext})
\item [\fort{nbndry}]
      Index of last boundary element in the section.
      Set to zero if the elements are unsorted.
      (\textcolor{input}{Input} for \fort{cg\_section\_write};
      \textcolor{output}{\textit{output}} for \fort{cg\_section\_read})
\item [\fort{nsections}]
      Lower range index (eg., \fort{imin, jmin, kmin}).
      (\textcolor{output}{\textit{Output}})
\item [\fort{S}]
      Element section index, where $1 \leq \text{\fort{S}} \leq \text{\fort{nsections}}$.
      (\textcolor{input}{Input} for \fort{cg\_parent\_data\_write},
      \fort{cg\_section\_read}, \fort{cg\_ElementDataSize},
      \fort{cg\_elements\_read};
      \textcolor{output}{\textit{output}} for \fort{cg\_section\_write})
\item [\fort{parent\_flag}]
      Flag indicating if the parent data are defined.
      If the parent data exist, \fort{parent\_flag} is set to 1;
      otherwise it is set to 0.
      (\textcolor{output}{\textit{Output}})
\item [\fort{ElementDataSize}]
      Number of element connectivity data values.
      (\textcolor{output}{\textit{Output}})
\item [\fort{Elements}]
      Element connectivity data.
      (\textcolor{input}{Input} for \fort{cg\_section\_write};
      \textcolor{output}{\textit{output}} for \fort{cg\_elements\_read})
\item [\fort{ParentData}]
      For boundary or interface elements, this array contains
      information on the cell(s) and cell face(s) sharing the element.
      (\textcolor{output}{\textit{Output}})
\item [\fort{npe}]
      Number of nodes for an element of type \fort{type}.
      (\textcolor{output}{\textit{Output}})
\item [\fort{ier}]
      Error status.
      (\textcolor{output}{\textit{Output}})
\end{Ventryi}

If the specified \fort{Elements\_t} node doesn't yet exist,
it may be created using either \fort{cg\_sec\-tion\_write} or
\fort{cg\_section\_partial\_write}.
If the \fort{Elements\_t} node does exist (created by using either
\fort{cg\_section\_write} or \fort{cg\_section\_partial\_write}),
calls to \fort{cg\_section\_partial\_write} will replace the existing
connectivity data with the union of the existing and new data, with the
new data overwriting the old data for common indices.
The resulting index range for the new data must be continuous.

The function \fort{cg\_parent\_data\_write} will automatically write the
full range of parent data (i.e., the entire \fort{ParentData} array).
If parent data does not exist for the specified \fort{Elements\_t}
node, the first use of \fort{cg\_parent\_data\_partial\_write} will add
it.
If parent data does exist (created by using either
\fort{cg\_parent\_data\_write} or \fort{cg\_par\-ent\_data\_par\-tial\_write}),
calls to \fort{cg\_par\-ent\_data\_par\-tial\_write} will replace the existing
parent data with the union of the existing and new data, with the new
data overwriting the old data for common indices.
The resulting index range for the new data must be continuous.

The ``partial write'' capability described above requires that the file be
opened in ``write mode''.
For both \fort{cg\_section\_partial\_write} and
\fort{cg\_parent\_data\_partial\_write}, if the file was opened in ``modify
mode'', the existing connectivity or parent data will first be deleted,
then replaced by the new data.

The function \fort{cg\_section\_read\_ext} returns the same information
as \fort{cg\_section\_read}, plus the index range for the parent data.

The function \fort{cg\_elements\_read} returns all of the element
connectivity and parent data.
Specified subsets of the element connectivity and parent data may be
read using \fort{cg\_elements\_read\_ext}.

\subsection{Axisymmetry}
\label{s:axisymmetry}

\noindent
\textit{Node}: \fort{Axisymmetry\_t}

\begin{fctbox}
\textcolor{output}{\textit{ier}} = cg\_axisym\_write(\textcolor{input}{int fn}, \textcolor{input}{int B}, \textcolor{input}{float *ReferencePoint}, & - w m \\
~~~~~~\textcolor{input}{float *AxisVector}); & \\
\textcolor{output}{\textit{ier}} = cg\_axisym\_read(\textcolor{input}{int fn}, \textcolor{input}{int B}, \textcolor{output}{\textit{float *ReferencePoint}}, & r - m \\
~~~~~~\textcolor{output}{\textit{float *AxisVector}}); & \\
\hline
call cg\_axisym\_write\_f(\textcolor{input}{fn}, \textcolor{input}{B}, \textcolor{input}{ReferencePoint}, \textcolor{input}{AxisVector}, \textcolor{output}{\textit{ier}}) & - w m \\
call cg\_axisym\_read\_f(\textcolor{input}{fn}, \textcolor{input}{B}, \textcolor{output}{\textit{ReferencePoint}}, \textcolor{output}{\textit{AxisVector}}, \textcolor{output}{\textit{ier}}) & r - m \\
\end{fctbox}

\noindent
\textbf{\textcolor{input}{Input}/\textcolor{output}{\textit{Output}}}

\begin{Ventryi}{\fort{ReferencePoint}}\raggedright
\item [\fort{fn}]
      CGNS file index number.
      (\textcolor{input}{\textit{Input}})
\item [\fort{B}]
      Base index number, where $1 \leq \text{\fort{B}} \leq \text{\fort{nbases}}$.
      (\textcolor{input}{\textit{Input}})
\item [\fort{ReferencePoint}]
      Origin used for defining the axis of rotation.
      (\textcolor{input}{Input} for \fort{cg\_axisym\_write};
      \textcolor{output}{\textit{output}} for \fort{cg\_axisym\_read})
\item [\fort{AxisVector}]
      Direction cosines of the axis of rotation, through the reference
      point.
      (\textcolor{input}{Input} for \fort{cg\_axisym\_write};
      \textcolor{output}{\textit{output}} for \fort{cg\_axisym\_read})
\item [\fort{ier}]
      Error status.
      (\textcolor{output}{\textit{Output}})
\end{Ventryi}

This node can only be used for a bi-dimensional model, i.e.,
\fort{PhysicalDimension} must equal two.

\subsection{Rotating Coordinates}
\label{s:rotatingcoordinates}

\noindent
\textit{Node}: \fort{RotatingCoordinates\_t}

\begin{fctbox}
\textcolor{output}{\textit{ier}} = cg\_rotating\_write(\textcolor{input}{float *RotationRateVector}, & - w m \\
~~~~~~\textcolor{input}{float *RotationCenter}); & \\
\textcolor{output}{\textit{ier}} = cg\_rotating\_read(\textcolor{output}{\textit{float *RotationRateVector}}, & r - m \\
~~~~~~\textcolor{output}{\textit{float *RotationCenter}}); & \\
\hline
call cg\_rotating\_write\_f(\textcolor{input}{RotationRateVector}, \textcolor{input}{RotationCenter}, \textcolor{output}{\textit{ier}}) & - w m \\
call cg\_rotating\_read\_f(\textcolor{output}{\textit{RotationRateVector}}, \textcolor{output}{\textit{RotationCenter}}, \textcolor{output}{\textit{ier}}) & r - m \\
\end{fctbox}

\noindent
\textbf{\textcolor{input}{Input}/\textcolor{output}{\textit{Output}}}

\begin{Ventryi}{\fort{RotationRateVector}}\raggedright
\item [\fort{RotationRateVector}]
      Components of the angular velocity of the grid about the center
      of rotation.
      (\textcolor{input}{Input} for \fort{cg\_rotating\_write};
      \textcolor{output}{\textit{output}} for \fort{cg\_rotating\_read})
\item [\fort{RotationCenter}]
      Coordinates of the center of rotation.
      (\textcolor{input}{Input} for \fort{cg\_rotating\_write};
      \textcolor{output}{\textit{output}} for \fort{cg\_rotating\_read})
\item [\fort{ier}]
      Error status.
      (\textcolor{output}{\textit{Output}})
\end{Ventryi}
