\section{Location and Position}
\label{s:location}
\thispagestyle{plain}

\subsection{Grid Location}
\label{s:gridlocation}

\noindent
\textit{Node}: \fort{GridLocation\_t}

\begin{fctbox}
\textcolor{output}{\textit{ier}} = cg\_gridlocation\_write(\textcolor{input}{GridLocation\_t GridLocation}); & - w m \\
\textcolor{output}{\textit{ier}} = cg\_gridlocation\_read(\textcolor{output}{\textit{GridLocation\_t *GridLocation}}); & r - m \\
\hline
call cg\_gridlocation\_write\_f(\textcolor{input}{GridLocation}, \textcolor{output}{\textit{ier}}) & - w m \\
call cg\_gridlocation\_read\_f(\textcolor{output}{\textit{GridLocation}}, \textcolor{output}{\textit{ier}}) & r - m \\
\end{fctbox}

\noindent
\textbf{\textcolor{input}{Input}/\textcolor{output}{\textit{Output}}}

\begin{Ventryi}{\fort{GridLocation}}\raggedright
\item [\fort{GridLocation}]
      Location in the grid.
      The admissible locations are \fort{Null}, \fort{UserDefined},
      \fort{Vertex}, \fort{CellCenter}, \fort{FaceCenter},
      \fort{IFaceCenter}, \fort{JFaceCenter}, \fort{KFaceCenter},
      and \fort{EdgeCenter}.
      (\textcolor{input}{Input} for \fort{cg\_gridlocation\_write};
      \textcolor{output}{\textit{output}} for \fort{cg\_gridlocation\_read})
\item [\fort{ier}]
      Error status.
      (\textcolor{output}{\textit{Output}})
\end{Ventryi}

\subsection{Point Sets}
\label{s:ptset}

\noindent
\textit{Node}: \fort{IndexArray\_t}, \fort{IndexRange\_t}

\begin{fctbox}
\textcolor{output}{\textit{ier}} = cg\_ptset\_write(\textcolor{input}{PointSetType\_t *ptset\_type}, \textcolor{input}{int npnts}, \textcolor{input}{int *pnts}); & - w m \\
\textcolor{output}{\textit{ier}} = cg\_ptset\_info(\textcolor{output}{\textit{PointSetType\_t *ptset\_type}}, \textcolor{output}{\textit{int *npnts}}); & r - m \\
\textcolor{output}{\textit{ier}} = cg\_ptset\_read(\textcolor{output}{\textit{int *pnts}}); & r - m \\
\hline
call cg\_ptset\_write\_f(\textcolor{input}{ptset\_type}, \textcolor{input}{npnts}, \textcolor{input}{pnts}, \textcolor{output}{\textit{ier}}) & - w m \\
call cg\_ptset\_info\_f(\textcolor{output}{\textit{ptset\_type}}, \textcolor{output}{\textit{npnts}}, \textcolor{output}{\textit{ier}}) & r - m \\
call cg\_ptset\_read\_f(\textcolor{output}{\textit{pnts}}, \textcolor{output}{\textit{ier}}) & r - m \\
\end{fctbox}

\noindent
\textbf{\textcolor{input}{Input}/\textcolor{output}{\textit{Output}}}

\begin{Ventryi}{\fort{ptset\_type}}\raggedright
\item [\fort{ptset\_type}]
      The point set type; either \fort{PointRange} for a range of
      points or cells, or \fort{PointList} for a list of discrete
      points or cells.
      (\textcolor{input}{Input} for \fort{cg\_ptset\_write};
      \textcolor{output}{\textit{output}} for \fort{cg\_ptset\_info})
\item [\fort{npnts}]
      The number of points or cells in the point set.
      For a point set type of \fort{PointRange}, \fort{npnts} is
      always two.
      For a point set type of \fort{PointList}, \fort{npnts} is
      the number of points or cells in the list.
      (\textcolor{input}{Input} for \fort{cg\_ptset\_write};
      \textcolor{output}{\textit{output}} for \fort{cg\_ptset\_info})
\item [\fort{pnts}]
      The array of point or cell indices defining the point set.
      There should be \fort{npnts} values, each of dimension
      \fort{IndexDimension}
      (i.e., 1 for unstructured grids, and 2 or 3 for structured grids
      with 2-D or 3-D elements, respectively).
      (\textcolor{input}{Input} for \fort{cg\_ptset\_write};
      \textcolor{output}{\textit{output}} for \fort{cg\_ptset\_read})
\item [\fort{ier}]
      Error status.
      (\textcolor{output}{\textit{Output}})
\end{Ventryi}

These functions may be used to write and read point set data (i.e., an
\fort{IndexArray\_t} node named \fort{PointList}, or an
\fort{IndexRange\_t} node named \fort{PointRange}).
They are only applicable at nodes that are descendents of a
\fort{Zone\_t} node.

\subsection{Rind Layers}
\label{s:rind}

\noindent
\textit{Node}: \fort{Rind\_t}

\begin{fctbox}
\textcolor{output}{\textit{ier}} = cg\_rind\_write(\textcolor{input}{int *RindData}); & - w m \\
\textcolor{output}{\textit{ier}} = cg\_rind\_read(\textcolor{output}{\textit{int *RindData}}); & r - m \\
\hline
call cg\_rind\_write\_f(\textcolor{input}{RindData}, \textcolor{output}{\textit{ier}}) & - w m \\
call cg\_rind\_read\_f(\textcolor{output}{\textit{RindData}}, \textcolor{output}{\textit{ier}}) & r - m \\
\end{fctbox}

\noindent
\textbf{\textcolor{input}{Input}/\textcolor{output}{\textit{Output}}}

\begin{Ventryi}{\fort{RindData}}\raggedright
\item [\fort{RindData}]
      Number of rind layers for each computational direction (structured
      grid) or number of rind points or elements (unstructured grid).
      (\textcolor{input}{Input} for \fort{cg\_rind\_write};
      \textcolor{output}{\textit{output}} for \fort{cg\_rind\_read})
\item [\fort{ier}]
      Error status.
      (\textcolor{output}{\textit{Output}})
\end{Ventryi}

When writing rind data for elements, \texttt{cg\_section\_write} must be
called first (see \autoref{s:elements}), followed by \texttt{cg\_goto}
(\autoref{s:navigating}) to access the \texttt{Elements\_t} node, and
then \texttt{cg\_rind\_write}.
