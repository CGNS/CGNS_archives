\section{Navigating a CGNS File}
\label{s:navigating}
\thispagestyle{plain}

\subsection{Accessing a Node}
\label{s:goto}

\begin{fctbox}
\textcolor{output}{\textit{ier}} = cg\_goto(\textcolor{input}{int fn}, \textcolor{input}{int B}, ..., \textcolor{input}{"end"}); & r w m \\
\hline
call cg\_goto\_f(\textcolor{input}{fn}, \textcolor{input}{B}, \textcolor{output}{\textit{ier}}, ..., \textcolor{input}{'end'}) & r w m \\
\end{fctbox}

\noindent
\textbf{\textcolor{input}{Input}/\textcolor{output}{\textit{Output}}}

\begin{Ventryi}{\fort{end}}\raggedright
\item [\fort{fn}]
      CGNS file index number.
      (\textcolor{input}{Input})
\item [\fort{B}]
      Base index number, where $1 \leq \text{\fort{B}} \leq \text{\fort{nbases}}$.
      (\textcolor{input}{Input})
\item [\fort{...}]
      Variable argument list used to specify the path to a node.
      It is composed of an unlimited list of pair-arguments.
      Each pair-argument takes the form \fort{CGNS\_NodeLabel, NodeIndex}.
      An example is \fort{Zone\_t, ZoneIndex}.

      There is one exception to this rule.
      When accessing a \fort{BCData\_t} node, the index must be set to
      either \fort{Dirichlet} or \fort{Neumann} since only these
      two types are allowed.
      (Note that \fort{Dirichlet} and \fort{Neumann} are defined in
      the include files \textit{cgnslib.h} and \textit{cgnslib\_f.h}).
      See the example below.
      (\textcolor{input}{Input})
\item [\fort{end}]
      The character string \fort{"end"} (or \fort{'end'} for the Fortran
      function) must be the last argument.
      It is used to indicate the end of the argument list.
      (\textcolor{input}{Input})
\item [\fort{ier}]
      Error status.
      The possible values, with the corresponding C names (or Fortran
      parameters) defined in \textit{cgnslib.h} (or \textit{cgnslib\_f.h})
      are listed below.

      \begin{tabular}{@{}c l}
         \uline{\ital{Value}} & \uline{\ital{Name/Parameter}} \\
         0 & \fort{CG\_OK} \\
         1 & \fort{CG\_ERROR} \\
         2 & \fort{CG\_NODE\_NOT\_FOUND} \\
         3 & \fort{CG\_INCORRECT\_PATH}
      \end{tabular}

      For non-zero values, an error message may be printed using
      \fort{cg\_error\_print()}, as described in \autoref{s:error}.
      (\textcolor{output}{\textit{Output}})
\end{Ventryi}

This function allows access to any parent-type nodes in a CGNS file.
A parent-type node is one that can have children.
Nodes that cannot have children, like \fort{Descriptor\_t}, are
not supported by this function.

\noindent
\uline{\textit{Example}}
\begin{indlefttt}
ier = cg\_goto(fn, B, "Zone\_t", Z, "FlowSolution\_t", F, "end");
call cg\_goto\_f(fn, B, ier, 'Zone\_t', Z, 'GasModel\_t', 1, 'DataArray\_t',
               A, 'end')
call cg\_goto\_f(fn, B, 'Zone\_t', Z, 'ZoneBC\_t', 1, 'BC\_t', BC,
               'BCDataSet\_t', S, 'BCData\_t', Dirichlet)
\end{indlefttt}

\subsection{Deleting a Node}
\label{s:delete}

\begin{fctbox}
\textcolor{output}{\textit{ier}} = cg\_delete\_node(\textcolor{input}{char *NodeName}); & - - m \\
\hline
call cg\_delete\_node\_f(\textcolor{input}{NodeName}, \textcolor{output}{\textit{ier}}) & - - m \\
\end{fctbox}

\noindent
\textbf{\textcolor{input}{Input}/\textcolor{output}{\textit{Output}}}


\begin{Ventryi}{\fort{NodeName}}\raggedright
\item [\fort{NodeName}]
      Name of the child to be deleted.
      (\textcolor{input}{Input})
\item [\fort{ier}]
      Error status.
      (\textcolor{output}{\textit{Output}})
\end{Ventryi}

The function \fort{cg\_delete\_node} is used is conjunction with
\fort{cg\_goto}.
Once positioned at a parent node with \fort{cg\_goto}, a child of
this node can be deleted with \fort{cg\_delete\_node}.
This function requires a single argument, \fort{NodeName}, which is
the name of the child to be deleted.

Since the highest level that can be pointed to with \fort{cg\_goto} is
a base node for a CGNS database
(\fort{CGNSBase\_t}), the
highest-level nodes that can be deleted are the children of a
\fort{CGNSBase\_t} node.
In other words, nodes located directly under the ADF (or HDF) root node
(\fort{CGNSBase\_t} and
\fort{CGNSLibraryVersion\_t})
can not be deleted with \fort{cg\_delete}.

A few other nodes are not allowed to be deleted from the database
because these are required nodes as defined by the SIDS, and deleting
them would make the file non-CGNS compliant.
These are:
\begin{itemize*}
\item Under \fort{Zone\_t}:
      \fort{ZoneType} 
\item Under \fort{GridConnectivity1to1\_t}:
      \fort{PointRange, PointRangeDonor, Transform}
\item Under \fort{OversetHoles\_t}:
      \fort{PointList} and any \fort{IndexRange\_t}
\item Under \fort{GridConnectivity\_t}:
      \fort{PointRange, PointList, CellListDonor, PointListDonor}
\item Under \fort{BC\_t}:
      \fort{PointList, PointRange}
\item Under \fort{GeometryReference\_t}:
      \fort{GeometryFile, GeometryFormat}
\item Under \fort{Elements\_t}:
      \fort{ElementRange, ElementConnectivity}
\item Under \fort{Gravity\_t}:
      \fort{GravityVector}
\item Under \fort{Axisymmetry\_t}:
      \fort{AxisymmetryReferencePoint, AxisymmetryAxisVector}
\item Under \fort{RotatingCoordinates\_t}:
      \fort{RotationCenter, RotationRateVector}
\item Under \fort{Periodic\_t}:
      \fort{RotationCenter, RotationAngle, Translation}
\item Under \fort{AverageInterface\_t}:
      \fort{AverageInterfaceType}
\item Under \fort{WallFunction\_t}:
      \fort{WallFunctionType}
\item Under \fort{Area\_t}:
      \fort{AreaType, SurfaceArea, RegionName}
\end{itemize*}

When a child node is deleted, both the database and the file on disk are
updated to remove the node.
One must be careful not to delete a node from within a loop of that node
type.
For example, if the number of zones below a \fort{CGNSBase\_t} node is
\fort{nzones}, a zone should never be deleted from within a zone loop!
By deleting a zone, the total number of zones (\fort{nzones}) changes,
as well as the zone indexing.
Suppose for example that \fort{nzones} is 5, and that the third zone
is deleted.
After calling \fort{cg\_delete\_node}, \fort{nzones} is changed to 4,
and the zones originally indexed 4 and 5 are now indexed 3 and 4.
