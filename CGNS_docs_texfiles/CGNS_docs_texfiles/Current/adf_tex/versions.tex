\section{ADF File Version Control Numbering}
\label{s:versions}
\thispagestyle{plain}

The ADF file version control number scheme is described below.
The format for the version number is a field of six digits or characters:
\begin{indlefttt}
AXXxxx
\end{indlefttt}
where:

\begin{Ventryi}{\fort{xxx}}
\item [\fort{A}]
      Major revision number.
      Major internal structure changes.
      This number is not expected to change very often, if at all,
      because the backward compatibility is only available by explicit
      policy decision.

      One alphabetic character.

      Range of values: \fort{A-Z}, \fort{a-z}

      In the unlikely event of reaching \fort{z}, then use any other
      unused printable ASCII character, except the blank or symbols
      used by the ``\textit{what}'' command 
\begin{indlefttt}
@()#~>\textbackslash
\end{indlefttt}
\item [\fort{XX}]
      Minor revision number.
      New features, minor changes, and bug fixes.
      Backward but not forward compatible.

      Two-digit hexadecimal number (uppercase letters).

      Range of values: \fort{00-FF}

      Reset to \fort{00} with changes in major revision number.
\item [\fort{xxx}]
      Incremental number.
      Incremented with every new version of the library (even if no
      changes are made to the file format).
      Files are forward and backward compatible.

      Three-digit hexadecimal number (lowercase letters).

      Range of values: \fort{000-fff}

      Does not reset.
\end{Ventryi}

\noindent
The following definitions are used:

\begin{Ventryi}{Backward compatible}
\item [Forward compatible]
      Older versions of libraries can read and write to files created
      by newer versions of libraries.
\item [Backward compatible]
      Newer versions of libraries can read and write to files created
      by older versions of libraries.
\end{Ventryi}

