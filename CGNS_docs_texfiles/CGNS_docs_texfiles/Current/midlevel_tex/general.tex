\section{General Remarks}
\label{s:general}
\thispagestyle{plain}

\subsection{Acquiring the Software and Documentation}

The CGNS Mid-Level Library may be downloaded from SourceForge, at
{\itshape\url{http://sourceforge.net/projects/cgns}}.
This manual, as well as the other CGNS documentation, is available in
both HTML and PDF format from the CGNS documentation web site, at
{\itshape\url{http://www.grc.nasa.gov/www/cgns/CGNS_docs_current/}}.

\subsection{Organization of This Manual}

The sections that follow describe the Mid-Level Library functions
in detail.
The first three sections cover some basic file operations (i.e.,
opening and closing a CGNS file, and some configuration options)
(\autoref{s:fileops}), accessing a specific node in a CGNS database
(\autoref{s:navigating}), and error handling (\autoref{s:error}).
The remaining sections describe the functions used to read, write, and
modify nodes and data in a CGNS database.
These sections basically follow the organization used in
the ``Detailed CGNS Node Descriptions'' section of the
\href{../filemap/filemap.pdf}{SIDS File Mapping} manual.

At the start of each sub-section is a \textit{Node} line, listing the
the applicable CGNS node label.

Next is a table illustrating the syntax for the Mid-Level Library
functions.
The C functions are shown in the top half of the table, followed by the
corresponding Fortran routines in the bottom half of the table.
Input variables are shown in an \textcolor{input}{\texttt{upright blue}}
font, and output variables are shown in a
\textcolor{output}{\texttt{\textit{slanted red}}} font.
As of Version 3.1, some of the arguments to the Mid-Level Library have
changed from \texttt{int} to \texttt{cgsize\_t} in order to support 64-bit data.
For each function, the right-hand column lists the modes (read, write,
and/or modify) applicable to that function.

The input and output variables are then listed and defined.

\subsection{Language}

The CGNS Mid-Level Library is written in C, but each function has a
Fortran counterpart.
All function names start with ``\texttt{cg\_}''.
The Fortran functions have the same name as their C counterpart with the
addition of the suffix ``\texttt{\_f}''.

\subsection{Character Strings}

All data structure names and labels in CGNS are limited to 32
characters.
When reading a file, it is advised to pre-allocate the
character string variables to 32 characters in Fortran, and 33
in C (to include the string terminator).
Other character strings, such as the CGNS file name or
descriptor text, are unlimited in length.
The space for unlimited length character strings will be created by the
Mid-Level Library; it is then the responsibility of the application to
release this space by a call to \texttt{cg\_free}, described in
\autoref{s:free}.

\subsection{Error Status}

All C functions return an integer value representing the error
status.
All Fortran functions have an additional parameter, \texttt{ier},
which contains the value of the error status.
An error status different from zero implies that an error occured.
The error message can be printed using the error
handling functions of the CGNS library, described in \autoref{s:error}.
The error codes are coded in the C and Fortran include files
\textit{cgnslib.h} and \textit{cgnslib\_f.h}.

\subsection{Typedefs}
\label{s:typedefs}

Beginning with version 3.1, two new typedef variables have been
introduced to support 64-bit mode. The \texttt{cglong\_t} typedef
is always a 64-bit integer, and \texttt{cgsize\_t} will be either
a 32-bit or 64-bit integer depending on how the library was built.
Many of the C functions in the MLL have been changed
to to use \texttt{cgsize\_t} instead of \texttt{int} in the arguments.
These functions include any that may exceed the \texttt{2Gb} limit
of an \texttt{int}, e.g. zone dimensions, element data, boundary
conditions, and connectivity. In Fortran, all integer data is taken
to be \texttt{integer*4} for 32-bit and \texttt{integer*8} for 64-bit
builds.

Several types of variables are defined using typedefs in the
\textit{cgnslib.h} file.
These are intended to facilitate the implementation of
CGNS in C.
These variable types are defined as an enumeration of key
words admissible for any variable of these types.
The file \textit{cgnslib.h} must be included in any C application
programs which use these data types.

In Fortran, the same key words are defined as integer parameters in the
include file \textit{cgnslib\_f.h}.
Such variables should be declared as \texttt{integer} in Fortran
applications.
The file \textit{cgnslib\_f.h} must be included in any
Fortran application using these key words.

Note that the first two enumerated values in these lists,
\textit{xxxNull} and \textit{xxxUserDefined}, are only available
in the C interface, and are provided in the advent that your
C compiler does strict type checking. In Fortran, these values
are replaced by the numerically equivalent \texttt{CG\_Null}
and \texttt{CG\_UserDefined}. These values are also defined in the
C interface, thus either form may be used. The function prototypes
for the MLL use \texttt{CG\_Null} and \texttt{CG\_UserDefined},
rather than the more specific values.

The list of enumerated values (key words) for each of these variable
types (typedefs) are:
        
{\raggedright
\begin{Ventryi}{\texttt{AverageInterfaceType\_t}}
   \item [\texttt{ZoneType\_t}]
         \texttt{ZoneTypeNull, ZoneTypeUserDefined, Structured, Unstructured}
   \item [\texttt{ElementType\_t}]
         \texttt{ElementTypeNull, ElementTypeUserDefined, NODE, BAR\_2, BAR\_3, TRI\_3, TRI\_6, QUAD\_4,
         QUAD\_8, QUAD\_9, TETRA\_4, TETRA\_10, PYRA\_5,
         PYRA\_14, PENTA\_6, PENTA\_15, PENTA\_18, HEXA\_8,
         HEXA\_20, HEXA\_27, MIXED, PYRA\_13, NGON\_n, NFACE\_n}
   \item [\texttt{DataType\_t}]
         \texttt{DataTypeNull, DataTypeUserDefined, Integer, RealSingle, RealDouble, Character}
   \item [\texttt{DataClass\_t}]
         \texttt{DataClassNull, DataClassUserDefined, Dimensional, NormalizedByDimensional,
         NormalizedByUnknownDimensional, NondimensionalParameter,
         DimensionlessConstant}
   \item [\texttt{MassUnits\_t}]
         \texttt{MassUnitsNull, MassUnitsNullUserDefined, Kilogram, Gram, Slug, PoundMass}
   \item [\texttt{LengthUnits\_t}]
         \texttt{LengthUnitsNull, LengthUnitsUserDefined, Meter, Centimeter, Millimeter, Foot,
         Inch}
   \item [\texttt{TimeUnits\_t}]
         \texttt{TimeUnitsNull, TimeUnitsUserDefined, Second}
   \item [\texttt{TemperatureUnits\_t}]
         \texttt{TemperatureUnitsNull, TemperatureUnitsUserDefined, Kelvin, Celsius, Rankine,
         Fahrenheit}
   \item [\texttt{AngleUnits\_t}]
         \texttt{AngleUnitsNull, AngleUnitsUserDefined, Degree, Radian}
   \item [\texttt{ElectricCurrentUnits\_t}]
         \texttt{ElectricCurrentUnitsNull, ElectricCurrentUnitsUserDefined, Ampere, Abampere, Statampere, Edison,
         auCurrent}
   \item [\texttt{SubstanceAmountUnits\_t}]
         \texttt{SubstanceAmountUnitsNull, SubstanceAmountUnitsUserDefined, Mole, Entities, StandardCubicFoot,
         StandardCubicMeter}
   \item [\texttt{LuminousIntensityUnits\_t}]
         \texttt{LuminousIntensityUnitsNull, LuminousIntensityUnitsUserDefined, Candela, Candle, Carcel, Hefner, Violle}
   \item [\texttt{GoverningEquationsType\_t}]
         \texttt{GoverningEquationsTypeNull, GoverningEquationsTypeUserDefined, FullPotential, Euler, NSLaminar,
         NSTurbulent, NSLaminarIncompressible,
         NSTurbulentIncompressible}
   \item [\texttt{ModelType\_t}]
         \texttt{ModelTypeNull, ModelTypeUserDefined, Ideal, VanderWaals, Constant,
         PowerLaw, SutherlandLaw, ConstantPrandtl, EddyViscosity,
         ReynoldsStress, ReynoldsStressAlgebraic,
         Algebraic\_BaldwinLomax, Algebraic\_CebeciSmith,
         HalfEquation\_JohnsonKing, OneEquation\_BaldwinBarth,
         OneEquation\_SpalartAllmaras, TwoEquation\_JonesLaunder,
         TwoEquation\_MenterSST, TwoEquation\_Wilcox,
	 CaloricallyPerfect, ThermallyPerfect,
	 ConstantDensity, RedlichKwong,
	 Frozen, ThermalEquilib, ThermalNonequilib,
	 ChemicalEquilibCurveFit, ChemicalEquilibMinimization,
	 ChemicalNonequilib, EMElectricField, EMMagneticField,
         EMConductivity, Voltage, Interpolated,
         Equilibrium\_LinRessler, Chemistry\_LinRessler}
   \item [\texttt{GridLocation\_t}]
         \texttt{GridLocationNull, GridLocationUserDefined, Vertex, IFaceCenter, CellCenter, JFaceCenter,
         FaceCenter, KFaceCenter, EdgeCenter}
   \item [\texttt{GridConnectivityType\_t}]
         \texttt{GridConnectivityTypeNull, GridConnectivityTypeUserDefined, Overset, Abutting, Abutting1to1}
   \item [\texttt{PointSetType\_t}]
         \texttt{PointSetTypeNull, PointSetTypeUserDefined, PointList, PointRange, PointListDonor, PointRangeDonor,
         ElementList, ElementRange, CellListDonor}
   \item [\texttt{BCType\_t}]
         \texttt{BCTypeNull, BCTypeUserDefined, BCAxisymmetricWedge,
         BCDegenerateLine, BCExtrapolate,
         BCDegeneratePoint, BCDirichlet, BCFarfield, BCNeumann,
         BCGeneral, BCInflow, BCOutflow, BCInflowSubsonic,
         BCOutflowSubsonic, BCInflowSupersonic,
         BCOutflowSupersonic, BCSymmetryPlane, BCTunnelInflow,
         BCSymmetryPolar, BCTunnelOutflow, BCWallViscous,
         BCWall, BCWallViscousHeatFlux, BCWallInviscid,
         BCWallViscousIsothermal, FamilySpecified}
   \item [\texttt{BCDataType\_t}]
         \texttt{BCDataTypeNull, BCDataTypeUserDefined, Dirichlet, Neumann}
   \item [\texttt{RigidGridMotionType\_t}]
         \texttt{RigidGridMotionTypeNull, RigidGridMotionTypeUserDefined, ConstantRate, VariableRate}
   \item [\texttt{ArbitraryGridMotionType\_t}]
         \texttt{ArbitraryGridMotionTypeNull, ArbitraryGridMotionTypeUserDefined, NonDeformingGrid, DeformingGrid}
   \item [\texttt{SimulationType\_t}]
         \texttt{SimulationTypeNull, SimulationTypeUserDefined, TimeAccurate, NonTimeAccurate}
   \item [\texttt{WallFunctionType\_t}]
         \texttt{WallFunctionTypeNull, WallFunctionTypeUserDefined, Generic}
   \item [\texttt{AreaType\_t}]
         \texttt{AreaTypeNull, AreaTypeUserDefined, BleedArea, CaptureArea}
   \item [\texttt{AverageInterfaceType\_t}]
         \texttt{AverageInterfaceTypeNull, AverageInterfaceTypeUserDefined, AverageAll, AverageCircumferential, AverageRadial, AverageI,
	 AverageJ, AverageK}
\end{Ventryi}
}

\subsection{Character Names For Typedefs}

The CGNS library defines character arrays which map the typedefs above
to character strings.
These are global arrays dimensioned to the size of each list of
typedefs.
To retrieve a character string representation of a typedef, use the
typedef value as an index to the appropiate character array.
For example, to retrieve the string ``\texttt{Meter}''
for the \texttt{LengthUnits\_t Meter} typedef, use
\texttt{LengthUnitsName[Meter]}.
Functions are available to retrieve these names without the need for
direct global data access.
These functions also do bounds checking on the input, and if out of
range, will return the string ``\texttt{<invalid>}''.
An additional benefit is that these will work from within a Windows DLL,
and are thus the recommended access technique.
The routines have the same name as the global data arrays, but with a
``\texttt{cg\_}'' prepended.
For the example above, use ``\texttt{cg\_LengthUnitsName(Meter)}''.

\begin{table}
\begin{center}
\begin{tabular}{l}
\hline
Typedef Name Access Functions \\
\hline
\textcolor{output}{const char *name} = cg\_MassUnitsName(\textcolor{input}{MassUnits\_t type}); \\
\textcolor{output}{const char *name} = cg\_LengthUnitsName(\textcolor{input}{LengthUnits\_t type}); \\
\textcolor{output}{const char *name} = cg\_TimeUnitsName(\textcolor{input}{TimeUnits\_t type}); \\
\textcolor{output}{const char *name} = cg\_TemperatureUnitsName(\textcolor{input}{TemperatureUnits\_t type}); \\
\textcolor{output}{const char *name} = cg\_ElectricCurrentUnitsName(\textcolor{input}{ElectricCurrentUnits\_t type}); \\
\textcolor{output}{const char *name} = cg\_SubstanceAmountUnitsName(\textcolor{input}{SubstanceAmountUnits\_t type}); \\
\textcolor{output}{const char *name} = cg\_LuminousIntensityUnitsName(\textcolor{input}{LuminousIntensityUnits\_t type}); \\
\textcolor{output}{const char *name} = cg\_DataClassName(\textcolor{input}{DataClass\_t type}); \\
\textcolor{output}{const char *name} = cg\_GridLocationName(\textcolor{input}{GridLocation\_t type}); \\
\textcolor{output}{const char *name} = cg\_BCDataTypeName(\textcolor{input}{BCDataType\_t type}); \\
\textcolor{output}{const char *name} = cg\_GridConnectivityTypeName(\textcolor{input}{GridConnectivityType\_t type}); \\
\textcolor{output}{const char *name} = cg\_PointSetTypeName(\textcolor{input}{PointSetType\_t type}); \\
\textcolor{output}{const char *name} = cg\_GoverningEquationsTypeName(\textcolor{input}{GoverningEquationsType\_t type}); \\
\textcolor{output}{const char *name} = cg\_ModelTypeName(\textcolor{input}{ModelType\_t type}); \\
\textcolor{output}{const char *name} = cg\_BCTypeName(\textcolor{input}{BCType\_t type}); \\
\textcolor{output}{const char *name} = cg\_DataTypeName(\textcolor{input}{DataType\_t type}); \\
\textcolor{output}{const char *name} = cg\_ElementTypeName(\textcolor{input}{ElementType\_t type}); \\
\textcolor{output}{const char *name} = cg\_ZoneTypeName(\textcolor{input}{ZoneType\_t type}); \\
\textcolor{output}{const char *name} = cg\_RigidGridMotionTypeName(\textcolor{input}{RigidGridMotionType\_t type}); \\
\textcolor{output}{const char *name} = cg\_ArbitraryGridMotionTypeName(\textcolor{input}{ArbitraryGridMotionType\_t type}); \\
\textcolor{output}{const char *name} = cg\_SimulationTypeName(\textcolor{input}{SimulationType\_t type}); \\
\textcolor{output}{const char *name} = cg\_WallFunctionTypeName(\textcolor{input}{WallFunctionType\_t type}); \\
\textcolor{output}{const char *name} = cg\_AreaTypeName(\textcolor{input}{AreaType\_t type}); \\
\textcolor{output}{const char *name} = cg\_AverageInterfaceTypeName(\textcolor{input}{AverageInterfaceType\_t type}); \\
\hline
\end{tabular}
\end{center}
\end{table}
 
\clearpage

\subsection{64-bit C Portability and Issues}

If you use the \texttt{cgsize\_t} data type in new code,
it will work in both 32 and 64-bit compilation modes.
In order to support CGNS versions prior to 3.1, you may also
want to add something like this to your code:

\noindent \texttt{\#if CGNS\_VERSION < 3100}  \\
\noindent \texttt{\#define cgsize\_t int}  \\
\noindent \texttt{\#endif}

Existing code that uses \texttt{int} will not work with a
CGNS 3.1 library compiled in 64-bit mode. You may want
to add something like this to your code:

\noindent \texttt{\#if CGNS\_VERSION >= 3100 \&\& CG\_BUILD\_64BIT} \\
\noindent \texttt{\#error does not work in 64 bit mode} \\
\noindent \texttt{\#endif}

\noindent or modify your code to use \texttt{cgsize\_t}.

\subsection{64-bit Fortran Portability and Issues}

All integer arguments in the Fortran interface - including
enumerated values (enums) - are taken to be
\texttt{integer*4} in 32-bit mode and \texttt{integer*8}
in 64-bit mode. If you have used default or implicit integers in
your Fortran code, it should port to 64-bit mode in most cases by
simply turning on your compiler option that promotes implicit
integers to \texttt{integer*8}. If you have explicitly defined your
integers as \texttt{integer*4}, your code will not work in 64-bit mode.
In that case, you will either need to change them to \texttt{integer}
(recommended for portability) or \texttt{integer*8}.

A new integer parameter has been added to the \textit{cgnslib\_f.h}
header, \texttt{CG\_BUILD\_64BIT}, which will be set
to 1 in 64-bit mode and 0 otherwise. You may use this parameter
to check at run time if the CGNS library has been compiled in 64-bit
mode or not, as in:

\noindent \texttt{if (CG\_BUILD\_64BIT .ne. 0) then} \\
\indent \texttt{print \*,'will not work in 64-bit mode'} \\
\indent \texttt{stop} \\
\noindent \texttt{endif}

If you are using a CGNS library prior to version 3.1, this parameter will
not be defined and you will need to rely on your compiler initializing all
undefined values to 0 (not always the case) for this test to work.

If your compiler supports automatic promotion of integers, and you use
implicit integers, your code should port to 64-bit with the following exception.

If you use an \texttt{Integer} data type in any routine that takes
a data type specification, and an implicit integer for the data, the code
will fail when compiled in 64-bit mode with automatic integer promotion.
An example of this would be:

\noindent \texttt{integer dim} \\
\noindent \texttt{integer data(dim)} \\
\noindent \texttt{call cg\_array\_write\_f('array',Integer,1,dim,data)} \\

This is because the MLL interprets the \texttt{Integer} data type
as \texttt{integer*4} regardless of the compilation mode. The compiler,
however, has automatically promoted \texttt{data} to be
\texttt{integer*8}. What you will need to do to prevent this problem,
is to either explicitly define \texttt{data} as in:

\noindent \texttt{integer dim} \\
\noindent \texttt{integer*4 data(dim)} \\
\noindent \texttt{call cg\_array\_write\_f('array',Integer,1,dim,data)} \\

\noindent or

\noindent \texttt{integer dim} \\
\noindent \texttt{integer*8 data(dim)} \\
\noindent \texttt{call cg\_array\_write\_f('array',LongInteger,1,dim,data)} \\

\noindent or test on \texttt{CG\_BUILD\_64BIT} as in:

\noindent \texttt{integer dim} \\
\noindent \texttt{integer data(dim)} \\
\noindent \texttt{if (CG\_BUILD\_64BIT .eq. 0) then} \\
\indent  \texttt{call cg\_array\_write\_f('array',Integer,1,dim,data)} \\
\noindent \texttt{else} \\
\indent  \texttt{call cg\_array\_write\_f('array',LongInteger,1,dim,data)} \\
\noindent \texttt{endif}

The last 2 options will only work with CGNS Version 3.1, since
\texttt{LongInteger} and \texttt{CG\_BUILD\_64BIT} are
not defined in previous versions.

You may also need to be careful when using integer constants as arguments
in 64-bit mode. If your compiler automatically promotes integer constants
to \texttt{integer*8}, then there is no problem. This is probably
the case if your compiler supports implicit integer promotion. If not,
then the constants will be \texttt{integer*4}, and your code
will not work in 64-bit mode. In that case you will need to do something like:

\noindent \texttt{integer*8 one,dim} \\
\noindent \texttt{integer*4 data(dim)} \\
\noindent \texttt{one = 1} \\
\noindent \texttt{call cg\_array\_write\_f('array',Integer,one,dim,data)}
