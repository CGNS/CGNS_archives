\section{Time-Dependent Data}
\label{s:timedep}
\thispagestyle{plain}

\subsection{Base Iterative Data}
\label{s:biter}

\noindent
\textit{Node}: \texttt{BaseIterativeData\_t}

\begin{fctbox}
\textcolor{output}{\textit{ier}} = cg\_biter\_write(\textcolor{input}{int fn}, \textcolor{input}{int B}, \textcolor{input}{char *BaseIterName}, \textcolor{input}{int Nsteps}); & - w m \\
\textcolor{output}{\textit{ier}} = cg\_biter\_read(\textcolor{input}{int fn}, \textcolor{input}{int B}, \textcolor{output}{\textit{char *BaseIterName}}, \textcolor{output}{\textit{int *Nsteps}}); & r - m \\
\hline
call cg\_biter\_write\_f(\textcolor{input}{fn}, \textcolor{input}{B}, \textcolor{input}{BaseIterName}, \textcolor{input}{Nsteps}, \textcolor{output}{\textit{ier}}) & - w m \\
call cg\_biter\_read\_f(\textcolor{input}{fn}, \textcolor{input}{B}, \textcolor{output}{\textit{BaseIterName}}, \textcolor{output}{\textit{Nsteps}}, \textcolor{output}{\textit{ier}}) & r - m \\
\end{fctbox}

\noindent
\textbf{\textcolor{input}{Input}/\textcolor{output}{\textit{Output}}}

\noindent (Note that for Fortran calls, all integer arguments are integer*4 in 32-bit mode and integer*8 in 64-bit mode.
See ``64-bit Fortran Portability and Issues" section.)

\begin{Ventryi}{\texttt{BaseIterName}}\raggedright
\item [\texttt{fn}]
      CGNS file index number.
      (\textcolor{input}{Input})
\item [\texttt{B}]
      Base index number, where $1 \leq \text{\texttt{B}} \leq \text{\texttt{nbases}}$.
      (\textcolor{input}{Input})
\item [\texttt{BaseIterName}]
      Name of the \texttt{BaseIterativeData\_t} node.
      (\textcolor{input}{Input} for \texttt{cg\_biter\_write};
      \textcolor{output}{\textit{output}} for \texttt{cg\_biter\_read})
\item [\texttt{Nsteps}]
      Number of time steps or iterations.
      (\textcolor{input}{Input} for \texttt{cg\_biter\_write};
      \textcolor{output}{\textit{output}} for \texttt{cg\_biter\_read})
\item [\texttt{ier}]
      Error status.
      (\textcolor{output}{\textit{Output}})
\end{Ventryi}

\subsection{Zone Iterative Data}
\label{s:ziter}

\noindent
\textit{Node}: \texttt{ZoneIterativeData\_t}

\begin{fctbox}
\textcolor{output}{\textit{ier}} = cg\_ziter\_write(\textcolor{input}{int fn}, \textcolor{input}{int B}, \textcolor{input}{int Z}, \textcolor{input}{char *ZoneIterName}); & - w m \\
\textcolor{output}{\textit{ier}} = cg\_ziter\_read(\textcolor{input}{int fn}, \textcolor{input}{int B}, \textcolor{input}{int Z}, \textcolor{output}{\textit{char *ZoneIterName}}); & r - m \\
\hline
call cg\_ziter\_write\_f(\textcolor{input}{fn}, \textcolor{input}{B}, \textcolor{input}{Z}, \textcolor{input}{ZoneIterName}, \textcolor{output}{\textit{ier}}) & - w m \\
call cg\_ziter\_read\_f(\textcolor{input}{fn}, \textcolor{input}{B}, \textcolor{input}{Z}, \textcolor{output}{\textit{ZoneIterName}}, \textcolor{output}{\textit{ier}}) & r - m \\
\end{fctbox}

\noindent
\textbf{\textcolor{input}{Input}/\textcolor{output}{\textit{Output}}}

\noindent (Note that for Fortran calls, all integer arguments are integer*4 in 32-bit mode and integer*8 in 64-bit mode.
See ``64-bit Fortran Portability and Issues" section.)

\begin{Ventryi}{\texttt{ZoneIterName}}\raggedright
\item [\texttt{fn}]
      CGNS file index number.
      (\textcolor{input}{Input})
\item [\texttt{B}]
      Base index number, where $1 \leq \text{\texttt{B}} \leq \text{\texttt{nbases}}$.
      (\textcolor{input}{Input})
\item [\texttt{Z}]
      Family index number, where $1 \leq \text{\texttt{Z}} \leq \text{\texttt{nzones}}$.
      (\textcolor{input}{Input})
\item [\texttt{ZoneIterName}]
      Name of the \texttt{ZoneIterativeData\_t} node.
      (\textcolor{input}{Input} for \texttt{cg\_ziter\_write};
      \textcolor{output}{\textit{output}} for \texttt{cg\_ziter\_read})
\item [\texttt{ier}]
      Error status.
      (\textcolor{output}{\textit{Output}})
\end{Ventryi}

\subsection{Rigid Grid Motion}
\label{s:rigid}

\noindent
\textit{Node}: \texttt{RigidGridMotion\_t}

\begin{fctbox}
\textcolor{output}{\textit{ier}} = cg\_rigid\_motion\_write(\textcolor{input}{int fn}, \textcolor{input}{int B}, \textcolor{input}{int Z}, & - w m \\
~~~~~~\textcolor{input}{char *RigidGridMotionName}, & \\
~~~~~~\textcolor{input}{RigidGridMotionType\_t RigidGridMotionType}, \textcolor{output}{\textit{int *R}}); & \\
\textcolor{output}{\textit{ier}} = cg\_n\_rigid\_motions(\textcolor{input}{int fn}, \textcolor{input}{int B}, \textcolor{input}{int Z}, \textcolor{output}{\textit{int *n\_rigid\_motions}}); & r - m \\
\textcolor{output}{\textit{ier}} = cg\_rigid\_motion\_read(\textcolor{input}{int fn}, \textcolor{input}{int B}, \textcolor{input}{int Z}, \textcolor{input}{int R}, & r - m \\
~~~~~~\textcolor{output}{\textit{char *RigidGridMotionName}}, & \\
~~~~~~\textcolor{output}{\textit{RigidGridMotionType\_t RigidGridMotionType}}); & \\
\hline
call cg\_rigid\_motion\_write\_f(\textcolor{input}{fn}, \textcolor{input}{B}, \textcolor{input}{Z}, \textcolor{input}{RigidGridMotionName}, & - w m \\
~~~~~\textcolor{input}{RigidGridMotionType}, \textcolor{output}{\textit{R}}, \textcolor{output}{\textit{ier}}) & \\
call cg\_n\_rigid\_motions\_f(\textcolor{input}{fn}, \textcolor{input}{B}, \textcolor{input}{Z}, \textcolor{output}{\textit{n\_rigid\_motions}}, \textcolor{output}{\textit{ier}}) & r - m \\
call cg\_rigid\_motion\_read\_f(\textcolor{input}{fn}, \textcolor{input}{B}, \textcolor{input}{Z}, \textcolor{input}{R}, \textcolor{output}{\textit{RigidGridMotionName}}, & r - m \\
~~~~~\textcolor{output}{\textit{RigidGridMotionType}}, \textcolor{output}{\textit{ier}}) & \\
\end{fctbox}

\noindent
\textbf{\textcolor{input}{Input}/\textcolor{output}{\textit{Output}}}

\noindent (Note that for Fortran calls, all integer arguments are integer*4 in 32-bit mode and integer*8 in 64-bit mode.
See ``64-bit Fortran Portability and Issues" section.)

\begin{Ventryi}{\texttt{RigidGridMotionName}}\raggedright
\item [\texttt{fn}]
      CGNS file index number.
      (\textcolor{input}{Input})
\item [\texttt{B}]
      Base index number, where $1 \leq \text{\texttt{B}} \leq \text{\texttt{nbases}}$.
      (\textcolor{input}{Input})
\item [\texttt{Z}]
      Family index number, where $1 \leq \text{\texttt{Z}} \leq \text{\texttt{nzones}}$.
      (\textcolor{input}{Input})
\item [\texttt{RigidGridMotionName}]
      Name of the \texttt{RigidGridMotion\_t} node.
      (\textcolor{input}{Input} for \texttt{cg\_rigid\_motion\_write};
      \textcolor{output}{\textit{output}} for \texttt{cg\_rigid\_motion\_read})
\item [\texttt{RigidGridMotionType}]
      Type of rigid grid motion.
      The admissible types are \texttt{CG\_Null}, \texttt{CG\_UserDefined},
      \texttt{ConstantRate}, and \texttt{VariableRate}.
      (\textcolor{input}{Input} for \texttt{cg\_rigid\_motion\_write};
      \textcolor{output}{\textit{output}} for \texttt{cg\_rigid\_motion\_read})
\item [\texttt{n\_rigid\_motions}]
      Number of \texttt{RigidGridMotion\_t} nodes under zone \texttt{Z}.
      (\textcolor{output}{\textit{Output}})
\item [\texttt{R}]
      Rigid rotation index number, where $1 \leq \text{\texttt{R}} \leq \text{\texttt{n\_rigid\_motions}}$.
      (\textcolor{input}{Input} for \texttt{cg\_rigid\_motion\_read};
      \textcolor{output}{\textit{output}} for \texttt{cg\_rigid\_motion\_write})
\item [\texttt{ier}]
      Error status.
      (\textcolor{output}{\textit{Output}})
\end{Ventryi}

\newpage
\subsection{Arbitrary Grid Motion}
\label{s:arbitrary}

\noindent
\textit{Node}: \texttt{ArbitraryGridMotion\_t}

\begin{fctbox}
\textcolor{output}{\textit{ier}} = cg\_arbitrary\_motion\_write(\textcolor{input}{int fn}, \textcolor{input}{int B}, \textcolor{input}{int Z}, & - w m \\
~~~~~~\textcolor{input}{char *ArbitraryGridMotionName}, & \\
~~~~~~\textcolor{input}{ArbitraryGridMotionType\_t ArbitraryGridMotionType}, \textcolor{output}{\textit{int *A}}); & \\
\textcolor{output}{\textit{ier}} = cg\_n\_arbitrary\_motions(\textcolor{input}{int fn}, \textcolor{input}{int B}, \textcolor{input}{int Z}, & r - m \\
~~~~~~\textcolor{output}{\textit{int *n\_arbitrary\_motions}}); & \\
\textcolor{output}{\textit{ier}} = cg\_arbitrary\_motion\_read(\textcolor{input}{int fn}, \textcolor{input}{int B}, \textcolor{input}{int Z}, \textcolor{input}{int A}, & r - m \\
~~~~~~\textcolor{output}{\textit{char *ArbitraryGridMotionName}}, & \\
~~~~~~\textcolor{output}{\textit{ArbitraryGridMotionType\_t ArbitraryGridMotionType}}); & \\
\hline
call cg\_arbitrary\_motion\_write\_f(\textcolor{input}{fn}, \textcolor{input}{B}, \textcolor{input}{Z}, \textcolor{input}{ArbitraryGridMotionName}, & - w m \\
~~~~~\textcolor{input}{ArbitraryGridMotionType}, \textcolor{output}{\textit{A}}, \textcolor{output}{\textit{ier}}) & \\
call cg\_n\_arbitrary\_motions\_f(\textcolor{input}{fn}, \textcolor{input}{B}, \textcolor{input}{Z}, \textcolor{output}{\textit{n\_arbitrary\_motions}}, \textcolor{output}{\textit{ier}}) & r - m \\
call cg\_arbitrary\_motion\_read\_f(\textcolor{input}{fn}, \textcolor{input}{B}, \textcolor{input}{Z}, \textcolor{input}{A}, \textcolor{output}{\textit{ArbitraryGridMotionName}}, & r - m \\
~~~~~\textcolor{output}{\textit{ArbitraryGridMotionType}}, \textcolor{output}{\textit{ier}}) & \\
\end{fctbox}

\noindent
\textbf{\textcolor{input}{Input}/\textcolor{output}{\textit{Output}}}

\noindent (Note that for Fortran calls, all integer arguments are integer*4 in 32-bit mode and integer*8 in 64-bit mode.
See ``64-bit Fortran Portability and Issues" section.)

\begin{Ventryi}{\texttt{ArbitraryGridMotionName}}\raggedright
\item [\texttt{fn}]
      CGNS file index number.
      (\textcolor{input}{Input})
\item [\texttt{B}]
      Base index number, where $1 \leq \text{\texttt{B}} \leq \text{\texttt{nbases}}$.
      (\textcolor{input}{Input})
\item [\texttt{Z}]
      Family index number, where $1 \leq \text{\texttt{Z}} \leq \text{\texttt{nzones}}$.
      (\textcolor{input}{Input})
\item [\texttt{ArbitraryGridMotionName}]
      Name of the \texttt{ArbitraryGridMotion\_t} node.
      (\textcolor{input}{Input} for \texttt{cg\_arbitrary\_motion\_write};
      \textcolor{output}{\textit{output}} for \texttt{cg\_arbitrary\_motion\_read})
\item [\texttt{ArbitraryGridMotionType}]
      Type of arbitrary grid motion.
      The admissible types are \texttt{CG\_Null}, \texttt{CG\_UserDefined},
      \texttt{NonDeformingGrid}, and \texttt{DeformingGrid}.
      (\textcolor{input}{Input} for \texttt{cg\_arbitrary\_motion\_write};
      \textcolor{output}{\textit{output}} for \texttt{cg\_arbitrary\_motion\_read})
\item [\texttt{n\_arbitrary\_motions}]
      Number of \texttt{ArbitraryGridMotion\_t} nodes under zone \texttt{Z}.
      (\textcolor{output}{\textit{Output}})
\item [\texttt{A}]
      Arbitrary grid motion index number, where $1 \leq \text{\texttt{A}} \leq \text{\texttt{n\_arbitrary\_motions}}$.
      (\textcolor{input}{Input} for \texttt{cg\_arbitrary\_motion\_read};
      \textcolor{output}{\textit{output}} for \texttt{cg\_arbitrary\_motion\_write})
\item [\texttt{ier}]
      Error status.
      (\textcolor{output}{\textit{Output}})
\end{Ventryi}

\newpage
\subsection{Zone Grid Connectivity}
\label{s:zconn}

\noindent
\textit{Node}: \texttt{ZoneGridConnectivity\_t}

\begin{fctbox}
\textcolor{output}{\textit{ier}} = cg\_nzconns(\textcolor{input}{int fn}, \textcolor{input}{int B}, \textcolor{input}{int Z}, \textcolor{output}{\textit{int *nzconns}}); & r - m \\
\textcolor{output}{\textit{ier}} = cg\_zconn\_read(\textcolor{input}{int fn}, \textcolor{input}{int B}, \textcolor{input}{int Z}, \textcolor{input}{int ZC}, \textcolor{output}{\textit{char *zcname}}); & r - m \\
\textcolor{output}{\textit{ier}} = cg\_zconn\_write(\textcolor{input}{int fn}, \textcolor{input}{int B}, \textcolor{input}{int Z}, \textcolor{input}{const char *zcname}, & - w m \\
~~~~~~\textcolor{output}{\textit{int *ZC}}); & \\
\textcolor{output}{\textit{ier}} = cg\_zconn\_set(\textcolor{input}{int fn}, \textcolor{input}{int B}, \textcolor{input}{int Z}, \textcolor{input}{int ZC}); & r w m \\
\textcolor{output}{\textit{ier}} = cg\_zconn\_get(\textcolor{input}{int fn}, \textcolor{input}{int B}, \textcolor{input}{int Z}, \textcolor{output}{\textit{int *ZC}}); & r w m \\
\hline
call cg\_nzconns\_f(\textcolor{input}{fn}, \textcolor{input}{B}, \textcolor{input}{Z}, \textcolor{output}{\textit{nzconns}}, \textcolor{output}{\textit{ier}}) & r - m \\
call cg\_zconn\_read\_f(\textcolor{input}{fn}, \textcolor{input}{B}, \textcolor{input}{Z}, \textcolor{input}{ZC}, \textcolor{output}{\textit{zcname}}, \textcolor{output}{\textit{ier}}) & r - m \\
call cg\_zconn\_write\_f(\textcolor{input}{fn}, \textcolor{input}{B}, \textcolor{input}{Z}, \textcolor{input}{zcname}, \textcolor{output}{\textit{ZC}}, \textcolor{output}{\textit{ier}}) & - w m \\
call cg\_zconn\_set\_f(\textcolor{input}{fn}, \textcolor{input}{B}, \textcolor{input}{Z}, \textcolor{input}{ZC}, \textcolor{output}{\textit{ier}}) & r w m \\
call cg\_zconn\_get\_f(\textcolor{input}{fn}, \textcolor{input}{B}, \textcolor{input}{Z}, \textcolor{output}{\textit{ZC}}, \textcolor{output}{\textit{ier}}) & r w m \\
\end{fctbox}

\noindent
\textbf{\textcolor{input}{Input}/\textcolor{output}{\textit{Output}}}

\noindent (Note that for Fortran calls, all integer arguments are integer*4 in 32-bit mode and integer*8 in 64-bit mode.
See ``64-bit Fortran Portability and Issues" section.)

\begin{Ventryi}{\texttt{nzconns}}\raggedright
\item [\texttt{fn}]
      CGNS file index number.
      (\textcolor{input}{Input})
\item [\texttt{B}]
      Base index number, where $1 \leq \text{\texttt{B}} \leq \text{\texttt{nbases}}$.
      (\textcolor{input}{Input})
\item [\texttt{Z}]
      Family index number, where $1 \leq \text{\texttt{Z}} \leq \text{\texttt{nzones}}$.
      (\textcolor{input}{Input})
\item [\texttt{ZC}]
      Zone grid connectivity index number, where $1 \leq \text{\texttt{ZC}} \leq \text{\texttt{nzconns}}$.
      (\textcolor{input}{Input} for \texttt{cg\_zconn\_read} and
      \texttt{cg\_zconn\_set};
      \textcolor{output}{\textit{output}} for \texttt{cg\_zconn\_write}
      and \texttt{cg\_zconn\_get})
\item [\texttt{nzconns}]
      Number of \texttt{ZoneGridConnectivity\_t} nodes under Zone \texttt{Z}.
      (\textcolor{output}{Output})
\item [\texttt{zcname}]
      Name of the \texttt{ZoneGridConnectivity\_t} node.
      (\textcolor{input}{Input} for \texttt{cg\_zconn\_write};
      \textcolor{output}{\textit{output}} for \texttt{cg\_zconn\_read})
\item [\texttt{ier}]
      Error status.
      (\textcolor{output}{\textit{Output}})
\end{Ventryi}

This allows for the specification of multiple \texttt{ZoneGridConnectivity\_t}
nodes. If these functions are not used, or \texttt{cg\_zconn\_write}
is called once with a \texttt{zcname} of \texttt{ZoneGridConnectivity},
then there will be no differences from previous versions of the
CGNS library. However, with multiple \texttt{ZoneGridConnectivity\_t}
nodes, there is an implicit \textit{current} \texttt{ZoneGridConnectivity\_t}
node on which subsequent grid connectivity functions will operate,
i.e. \texttt{cg\_conn\_read/write} (\autoref{s:generalized}) or
\texttt{cg\_1to1\_read/write} (\autoref{s:1to1}).

The functions \texttt{cg\_zconn\_read} and \texttt{cg\_zconn\_write}
will implicity set the \textit{current} \texttt{ZoneGridConnectivity\_t}
node, while \texttt{cg\_zconn\_set} explicitly sets it. The functions
\texttt{cg\_nzconns} and \texttt{cg\_zconn\_get} do not change it.

The time-dependent changes to the connectivities may then be recorded
in the \texttt{ZoneInterativeData\_t} (\autoref{s:ziter}) node as an
array of \texttt{ZoneGridConnectivityPointers}.

