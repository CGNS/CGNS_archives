\section{Data I/O Routines}
\label{s:dataio}

\begin{fctbox}
\textcolor{output}{\textit{ier}} = cgio\_read\_data(\textcolor{input}{int cgio\_num}, \textcolor{input}{double id}, \textcolor{input}{const cgsize\_t *s\_start}, & r w m \\
~~~~~~~\textcolor{input}{const cgsize\_t *s\_end}, \textcolor{input}{const cgsize\_t *s\_stride}, \textcolor{input}{int m\_num\_dims}, & \\
~~~~~~~\textcolor{input}{const cgsize\_t *m\_dims}, \textcolor{input}{const cgsize\_t *m\_start}, & \\
~~~~~~~\textcolor{input}{const cgsize\_t *m\_end}, \textcolor{input}{const cgsize\_t *m\_stride}, \textcolor{output}{\textit{void *data}}); & \\
\textcolor{output}{\textit{ier}} = cgio\_read\_all\_data(\textcolor{input}{int cgio\_num}, \textcolor{input}{double id}, \textcolor{output}{\textit{void *data}}); & r w m \\
\textcolor{output}{\textit{ier}} = cgio\_read\_block\_data(\textcolor{input}{int cgio\_num}, \textcolor{input}{double id}, \textcolor{input}{cgsize\_t b\_start}, & r w m \\
~~~~~~~\textcolor{input}{cgsize\_t b\_end}, \textcolor{output}{\textit{void *data}}); & \\
\textcolor{output}{\textit{ier}} = cgio\_write\_data(\textcolor{input}{int cgio\_num}, \textcolor{input}{double id}, \textcolor{input}{const cgsize\_t *s\_start}, & - w m \\
~~~~~~~\textcolor{input}{const cgsize\_t *s\_end}, \textcolor{input}{const cgsize\_t *s\_stride}, \textcolor{input}{int m\_num\_dims}, & \\
~~~~~~~\textcolor{input}{const cgsize\_t *m\_dims}, \textcolor{input}{const cgsize\_t *m\_start}, & \\
~~~~~~~\textcolor{input}{const cgsize\_t *m\_end}, \textcolor{input}{const cgsize\_t *m\_stride}, \textcolor{input}{void *data}); & \\
\textcolor{output}{\textit{ier}} = cgio\_write\_all\_data(\textcolor{input}{int cgio\_num}, \textcolor{input}{double id}, \textcolor{input}{void *data}); & - w m \\
\textcolor{output}{\textit{ier}} = cgio\_write\_block\_data(\textcolor{input}{int cgio\_num}, \textcolor{input}{double id}, \textcolor{input}{cgsize\_t b\_start}, & - w m \\
~~~~~~~\textcolor{input}{cgsize\_t b\_end}, \textcolor{input}{void *data}); & \\
\hline
call cgio\_read\_data\_f(\textcolor{input}{cgio\_num}, \textcolor{input}{id}, \textcolor{input}{s\_start}, \textcolor{input}{s\_end}, \textcolor{input}{s\_stride},  & r w m \\
~~~~~~\textcolor{input}{m\_num\_dims}, \textcolor{input}{m\_dims}, \textcolor{input}{m\_start}, \textcolor{input}{m\_end}, \textcolor{input}{m\_stride}, \textcolor{output}{\textit{data}}, \textcolor{output}{\textit{ier}}) & \\
call cgio\_read\_all\_data\_f(\textcolor{input}{cgio\_num}, \textcolor{input}{id}, \textcolor{output}{\textit{data}}, \textcolor{output}{\textit{ier}}) & r w m \\
call cgio\_read\_block\_data\_f(\textcolor{input}{cgio\_num}, \textcolor{input}{id}, \textcolor{input}{b\_start}, \textcolor{input}{b\_end}, \textcolor{output}{\textit{data}}, \textcolor{output}{\textit{ier}}) & r w m \\
call cgio\_write\_data\_f(\textcolor{input}{cgio\_num}, \textcolor{input}{id}, \textcolor{input}{s\_start}, \textcolor{input}{s\_end}, \textcolor{input}{s\_stride}, & - w m \\
~~~~~~\textcolor{input}{m\_num\_dims}, \textcolor{input}{m\_dims}, \textcolor{input}{m\_start}, \textcolor{input}{m\_end}, \textcolor{input}{m\_stride}, \textcolor{input}{data}, \textcolor{output}{\textit{ier}}) & \\
call cgio\_write\_all\_data\_f(\textcolor{input}{cgio\_num}, \textcolor{input}{id}, \textcolor{input}{data}, \textcolor{output}{\textit{ier}}) & - w m \\
call cgio\_write\_block\_data\_f(\textcolor{input}{cgio\_num}, \textcolor{input}{id}, \textcolor{input}{b\_start}, \textcolor{input}{b\_end}, \textcolor{input}{data}, \textcolor{output}{\textit{ier}}) & - w m \\
\end{fctbox}

\noindent
\textbf{\textcolor{input}{Input}/\textcolor{output}{\textit{Output}}}

\begin{Ventryi}{\texttt{m\_num\_dims}}\raggedright
\item [\texttt{cgio\_num}]
      Database identifier.
\item [\texttt{id}]
      Node identifier.
\item [\texttt{s\_start}]
      Starting indices for data in the database.
      Fortran indexing is used (starting at 1).
\item [\texttt{s\_end}]
      Ending indices for data in the database.
      Fortran indexing is used (starting at 1).
\item [\texttt{s\_stride}]
      Step increment for data in the database.
\item [\texttt{m\_num\_dims}]
      Number of dimensions for data in memory.
\item [\texttt{m\_dims}]
      Dimension values for data in memory.
\item [\texttt{m\_start}]
      Starting indices for data in memory.
      Fortran indexing is used (starting at 1).
\item [\texttt{m\_end}]
      Ending indices for data in memory.
      Fortran indexing is used (starting at 1).
\item [\texttt{m\_stride}]
      Step increment for data in memory.
\item [\texttt{data}]
      Array of data to be read or written.
\item [\texttt{b\_start}]
      Starting offset (index) for the data in the database.
      Fortran indexing is used (starting at 1).
\item [\texttt{b\_end}]
      Ending offset (index) for the data in the database.
      Fortran indexing is used (starting at 1).
\item [\texttt{ier}]
      Error status.
\end{Ventryi}

\subsection{Function Descriptions}

\subsubsection{\texttt{cgio\_read\_data}} \label{read_data}
    \noindent
    This routine provides general purpose read capabilities from the
    node identified by \texttt{id} in the database given by \texttt{cgio\_num}.
    It allows for a general specification of the starting location within
    the data as well as fixed step lengths (strides) through the data from
    the initial position.
    This capability works for both the data on disk and the data being
    stored in memory. One set of vectors (\texttt{s\_start},
    \texttt{s\_end} and \texttt{s\_stride}) are used to describe the mapping
    of the data within the node, and a second set of vectors
    (\texttt{m\_start}, \texttt{m\_end} and \texttt{m\_stride}) are used to
    describe the mapping of the desired data within memory.

    \noindent
    The memory dimensions are given by \texttt{m\_num\_dims} and \texttt{m\_dims}.
    There is no requirement that the node dimensions and memory dimensions
    match, only that the total number of values to be read are the same
    for the node and memory specifications.

    \noindent
    The data are stored in both memory and on disk in ``Fortran ordering.''
    That is, the first index varies the fastest, and indexing starts at 1.
    Negative indexing is not allowed.

    \noindent
    Be careful when writing data using
    \texttt{cgio\_write\_all\_data} and then using
    \texttt{cgio\_read\_data} to randomly access the data.
    \texttt{cgio\_write\_all\_data} takes a starting address in memory and
    writes \textit{N} words to disk, making no assumption as to the order of
    the data.
    \texttt{cgio\_read\_data} assumes that the data have Fortran-like
    ordering to navigate through the data in memory and on disk.
    It assumes that the first dimension varies the fastest.
    It would be easy for a C program to use the default array ordering
    (last dimension varying fastest) and write the data out using
    \texttt{cgio\_write\_all\_data}. Then another program might use
    \texttt{cgio\_read\_data} to access a subsection of the data, and the
    routine would not return what was expected.

    \noindent
    There can be a significant performance penalty for using
    \texttt{cgio\_read\_data} when compared with
    \texttt{cgio\_read\_all\_data}.
    If performance is a major consideration, it is best to organize
    data to take advantage of the speed of \texttt{cgio\_read\_all\_data}.

    \noindent
    The function returns 0 on success, else an error code.

\subsubsection{\texttt{cgio\_read\_all\_data}} \label{read_all_data}
    \noindent
    Reads all the data from the node identified by \texttt{id} in the
    database given by \texttt{cgio\_num}. On success, the function
    returns 0 and the data in \texttt{data}, else an error code is returned.
    \textit{Note}: Data is returned in Fortran indexing order.

\subsubsection{\texttt{cgio\_read\_block\_data}} \label{read_block_data}
    \noindent
    Reads a contiguous block of data from the node identified by \texttt{id}
    in the database given by \texttt{cgio\_num}. On success, the function
    returns 0 and the data in \texttt{data}, else an error code is returned.
    The starting index is given by \texttt{b\_start} and the end by
    \texttt{b\_end}.
    \textit{Note}: Fortran indexing order for multi-dimensional data is used
    when computing the starting and ending locations.

\subsubsection{\texttt{cgio\_write\_data}} \label{write_data}
    \noindent
    This function is similar to \texttt{cgio\_read\_data},
    but writes the data from memory to the node.

\subsubsection{\texttt{cgio\_write\_all\_data}} \label{write_all_data}
    \noindent
    This function is similar to \texttt{cgio\_read\_all\_data},
    but writes the data from memory to the node.

\subsubsection{\texttt{cgio\_write\_block\_data}} \label{write_block_data}
    \noindent
    This function is similar to \texttt{cgio\_read\_block\_data},
    but writes the data from memory to the node.

