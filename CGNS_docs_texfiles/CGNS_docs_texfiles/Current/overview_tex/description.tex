\section{General Description}
\label{s:description}
\thispagestyle{plain}

A CGNS database describes the current state of one or more entire CFD
(Computational Fluid Dynamics) problems, including the following:

\begin{itemize*}
   \item grid
   \item flowfield
   \item boundary conditions
   \item topological connection information
   \item auxiliary data (e.g., nondimensionalization parameters,
         reference states)
\end{itemize*}

Not all of these data need to be present at any particular time.
The overall view is that of a shared database that can be accessed
by the various software tools common to CFD, such as solvers, grid
generators, field visualizers, and postprocessors.
Each of these ``applications'' serves as an editor of the data, adding to,
modifying, or interpreting it according to that application's specific
role.

CGNS conventions and software provide for the recording of a complete
and flexible problem description.
The exact meaning of a subsonic inflow boundary condition, for example,
can be described in complete detail if desired.
User comments can be included nearly anywhere, affording the
opportunity, for instance, for date stamping or history information to
be included.
Dimension and sizing information is carefully defined.
Any number of flow variables may be recorded, with or without standard
names, and it is also possible to add user-defined or site-specific
data.
These features afford the opportunity for applications to perform
extensive error checking if desired.

Because of this generality, CGNS provides for the recording of much more
descriptive information than current applications normally use.
However, the provisions for this data are layered so that much of it is
optional.
It should be practical to convert most current applications to CGNS with
little or no conceptual change, retaining the option to take advantage
of more detailed descriptions as that becomes desirable.

CGNS specifications currently cover the bulk of CFD data that one might
wish to exchange among sites or applications; for instance, nearly any
type of field data can be recorded, and, based on its name, found and
understood by any code that needs it.
Global data (e.g., freestream Mach number, Reynolds number, angle
of attack) and physical modeling instructions (e.g., thin layer
assumptions, turbulence model) may be specified.
Nevertheless, there are items specific to individual applications for
which there is currently no specification within CGNS.
Most commonly, these are operational instructions, such as number of
sweeps, solution method, multigrid directives, and so on.
Owing to the miscellaneous nature of this data, there has been no
attempt to codify it within a global standard.
It is therefore expected that many applications will continue to require
small user-generated input files, presumably in ASCII format.

CGNS itself does not initiate action or undertake any function normally
handled by the operating system.
The user still performs CFD tasks according to existing processes.
This includes selecting the computing platform, maintaining the files,
and launching the applications.

However, the ease of communication between applications that CGNS
provides should motivate the development of new batch and interactive
mechanisms for the convenient application of CFD tools.
