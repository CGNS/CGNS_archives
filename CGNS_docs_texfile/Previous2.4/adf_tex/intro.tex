\section{Introduction}
\label{s:intro}
\thispagestyle{plain}

Advanced Data Format (ADF) is a library of basic database management
and I/O subroutines that implements a relatively simple hierarchical
database.
ADF is written in ANSI C to enhance portability of the software, and the
design of the database allows for portability of files from platform to
platform.
There is also a Fortran interface.
The files are self-describing (i.e., it is possible to browse files and
determine their contents) and extensible (i.e., many different pieces of
software on different platforms may add or modify information).

The routines allow the user to construct a tree structure with their data.
(See \autoref{f:example} on p.~\pageref*{f:example}.)
This structure is very similar to the directory structures of the UNIX
or DOS operating systems.
ADF also allows links between nodes within the same file or to different
files.
This feature works somewhat like ``soft links'' in the UNIX operating
system.
The major difference between the aforementioned directory structures and
ADF is that the nodes not only contain information about their children
(next lower-level nodes) but may also contain data.

The installation package includes the source code to ADF, the Fortran
interface, sample files, and a simple file browser.

\subsection{History of Project}

ADF was developed as part of the CFD
General Notation System(CGNS) project.
The purpose of the CGNS project is to define the data models for
Navier-Stokes based Computational Fluid Dynamics (CFD) technology,
develop a set of standard interface data structures for those data
models, and develop the software that will allow implementation of those
data structures in existing and future CFD analysis tools.
The CGNS system consists of a collection of conventions, and software
conforming to these conventions, for the storage and retrieval of CFD
data.
Adherence to these conventions is intended to facilitate the exchange of
CFD data between sites, between application codes such as solvers and
grid generators, and across computing platforms.

Once the data models (called the \href{../sids/sids.pdf}{Standard
Interface Data Structures}, or SIDS) were defined, a project was started
to write software that could faithfully reproduce that information on
disk.
ADF is the result of that project.
While ADF was developed specifically for the CFD process, it is quite
general and has no built-in knowledge of CFD; therefore, it should
be applicable to storing any type of data that lends itself to a
hierarchical definition.

\subsection{Other Software Interfaces and ADF}

Two other software packages were investigated as part of this project.
The first database interface investigated was the
Hierarchical Data Format (HDF)
developed at the National Center for
Supercomputing Applications at the University of Illinois.
This database system has a large user base and support and has been
in existence more than 6 years with utilities and graphical routines
written with both a C and a Fortran interface.
The limitation of HDF was that it was not truly hierarchical despite its
name.
Any hierarchy has to be built using naming conventions.
Since the CGNS data models indicated a natural hierarchical structure,
it seemed appropriate to develop database software that worked in that
mode by design.
The ADF design considerations are summarized in
\hyperref[s:design]{Appendix~\ref*{s:design}}.

The second was the
Common File Format
(CFF) developed by McDonnell Douglas Aerospace.
CFF is a second-generation database management system that provides a
unifying file structure for CFD data.
The purpose of the Common File is to insulate the user from the myriad
of different computer types that make up computer systems so that the
user or application programmer may process a file from another machine
without performing explicit conversions.
CFF was written in Fortran; however, it was felt that portability and
extensibility could be enhanced using C.
Much of the experience gained from the McDonnell Douglas group is
incorporated into ADF due to the cooperative efforts of personnel from
the CFD group at McDonnell Douglas Aerospace.

\subsection{Organization of Manual}

The main section of this manual explains the basics of the
hierarchical structure of an ADF database.
In addition, the concept of a node as
the basic building block of the hierarchy is developed in detail.
The remaining sections of this manual are extensive.
They provide a glossary of terms and
conventions, as well as information related to the
ADF version releases,
version control numbering, and
architectures that are supported by ADF.
There are two examples, one in Fortran and one in C, in
\hyperref[s:sampleFortran]{Appendix~\ref*{s:sampleFortran}} and
\hyperref[s:sampleC]{Appendix~\ref*{s:sampleC}} respectively, that
implement the structure illustrated in \autoref{f:example} on
p.~\pageref*{f:example}.
These examples should help familiarize the new user with ADF.
Design considerations, file optimization, and portability issues are
also discussed.
The individual ADF core routines are described in detail in
\hyperref[s:subs]{Appendix~\ref*{s:subs}}.
They are categorized into
database-level routines,
data structure and management subroutines,
data query subroutines,
data I/O subroutines, and some
miscellaneous utility subroutines.
Numerous examples are included to clarify the use of each subroutine.
\hyperref[s:errors]{Appendix~\ref*{s:errors}} provides a summary of ADF
error messages, and \hyperref[s:defaults]{Appendix~\ref*{s:defaults}}
lists default values for various parameters and limits on dimensions of
arrays.
