\section{Solution Data}
\label{s:solution}
\thispagestyle{plain}

\subsection{Flow Solution}
\label{s:flowsolution}

\noindent
\textit{Node}: \texttt{FlowSolution\_t}

\begin{fctbox}
\textcolor{output}{\textit{ier}} = cg\_sol\_write(\textcolor{input}{int fn}, \textcolor{input}{int B}, \textcolor{input}{int Z}, \textcolor{input}{char *solname}, & - w m \\
~~~~~~\textcolor{input}{GridLocation\_t location}, \textcolor{output}{\textit{int *S}}); & \\
\textcolor{output}{\textit{ier}} = cg\_nsols(\textcolor{input}{int fn}, \textcolor{input}{int B}, \textcolor{input}{int Z}, \textcolor{output}{\textit{int *nsols}}); & r - m \\
\textcolor{output}{\textit{ier}} = cg\_sol\_info(\textcolor{input}{int fn}, \textcolor{input}{int B}, \textcolor{input}{int Z}, \textcolor{input}{int S}, \textcolor{output}{\textit{char *solname}}, & r - m \\
~~~~~~\textcolor{output}{\textit{GridLocation\_t *location}}); & \\
\hline
call cg\_sol\_write\_f(\textcolor{input}{fn}, \textcolor{input}{B}, \textcolor{input}{Z}, \textcolor{input}{solname}, \textcolor{input}{location}, \textcolor{output}{\textit{S}}, \textcolor{output}{\textit{ier}}) & - w m \\
call cg\_nsols\_f(\textcolor{input}{fn}, \textcolor{input}{B}, \textcolor{input}{Z}, \textcolor{output}{\textit{nsols}}, \textcolor{output}{\textit{ier}}) & r - m \\
call cg\_sol\_info\_f(\textcolor{input}{fn}, \textcolor{input}{B}, \textcolor{input}{Z}, \textcolor{input}{S}, \textcolor{output}{\textit{solname}}, \textcolor{output}{\textit{location}}, \textcolor{output}{\textit{ier}}) & r - m \\
\end{fctbox}

\noindent
\textbf{\textcolor{input}{Input}/\textcolor{output}{\textit{Output}}}

\begin{Ventryi}{\texttt{location}}\raggedright
\item [\texttt{fn}]
      CGNS file index number.
      (\textcolor{input}{Input})
\item [\texttt{B}]
      Base index number, where $1 \leq \text{\texttt{B}} \leq \text{\texttt{nbases}}$.
      (\textcolor{input}{Input})
\item [\texttt{Z}]
      Zone index number, where $1 \leq \text{\texttt{Z}} \leq \text{\texttt{nzones}}$.
      (\textcolor{input}{Input})
\item [\texttt{S}]
      Flow solution index number, where $1 \leq \text{\texttt{S}} \leq \text{\texttt{nsols}}$.
      (\textcolor{input}{Input} for \texttt{cg\_sol\_info};
      \textcolor{output}{\textit{output}} for \texttt{cg\_sol\_write})
\item [\texttt{nsols}]
      Number of flow solutions for zone \texttt{Z}.
      (\textcolor{output}{\textit{Output}})
\item [\texttt{solname}]
      Name of the flow solution.
      (\textcolor{input}{Input} for \texttt{cg\_sol\_write};
      \textcolor{output}{\textit{output}} for \texttt{cg\_sol\_info})
\item [\texttt{location}]
      Grid location where the solution is recorded.
      The current admissible locations are \texttt{Vertex},
      \texttt{CellCenter}, \texttt{IFaceCenter}, \texttt{JFaceCenter},
      and \texttt{KFaceCenter}.
      (\textcolor{input}{Input} for \texttt{cg\_sol\_write};
      \textcolor{output}{\textit{output}} for \texttt{cg\_sol\_info})
\item [\texttt{ier}]
      Error status.
      (\textcolor{output}{\textit{Output}})
\end{Ventryi}

The above functions are used to create, and get information about,
\texttt{FlowSolution\_t} nodes.

\begin{fctbox}
\textcolor{output}{\textit{ier}} = cg\_field\_write(\textcolor{input}{int fn}, \textcolor{input}{int B}, \textcolor{input}{int Z}, \textcolor{input}{int S}, \textcolor{input}{DataType\_t datatype}, & - w m \\
~~~~~~\textcolor{input}{char *fieldname}, \textcolor{input}{void *solution\_array}, \textcolor{output}{\textit{int *F}}); & \\
\textcolor{output}{\textit{ier}} = cg\_field\_partial\_write(\textcolor{input}{int fn}, \textcolor{input}{int B}, \textcolor{input}{int Z}, \textcolor{input}{int S}, & - w m \\
~~~~~~\textcolor{input}{DataType\_t datatype}, \textcolor{input}{char *fieldname}, \textcolor{input}{int *range\_min}, & \\
~~~~~~\textcolor{input}{int *range\_max}, \textcolor{input}{void *solution\_array}, \textcolor{output}{\textit{int *F}}); & \\
\textcolor{output}{\textit{ier}} = cg\_nfields(\textcolor{input}{int fn}, \textcolor{input}{int B}, \textcolor{input}{int Z}, \textcolor{input}{int S}, \textcolor{output}{\textit{int *nfields}}); & r - m \\
\textcolor{output}{\textit{ier}} = cg\_field\_info(\textcolor{input}{int fn}, \textcolor{input}{int B}, \textcolor{input}{int Z}, \textcolor{input}{int S}, \textcolor{input}{int F}, & r - m \\
~~~~~~\textcolor{output}{\textit{DataType\_t *datatype}}, \textcolor{output}{\textit{char *fieldname}}); & \\
\textcolor{output}{\textit{ier}} = cg\_field\_read(\textcolor{input}{int fn}, \textcolor{input}{int B}, \textcolor{input}{int Z}, \textcolor{input}{int S}, \textcolor{input}{char *fieldname}, & r - m \\
~~~~~~\textcolor{input}{DataType\_t datatype}, \textcolor{input}{int *range\_min}, \textcolor{input}{int *range\_max}, & \\
~~~~~~\textcolor{output}{\textit{void *solution\_array}}); & \\
\hline
call cg\_field\_write\_f(\textcolor{input}{fn}, \textcolor{input}{B}, \textcolor{input}{Z}, \textcolor{input}{S}, \textcolor{input}{datatype}, \textcolor{input}{fieldname}, \textcolor{input}{solution\_array}, & - w m \\
~~~~~\textcolor{output}{\textit{F}}, \textcolor{output}{\textit{ier}}) & \\
call cg\_field\_partial\_write\_f(\textcolor{input}{fn}, \textcolor{input}{B}, \textcolor{input}{Z}, \textcolor{input}{S}, \textcolor{input}{datatype}, \textcolor{input}{fieldname}, & - w m \\
~~~~~\textcolor{input}{range\_min}, \textcolor{input}{range\_max}, \textcolor{input}{solution\_array}, \textcolor{output}{\textit{F}}, \textcolor{output}{\textit{ier}}) & \\
call cg\_nfields\_f(\textcolor{input}{fn}, \textcolor{input}{B}, \textcolor{input}{Z}, \textcolor{input}{S}, \textcolor{output}{\textit{nfields}}, \textcolor{output}{\textit{ier}}) & r - m \\
call cg\_field\_info\_f(\textcolor{input}{fn}, \textcolor{input}{B}, \textcolor{input}{Z}, \textcolor{input}{S}, \textcolor{input}{F}, \textcolor{output}{\textit{datatype}}, \textcolor{output}{\textit{fieldname}}, \textcolor{output}{\textit{ier}}) & r - m \\
call cg\_field\_read\_f(\textcolor{input}{fn}, \textcolor{input}{B}, \textcolor{input}{Z}, \textcolor{input}{S}, \textcolor{input}{fieldname}, \textcolor{input}{datatype}, \textcolor{input}{range\_min}, & r - m \\
~~~~~\textcolor{input}{range\_max}, \textcolor{output}{\textit{solution\_array}}, \textcolor{output}{\textit{ier}}) & \\
\end{fctbox}

\noindent
\textbf{\textcolor{input}{Input}/\textcolor{output}{\textit{Output}}}

\begin{Ventryi}{\texttt{solution\_array}}\raggedright
\item [\texttt{fn}]
      CGNS file index number.
      (\textcolor{input}{Input})
\item [\texttt{B}]
      Base index number, where $1 \leq \text{\texttt{B}} \leq \text{\texttt{nbases}}$.
      (\textcolor{input}{Input})
\item [\texttt{Z}]
      Zone index number, where $1 \leq \text{\texttt{Z}} \leq \text{\texttt{nzones}}$.
      (\textcolor{input}{Input})
\item [\texttt{S}]
      Flow solution index number, where $1 \leq \text{\texttt{S}} \leq \text{\texttt{nsols}}$.
      (\textcolor{input}{Input})
\item [\texttt{F}]
      Solution array index number, where $1 \leq \text{\texttt{F}} \leq \text{\texttt{nfields}}$.
      (\textcolor{input}{Input} for \texttt{cg\_field\_info};
      \textcolor{output}{\textit{output}} for \texttt{cg\_field\_write})
\item [\texttt{nfields}]
      Number of data arrays in flow solution \texttt{S}.
      (\textcolor{output}{\textit{Output}})
\item [\texttt{datatype}]
      Data type in which the solution array is written.
      Admissible data types for a solution array are \texttt{Integer},
      \texttt{RealSingle}, and \texttt{RealDouble}.
      (\textcolor{input}{Input} for \texttt{cg\_field\_write},
      \texttt{cg\_field\_read};
      \textcolor{output}{\textit{output}} for \texttt{cg\_field\_info})
\item [\texttt{fieldname}]
      Name of the solution array.
      It is strongly advised to use the SIDS nomenclature conventions
      when naming the solution arrays to insure file compatibility.
      (\textcolor{input}{Input} for \texttt{cg\_field\_write},
      \texttt{cg\_field\_read};
      \textcolor{output}{\textit{output}} for \texttt{cg\_field\_info})
\item [\texttt{range\_min}]
      Lower range index (eg., \texttt{imin, jmin, kmin}).
      (\textcolor{input}{Input})
\item [\texttt{range\_max}]
      Upper range index (eg., \texttt{imax, jmax, kmax}).
      (\textcolor{input}{Input})
\item [\texttt{solution\_array}]
      Array of solution values for the range prescribed.
      (\textcolor{input}{Input} for \texttt{cg\_field\_write};
      \textcolor{output}{\textit{output}} for \texttt{cg\_field\_read})
\item [\texttt{ier}]
      Error status.
      (\textcolor{output}{\textit{Output}})
\end{Ventryi}

The above functions are used to read and write solution arrays stored
below a \texttt{FlowSolution\_t} node.

When writing, the function \texttt{cg\_field\_write} will
automatically write the full range of the solution (i.e., the entire
\texttt{solution\_array}).
The function \texttt{cg\_field\_partial\_write} may be used to write
only a subset of \texttt{solution\_array}.
When using the partial write, any existing data from \texttt{range\_min}
to \texttt{range\_max} will be overwritten by the new values.
All other values will not be affected.

The function \texttt{cg\_field\_read} returns the solution array
\texttt{fieldname}, for the range prescribed by \texttt{range\_min} and
\texttt{range\_max}.
The array is returned to the application in the data type requested in
\texttt{datatype}.
This data type does not need to be the same as the one in which the data is
stored in the file.
A solution array stored as double precision in the CGNS file can be
returned to the application as single precision, or vice versa.

In Fortran, when using \texttt{cg\_field\_read\_f} to read a 2D or 3D
solution, the extent of each dimension of \texttt{solution\_array} must
be consistent with the requested range.
When reading a 1D solution, the declared size can be larger than the
requested range.
For example, for a 2D zone with $100 \times 50$ vertices, if
\texttt{range\_min} and \texttt{range\_max} are set to (11,11) and (20,20)
to read a subset of the solution, then \texttt{solution\_array} must be
dimensioned (10,10).
If \texttt{solution\_array} is declared larger (e.g., (100,50)) the
indices for the returned array values will be wrong.

\noindent
\textbf{Note:} In CGNS 2.5, partial reads and writes are only supported
when ADF is being used as the underlying database manager.
They will not work correctly with HDF5.

\subsection{Discrete Data}
\label{s:discretedata}

\noindent
\textit{Node}: \texttt{DiscreteData\_t}

\begin{fctbox}
\textcolor{output}{\textit{ier}} = cg\_discrete\_write(\textcolor{input}{int fn}, \textcolor{input}{int B}, \textcolor{input}{int Z}, \textcolor{input}{char *DiscreteName}, & - w m \\
~~~~~~\textcolor{output}{\textit{int *D}}); & \\
\textcolor{output}{\textit{ier}} = cg\_ndiscrete(\textcolor{input}{int fn}, \textcolor{input}{int B}, \textcolor{input}{int Z}, \textcolor{output}{\textit{int *ndiscrete}}); & r - m \\
\textcolor{output}{\textit{ier}} = cg\_discrete\_read(\textcolor{input}{int fn}, \textcolor{input}{int B}, \textcolor{input}{int Z}, \textcolor{input}{int D}, & r - m \\
~~~~~~\textcolor{output}{\textit{char *DiscreteName}}); & \\
\hline
call cg\_discrete\_write\_f(\textcolor{input}{fn}, \textcolor{input}{B}, \textcolor{input}{Z}, \textcolor{input}{DiscreteName}, \textcolor{output}{\textit{D}}, \textcolor{output}{\textit{ier}}) & - w m \\
call cg\_ndiscrete\_f(\textcolor{input}{fn}, \textcolor{input}{B}, \textcolor{input}{Z}, \textcolor{output}{\textit{ndiscrete}}, \textcolor{output}{\textit{ier}}) & r - m \\
call cg\_discrete\_read\_f(\textcolor{input}{fn}, \textcolor{input}{B}, \textcolor{input}{Z}, \textcolor{input}{D}, \textcolor{output}{\textit{DiscreteName}}, \textcolor{output}{\textit{ier}}) & r - m \\
\end{fctbox}

\noindent
\textbf{\textcolor{input}{Input}/\textcolor{output}{\textit{Output}}}

\begin{Ventryi}{\texttt{DiscreteName}}\raggedright
\item [\texttt{fn}]
      CGNS file index number.
      (\textcolor{input}{Input})
\item [\texttt{B}]
      Base index number, where $1 \leq \text{\texttt{B}} \leq \text{\texttt{nbases}}$.
      (\textcolor{input}{Input})
\item [\texttt{Z}]
      Zone index number, where $1 \leq \text{\texttt{Z}} \leq \text{\texttt{nzones}}$.
      (\textcolor{input}{Input})
\item [\texttt{D}]
      Discrete data index number, where $1 \leq \text{\texttt{D}} \leq \text{\texttt{ndiscrete}}$.
      (\textcolor{input}{Input} for \texttt{cg\_discrete\_read};
      \textcolor{output}{\textit{output}} for \texttt{cg\_discrete\_write})
\item [\texttt{ndiscrete}]
      Number of \texttt{DiscreteData\_t} data structures under zone \texttt{Z}.
      (\textcolor{output}{\textit{Output}})
\item [\texttt{DiscreteName}]
      Name of \texttt{DiscreteData\_t} data structure.
      (\textcolor{input}{Input} for \texttt{cg\_discrete\_write};
      \textcolor{output}{\textit{output}} for \texttt{cg\_discrete\_read})
\item [\texttt{ier}]
      Error status.
      (\textcolor{output}{\textit{Output}})
\end{Ventryi}

\texttt{DiscreteData\_t} nodes are intended for the storage of fields
of data not usually identified as part of the flow solution, such as
fluxes or equation residuals.
