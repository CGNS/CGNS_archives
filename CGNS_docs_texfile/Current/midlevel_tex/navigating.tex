\section{Navigating a CGNS File}
\label{s:navigating}
\thispagestyle{plain}

\subsection{Accessing a Node}
\label{s:goto}

\begin{fctbox}
\textcolor{output}{\textit{ier}} = cg\_goto(\textcolor{input}{int fn}, \textcolor{input}{int B}, ..., \textcolor{input}{"end"}); & r w m \\
\textcolor{output}{\textit{ier}} = cg\_gorel(\textcolor{input}{int fn}, ..., \textcolor{input}{"end"}); & r w m \\
\textcolor{output}{\textit{ier}} = cg\_gopath(\textcolor{input}{int fn}, \textcolor{input}{const char *path}); & r w m \\
\textcolor{output}{\textit{ier}} = cg\_golist(\textcolor{input}{int fn}, \textcolor{input}{int B}, \textcolor{input}{int depth}, \textcolor{input}{char **label}, \textcolor{input}{int *index}); & r w m \\
\textcolor{output}{\textit{ier}} = cg\_where(\textcolor{output}{\textit{int *fn}}, \textcolor{output}{\textit{int *B}}, \textcolor{output}{\textit{int *depth}}, \textcolor{output}{\textit{char **label}}, & r w m \\
~~~~~~~~~~~~~~~~~\textcolor{output}{\textit{int *index}}); & \\
\hline
call cg\_goto\_f(\textcolor{input}{fn}, \textcolor{input}{B}, \textcolor{output}{\textit{ier}}, ..., \textcolor{input}{'end'}) & r w m \\
call cg\_gorel\_f(\textcolor{input}{fn}, \textcolor{output}{\textit{ier}}, ..., \textcolor{input}{'end'}) & r w m \\
call cg\_gopath\_f(\textcolor{input}{fn}, \textcolor{input}{path}, \textcolor{output}{\textit{ier}}) & r w m \\
\end{fctbox}

\noindent
\textbf{\textcolor{input}{Input}/\textcolor{output}{\textit{Output}}}

\noindent (Note that for Fortran calls, all integer arguments are integer*4 in 32-bit mode and integer*8 in 64-bit mode.
See ``64-bit Fortran Portability and Issues" section.)

\begin{Ventryi}{\texttt{depth}}\raggedright
\item [\texttt{fn}]
      CGNS file index number.
      (\textcolor{input}{Input} except for \texttt{cg\_where})
\item [\texttt{B}]
      Base index number, where $1 \leq \text{\texttt{B}} \leq \text{\texttt{nbases}}$.
      (\textcolor{input}{Input} except for \texttt{cg\_where})
\item [\texttt{...}]
      Variable argument list used to specify the path to a node.
      It is composed of an unlimited list of pair-arguments
      identifying each node in the path.
      Nodes may be identified by their label or name.
      Thus, a pair-argument may be of the form
\begin{alltt}
   "CGNS\_NodeLabel", \textit{NodeIndex}
\end{alltt}
      where \texttt{CGNS\_NodeLabel} is the node label and
      \textit{NodeIndex} is the node index, or
\begin{alltt}
   "CGNS\_NodeName", 0
\end{alltt}
      where \texttt{CGNS\_NodeName} is the node name.
      The 0 in the second form is required, to indicate that a node
      name is being specified rather than a node label.
      In addition, a pair-argument may be specified as
\begin{alltt}
   "..", 0
\end{alltt}
      indicating the parent of the current node.
      The different pair-argument forms may be intermixed in the same
      function call.

      There is one exception to this rule.
      When accessing a \texttt{BCData\_t} node, the index must be set to
      either \texttt{Dirichlet} or \texttt{Neumann} since only these
      two types are allowed.
      (Note that \texttt{Dirichlet} and \texttt{Neumann} are defined in
      the include files \textit{cgnslib.h} and \textit{cgnslib\_f.h}).
      Since "\texttt{Dirichlet}" and "\texttt{Neuman}" are also the
      names for these nodes, you may also use the \texttt{"Dirichlet", 0}
      or \texttt{"Neuman", 0} to access the node.
      See the example below.
      (\textcolor{input}{Input})
\item [\texttt{end}]
      The character string \texttt{"end"} (or \texttt{'end'} for the Fortran
      function) must be the last argument.
      It is used to indicate the end of the argument list.
      You may also use the empty string, \texttt{""} (\texttt{{'}{'}} for
      Fortran), or the \texttt{NULL} string in C, to terminate the
      list.
      (\textcolor{input}{Input})
\item [\texttt{path}]
      The pathname for the node to go to.
      If a position has been already set, this may be a relative
      path, otherwise it is an absolute path name, starting with
      \texttt{"/Basename"}, where \texttt{Basename} is the base under
      which you wish to move.
      (\textcolor{input}{Input})
\item [\texttt{depth}]
      Depth of the path list.
      The maximum depth is defined in \textit{cgnslib.h} by
      \texttt{CG\_MAX\_GOTO\_DEPTH}, and is currently equal to 20.
      (\textcolor{input}{Input} for \texttt{cg\_golist};
      \textcolor{output}{\textit{output}} for \texttt{cg\_where})
\item [\texttt{label}]
      Array of node labels for the path.
      This argument may be passed as \texttt{NULL} to
      \texttt{cg\_where()}, otherwise it must be dimensioned by the
      calling program.
      The maximum size required is \texttt{label[MAX\_GO\_TO\_DEPTH][33]}.
      You may call \texttt{cg\_where()} with both \texttt{label} and
      \texttt{index} set to \texttt{NULL} in order to get the current
      depth, then dimension to that value.
      (\textcolor{input}{Input} for \texttt{cg\_golist};
      \textcolor{output}{\textit{output}} for \texttt{cg\_where})
\item [\texttt{index}]
      Array of node indices for the path.
      This argument may be passed as \texttt{NULL} to
      \texttt{cg\_where()}, otherwise it must be dimensioned by the
      calling program.
      The maximum size required is \texttt{index[MAX\_GO\_TO\_DEPTH]}.
      You may call \texttt{cg\_where()} with both \texttt{label} and
      \texttt{index} set to \texttt{NULL} in order to get the current
      depth, then dimension to that value.
      (\textcolor{input}{Input} for \texttt{cg\_golist};
      \textcolor{output}{\textit{output}} for \texttt{cg\_where})
\item [\texttt{ier}]
      Error status.
      The possible values, with the corresponding C names (or Fortran
      parameters) defined in \textit{cgnslib.h} (or \textit{cgnslib\_f.h})
      are listed below.

      \begin{tabular}{@{}c l}
         \uline{\ital{Value}} & \uline{\ital{Name/Parameter}} \\
         0 & \texttt{CG\_OK} \\
         1 & \texttt{CG\_ERROR} \\
         2 & \texttt{CG\_NODE\_NOT\_FOUND} \\
         3 & \texttt{CG\_INCORRECT\_PATH}
      \end{tabular}

      For non-zero values, an error message may be printed using
      \texttt{cg\_error\_print()}, as described in \autoref{s:error}.
      (\textcolor{output}{\textit{Output}})
\end{Ventryi}

This function allows access to any parent-type nodes in a CGNS file.
A parent-type node is one that can have children.
Nodes that cannot have children, like \texttt{Descriptor\_t}, are
not supported by this function.

\noindent
\uline{\textit{Examples}}

To illustrate the use of the above routines, assume you have a file with
CGNS index number \texttt{filenum}, a base node named \texttt{Base}
with index number \texttt{basenum}, 2 zones (named \texttt{Zone1}
and \texttt{Zone2}, with indices 1 and 2), and user-defined data
(\texttt{User}, index 1) below each zone.
To move to the user-defined data node under zone 1, you may use any of
the following:
\begin{alltt}
   cg\_goto(filenum, basenum, "Zone\_t", 1, "UserDefinedData\_t", 1, NULL);
   cg\_goto(filenum, basenum "Zone1", 0, "UserDefinedData\_t", 1, NULL);
   cg\_goto(filenum, basenum, "Zone\_t", 1, "User", 0, NULL);
   cg\_goto(filenum, basenum, "Zone1", 0, "User", 0, NULL);
   cg\_gopath(filenum, "/Base/Zone1/User");
\end{alltt}
Now, to change to the user-defined data node under zone 2, you may use
the full path specification as above, or else a relative path, using one
of the following:
\begin{alltt}
   cg\_gorel(filenum, "..", 0, "..", 0, "Zone\_t", 2, "UserDefinedData\_t", 1, NULL);
   cg\_gorel(filenum, "..", 0, "..", 0, "Zone2", 0, "UserDefinedData\_t", 1, NULL);
   cg\_gorel(filenum, "..", 0, "..", 0, "Zone\_t", 2, "User", 0, NULL);
   cg\_gorel(filenum, "..", 0, "..", 0, "Zone2", 0, "User", 0, NULL);
   cg\_gopath(filenum, "../../Zone2/User");
\end{alltt}

Shown below are some additional examples of various uses of these
routines, in both C and Fortran, where \texttt{fn}, \texttt{B},
\texttt{Z}, etc., are index numbers.

\begin{alltt}
   ier = cg\_goto(fn, B, "Zone\_t", Z, "FlowSolution\_t", F, "..", 0, "MySolution",
                 0, "end");

   ier = cg\_gorel(fn, "..", 0, "FlowSolution\_t", F, NULL);

   ier = cg\_gopath(fn, "/MyBase/MyZone/MySolution");

   ier = cg\_gopath(fn, "../../MyZoneBC");

   call cg\_goto\_f(fn, B, ier, 'Zone\_t', Z, 'GasModel\_t', 1, 'DataArray\_t',
                  A, 'end')

   call cg\_goto\_f(fn, B, ier, 'Zone\_t', Z, 'ZoneBC\_t', 1, 'BC\_t', BC,
                  'BCDataSet\_t', S, 'BCData\_t', Dirichlet, 'end')

   call cg\_gorel\_f(fn, ier, '..', 0, 'Neumann', 0, '')

   call cg\_gopath\_f(fn, '../../MyZoneBC', ier)
\end{alltt}

\subsection{Deleting a Node}
\label{s:delete}

\begin{fctbox}
\textcolor{output}{\textit{ier}} = cg\_delete\_node(\textcolor{input}{char *NodeName}); & - - m \\
\hline
call cg\_delete\_node\_f(\textcolor{input}{NodeName}, \textcolor{output}{\textit{ier}}) & - - m \\
\end{fctbox}

\noindent
\textbf{\textcolor{input}{Input}/\textcolor{output}{\textit{Output}}}

\noindent (Note that for Fortran calls, all integer arguments are integer*4 in 32-bit mode and integer*8 in 64-bit mode.
See ``64-bit Fortran Portability and Issues" section.)

\begin{Ventryi}{\texttt{NodeName}}\raggedright
\item [\texttt{NodeName}]
      Name of the child to be deleted.
      (\textcolor{input}{Input})
\item [\texttt{ier}]
      Error status.
      (\textcolor{output}{\textit{Output}})
\end{Ventryi}

The function \texttt{cg\_delete\_node} is used is conjunction with
\texttt{cg\_goto}.
Once positioned at a parent node with \texttt{cg\_goto}, a child of
this node can be deleted with \texttt{cg\_delete\_node}.
This function requires a single argument, \texttt{NodeName}, which is
the name of the child to be deleted.

Since the highest level that can be pointed to with \texttt{cg\_goto} is
a base node for a CGNS database
(\texttt{CGNSBase\_t}), the
highest-level nodes that can be deleted are the children of a
\texttt{CGNSBase\_t} node.
In other words, nodes located directly under the ADF (or HDF) root node
(\texttt{CGNSBase\_t} and
\texttt{CGNSLibraryVersion\_t})
can not be deleted with \texttt{cg\_delete}.

A few other nodes are not allowed to be deleted from the database
because these are required nodes as defined by the SIDS, and deleting
them would make the file non-CGNS compliant.
These are:
\begin{itemize*}
\item Under \texttt{Zone\_t}:
      \texttt{ZoneType} 
\item Under \texttt{GridConnectivity1to1\_t}:
      \texttt{PointRange, PointRangeDonor, Transform}
\item Under \texttt{OversetHoles\_t}:
      \texttt{PointList} and any \texttt{IndexRange\_t}
\item Under \texttt{GridConnectivity\_t}:
      \texttt{PointRange, PointList, CellListDonor, PointListDonor}
\item Under \texttt{BC\_t}:
      \texttt{PointList, PointRange}
\item Under \texttt{GeometryReference\_t}:
      \texttt{GeometryFile, GeometryFormat}
\item Under \texttt{Elements\_t}:
      \texttt{ElementRange, ElementConnectivity}
\item Under \texttt{Gravity\_t}:
      \texttt{GravityVector}
\item Under \texttt{Axisymmetry\_t}:
      \texttt{AxisymmetryReferencePoint, AxisymmetryAxisVector}
\item Under \texttt{RotatingCoordinates\_t}:
      \texttt{RotationCenter, RotationRateVector}
\item Under \texttt{Periodic\_t}:
      \texttt{RotationCenter, RotationAngle, Translation}
\item Under \texttt{AverageInterface\_t}:
      \texttt{AverageInterfaceType}
\item Under \texttt{WallFunction\_t}:
      \texttt{WallFunctionType}
\item Under \texttt{Area\_t}:
      \texttt{AreaType, SurfaceArea, RegionName}
\end{itemize*}

When a child node is deleted, both the database and the file on disk are
updated to remove the node.
One must be careful not to delete a node from within a loop of that node
type.
For example, if the number of zones below a \texttt{CGNSBase\_t} node is
\texttt{nzones}, a zone should never be deleted from within a zone loop!
By deleting a zone, the total number of zones (\texttt{nzones}) changes,
as well as the zone indexing.
Suppose for example that \texttt{nzones} is 5, and that the third zone
is deleted.
After calling \texttt{cg\_delete\_node}, \texttt{nzones} is changed to 4,
and the zones originally indexed 4 and 5 are now indexed 3 and 4.
