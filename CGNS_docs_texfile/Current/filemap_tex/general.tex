\section{General CGNS File Mapping Concepts}
\label{s:general}
\thispagestyle{plain}

This document describes the general concepts behind and structure of
a CGNS Database. The implementation details of reading and writing to a
storage medium are left to an underlying database manager. The
\href{../cgio/cgio.pdf}{CGIO User's Guide} provides the core
routines to access the database manager at that level.

Since CGNS was originally developed based on the
\href{../adf/adf.pdf}{ADF} (Advanced Data Format) as the
database manager, much of this documentation parallels those concepts.
With the introduction of the \href{../hdf5/hdf5.pdf}{HDF5}
(Hierarchical Data Format) database manager, the concepts in this
document have been "mapped" to HDF5.

We first describe the structure of the database,
and the nodes from which it is constructed,
followed by a description of the overall layout
of the tree structure itself.

The section Detailed CGNS Node Descriptions (\autoref{s:nodes})
describes the exact conventions for each type of data.

\subsection{The Conceptual Structure of a CGNS File}
\label{s:filetructure}

Although the physical structure of an CGNS file in storage is (or should
be) of little concern to users of CGNS, an understanding of its logical
or conceptual structure is essential. This structure determines its
suitability for the type of data at hand and is reflected in all of the
low-level database access routines.

An CGNS file consists entirely of a collection of elements called
nodes. These nodes are arranged in a tree structure which is logically
similar to a UNIX file system. The nodes are said to be connected in a
``child-parent'' relationship according to the following simple rules:

\begin{enumerate}
\item Each node may have any number of child nodes.
\item Except for one node, called the root, each node is the child of
      exactly one other node, called its parent.
\item The root node has no parent.
\end{enumerate}

Every node in a CGNS file has exactly the same internal structure. Each
node contains identifying information, pointers to any children, and,
optionally, data.

When a CGNS file is opened (by an appropriate database manager),
information is returned to the calling program which is sufficient to
access the root node. It is then the responsibility of the program to
search the tree for whatever information is required, or to add to the
tree any information it wishes to store.

There is a special kind of node called a link, which serves as a pointer
to a node in another CGNS file, or in another part of the same CGNS
file. The tree structure at and below the node to which the link points
is available as if that node were present instead of the link. This
allows a CGNS file to span multiple physical files, and also allows a
portion of one CGNS file to be referenced by several other CGNS files.

\subsection{The Structure of a CGNS Node}
\label{s:nodestructure}

The File Mapping specifies not only the location of the node at which
a particular kind of data is to be stored, but also how the internal
structure of the node is to be used. Each node contains a number of
fields into which data may be recorded. They
are:

\begin{itemize*}
\item The Node Name
\item The Label
\item The Data Type
\item The Number of Dimensions
\item The Dimension Values
\item The Data
\end{itemize*}
In addition, two other entities associated with the nodes are managed by
the underlying database manager itself. These are:
\begin{itemize*}
\item The Node ID
\item The Child Table
\end{itemize*}

\subsubsection{The Node Name}

The node Name is a 32-byte character field which is user
controllable. Its general use is to distinguish among the children of a
given node; consequently, no two children of the same parent may have
the same Name.

The Name may be left to the choice of the user, or it may be
specified by the \href{../sids/sids.pdf}{SIDS}.
At the levels of the tree nearest the root,
the (end-)user is free to set the Name to distinguish among like
objects in the case at hand. For example, in a multizone problem, nodes
associated with different zones might be named ``\texttt{UnderLeftWing}''
or ``\texttt{AboveForwardFuselage}''. At this level, it is generally
not possible to identify a collection of names which are likely to
recur. This means that the naming of high level objects does not
require standardization, and the SIDS are silent regarding the naming
convention.

Because every node must be given a name when it is opened, default
names, based on the node Label, are provided by convention.
The \href{../midlevel/midlevel.pdf}{CGNS Midlevel Library} will
record the default names if none is provided by the user.
The precise formula is given in the Label section below.

At levels of the tree farther from the root, the SIDS often specify
the name. There is, for example, a commonly encountered collection of
flow variables whose general meaning is widely understood. In this
case, standardizing the name conveys precise information. Thus the SIDS
specify, for instance, that a node containing static internal energy
per unit mass should have the Name ``\texttt{EnergyInternal}''.
Adherence to these naming
conventions guarantees uniform meaning of the data from site to site
and user to user.
Of course, there may be a desire to store quantities for which no naming
convention is specified.
In this case any suitable name can be used, but there is no guarantee of
proper interpretation by anyone unaware of the choice.

\subsubsection{The Label}

The Label is a 32-byte character field which is used
to identify the structure of the included data.
It is common for the various
children of a single parent to store different instances of the same
structure. Therefore, there is no prohibition against more than one
child of the same parent having the same Label.

Within CGNS, nearly all labels reflect C-style type definitions
(``typedefs'') specified by the SIDS, and end in the characters
``\texttt{\_t}''. Some ``Leaf'' nodes (i.e. nodes that have no children)
do not represent higher level CGNS structures and therefore have labels
that do not follow the ``\texttt{\_t}'' convention. At this writing, all
such nodes have the type \texttt{int[]}, i.e., integer array, a type
already recognized in C, for which a separate type definition would be
artificial. Such nodes are generally located by the software through
their names, which are specified by the SIDS, rather than through their
labels.

The Label generally indicates the role of the data at and below the
node in the context of CFD. Nodes which are entry points to data for a
particular zone, for example, have the Label ``\texttt{Zone\_t}''.

Parent nodes often have a number of children each containing data for
different instances of the same structure. Multiple children of the same
parent therefore often have the same Label. It is customary for software
to conduct searches which depend on the Label, e.g., to determine the
number of zones in a problem. The software will fail if the conventions
regarding Labels are not observed.

Labels are also used to build default node Names. These are derived from
the Label by dropping the characters ``\texttt{\_t}'' and substituting the
smallest positive integer resulting in a unique name among children
of the same parent. For example, the first default Name for a node
of type \texttt{Zone\_t} will be ``\texttt{Zone1}''; the second will be
``\texttt{Zone2}''; and so on.

\subsubsection{The Data Type}

The Data Type is a 2-byte character field which specifies the type and
precision of any data which is stored in the data field. The data types
allowed by CGNS are:

\begin{table}[htbp]
\centering
\caption[Data Types]{\textbf{Data Types}}
\label{t:datatype}
\begin{tabular}{l >{\quad}l}
\\ \hline\hline \\*[-2ex]
\bold{Data Type} & \bold{Notation}
\\*[1ex] \hline\hline \\*[-2ex]
No Data             & \fort{MT} \\
Link                & \fort{LK} \\
Integer 32          & \fort{I4} \\
Integer 64          & \fort{I8} \\
Real 32             & \fort{R4} \\
Real 64             & \fort{R8} \\
Character           & \fort{C1}
\\*[1ex] \hline\hline
\end{tabular}
\end{table}

The \texttt{MT} node contains no data, thus the number of dimensions and
dimension values are ignored. This node is typically used as a container
for children nodes.

The \texttt{LK} node is a link to a node within within the current or
an external database. The actual data stored with the node is a file name
and node path name to the link (as \texttt{C1} data). This information
is available only through the core link access routines. A normal
read of a \texttt{LK} node will will return the node properties
(with the exception of the Name) from the linked-to node.

The specification of data types within the File Mapping allows for the
probability that files written under different circumstances, or on
different platforms, may differ in precision or format. It is the
responsibility of the database manager to provide any data conversions
required, and to ensure precision of the data.

\subsubsection{The Number of Dimensions}

The Data portion of a node is designed to store multi-dimensional arrays
of data, each element of which is presumed to be of the Data Type
specified. The Number of Dimensions specifies the number of integers
required to reference a single datum within the array. The limit for
the Number of Dimensions is 12.

Whenever data is stored at a node, it is in the form of a single array
of elements of a single date type. \textit{Note} that Fortran-indexing
conventions are  used within CGNS. For multi-dimensional data, this means
that the first index of the array varies the fastest, and indexing
starts at 1. This is important to understand when reading the data
into multi-dimensional ``C'' arrays, since the indexing order is opposite
to that as used by CGNS.

Another thing to note, is that character data (\texttt{C1}) is not
stored with a ``C'' terminating `0' character, but rather is padded out
with blanks (space character) as in Fortran character data. The MLL,
however, will add the terminating-`0' to the strings when returning
this data to a ``C'' application.

\subsubsection{The Dimension Values}

The Dimension Values are a list of integers expressing the actual sizes
of the stored array in each of the dimensions specified. With the
introduction of 64-bit capability with CGNS Version 3.1.3, these
dimensions are stored internally as 64-bit integers, regardless of the
compilation mode (32 or 64-bit mode). They are returned to the
application, however, as \ital{cgsize\_t}.

In the case of rectangular
arrays of physical data, the dimension values are set to the actual
sizes in physical space. Note that these sizes often depend on whether
the values are associated with grid nodes, cell centers or other
physical locations with respect to the grid. In any event, they refer to
the amount of data actually stored, not to any larger array from which
it may have been extracted.

In the case of list data, the dimension value is the length of the
list. Lists of characters may contain termination bytes such as
newline characters; this means an entire document can be
stored in the data field.

\subsubsection{The Data}

The portion of the node holding the actual stored data array.

CGNS imposes no conventions on the data itself.
\textit{Note} that it is a responsibility of the CGNS software to ensure
that the amount and type of stored data agrees with the specification of
the data type, number of dimensions, and dimension values.

\subsubsection{The Node ID}

The node ID is a unique identifier assigned to each existing node by
a database manager when the file containing it is opened, and to new
nodes as they are created. Core routine inquiries generally return node
IDs as a result and accept node IDs as input. By building a table of IDs,
calling software can subsequently access specific nodes without a
further search. The Node ID is a 64-bit real and is not under user control.

\subsubsection{The Child Table}

The database manager maintains a table recording the
children of each node, which is adjusted when children are added or
deleted

Children may be identified by their names and labels, and, thence,
by their node IDs once these have been determined. There is no provision
for the notion of order among children. In particular, the order of a
list of children returned by the database manager may or may not have
anything to do with the order in which they
were inserted in the file. However, the order returned is consistent
from call to call provided the file has not been closed and the node
structure has not been modifed.

Note that there is no \emph{parent} table; that is, a node has no direct
knowledge of its parent. Since calling software must open the file
from the root, it presumably cannot access a child without having
first accessed the parent.  It is the responsibility of the calling
software to record the node ID of the parent if this information will be
required.

The Child Table is completely controlled by the database manager,
and it's role is exactly the same for CGNS. CGNS software accesses and
modifies the child table only through calls to the database manager.

In addition to the meaning of attributes of individual nodes,
the File Mapping specifies the relations between nodes in a CGNS
database. Consequently, the File Mapping determines what kinds of nodes
will lie in the child table.

\subsection{Use of Nodes in CGNS}
\label{s:nodeuse}

There are certain attributes of nodes which are
derived from context, i.e., which the node possesses by virtue of its
location within a CGNS database. These are often needed by CGNS to
properly interpret the data; namely,

\begin{itemize*}
\item Cardinality
\item Parameters
\item Functions
\end{itemize*}

\subsubsection{Cardinality}
\label{s:cardinality}

The \emph{cardinality} of a CGNS node is the number of nodes of the same
label permitted at one point in the tree, i.e., as children of the same
parent. It consists of both lower and upper limits.

Since the notion of a CGNS database allows for work in progress, the
lower limit is generally zero (although the database may be of little
use until certain nodes are filled). The upper limit is usually either
one or many ($N$).

\subsubsection{Parameters}

CGNS relies on the fact that nodes cannot be found except by
following the pointers from their parents.  This means that software
accessing a node has had an opportunity to note all the data above that
node in the tree. Therefore, nodes do not repeat within themselves
information which is necessary for their interpretation but which is
available at a higher level.

A datum which is necessary for the proper interpretation of a
node but which is derived from its ancestors is referred to as a
\textit{structure parameter}.

\subsubsection{Functions}
\label{s:functions}

Occasionally the proper interpretation of a node depends on
an implicitly understood \emph{function} of its structure
parameters. Usually these relate to the actual amount of data
stored. Several of these functions are defined in the SIDS and
referenced in this document.

\subsection{CGNS Databases}
\label{s:cgns_databases}

\subsubsection{Definition of a CGNS Database}

A CGNS database is defined by the existence of a
node or nodes which conforms to the specifications
given below for a node of type ``\fort{CGNSBase\_t}''.
This node is conceptually the root of the CGNS database. Because it is
created and controlled by the user, it is not the actual root of the
database file.
However, current CGNS conventions require that it be located directly below
the database root node which is identified by the name ``/''.

Further, by the mechanism of links, a CGNS database may span multiple
files. Thus there is no notion of a CGNS \emph{file}, only of a CGNS
\emph{database} implemented within one or more \emph{files}.

By virtue of its intended use, a CGNS database is dynamic in
that its content at any time reflects the current state of a CFD
problem of interest. For example, after the completion of a grid
generation procedure, a CGNS file may contain a grid but no boundary
conditions. Therefore, beyond the occurrence of a \fort{CGNSBase\_t}
node, there is no minimum content required in a CGNS database.

Conversely, there is no proscription against the inclusion, anywhere
within a CGNS database, of nodes of any form
whatsoever, provided only that their naming and labeling does not mimic
CGNS conventions. Such ``non-CGNS'' nodes, and those below them in the
tree, are not regarded as part of the CGNS database. CGNS software will
not detect the existence of non-CGNS nodes.

We may therefore take the following as a definition of a CGNS database:
\begin{quote}\itshape%
A CGNS database is a subtree of a file or files which is rooted at
a node with label ``\fort{CGNSBase\_t}'' and which conforms to the SIDS data
model as implemented by the SIDS File Mapping.
\end{quote}

\subsubsection{Location of CGNS Databases within a File}

An file may contain more than one \fort{CGNSBase\_t} node; i.e.,
there may be more than one CGNS database rooted within the same file.
CGNS software accepts the \emph{name} of the desired database as an
argument, and will locate the correct \fort{CGNSBase\_t} node within the
specified file.
Obviously, each \fort{CGNSBase\_t} node in a single file must have a
unique name.

A CGNS database may link to CGNS nodes in the same or other files.
Thus, for example, a CGNS database may reference the grid from
another CGNS database without physically copying the the information.
In this case, the structure of the file into which the link is made
is invisible except below the node to which the link is made.

\subsubsection{File Management}

Beyond \textit{Open} and \textit{Close} neither CGNS or the database
manager provides
any file management facilities. The user is responsible for ensuring
that:

\begin{itemize}
\item The file containing the root of the required database is
      available and its permissions are properly set at runtime.
\item If links are made to other files, including any not under the
      user's direct control, these are also available at runtime.
\item No file is opened for writing by more than one program at a time.
\end{itemize}

It is possible, within CGNS, to protect files from inadvertent writing
by opening them as ``read only''.

\subsection{Internal Organization of a CGNS Database}
\label{s:internal}

\subsubsection{The \texttt{CGNSBase\_t} Node}

At the highest level of the tree defining a CGNS database there is
always a node labeled ``\fort{CGNSBase\_t}''. The name of this
node is user defined, and serves essentially as the name of the
database itself. This name is used by the CGNS software to open the
database.

\subsubsection{The \texttt{CGNSLibraryVersion\_t} Node}

An file may also contain other nodes below the root node beside
\fort{CGNSBase\_t}, but these are \emph{not} officially part of the
CGNS database and will not be recognized by most CGNS software.
One exception to this is a node called \fort{CGNSLibraryVersion\_t},
which is a child of the root node.
This node stores the version number of the CGNS standard with which
the file is consistent, and is created automatically when the file is
created or modified using the CGNS Mid-Level Library.
Officially, the CGNS version number is not a part of the CGNS
database (because it is not located below \fort{CGNSBase\_t}).
But because the Mid-Level Library software makes use of it, the node is
included in this document.

\subsubsection{Topological Basis of CGNS Database Organization}

Below the root, the organization of a CGNS database reflects the problem
topology.
Omitting detail, \autoref{f:example_hierarchy} shows the overall
structure of the CGNS file.
Below the root node is the \fort{CGNSLibraryVersion\_t} node, and
one or more \fort{CGNSBase\_t} nodes.
Each \fort{CGNSBase\_t} node is the root of a CGNS database.

At the next level below a \fort{CGNSBase\_t} node are general
specifications which apply to the problem globally, such as reference
states, units, and so on.
At this level we also find a collection of nodes labeled
``\fort{Zone\_t}''.
The tree below each of these holds all the data local to one of the
various zones or subdomains which constitute the problem.

Beneath each \fort{Zone\_t} node there are nodes whose subtrees
store: the grid (labeled \fort{GridCoordinates\_t}); flowfields
(\fort{FlowSolution\_t}); boundary conditions (\fort{ZoneBC\_t});
information about the geometrical connection to other zones
(\fort{GridConnectivity\_t}); and information defining time-dependent
data. Below these there may be additional nodes containing yet more
geometrically local information. For example, under the \fort{ZoneBC\_t}
node there are nodes defining individual boundary conditions on portions
of faces of the zone (\fort{BC\_t}).

Certain types of nodes originally specified at a high level are
optionally repeated below. For example, immediately below a
\fort{Zone\_t} node we may find another \fort{ReferenceState\_t}
node (see \hyperref[s:figures]{Appendix~\ref*{s:figures}},
\autoref{f:Zone}). The CGNS convention is that such a node overrides
(for the associated portion of the topology only) any data found at a
higher level.

\subsubsection{Topics Not Currently Covered}
\label{s:notcovered}

No specification of the kind represented by this file mapping can ever
be complete. However, it is worth noting that there are certain entities
common in CFD which are not currently specified by the file mapping.

Within nodes of type \fort{FlowSolution\_t}, the current file mapping
permits the storage of fields of any number of dependent variables.
In addition to those whose names are specified in the SIDS the user may
add any desired quantities, naming them appropriately.
Names that are not currently codified in the SIDS will not be
common between practitioners without separate communication.

Obviously any sort of physical field could be stored in a
\fort{FlowSolution\_t} node.
The problem with using CGNS for such applications lies in the probable
need to specify additional physical information.
Standardizing this information is tantamount to extending the SIDS and
File Mapping to the disciplines in question.

Similarly, if a reacting flow problem requires the specification of rate
tables or catalytic wall boundary conditions, extensions to the SIDS and
File mapping will be needed.
