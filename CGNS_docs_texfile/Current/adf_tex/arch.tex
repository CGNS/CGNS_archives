\section{ADF File System Architectures}
\label{s:arch}
\thispagestyle{plain}

The following platform architectures have been tested and used both for
functionality testing of the ADF core software libraries and for testing
and running the prototype.

\begin{longtable}{l >{\quad}l >{\quad}l >{\quad}l}
\caption[Platform Architectures]{\textbf{Platform Architectures}}
\label{t:arch}
\\ \hline\hline \\*[-2ex]
\bold{Release} & \bold{Machine} & \bold{OS Version} & \bold{Native Format}
\\*[1ex] \hline\hline \\*[-2ex]
A01 & Cray          & Unicos 8.0  & Unicos 8.0 \\
A01 & SGI/IRIS      & 4.0.5       & IEEE Big Endian\footnote{
   In the table, ``Endian'' refers to the ordering of bytes in a
   multi-byte number.
   Big endian is a computer architecture in which, within a given
   multi-byte numeric representation, the most significant byte has the
   lowest address (the word is stored ``big-end-first'').
   Little endian is a computer architecture in which, within a given 16-
   or 32-bit word, bytes at lower addresses have lower significance (the
   word is stored ``little-end-first'').} \\
B01 & HP            & 9.05        & IEEE Big Endian \\
B01 & SGI/IRIS      & 5.03        & IEEE Little Endian \\
C00 & Intel Paragon & ---         & IEEE Little Endian \\
C00 & Dec Alpha     & ---         & IEEE Little Endian \\
C00 & SGI/IRIS      & 6.2         & IEEE Big Endian \\
C00 & Cray T90      & Unicos 9.02 & Cray Format
\\*[1ex] \hline\hline
\end{longtable}
