\section{Hierarchical Structures}
\label{s:topo}
\thispagestyle{plain}

This section presents the structure-type definitions for the top levels
of the CGNS hierarchy.
As stated in \autoref{s:design}, the hierarchy is topologically
based, where the overall organization is by zones.
All information pertaining to a given zone, including grid coordinates,
flow solution, and other related data, is contained within that zone's
structure entity.
\autoref{f:hierarchy} depicts this topologically based hierarchy.
The CGNS version number is described in \autoref{s:CGNSVersion}.
The CGNS database entry level structure type is defined in
\autoref{s:CGNSBase}, and the zone structure is defined in
\autoref{s:Zone}.
This section concludes with a discussion of globally applicable data.

\subsection{CGNS Version}
\label{s:CGNSVersion}

CGNS is an evolving standard.
Although great care is taken to make CGNS databases backward-compatible
with previous versions whenever possible, new nodes and new features are
still being added that make them non-forward-compatible.
To address this issue, each new version of the standard is labeled with
a version number that should be written in the file.
This version number corresponds to the version of the SIDS and is an
essential part of the file containing the CGNS database.
The file can not be interpreted properly without knowledge of this
version number.

Physically, this version number is located directly under the root node
of the file.
The \textit{SIDS File Mapping Manual}
defines this location more precisely.

Historically, the version number was used to describe the version of the
Mid-Level Library used to write or modify the file.
The corresponding node was thus named \fort{CGNSLibraryVersion\_t}.
With the advent of new libraries that can read and write CGNS databases,
the node is now defined as the version of the CGNS standard.
The Mid-Level Library modifies its interpretation of node data according
to the CGNS version number, and other libraries should also.

\subsection{CGNS Entry Level Structure Definition: \texttt{CGNSBase\_t}}
\label{s:CGNSBase}

The highest level structure in a CGNS database is |CGNSBase_t|.
It contains the cell dimension and physical dimension of the
computational grid and lists of zones and families making up the domain.
Globally applicable information, including a reference state, a set of
flow equations, dimensional units, time step or iteration information,
and convergence history are also attached.
In addition, structures for describing or annotating the database are
also provided; these same descriptive mechanisms are provided for
structures at all levels of the hierarchy.
\begin{alltt}
  CGNSBase\_t :=
    \{
    List( Descriptor\_t Descriptor1 ... DescriptorN ) ;                      (o)

    int CellDimension ;                                                     (r)
    int PhysicalDimension ;                                                 (r)

    BaseIterativeData\_t BaseIterativeData ;                                 (o)

    List( Zone\_t<CellDimension, PhysicalDimension> Zone1 ... ZoneN ) ;      (o)

    ReferenceState\_t ReferenceState ;                                       (o)

    Axisymmetry\_t Axisymmetry ;                                             (o)

    RotatingCoordinates\_t RotatingCoordinates ;                             (o)

    Gravity\_t Gravity ;                                                     (o)

    SimulationType\_t SimulationType ;                                       (o)

    DataClass\_t DataClass ;                                                 (o)

    DimensionalUnits\_t DimensionalUnits ;                                   (o)

    FlowEquationSet\_t<CellDimension> FlowEquationSet ;                      (o)

    ConvergenceHistory\_t GlobalConvergenceHistory ;                         (o)

    List( IntegralData\_t IntegralData1... IntegralDataN ) ;                 (o)

    List( Family\_t Family1... FamilyN ) ;                                   (o)

    List( UserDefinedData\_t UserDefinedData1 ... UserDefinedDataN ) ;       (o)
    \} ;
\end{alltt}

\begin{notes}
\item
 Default names for the \fort{Descriptor\_t}, \fort{Zone\_t},
 \fort{IntegralData\_t}, \fort{Family\_t}, and
 \fort{UserDefinedData\_t}
 lists are as shown; users may choose other legitimate names.
 Legitimate names must be unique at this level and shall not
 include the names \fort{Axisymmetry}, \fort{BaseIterativeData},
 \fort{DataClass}, \fort{DimensionalUnits}, \fort{FlowEquationSet},
 \fort{GlobalConvergenceHistory}, \fort{Gravity}, \fort{ReferenceState},
 \fort{RotatingCoordinates}, or \fort{SimulationType}.
\item
 The number of entities of type \fort{Zone\_t} defines the number of zones
 in the domain.
\item
 \fort{CellDimension} and \fort{PhysicalDimension} are the only required
 fields.
 The \fort{Descriptor\_t}, \fort{Zone\_t} and \fort{IntegralData\_t}
 lists may be empty, and all other optional fields absent.
\end{notes}

Note that we make the distinction between the following:

\begin{Ventryi}{\fort{PhysicalDimension}}
   \item [\fort{IndexDimension}]
         Number of different indices required to
         reference a node (e.g., $1 = i$, $2 = i,j$, $3 = i,j,k$)
   \item [\fort{CellDimension}]
         Dimensionality of the cell in the mesh (e.g., 3 for a volume cell,
         2 for a face cell)
   \item [\fort{PhysicalDimension}]
         Number of coordinates required to define a node position
         (e.g., 1 for 1-D, 2 for 2-D, 3 for 3-D)
\end{Ventryi}

These three dimensions may differ depending on the mesh.
For example, an unstructured triangular surface mesh representing the
wet surface of an aircraft will have:

\begin{itemize}
   \item \fort{IndexDimension} $=$ 1 (always for unstructured)
   \item \fort{CellDimension} $=$ 2  (face elements)
   \item \fort{PhysicalDimension} $=$ 3  (needs $x$, $y$, $z$ coordinates
         since it is a 3D surface)
\end{itemize}

For a structured zone, the quantities \fort{IndexDimension} and
\fort{CellDimension} are always equal.
For an unstructured zone, \fort{IndexDimension} always equals 1.
Therefore, storing \fort{CellDimension} at the \fort{CGNSBase\_t} level
will automatically define the \fort{IndexDimension} value for each zone.

On the other hand we assume that all zones of the base have the same
\fort{CellDimension}, e.g., if \fort{CellDimension} is 3, all zones
must be composed of 3D cells within the \fort{CGNSBase\_t}.

We need \fort{IndexDimension} for both structured and
unstructured zones in order to use original data structures
such as \fort{GridCoordinates\_t}, \fort{FlowSolution\_t},
\fort{DiscreteData\_t}, etc.
\fort{CellDimension} is necessary to express the interpolants in
\fort{ZoneConnectivity} with an unstructured zone (mismatch or overset
connectivity).
When the cells are bidimensional, two interpolants per node are required,
whereas when the cells are tridimensional, three interpolants per node must
be provided.
\fort{PhysicalDimension} becomes useful when expressing quantities
such as the \fort{InwardNormalList} in the \fort{BC\_t} data structure.
It's possible to have a mesh where \fort{IndexDimension} $=$ 2 but the
normal vectors still require $x$, $y$, $z$ components in order to be
properly defined.
Consider, for example, a structured surface mesh in the 3D space.

Information about the number of time steps or iterations being recorded,
and the time and/or iteration values at each step, may be contained in
the \fort{BaseIterativeData} structure.

Data specific to each zone in a multizone case is contained in the
list of |Zone_t| structure entities.

Reference data applicable to the entire CGNS database is contained in the
|ReferenceState| structure; quantities such as Reynolds number and
freestream Mach number are contained here (for external flow problems).

\fort{Axisymmetry} may be used to specify the axis of rotation and the
circumferential extent for an axisymmetric database.

If a rotating coordinate system is being used, the rotation
center and rotation rate vector may be specified using the
\fort{RotatingCoordinates} structure.

\fort{Gravity} may be used to define the gravitational vector.

\fort{SimulationType} is an enumeration type identifying the type
of simulation.

\begin{alltt}
  SimulationType\_t := Enumeration (
    SimulationTypeNull,
    SimulationTypeUserDefined,
    TimeAccurate,
    NonTimeAccurate ) ;
\end{alltt}

|DataClass| describes the global default for the class of data
contained in the CGNS database.
If the CGNS database contains dimensional data (e.g., velocity with units
of m/s), \fort{DimensionalUnits} may be used to describe the system of
units employed.

|FlowEquationSet| contains a description of the governing flow
equations associated with the entire CGNS database.  This structure contains
information on the general class of governing equations (e.g., Euler or
Navier-Stokes), equation sets required for closure, including turbulence
modeling and equations of state, and constants associated with the
equations.

|DataClass|, |DimensionalUnits|, |ReferenceState| and
|FlowEquationSet| have special function in the CGNS hierarchy.  They
are globally applicable throughout the database, but their precedence may
be superseded by local entities (e.g., within a given zone).  The scope of
these entities and the rules for determining precedence are treated in 
\autoref{s:precedence}.

Globally relevant convergence history information is contained in
|GlobalConvergenceHistory|.  This convergence information includes total
configuration forces, moments, and global residual and solution-change
norms taken over all the zones.

Miscellaneous global data may be contained in the |IntegralData_t| list.
Candidates for inclusion here are global forces and moments.

The \fort{Family\_t} data structure, defined in \autoref{s:Family},
is used to record geometry reference data.
It may also include boundary conditions linked to geometry patches.
For the purpose of defining material properties, families may also be
defined for groups of elements.
The family-mesh association is defined under the \fort{Zone\_t} and
\fort{BC\_t} data structures by specifying the family name corresponding
to a zone or a boundary patch.

The \fort{UserDefinedData\_t} data structure allows arbitrary
user-defined data to be stored in \fort{Descriptor\_t} and
\fort{DataArray\_t} children without the restrictions or implicit
meanings imposed on these node types at other node locations.

\subsection{Zone Structure Definition: \texttt{Zone\_t}} 
\label{s:Zone}

The |Zone_t| structure contains all information pertinent to an individual
zone.
This information includes the zone type, the number of cells and
vertices making up the grid in that zone, the physical coordinates
of the grid vertices, grid motion information, the family, the flow
solution, zone interface connectivity, boundary conditions, and zonal
convergence history data.
Zonal data may be recorded at multiple time steps or iterations.
In addition, this structure contains a reference state, a set of flow
equations and dimensional units that are all unique to the zone.
For unstructured zones, the element connectivity may also be recorded.

\begin{alltt}	
  ZoneType\_t := Enumeration(
    ZoneTypeNull,
    ZoneTypeUserDefined,
    Structured,
    Unstructured ) ;

  Zone_t< int CellDimension, int PhysicalDimension > :=
    \{
    List( Descriptor_t Descriptor1 ... DescriptorN ) ;                      (o)

    ZoneType_t ZoneType ;                                                   (r)

    int[IndexDimension] VertexSize ;                                        (r)
    int[IndexDimension] CellSize ;                                          (r)
    int[IndexDimension] VertexSizeBoundary ;                                (o/d)

    List( GridCoordinates_t<IndexDimension, VertexSize>
          GridCoordinates MovedGrid1 ... MovedGridN ) ;                     (o)

    List( Elements_t Elements1 ... ElementsN ) ;                            (o)

    List( RigidGridMotion_t RigidGridMotion1 ... RigidGridMotionN ) ;       (o)

    List( ArbitraryGridMotion_t<IndexDimension, VertexSize, CellSize>
          ArbitraryGridMotion1 ... ArbitraryGridMotionN ) ;                 (o)

    FamilyName_t FamilyName ;                                               (o)

    List( AdditionalFamilyName\_t AddFamilyName1 ... AddFamilyNameN ) ;     (o)

    List( FlowSolution_t<CellDimension, IndexDimension, VertexSize, CellSize> 
          FlowSolution1 ... FlowSolutionN ) ;                               (o)

    List( DiscreteData_t<CellDimension, IndexDimension, VertexSize, CellSize> 
          DiscreteData1 ... DiscreteDataN ) ;                               (o)

    List( IntegralData_t IntegralData1 ... IntegralDataN ) ;                (o)

    List( ZoneGridConnectivity_t<IndexDimension, CellDimension>
          ZoneGridConnectivity1 ... ZoneGridConnectivityN ;                 (o)

    List( ZoneSubRegion_t<IndexDimension, VertexSize, CellSize>
          ZoneSubRegion1 ... ZoneSubRegion ) ;                              (o)

    ZoneBC_t<CellDimension, IndexDimension, PhysicalDimension> ZoneBC ;     (o)

    ZoneIterativeData_t<NumberOfSteps> ZoneIterativeData ;                  (o)

    ReferenceState_t ReferenceState ;                                       (o)

    RotatingCoordinates\_t RotatingCoordinates ;                             (o)

    DataClass_t DataClass ;                                                 (o)

    DimensionalUnits_t DimensionalUnits ;                                   (o)

    FlowEquationSet_t<CellDimension> FlowEquationSet ;                      (o)

    ConvergenceHistory_t ZoneConvergenceHistory ;                           (o)

    List( UserDefinedData\_t UserDefinedData1 ... UserDefinedDataN ) ;       (o)

    int Ordinal ;                                                           (o)
    \} ;
\end{alltt}

\begin{notes}
\item
 Default names for the \texttt{Descriptor\_t},
 \texttt{Elements\_t}, \texttt{RigidGridMotion\_t}, \texttt{ArbitraryGridMotion\_t},
 \texttt{FlowSolution\_t}, \texttt{DiscreteData\_t}, \texttt{IntegralData\_t}, and
 \texttt{UserDefinedData\_t}
 lists are as shown; users may choose other legitimate names.
 Legitimate names must be unique within a given instance of \texttt{Zone\_t}
 and shall not include the names
 \texttt{DataClass}, \texttt{DimensionalUnits}, \texttt{FamilyName},
 \texttt{FlowEquationSet}, \texttt{GridCoordinates}, \texttt{Ordinal},
 \texttt{Reference\-State}, \texttt{RotatingCoordinates}, \texttt{ZoneBC},
 \texttt{ZoneConvergenceHistory}, \texttt{ZoneGridConnectivity},
 \texttt{ZoneIterativeData}, or \texttt{ZoneType}.
\item
 The original grid coordinates should have the name \texttt{GridCoordinates}.
 Default names for the remaining entities in the \texttt{GridCoordinates\_t}
 list are as shown; users may choose other legitimate names, subject to
 the restrictions listed in the previous note.
\item
 \texttt{ZoneType}, \texttt{VertexSize}, and \texttt{CellSize}
 are the only required fields within the \texttt{Zone\_t} structure.
\end{notes}

|Zone_t| requires the parameters |CellDimension| and
|PhysicalDimension|.
\fort{CellDimension}, along with the type of zone, determines
\fort{IndexDimension}; if the zone type is \fort{Unstructured},
\fort{IndexDimension} $=$ 1, and if the zone type is \fort{Structured}, 
\fort{IndexDimension} $=$ \fort{CellDimension}.
These three structure parameters identify the dimensionality of the
grid-size arrays.
One or more of them are passed on to the grid coordinates, flow
solution, interface connectivity, boundary condition and flow-equation
description structures.

\texttt{VertexSize} is the number of vertices in each index direction,
\texttt{and CellSize} is the number of cells in each direction.
For example, for structured grids in 3-D, \texttt{CellSize = VertexSize -
[1,1,1]}, and for unstructured grids in 3-D, \texttt{CellSize} is
simply the total number of 3-D cells.
\texttt{VertexSize} is the number of vertices defining ``the grid'' or
the domain (i.e., without rind points); \texttt{CellSize} is the number
of cells on the interior of the domain.
These two grid-size arrays are passed onto the grid-coordinate,
flow-solution and discrete-data substructures.

If the nodes are sorted between internal nodes and boundary nodes,
then the optional parameter \fort{VertexSizeBoundary} must be set equal
to the number of boundary nodes.
If the nodes are sorted, the grid coordinate vector must first include
the boundary nodes, followed by the internal nodes.
By default, \fort{VertexSizeBoundary} equals zero, meaning that the nodes
are unsorted.
This option is only useful for unstructured zones.
For structured zones, \fort{VertexSizeBoundary} always equals 0 in all
index directions.

The \fort{GridCoordinates\_t} structure defines ``the grid''; it contains
the physical coordinates of the grid vertices, and may optionally
contain physical coordinates of rind or ghost points.
The original grid is contained in \fort{GridCoordinates}.
Additional \fort{GridCoordinates\_t} data structures are allowed, to
store the grid at multiple time steps or iterations.

When the grid nodes are sorted, the \fort{DataArray\_t} in
\fort{GridCoordinates\_t} lists first the data for the boundary nodes,
then the data for the internal nodes.

The \fort{Elements\_t} data structure contains unstructured elements
data such as connectivity, element type, parent elements, etc.

The \fort{RigidGridMotion\_t} and \fort{ArbitraryGridMotion\_t} data
structures contain information defining rigid and arbitrary (i.e.,
deforming) grid motion.

\fort{FamilyName} identifies to which family a zone belongs.
Families may be used to define material properties.
Where multiple families are desired, \fort{AdditionalFamilyName} 
nodes may be used to specify them. 

Flow-solution quantities are contained in the list of |FlowSolution_t|
structures.
Each instance of the |FlowSolution_t| structure is only allowed to
contain data at a single grid location (vertices, cell-centers, etc.);
multiple |FlowSolution_t| structures are provided to store flow-solution
data at different grid locations, to record different solutions at
the same grid location, or to store solutions at multiple time steps
or iterations.
These structures may optionally contain solution data defined at rind
points.

Miscellaneous discrete field data is contained in the list of
|DiscreteData_t| structures.
Candidate information includes residuals, fluxes and other related
discrete data that is considered auxiliary to the flow solution.
Likewise, miscellaneous zone-specific global data, other than
reference-state data and convergence history information, is contained
in the list of |IntegralData_t| structures.
It is envisioned that these structures will be seldom used in practice
but are provided nonetheless.

The \fort{ZoneSubRegion\_t} node allows flowfield or other information
to be specified over a subset of the entire zone. 

For unstructured zones only, the node-based \fort{DataArray\_t}
vectors (\fort{GridLocation = Vertex}) in \fort{FlowSolution\_t} or
\fort{DiscreteData\_t} must follow exactly the same ordering as the
\fort{GridCoordinates} vector.
If the nodes are sorted (\fort{VertexSizeBoundary} $\neq 0$), the data
on the boundary nodes must be listed first, followed by the data on the
internal nodes.
Note that the order in which the node-based data are recorded must
follow exactly the node ordering in \fort{GridCoordinates\_t}, to be
able to associate the data to the correct nodes.
For element-based data (\fort{GridLocation = CellCenter}), the
\fort{FlowSolution\_t} or \fort{DiscreteData\_t} data arrays must list
the data values for each element, in the same order as the elements are
listed in \fort{ElementConnectivity}.

All interface connectivity information, including identification
of overset-grid holes, for a given zone is contained in
|ZoneGridConnectivity|.

All boundary condition information pertaining to a zone is contained in 
|ZoneBC_t|.

The \fort{ZoneIterativeData\_t} data structure may be used to record
pointers to zonal data at multiple time steps or iterations.

Reference-state data specific to an individual zone is contained in the
|ReferenceState| structure.  

\fort{RotatingCoordinates} may be used to specify the rotation center
and rotation rate vector of a rotating coordinate system.

|DataClass| defines the zonal default for the class of data contained in
the zone and its substructures.  If a zone contains dimensional data,
|DimensionalUnits| may be used to describe the system of
dimensional units employed.

If a set of flow equations are specific to a given zone, these may be
described in |FlowEquationSet|.  For example, if a single zone within
the domain is inviscid, whereas all other are turbulent, then this
zone-specific equation set could be used to describe the special zone.

|DataClass|, |DimensionalUnits|, |ReferenceState| and |FlowEquationSet| have
special function in the hierarchy.  They are applicable throughout a given
zone, but their precedence may be superseded by local entities contained in
the zone's substructures.  If any of these entities are present within a
given instance of |Zone_t|, they take precedence over the corresponding
global entities contained in database's |CGNSBase_t| entity.  These
precedence rules are further discussed in \autoref{s:precedence}.

Convergence history information applicable to the zone is contained in
|ZoneConvergenceHistory|; this includes residual and solution-change norms.

The \fort{UserDefinedData\_t} data structure allows arbitrary
user-defined data to be stored in \fort{Descriptor\_t} and
\fort{DataArray\_t} children without the restrictions or implicit
meanings imposed on these node types at other node locations.

|Ordinal| is user-defined and has no restrictions on the values that
it can contain.
It is included for backward compatibility to assist implementation
of the CGNS system into  applications whose I/O depends heavily on the
numbering of zones.
Since there are no restrictions on the values contained in |Ordinal|
(or that |Ordinal| is even provided), there is no guarantee that the
zones in an existing CGNS database will have sequential values from
1 to $N$ without holes or repetitions.
Use of |Ordinal| is discouraged and is on a user-beware basis.

\subsection{Precedence Rules and Scope Within the Hierarchy}
\label{s:precedence}

The dependence of a structure entity's information on data contained at 
higher levels of the hierarchy is typically explicitly expressed through
structure parameters.  For example, all arrays within |Zone_t| depend on
the dimensionality of the computational grid.  This dimensionality is
passed down to a |Zone_t| entity through a structure parameter in the
definition of |Zone_t|.

We have established an alternate dependency for a limited number
of entities that is not explicitly stated in the structure type
definitions.  These special situations include entities for describing
data class, system of dimensional units, reference states and flow
equation sets.  At each level of the hierarchy (where appropriate),
entities for describing this information are defined, and if present
they take precedence over all corresponding information existing at
higher levels of the CGNS hierarchy.  Essentially, we have established
globally applicable data with provisions for recursively overriding it
with local data.

Specifically, the entities that follow this alternate dependency are:
\begin{itemize}
\item |FlowEquationSet_t FlowEquationSet|,
\item |ReferenceState_t ReferenceState|,
\item |DataClass_t DataClass|,
\item |DimensionalUnits_t DimensionalUnits|.
\end{itemize}
\fort{FlowEquationSet} contains a description of the governing flow
equations (see \autoref{s:floweqn}); \fort{ReferenceState} describes a
set of reference state flow conditions (see \autoref{s:ReferenceState});
\fort{DataClass} defines the class of data (e.g., dimensional or
nondimensional---see \autoref{s:DataClass} and \autoref{s:data}); and
\fort{DimensionalUnits} specifies the system of units used for
dimensional data (see \autoref{s:DimensionalUnits}).

All of these entities may be defined within the highest level |CGNSBase_t|
structure, and if present in a given database, establish globally applicable
information; these may also be considered to be global defaults.  Each of
these four entities may also be defined within the |Zone_t| structure.  If
present in a given instance of |Zone_t|, they supersede the global data and
establish new defaults that apply only within that zone.

For example, if a different set of flow equations is solved within a
given zone than is solved in the rest of the flowfield, then this can be
conveyed through |FlowEquationSet|.
In this case, one |FlowEquationSet| entity would be placed within
|CGNSBase_t| to state the globally applicable flow equations, and a second
|FlowEquationSet| entity would be placed within the given zone (within its
instance of |Zone_t|); this second |FlowEquationSet| entity supersedes the
first only within the given zone.

In addition to its presence in |CGNSBase_t| and |Zone_t|,
|ReferenceState| may also be defined within the boundary-condition
structure types to establish reference states applicable to one or more
boundary-condition patches.  Actually, |ReferenceState| entities can be
defined at several levels of the boundary-condition hierarchy to allow
flexibility in setting the appropriate reference state conditions (see
\autoref{s:BCstruct} and subsequent sections).

\texttt{DataClass} and \texttt{DimensionalUnits} are used within
entities describing data arrays (see the \texttt{DataArray\_t} type
definition in \autoref{s:DataArray}).
They classify the data and specify its system of units if dimensional.  If
these entities are absent from a particular instance of |DataArray_t|, the
information is derived from appropriate global data.  |DataClass| and
|DimensionalUnits| are also declared in all intermediate structure types that
directly or indirectly contain |DataArray_t| entities.  Examples include
|GridCoordinates_t| (\autoref{s:Grid}), |FlowSolution_t|
(\autoref{s:FlowSolution}), |BC_t| (\autoref{s:BCdefn}) and |ReferenceState_t|
(\autoref{s:ReferenceState}).  The same precedence rules apply---lower-level
entities supersede higher-level entities.

It is envisioned that in practice, the use of globally applicable data will
be the norm rather than the exception.  It provides a measure of economy
throughout the CGNS database in many situations.  For example, when creating
a database where the vast majority of data arrays are dimensional and use a
consistent set of units, |DataClass| and |DimensionalUnits| can be set
appropriately at the |CGNSBase_t| level and thereafter omitted when
outputting data.
