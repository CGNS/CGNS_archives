\section{Governing Flow Equations}
\label{s:floweqn}
\thispagestyle{plain}

This section provides structure type definitions for describing the
governing flow-equation set associated with the database.
The description includes the general class of governing equations, the
turbulent closure equations, the gas and chemistry models, the
viscosity and thermal-conductivity models, and the electromagnetics
models.
Included with each equation description are associated constants.
The structure definitions attempt to balance the opposing requirements
for future growth and extensibility with initial ease of implementation.
Included in the final section (\autoref{s:flowexample}) are examples of
flow-equation sets.

The intended use of these structures initially is primarily for archival
purposes and to provide additional documentation of the flow solution.
If successful in this role, it is foreseeable that these flow-equation
structures may eventually be also used as inputs for grid generators,
flow solvers, and post-processors.

\subsection{Flow Equation Set Structure Definition: \texttt{FlowEquationSet\_t}}
\label{s:FlowEquationSet}

|FlowEquationSet_t| is a general description of the governing flow
equations.
It includes the dimensionality of the governing equations, and the
collection of specific equation-set descriptions covered in subsequent
sections.
It can be a child node of either \fort{CGNSBase\_t} or \fort{Zone\_t}
(or both).
\begin{alltt}
  FlowEquationSet\_t< int CellDimension > :=
    \{
    List( Descriptor\_t Descriptor1 ... DescriptorN ) ;                      (o)

    int EquationDimension ;                                                 (o)  
    
    GoverningEquations\_t<CellDimension> GoverningEquations ;                (o)

    GasModel\_t GasModel ;                                                   (o)

    ViscosityModel\_t ViscosityModel ;                                       (o)

    ThermalConductivityModel\_t ThermalConductivityModel ;                   (o)

    TurbulenceClosure\_t TurbulenceClosure ;                                 (o)

    TurbulenceModel\_t<CellDimension> TurbulenceModel ;                      (o)

    ThermalRelaxationModel\_t ThermalRelaxationModel ;                       (o)

    ChemicalKineticsModel\_t ChemicalKineticsModel ;                         (o)

    EMElectricFieldModel\_t EMElectricFieldModel ;                           (o)

    EMMagneticFieldModel\_t EMMagneticFieldModel ;                           (o)

    EMConductivityModel\_t EMConductivityModel ;                             (o)

    DataClass\_t DataClass ;                                                 (o)
                
    DimensionalUnits\_t DimensionalUnits ;                                   (o)

    List( UserDefinedData\_t UserDefinedData1 ... UserDefinedDataN ) ;       (o)
    \} ;
\end{alltt}

\begin{notes}
\item Default names for the \fort{Descriptor\_t} and
      \fort{UserDefinedData\_t} lists are as shown; users may choose
      other legitimate names.
      Legitimate names must be unique within a given instance of
      \fort{FlowEquationSet\_t} and shall not include the names
      \fort{EMConductivityModel}, \fort{EMElectricFieldModel},
      \fort{EMMagneticFieldModel}, \fort{EquationDimension},
      \fort{GoverningEquations}, \fort{GasModel}, \fort{ViscosityModel},
      \fort{ThermalConductivityModel}, \fort{TurbulenceClosure},
      \fort{TurbulenceModel}, \fort{ThermalRelaxationModel},
      \fort{ChemicalKineticsModel}, \fort{DataClass}, or
      \fort{Di\-men\-sion\-al\-Units}.
\item There are no required elements for \fort{FlowEquationSet\_t}.
\end{notes}

|FlowEquationSet_t| requires a single structure parameter,
\fort{CellDimension}, to identify the dimensionality of index arrays for
structured grids.
This parameter is passed onto several substructures.

\fort{EquationDimension} is the dimensionality of the governing
equations; it is the number of spatial variables describing the flow.
\fort{GoverningEquations} describes the general class of flow equations.
\fort{GasModel} describes the equation of state, and
\fort{ViscosityModel} and \fort{ThermalConductivityModel} describe the
auxiliary relations for molecular viscosity and the thermal conductivity
coefficient.
\fort{TurbulenceClosure} and \fort{TurbulenceModel} describe the
turbulent closure for the Reynolds-aver\-aged Navier-Stokes equations.
\fort{ThermalRelaxationModel} and \fort{ChemicalKineticsModel} describe
the equations used to model thermal relaxation and chemical kinetics.
\fort{EMElectricFieldModel}, \fort{EMMagneticFieldModel}, and
\fort{EMConductivityModel} describe the equations used to model
electromagnetics.

\fort{DataClass} defines the default for the class of data contained in
the flow-equation set.
For any data that is dimensional, \fort{DimensionalUnits} may be used to
describe the system of dimensional units employed.
If present, these two entities take precedence of all corresponding
entities at higher levels of the hierarchy.
These precedence rules are further discussed in \autoref{s:precedence}.

The \fort{UserDefinedData\_t} data structure allows arbitrary
user-defined data to be stored in \fort{Descriptor\_t} and
\fort{DataArray\_t} children without the restrictions or implicit
meanings imposed on these node types at other node locations.

\subsection{Governing Equations Structure Definition: \texttt{GoverningEquations\_t}}
\label{s:GoverningEquations}

|GoverningEquations_t| describes the class of governing flow equations
associated with the solution.  
\begin{alltt}
  GoverningEquationsType\_t := Enumeration(
    GoverningEquationsTypeNull,
    GoverningEquationsTypeUserDefined,
    FullPotential,
    Euler,
    NSLaminar,
    NSTurbulent,
    NSLaminarIncompressible,
    NSTurbulentIncompressible ) ;

  GoverningEquations\_t< int CellDimension > :=
    \{
    List( Descriptor\_t Descriptor1 ... DescriptorN ) ;                      (o)

    GoverningEquationsType\_t GoverningEquationsType ;                       (r)
    
    int[CellDimension*(CellDimension + 1)/2] DiffusionModel ;               (o)

    List( UserDefinedData\_t UserDefinedData1 ... UserDefinedDataN ) ;       (o)
    \} ;
\end{alltt}

\begin{notes}
\item
 Default names for the \fort{Descriptor\_t} and
 \fort{UserDefinedData\_t}
 lists are as shown; users may choose other legitimate names.
 Legitimate names must be unique within a given instance of
 \fort{GoverningEquations\_t} and shall not include the name
 \fort{DiffusionModel}.
\item
 |GoverningEquationsType| is the only required element.
\item
 The length of the |DiffusionModel| array is as follows: in 1-D it is
 |int[1]|; in 2-D it is |int[3]|; and in 3-D it is |int[6]|.
 For unstructured zones, \fort{DiffusionModel} is not supported, and
 should not be used.
\end{notes}

|GoverningEquations_t| requires a single structure parameter,
\fort{CellDimension}.
It is used to define the length of the array |DiffusionModel|.

|DiffusionModel| describes the viscous diffusion terms modeled in the flow
equations, and is applicable only to the Navier-Stokes equations
with structured grids.
Typically, thin-layer approximations include only the diffusion terms in
one or two computational-coordinate directions.
\fort{DiffusionModel} encodes the coordinate directions that include
second-derivative and cross-derivative diffusion terms.
The first \fort{CellDimension} elements are second-derivative terms and
the remainder elements are cross-derivative terms.
Allowed values for individual elements in the array |DiffusionModel| are
0 and 1; a value of 1 indicates the diffusion term is modeled, and 0
indicates that they are not modeled.
In 3-D, the encoding of |DiffusionModel| is as follows:
\begin{center}
\begin{tabular}{c >{\quad}l}
\hline\hline \\*[-2ex]
\bold{Element} & \bold{Modeled Terms}
\\*[1ex] \hline\hline \\*[-2ex]
$n = 1$ & Diffusion terms in $i$ ($\partial^2/\partial \xi^2$) \\
$n = 2$ & Diffusion terms in $j$ ($\partial^2/\partial \eta^2$) \\
$n = 3$ & Diffusion terms in $k$ ($\partial^2/\partial \zeta^2$) \\
$n = 4$ & Cross-diffusion terms in $i$-$j$ 
          ($\partial^2/\partial \xi \partial \eta$ and
           $\partial^2/\partial \eta \partial \xi$) \\
$n = 5$ & Cross-diffusion terms in $j$-$k$ 
          ($\partial^2/\partial \eta \partial \zeta$ and
           $\partial^2/\partial \zeta \partial \eta$) \\
$n = 6$ & Cross-diffusion terms in $k$-$i$ 
          ($\partial^2/\partial \zeta \partial \xi$ and
           $\partial^2/\partial \xi \partial \zeta$)
\\*[1ex] \hline\hline
\end{tabular}
\end{center}
where derivatives in the $i$, $j$ and $k$ computational-coordinates are
$\xi$, $\eta$ and $\zeta$, respectively.
The full Navier-Stokes equations in 3-D are indicated by
|DiffusionModel = [1,1,1,1,1,1]|, and the thin-layer equations
including only diffusion in the $j$-direction are |[0,1,0,0,0,0]|. 

The \fort{UserDefinedData\_t} data structure allows arbitrary
user-defined data to be stored in \fort{Descriptor\_t} and
\fort{DataArray\_t} children without the restrictions or implicit
meanings imposed on these node types at other node locations.

\subsection{Model Type Structure Definition: \
\texttt{ModelType\_t}}
\label{s:ModelType}

\fort{ModelType\_t} is a complete list of all models covered in subsequent
sections. A specific model will contain a subset of this enumeration.
\begin{alltt}
  GasModelType_t := Enumeration(
    ModelTypeNull, ModelTypeUserDefined,
    Ideal, VanderWaals, Constant, PowerLaw,
    SutherlandLaw, ConstantPrandtl, EddyViscosity,
    ReynoldsStress, ReynoldsStressAlgebraic,
    Algebraic_BaldwinLomax, Algebraic_CebeciSmith,
    HalfEquation_JohnsonKing, OneEquation_BaldwinBarth,
    OneEquation_SpalartAllmaras, TwoEquation_JonesLaunder,
    TwoEquation_MenterSST, TwoEquation_Wilcox,
    CaloricallyPerfect, ThermallyPerfect, ConstantDensity,
    RedlichKwong, Frozen, ThermalEquilib, ThermalNonequilib,
    ChemicalEquilibCurveFit, ChemicalEquilibMinimization,
    ChemicalNonequilib, EMElectricField, EMMagneticField,
    EMConductivity, Voltage, Interpolated,
    Equilibrium_LinRessler, Chemistry_LinRessler ) ;
\end{alltt}

\subsection{Thermodynamic Gas Model Structure Definition: \
\texttt{GasModel\_t}}
\label{s:GasModel}

\fort{GasModel\_t} describes the equation of state model used in the
governing equations to relate pressure, temperature and density.
The enumerated values for \texttt{GasModelType\_t} are a subset of the
\texttt{ModelType\_t} enumeration.
\begin{alltt}
  GasModelType\_t := Enumeration(
    ModelTypeNull,
    ModelTypeUserDefined,
    Ideal,
    VanderWaals,
    CaloricallyPerfect,
    ThermallyPerfect,
    ConstantDensity,
    RedlichKwong ) ;
\end{alltt}

\begin{alltt}
  GasModel_t :=
    \{
    List( Descriptor\_t Descriptor1 ... DescriptorN ) ;                      (o)

    GasModelType\_t GasModelType ;                                           (r)
    
    List( DataArray\_t<DataType, 1, 1> DataArray1 ... DataArrayN ) ;         (o)

    DataClass\_t DataClass ;                                                 (o)
                
    DimensionalUnits\_t DimensionalUnits ;                                   (o)

    List( UserDefinedData\_t UserDefinedData1 ... UserDefinedDataN ) ;       (o)
    \} ;
\end{alltt}

\begin{notes}
\item
 Default names for the \fort{Descriptor\_t}, \fort{DataArray\_t}, and
 \fort{UserDefinedData\_t}
 lists are as shown; users may choose other legitimate names.
 Legitimate names must be unique within a given instance of
 \fort{GasModel\_t} and shall not include the names \fort{DataClass} or
 \fort{Di\-men\-sion\-al\-Units}.
\item
 \fort{GasModelType} is the only required element.
\item
 The \fort{GasModelType} enumeration name \fort{Ideal} implies
 a calorically perfect single-component gas, but the more descriptive
 name \fort{CaloricallyPerfect} is generally preferred.
\end{notes}

For a perfect gas (\fort{GasModelType = CaloricallyPerfect}), the
pressure, temperature and density are related by,
$$
 p = \rho R T,
$$
where $R$ is the ideal gas constant.  Related quantities are the specific
heat at constant pressure ($c_p$), specific heat at constant volume ($c_v$)
and specific heat ratio ($\gamma = c_p/c_v$).  The gas constant and specific
heats are related by $R = c_p - c_v$.  Data-name identifiers associated with
the perfect gas law are listed in \autoref{t:id_perfect}.

\begin{table}[htbp]
\centering
\caption[Data-Name Identifiers for Perfect Gas]{\textbf{Data-Name Identifiers for Perfect Gas}}
\label{t:id_perfect}
\begin{tabular}{>{\ttfamily}l >{\quad}l >{\quad}c}
\\ \hline\hline \\*[-2ex]
\bold{Data-Name Identifier} & \bold{Description} & \bold{Units}
\\*[1ex] \hline\hline \\*[-2ex]
IdealGasConstant     & Ideal gas constant ($R$)                     &
   $\L^2/(\T^2 \TH)$ \\
SpecificHeatRatio    & Ratio of specific heats ($\gamma = c_p/c_v$) &
   - \\
SpecificHeatVolume   & Specific heat at constant volume ($c_v$)     &
   $\L^2/(\T^2 \TH)$ \\
SpecificHeatPressure & Specific heat at constant pressure ($c_p$)   &
   $\L^2/(\T^2 \TH)$
\\*[1ex] \hline\hline
\end{tabular}
\end{table}

If it is desired to specify any of these identifiers in a CGNS database,
they should be defined as \fort{DataArray}s under \fort{GasModel\_t}.

The dimensional units are defined as follows: $\M$ is mass, $\L$ is length,
$\T$ is time and $\TH$ is temperature.  These are further
described in \hyperref[s:dataname]{Appendix~\ref*{s:dataname}}.

|DataClass| defines the default for the class of data contained in the
thermodynamic gas model.  For any data that is dimensional,
|DimensionalUnits| may be used to describe the system of dimensional units
employed.  If present, these two entities take precedence of all
corresponding entities at higher levels of the hierarchy.
These precedence rules are further discussed in \autoref{s:precedence}.

The \fort{UserDefinedData\_t} data structure allows arbitrary
user-defined data to be stored in \fort{Descriptor\_t} and
\fort{DataArray\_t} children without the restrictions or implicit
meanings imposed on these node types at other node locations.

\subsection{Molecular Viscosity Model Structure Definition: \texttt{ViscosityModel\_t}} 

|ViscosityModel_t| describes the model for relating molecular
viscosity ($\mu$) to temperature.
The enumerated values for \texttt{ViscosityModelType\_t} are a subset of the
\texttt{ModelType\_t} enumeration.
\begin{alltt}
  ViscosityModelType\_t := Enumeration(
    ModelTypeNull,
    ModelTypeUserDefined,
    Constant,
    PowerLaw,
    SutherlandLaw ) ;

  ViscosityModel\_t :=
    \{
    List( Descriptor\_t Descriptor1 ... DescriptorN ) ;                      (o)

    ViscosityModelType\_t ViscosityModelType ;                               (r)
    
    List( DataArray\_t<DataType, 1, 1> DataArray1 ... DataArrayN ) ;         (o)

    DataClass\_t DataClass ;                                                 (o)
                
    DimensionalUnits\_t DimensionalUnits ;                                   (o)

    List( UserDefinedData\_t UserDefinedData1 ... UserDefinedDataN ) ;       (o)
    \} ;
\end{alltt}

\begin{notes}
\item
 Default names for the \fort{Descriptor\_t}, \fort{DataArray\_t}, and
 \fort{UserDefinedData\_t}
 lists are as shown; users may choose other legitimate names.
 Legitimate names must be unique within a given instance of
 \fort{ViscosityModel\_t} and shall not include the names
 \fort{DataClass} or \fort{DimensionalUnits}.
\item
 |ViscosityModelType| is the only required element.
\end{notes}

The molecular viscosity models are as follows: |Constant| states that
molecular viscosity is constant throughout the field and is equal to some
reference value ($\mu = \mu_{\rm ref}$); |PowerLaw| states that molecular
viscosity follows a power-law relation,
$$
 \mu = \mu_{\rm ref} \left( T \over T_{\rm ref} \right)^n
$$
and |SutherlandLaw| is Sutherland's Law for molecular viscosity,
$$
 \mu = \mu_{\rm ref} \left( T \over T_{\rm ref} \right)^{3/2} 
  {T_{\rm ref} + T_s \over T + T_s},
$$
where $T_s$ is the Sutherland's Law constant, and $\mu_{\rm ref}$ and
$T_{\rm ref}$ are the reference viscosity and temperature, respectively.
For air\footnote{White, F. M., {\it Viscous Fluid Flow}, McGraw-Hill, 1974,
p.~28-29}, the power-law exponent is $n = 0.666$, Sutherland's law constant
($T_s$) is 110.6 K, the reference temperature ($T_{\rm ref}$) is 273.15 K,
and the reference viscosity ($\mu_{\rm ref}$) is 
$1.716 \!\times\! 10^{-5}$ kg/(m-s).
The data-name identifiers for molecular viscosity models are defined in
\autoref{t:id_viscosity}.

\settowidth{\tmplengtha}{\fort{ViscosityModelType}}
\settowidth{\tmplengthb}{\fort{ViscosityMolecularReference}}
\settowidth{\tmplengthc}{$\M/(\L \T)$}
\setlength{\Pwidth}{\linewidth-8\tabcolsep-\tmplengtha-\tmplengthb-\tmplengthc}
\begin{table}[htbp]
\centering
\caption[Data-Name Identifiers for Molecular Viscosity Models]{\textbf{Data-Name Identifiers for Molecular Viscosity Models}}
\label{t:id_viscosity}
\begin{tabular}{>{\ttfamily}l >{\ttfamily}l >{\raggedright\arraybackslash}p{\Pwidth} c}
\\ \hline\hline \\*[-2ex]
ViscosityModelType & \bold{Data-Name Identifer} & \bold{Description} & \bold{Units}
\\*[1ex] \hline\hline \\*[-2ex]
PowerLaw      & PowerLawExponent
   & Power-law exponent ($n$) & - \\
SutherlandLaw & SutherlandLawConstant
   & Sutherland's Law constant ($T_s$)     & $\TH$ \\
\ital{All}    & TemperatureReference
   & Reference temperature ($T_{\rm ref}$) & $\TH$ \\
\ital{All}    & ViscosityMolecularReference
   & Reference viscosity ($\mu_{\rm ref}$) & $\M/(\L\T)$
\\*[1ex] \hline\hline
\end{tabular}
\end{table}

If it is desired to specify any of these identifiers in a CGNS
database, they should be defined as \fort{DataArray}s under
\fort{ViscosityModel\_t}.

|DataClass| defines the default for the class of data contained in the
molecular viscosity model.
For any data that is dimensional, |DimensionalUnits| may be used to
describe the system of dimensional units employed.
If present, these two entities take precedence of all corresponding
entities at higher levels of the hierarchy.
These precedence rules are further discussed in \autoref{s:precedence}.

The \fort{UserDefinedData\_t} data structure allows arbitrary
user-defined data to be stored in \fort{Descriptor\_t} and
\fort{DataArray\_t} children without the restrictions or implicit
meanings imposed on these node types at other node locations.

\subsection{Thermal Conductivity Model Structure Definition: \texttt{ThermalConductivityModel\_t}} 

|ThermalConductivityModel_t| describes the model for relating the
thermal-conductivity coefficient ($k$) to temperature.
The enumerated values for \texttt{ThermalConductivityModelType\_t} are a subset of the
\texttt{ModelType\_t} enumeration.
\begin{alltt}
  ThermalConductivityModelType\_t := Enumeration(
    ModelTypeNull,
    ModelTypeUserDefined,
    ConstantPrandtl,
    PowerLaw,
    SutherlandLaw ) ;

  ThermalConductivityModel\_t :=
    \{
    List( Descriptor\_t Descriptor1 ... DescriptorN ) ;                      (o)

    ThermalConductivityModelType\_t ThermalConductivityModelType ;           (r)
    
    List( DataArray\_t<DataType, 1, 1> DataArray1 ... DataArrayN ) ;         (o)

    DataClass\_t DataClass ;                                                 (o)
                
    DimensionalUnits\_t DimensionalUnits ;                                   (o)

    List( UserDefinedData\_t UserDefinedData1 ... UserDefinedDataN ) ;       (o)
    \} ;
\end{alltt}

\begin{notes}
\item
 Default names for the \fort{Descriptor\_t} and \fort{DataArray\_t}
 \fort{UserDefinedData\_t}
 lists are as shown; users may choose other legitimate names.
 Legitimate names must be unique within a given instance of
 \fort{ThermalConductivityModel\_t} and shall not include the names
 \fort{DataClass} or \fort{DimensionalUnits}.
\item
 \fort{ThermalConductivityModelType} is the only required element.
\end{notes}

The thermal-conductivity models parallel the molecular viscosity models.
|ConstantPrandtl| states that the Prandtl number ($Pr = \mu c_p/k$) is
constant and equal to some reference value.
|PowerLaw| relates $k$ to temperature via a power-law relation,
$$
 k = k_{\rm ref} \left( T \over T_{\rm ref} \right)^n.
$$
|SutherlandLaw| states the Sutherland's Law for thermal conductivity,
$$
 k = k_{\rm ref} \left( T \over T_{\rm ref} \right)^{3/2} 
  {T_{\rm ref} + T_s \over T + T_s},
$$
where $k_{\rm ref}$ is the reference thermal conductivity, $T_{\rm ref}$
is the reference temperature, and $T_s$ is the Sutherland's law constant.
For air\footnote{White, F. M., {\it Viscous Fluid Flow}, McGraw-Hill, 1974,
p.~32-33}, the Prandtl number is $Pr = 0.72$, the power-law exponent is
$n = 0.81$, Sutherland's law constant ($T_s$) is 194.4 K, the reference
temperature ($T_{\rm ref}$) is 273.15 K, and the reference thermal
conductivity ($k_{\rm ref}$) is $2.414 \!\times\! 10^{-2}$ kg-m/(s\tsup{3}-K).  
Data-name identifiers for thermal conductivity models are listed in
\autoref{t:id_thermal}.

\settowidth{\tmplengtha}{\fort{tivityModelType}}
\settowidth{\tmplengthb}{\fort{ThermalConductivityReference}}
\settowidth{\tmplengthc}{$\M \L/(\T^3 \TH)$}
\setlength{\Pwidth}{\linewidth-8\tabcolsep-\tmplengtha-\tmplengthb-\tmplengthc}
\begin{table}[htbp]
\centering
\caption[Data-Name Identifiers for Thermal Conductivity Models]{\textbf{Data-Name Identifiers for Thermal Conductivity Models}}
\label{t:id_thermal}
\begin{tabular}{>{\ttfamily}l >{\ttfamily}l >{\raggedright\arraybackslash}p{\Pwidth} c}
\\ \hline\hline \\*[-2ex]
ThermalConduc\textnormal{-} \\
tivityModelType & \spantwo{\bold{Data-Name Identifer}} &
   \spantwo{\bold{Description}} & \spantwo{\bold{Units}}
\\*[1ex] \hline\hline \\*[-2ex]
ConstantPrandtl & Prandtl &
   Prandtl number ($Pr$)                          & - \\
PowerLaw        & PowerLawExponent &
   Power-law exponent ($n$)                       & - \\
SutherlandLaw   & SutherlandLawConstant &
   Sutherland's Law constant ($T_s$)              & $\TH$ \\
\ital{All}      & TemperatureReference &
   Reference temperature ($T_{\rm ref}$)          & $\TH$ \\
\ital{All}      & ThermalConductivityReference &
   Reference thermal conductivity ($k_{\rm ref}$) & $\M \L/(\T^3 \TH)$
\\*[1ex] \hline\hline
\end{tabular}
\end{table}

If it is desired to specify any of these identifiers in a CGNS
database, they should be defined as \fort{DataArray}s under
\fort{ThermalConductivityModel\_t}.

|DataClass| defines the default for the class of data contained in the
thermal conductivity model.
For any data that is dimensional, |DimensionalUnits| may be used to
describe the system of dimensional units employed.
If present, these two entities take precedence of all corresponding
entities at higher levels of the hierarchy.
These precedence rules are further discussed in \autoref{s:precedence}.

The \fort{UserDefinedData\_t} data structure allows arbitrary
user-defined data to be stored in \fort{Descriptor\_t} and
\fort{DataArray\_t} children without the restrictions or implicit
meanings imposed on these node types at other node locations.

\subsection{Turbulence Structure Definitions}

This section presents structure definitions for describing the form of
closure used in the Reynolds-averaged (or Favre-averaged) Navier-Stokes
equations for determining the Reynolds stress terms.  Here ``turbulence
closure'' refers to eddy viscosity or other approximations for the
Reynolds stress terms, and ``turbulence model'' refers to the actual
algebraic or turbulence-transport equation models used.  To an extent
these are independent choices (e.g., using either an eddy viscosity
closure or an algebraic Reynolds-stress closure with a two-equation
model).

\subsubsection{Turbulence Closure Structure Definition: \texttt{TurbulenceClosure\_t}}

|TurbulenceClosure_t| describes the turbulence closure for the Reynolds
stress terms of the Navier-Stokes equations.
The enumerated values for \texttt{TurbulenceClosureType\_t} are a subset of the
\texttt{ModelType\_t} enumeration.

\begin{alltt}
  TurbulenceClosureType\_t := Enumeration(
    ModelTypeNull,
    ModelTypeUserDefined,
    EddyViscosity,
    ReynoldsStress,
    ReynoldsStressAlgebraic ) ;

  TurbulenceClosure\_t :=
    \{
    List( Descriptor\_t Descriptor1 ... DescriptorN ) ;                      (o)

    TurbulenceClosureType\_t TurbulenceClosureType ;                         (r)
    
    List( DataArray\_t<DataType, 1, 1> DataArray1 ... DataArrayN ) ;         (o)

    DataClass\_t DataClass ;                                                 (o)
                
    DimensionalUnits\_t DimensionalUnits ;                                   (o)

    List( UserDefinedData\_t UserDefinedData1 ... UserDefinedDataN ) ;       (o)
    \} ;
\end{alltt}

\begin{notes}
\item
 Default names for the \fort{Descriptor\_t}, \fort{DataArray\_t}, and
 \fort{UserDefinedData\_t}
 lists are as shown; users may choose other legitimate names.
 Legitimate names must be unique within a given instance of
 \fort{TurbulenceClosure\_t} and shall not include the names
 \fort{DataClass} or \fort{DimensionalUnits}.
\item
 \fort{TurbulenceClosureType} is the only required element.
\end{notes}

The different types of turbulent closure are as follows: |EddyViscosity| is
the Boussinesq eddy-viscosity closure, where the Reynolds stresses are
approximated as the product of an eddy viscosity ($\nu_t$) and the mean
strain tensor.
Using indicial notation, the relation is,
$$
  - \overline{u'_i u'_j} = \nu_t \left( \pdf{u_i}{x_j} + \pdf{u_j}{x_i} \right),
$$
where $- \overline{u'_i u'_j}$ are the Reynolds stresses; the notation is
further discussed in
\hyperref[s:dataname_flow]{Appendix~\ref*{s:dataname_flow}}.
\texttt{Rey\-nolds\-Stress} is no approximation of the Reynolds stresses.
\texttt{ReynoldsStressAlgebraic} is an algebraic approximation for the
Reynolds stresses based on some intermediate transport quantities.

Associated with the turbulent closure is a list of constants, where each
constant is described by a separate |DataArray_t| entity.
Constants associated with the eddy-viscosity closure are listed in
\autoref{t:id_closure}.

\begin{table}[htbp]
\centering
\caption[Data-Name Identifiers for Turbulence Closure]{\textbf{Data-Name Identifiers for Turbulence Closure}}
\label{t:id_closure}
\begin{tabular}{>{\ttfamily}l >{\quad}l >{\quad}c}
\\ \hline\hline \\*[-2ex]
\bold{Data-Name Identifier} & \bold{Description} & \bold{Units}
\\*[1ex] \hline\hline \\*[-2ex]
PrandtlTurbulent     & Turbulent Prandtl number ($\rho \nu_t c_p/k_t$) & -
\\*[1ex] \hline\hline
\end{tabular}
\end{table}

If it is desired to specify any of these identifiers in a CGNS
database, they should be defined as \fort{DataArray}s under
\fort{TurbulenceClosure\_t}.

|DataClass| defines the default for the class of data contained in the
turbulence closure.
For any data that is dimensional, |DimensionalUnits| may be used to
describe the system of dimensional units employed.
If present, these two entities take precedence of all corresponding
entities at higher levels of the hierarchy.
These precedence rules are further discussed in \autoref{s:precedence}.

The \fort{UserDefinedData\_t} data structure allows arbitrary
user-defined data to be stored in \fort{Descriptor\_t} and
\fort{DataArray\_t} children without the restrictions or implicit
meanings imposed on these node types at other node locations.

\subsubsection{Turbulence Model Structure Definition: \texttt{TurbulenceModel\_t}}

|TurbulenceModel\_t| describes the equation set used to model the
turbulence quantities.
The enumerated values for \texttt{TurbulenceModelType\_t} are a subset of the
\texttt{ModelType\_t} enumeration.
\begin{alltt}
  TurbulenceModelType\_t := Enumeration(
    ModelTypeNull,
    ModelTypeUserDefined,
    Algebraic\_BaldwinLomax,
    Algebraic\_CebeciSmith,
    HalfEquation\_JohnsonKing,
    OneEquation\_BaldwinBarth,
    OneEquation\_SpalartAllmaras,
    TwoEquation\_JonesLaunder,
    TwoEquation\_MenterSST,
    TwoEquation\_Wilcox ) ;

  TurbulenceModel\_t< int CellDimension > :=
    \{
    List( Descriptor\_t Descriptor1 ... DescriptorN ) ;                      (o)

    TurbulenceModelType\_t TurbulenceModelType ;                             (r)
    
    List( DataArray\_t<DataType, 1, 1> DataArray1 ... DataArrayN ) ;         (o)

    int[CellDimension*(CellDimension + 1)/2] DiffusionModel ;               (o)

    DataClass\_t DataClass ;                                                 (o)
                
    DimensionalUnits\_t DimensionalUnits ;                                   (o)

    List( UserDefinedData\_t UserDefinedData1 ... UserDefinedDataN ) ;       (o)
    \} ;
\end{alltt}

\begin{notes}
\item
 Default names for the \fort{Descriptor\_t} and \fort{DataArray\_t}
 \fort{UserDefinedData\_t}
 lists are as shown; users may choose other legitimate names.
 Legitimate names must be unique within a given instance of
 \fort{TurbulenceModel\_t} and shall not include the names
 \fort{DiffusionModel}, \fort{DataClass}, or \fort{DimensionalUnits}.
\item
 \fort{TurbulenceModelType} is the only required element.
\item
 The length of the |DiffusionModel| array is as follows: in 1-D it is
 |int[1]|; in 2-D it is |int[3]|; and in 3-D it is |int[6]|.
 For unstructured zones, \fort{DiffusionModel} is not supported, and
 should not be used.
\end{notes}

|TurbulenceModel_t| requires a single structure parameter,
\fort{CellDimension}.
It is used to define the length of the array |DiffusionModel|.
|DiffusionModel| describes the viscous diffusion terms included in the
turbulent transport model equations; the encoding of |DiffusionModel| is
described in \autoref{s:GoverningEquations}.

The \texttt{TurbulenceModelType} names currently listed correspond to
the following particular references.

\begin{Ventryi}{\texttt{OneEquation\_SpalartAllmaras}}
\item [\texttt{Algebraic\_BaldwinLomax}]
      Baldwin, B. S., and Lomax, H. (1978) ``Thin Layer Approximations
      and Algebraic Model for Separated Turbulent Flows,'' AIAA Paper
      78-257.
\item [\texttt{Algebraic\_CebeciSmith}]
      Cebeci, T., and Smith, A. M. O. (1974) \textit{Analysis of
      Turbulent Boundary Layers}, Academic Press, New York.
\item [\texttt{HalfEquation\_JohnsonKing}]
      Johnson, D., and King, L. (1985) ``A Mathematically Simple
      Turbulence Closure Model for Attached and Separated Turbulent
      Boundary Layers,'' AIAA Journal, Vol. 23, No. 11, pp. 1684--1692.
\item [\texttt{OneEquation\_BaldwinBarth}]
      Baldwin, B., and Barth, T. (1990) ``A One-Equation Turbulent
      Transport Model for High Reynolds Number Wall-Bounded Flows,'' NASA
      TM-102847.
\item [\texttt{OneEquation\_SpalartAllmaras}]
      Spalart, P. R., and Allmaras, S. R. (1994) ``A One-Equation
      Turbulence Model for Aerodynamic Flows,'' La Recherche
      Aerospatiale, Vol. 1, pp. 5--21.
\item [\texttt{TwoEquation\_JonesLaunder}]
      Jones, W., and Launder, B. (1972) ``The Prediction of
      Laminarization with a Two-Equation Model of Turbulence,''
      International Journal of Heat and Mass Transfer, Vol. 15,
      pp. 301--314.
\item [\texttt{TwoEquation\_MenterSST}]
      Menter, F. R. (1994) ``Two-Equation Eddy-Viscosity Turbulence
      Models for Engineering Application,'' AIAA Journal, Vol. 32,
      No. 8, pp. 1598--1605.
\item [\texttt{TwoEquation\_Wilcox}]
      Wilcox, D. C. (1994) \textit{Turbulence Modeling for CFD}, First
      Edition, DCW Industries, La Canada, California.
\end{Ventryi}

There is no formal mechanism for accounting for subsequent changes to
these models.
(For example, Wilcox later published 1998 and 2006 versions of his
$k$-$\omega$ model).
If it is a mere change to constant(s), then this could be described by
retaining the same \texttt{TurbulenceModelType} name and listing each
constant using a separate \texttt{DataArray\_t} entry.
If the change is more involved, then it is recommended to
employ \texttt{TurbulenceModelType = UserDefined} with a child
\texttt{Descriptor\_t} node giving details about it.

Associated with each choice of turbulence model may be a list of constants,
where each constant is described by a separate |DataArray_t| entity.
If used, the Data-Name Identifier of each constant should include the
turbulence model name, as well as the constant name (e.g.,
\fort{TurbulentSACb1}, \fort{TurbulentSSTCmu}, \fort{TurbulentKESigmak},
etc.).
However, no attempt is made here to formalize the names for all possible
turbulence models.

\fort{DataClass} defines the default for the class of data contained in the
turbulence model equation set.
For any data that is dimensional, \fort{DimensionalUnits} may be used to
describe the system of dimensional units employed.
If present, these two entities take precedence of all corresponding
entities at higher levels of the hierarchy.
These precedence rules are further discussed in \autoref{s:precedence}.

The \fort{UserDefinedData\_t} data structure allows arbitrary
user-defined data to be stored in \fort{Descriptor\_t} and
\fort{DataArray\_t} children without the restrictions or implicit
meanings imposed on these node types at other node locations.

\begin{example}{Spalart-Allmaras Turbulence Model}

Description for the eddy-viscosity closure and Spalart-Allmaras
turbulence model, including associated constants.
\begin{alltt}
  TurbulenceClosure\_t TurbulenceClosure =
    \{\{
    TurbulenceClosureType\_t TurbulenceClosureType = EddyViscosity ;

    DataArray\_t<real, 1, 1> PrandtlTurbulent = \{\{ 0.90 \}\} ;
    \}\} ;

  TurbulenceModel\_t TurbulenceModel = 
    \{\{
    TurbulenceModelType\_t TurbulenceModelType = OneEquation\_SpalartAllmaras ;

    DataArray\_t<real, 1, 1> TurbulentSACb1   = \{\{ 0.1355 \}\} ;
    DataArray\_t<real, 1, 1> TurbulentSACb2   = \{\{ 0.622 \}\} ;
    DataArray\_t<real, 1, 1> TurbulentSASigma = \{\{ 2/3 \}\} ;
    DataArray\_t<real, 1, 1> TurbulentSAKappa = \{\{ 0.41 \}\} ;
    DataArray\_t<real, 1, 1> TurbulentSACw1   = \{\{ 3.2391 \}\} ;
    DataArray\_t<real, 1, 1> TurbulentSACw2   = \{\{ 0.3 \}\} ;
    DataArray\_t<real, 1, 1> TurbulentSACw3   = \{\{ 2 \}\} ;
    DataArray\_t<real, 1, 1> TurbulentSACv1   = \{\{ 7.1 \}\} ;
    DataArray\_t<real, 1, 1> TurbulentSACt1   = \{\{ 1 \}\} ;
    DataArray\_t<real, 1, 1> TurbulentSACt2   = \{\{ 2 \}\} ;
    DataArray\_t<real, 1, 1> TurbulentSACt3   = \{\{ 1.2 \}\} ;
    DataArray\_t<real, 1, 1> TurbulentSACt4   = \{\{ 0.5 \}\} ;
    \}\} ;
\end{alltt}
Note that each |DataArray_t| entity is abbreviated.
\end{example}

\subsection{Thermal Relaxation Model Structure Definition: \
\texttt{ThermalRelaxationModelType\_t}}
\label{s:ThermalRelaxationModel}

\fort{ThermalRelaxationModel\_t} describes the equation set used to model
thermal relaxation quantities.
The enumerated values for \texttt{ThermalRelaxationModelType\_t} are a subset of the
\texttt{ModelType\_t} enumeration.
\begin{alltt}
  ThermalRelaxationModelType\_t := Enumeration(
    ModelTypeNull,
    ModelTypeUserDefined,
    Frozen,
    ThermalEquilib,
    ThermalNonequilib ) ;
\end{alltt}

\begin{alltt}
  ThermalRelaxationModel\_t :=
    \{
    List( Descriptor\_t Descriptor1 ... DescriptorN ) ;                      (o)

    ThermalRelaxationModelType\_t ThermalRelaxationModelType ;               (r)
    
    List( DataArray\_t<DataType, 1, 1> DataArray1 ... DataArrayN ) ;         (o)

    DataClass\_t DataClass ;                                                 (o)
                
    DimensionalUnits\_t DimensionalUnits ;                                   (o)

    List( UserDefinedData\_t UserDefinedData1 ... UserDefinedDataN ) ;       (o)
    \} ;
\end{alltt}

\begin{notes}
\item
 Default names for the \fort{Descriptor\_t}, \fort{DataArray\_t}, and
 \fort{UserDefinedData\_t}
 lists are as shown; users may choose other legitimate names.
 Legitimate names must be unique within a given instance of
 \fort{ThermalRelaxationModel\_t} and shall not include the names
 \fort{DataClass} or \fort{DimensionalUnits}.
\item
 \fort{ThermalRelaxationModelType} is the only required element.
\end{notes}

\fort{ThermalRelaxationModelType\_t} is an enumeration type describing
the type of thermal relaxation model.

\fort{DataArray\_t} data structures may be used to store data associated
with the thermal relaxation model.
\fort{DataClass} defines the default for the class of data being used.
For any data that is dimensional, \fort{DimensionalUnits} may be used to
describe the system of dimensional units employed.
If present, these two entities take precedence of all corresponding
entities at higher levels of the hierarchy.
These precedence rules are further discussed in \autoref{s:precedence}.

Additional information, if needed, may be stored using
\fort{Descriptor\_t} data structures.

The \fort{UserDefinedData\_t} data structure allows arbitrary
user-defined data to be stored in \fort{Descriptor\_t} and
\fort{DataArray\_t} children without the restrictions or implicit
meanings imposed on these node types at other node locations.

\subsection{Chemical Kinetics Structure Definition: \texttt{ChemicalKineticsModel\_t}}
\label{s:ChemicalKineticsModel}

\fort{ChemicalKineticsModel\_t} describes the equation set used to model
chemical kinetics quantities.
The enumerated values for \texttt{ChemicalKineticsModelType\_t} are a subset of the
\texttt{ModelType\_t} enumeration.
\begin{alltt}
  ChemicalKineticsModelType\_t := Enumeration(
    ModelTypeNull,
    ModelTypeUserDefined,
    Frozen,
    ChemicalEquilibCurveFit,
    ChemicalEquilibMinimization,
    ChemicalNonequilib ) ;
\end{alltt}

\begin{alltt}
  ChemicalKineticsModel\_t :=
    \{
    List( Descriptor\_t Descriptor1 ... DescriptorN ) ;                      (o)

    ChemicalKineticsModelType\_t ChemicalKineticsModelType ;                 (r)
    
    List( DataArray\_t<DataType, 1, 1> DataArray1 ... DataArrayN ) ;         (o)

    DataClass\_t DataClass ;                                                 (o)
                
    DimensionalUnits\_t DimensionalUnits ;                                   (o)

    List( UserDefinedData\_t UserDefinedData1 ... UserDefinedDataN ) ;       (o)
    \} ;
\end{alltt}

\begin{notes}
\item
 Default names for the \fort{Descriptor\_t}, \fort{DataArray\_t}, and
 \fort{UserDefinedData\_t}
 lists are as shown; users may choose other legitimate names.
 Legitimate names must be unique within a given instance of
 \fort{ChemicalKineticsModel\_t} and shall not include the names
 \fort{DataClass} or \fort{DimensionalUnits}.
\item
 \fort{ChemicalKineticsModelType} is the only required element.
\end{notes}

\fort{ChemicalKineticsModelType\_t} is an enumeration type describing
the type of chemical kinetics model.

\fort{DataArray\_t} data structures may be used to store data associated
with the chemical kinetics model.
Recommended data-name identifiers are listed in the following table.

\begin{table}[htbp]
\centering
\caption[Data-Name Identifiers for Chemical Kinetics Models]{\textbf{Data-Name Identifiers for Chemical Kinetics Models}}
\label{t:id_chemicalkinetics}
\begin{tabular}{>{\ttfamily}l >{\quad}l >{\quad}c}
\\ \hline\hline \\*[-2ex]
\bold{Data-Name Identifier} & \bold{Description} & \bold{Units}
\\*[1ex] \hline\hline \\*[-2ex]
MolecularWeight\textit{Symbol} & Molecular weight for species \textit{Symbol} &
   - \\
HeatOfFormation\textit{Symbol} & Heat of formation per unit mass for species \textit{Symbol} &
   $\L^2/\T^2$ \\
FuelAirRatio                   & Fuel/air mass ratio &
   - \\
ReferenceTemperatureHOF        & Reference temperature for the heat of formation &
   $\TH$
\\*[1ex] \hline\hline
\end{tabular}
\end{table}

The dimensional units are defined as follows: $\L$ is length, $\T$ is
time and $\TH$ is temperature.
These are further described in
\hyperref[s:dataname]{Appendix~\ref*{s:dataname}}.

The variable string \textit{Symbol} in the above data-name identifiers 
represents the chemical symbol for the desired species.
For example, \fort{H} represents hydrogen atoms, \fort{O} represents
oxygen atoms, \fort{H2} represents hydrogen molecules, \fort{H2O}
represents water molecules, and \fort{C3H5O3(NO2)3} represents
nitroglycerin molecules.
Any symbols off the periodic table of the elements can be used.
For charged molecules or particles, the word ``\fort{plus}'' or
``\fort{minus}'' should be spelled out in lower case.
For example, a CNO$+$ molecule should be denoted as \fort{CNOplus}.  

Other commonly used mixtures, that are usually not referred to by their
chemical symbols, are defined in the following table.
Individual users may define new names, but these may not be recognized
by other CGNS applications.
For consistency, additional names should be proposed as SIDS extensions.

\begin{table}[htbp]
\centering
\caption[Defined Names (Symbols) for Commonly Used Mixtures]{\textbf{Defined Names (Symbols) for Commonly Used Mixtures}}
\label{t:id_chemicalkineticssymbols}
\begin{tabular}{>{\ttfamily}l >{\quad}l}
\\ \hline\hline \\*[-2ex]
\bold{Symbol} & \bold{Mixture}
\\*[1ex] \hline\hline \\*[-2ex]
Air     & Generic air model \\
eminus  & Electrons\\
Fuel    & Generic fuel model \\
FuelAir & Generic fuel/air mixture \\
JP5     & JP5 jet fuel \\
JP7     & JP7 jet fuel \\
JP10    & JP10 jet fuel \\
Product & Generic fuel/air product of combustion \\
RP1     & RP1 rocket fuel
\\*[1ex] \hline\hline
\end{tabular}
\end{table}

\fort{DataClass} defines the default for the class of data being used.
For any data that is dimensional, \fort{DimensionalUnits} may be used to
describe the system of dimensional units employed.
If present, these two entities take precedence of all corresponding
entities at higher levels of the hierarchy,
following the standard precedence rules.

Additional information, if needed, may be stored using
\fort{Descriptor\_t} data structures.
For example, if CHEMKIN is used, it is recommended that a
\fort{Descriptor\_t} data structure be used to indicate this.
Reaction equations could also be specified using \fort{Descriptor\_t}
data structures.

The \fort{UserDefinedData\_t} data structure allows arbitrary
user-defined data to be stored in \fort{Descriptor\_t} and
\fort{DataArray\_t} children without the restrictions or implicit
meanings imposed on these node types at other node locations.

\subsection{Electromagnetics Structure Definitions}
\label{s:EM}

This section presents structure definitions for describing the electric
field, magnetic field, and conductivity models used for electromagnetic
flows.

\subsubsection{Electromagnetics Electric Field Model Structure Definition: \texttt{EMElectricFieldModel\_t}}

\fort{EMElectricFieldModel\_t} describes the electric field model used
for electromagnetic flows.
The enumerated values for \texttt{EMElectricFieldModelType\_t} are a subset of the
\texttt{ModelType\_t} enumeration.
\begin{alltt}
  EMElectricFieldModelType\_t := Enumeration(
    ModelTypeNull,
    ModelTypeUserDefined,
    Constant,
    Frozen,
    Interpolated,
    Voltage ) ;
\end{alltt}

\begin{alltt}
  EMElectricFieldModel\_t :=
    \{
    List( Descriptor\_t Descriptor1 ... DescriptorN ) ;                      (o)

    EMElectricFieldModelType\_t EMElectricFieldModelType ;                   (r)
    
    List( DataArray\_t<DataType, 1, 1> DataArray1 ... DataArrayN ) ;         (o)

    DataClass\_t DataClass ;                                                 (o)
                
    DimensionalUnits\_t DimensionalUnits ;                                   (o)

    List( UserDefinedData\_t UserDefinedData1 ... UserDefinedDataN ) ;       (o)
    \} ;
\end{alltt}

\begin{notes}
\item Default names for the \fort{Descriptor\_t}, \fort{DataArray\_t}, and
      \fort{UserDefinedData\_t}
      lists are as shown; users may choose other legitimate names.
      Legitimate names must be unique within a given instance of
      \fort{EMElectricFieldModel\_t} and shall not include the names
      \fort{DataClass} or \fort{DimensionalUnits}.
\item \fort{EMElectricFieldModelType} is the only required element.
\end{notes}

\fort{EMElectricFieldModelType\_t} is an enumeration type describing
the type of electric field model.

\fort{DataArray\_t} data structures may be used to store data associated
with the electric field model.
Recommended data-name identifiers are listed in \autoref{t:id_EM}.

\fort{DataClass} defines the default for the class of data contained in
the electric field model.
For any data that is dimensional, \fort{DimensionalUnits} may be used to
describe the system of dimensional units employed.
If present, these two entities take precedence of all corresponding
entities at higher levels of the hierarchy, following the standard
precedence rules.

The \fort{UserDefinedData\_t} data structure allows arbitrary
user-defined data to be stored in \fort{Descriptor\_t} and
\fort{DataArray\_t} children without the restrictions or implicit
meanings imposed on these node types at other node locations.

\subsubsection{Electromagnetics Magnetic Field Model Structure Definition: \texttt{EMMagneticFieldModel\_t}}

\fort{EMMagneticFieldModel\_t} describes the magnetic field model used
for electromagnetic flows.
The enumerated values for \texttt{EMMagneticFieldModelType\_t} are a subset of the
\texttt{ModelType\_t} enumeration.
\begin{alltt}
  EMMagneticFieldModelType\_t := Enumeration(
    ModelTypeNull,
    ModelTypeUserDefined,
    Constant,
    Frozen,
    Interpolated ) ;
\end{alltt}

\begin{alltt}
  EMMagneticFieldModel\_t :=
    \{
    List( Descriptor\_t Descriptor1 ... DescriptorN ) ;                      (o)

    EMMagneticFieldModelType\_t EMMagneticFieldModelType ;                   (r)
    
    List( DataArray\_t<DataType, 1, 1> DataArray1 ... DataArrayN ) ;         (o)

    DataClass\_t DataClass ;                                                 (o)
                
    DimensionalUnits\_t DimensionalUnits ;                                   (o)

    List( UserDefinedData\_t UserDefinedData1 ... UserDefinedDataN ) ;       (o)
    \} ;
\end{alltt}

\begin{notes}
\item Default names for the \fort{Descriptor\_t}, \fort{DataArray\_t}, and
      \fort{UserDefinedData\_t}
      lists are as shown; users may choose other legitimate names.
      Legitimate names must be unique within a given instance of
      \fort{EMMagneticFieldModel\_t} and shall not include the names
      \fort{DataClass} or \fort{DimensionalUnits}.
\item \fort{EMMagneticFieldModelType} is the only required element.
\end{notes}

\fort{EMMagneticFieldModelType\_t} is an enumeration type describing
the type of magnetic field model.

\fort{DataArray\_t} data structures may be used to store data associated
with the magnetic field model.
Recommended data-name identifiers are listed in \autoref{t:id_EM}.

\fort{DataClass} defines the default for the class of data contained in
the electric field model.
For any data that is dimensional, \fort{DimensionalUnits} may be used to
describe the system of dimensional units employed.
If present, these two entities take precedence of all corresponding
entities at higher levels of the hierarchy, following the standard
precedence rules.

The \fort{UserDefinedData\_t} data structure allows arbitrary
user-defined data to be stored in \fort{Descriptor\_t} and
\fort{DataArray\_t} children without the restrictions or implicit
meanings imposed on these node types at other node locations.

\subsubsection{Electromagnetics Conductivity Model Structure Definition: \texttt{EMConductivityModel\_t}}

\fort{EMConductivityModel\_t} describes the conductivity model used
for electromagnetic flows.
The enumerated values for \texttt{EMConductivityModelType\_t} are a subset of the
\texttt{ModelType\_t} enumeration.
\begin{alltt}
  EMConductivityModelType\_t := Enumeration(
    ModelTypeNull,
    ModelTypeUserDefined,
    Constant,
    Frozen,
    Equilibrium\_LinRessler,
    Chemistry\_LinRessler ) ;
\end{alltt}

\begin{alltt}
  EMConductivityModel\_t :=
    \{
    List( Descriptor\_t Descriptor1 ... DescriptorN ) ;                      (o)

    EMConductivityModelType\_t EMConductivityModelType ;                     (r)
    
    List( DataArray\_t<DataType, 1, 1> DataArray1 ... DataArrayN ) ;         (o)

    DataClass\_t DataClass ;                                                 (o)
                
    DimensionalUnits\_t DimensionalUnits ;                                   (o)

    List( UserDefinedData\_t UserDefinedData1 ... UserDefinedDataN ) ;       (o)
    \} ;
\end{alltt}

\begin{notes}
\item Default names for the \fort{Descriptor\_t}, \fort{DataArray\_t}, and
      \fort{UserDefinedData\_t}
      lists are as shown; users may choose other legitimate names.
      Legitimate names must be unique within a given instance of
      \fort{EMConductivityModel\_t} and shall not include the names
      \fort{DataClass} or \fort{DimensionalUnits}.
\item \fort{EMConductivityModelType} is the only required element.
\end{notes}

\fort{EMConductivityModelType\_t} is an enumeration type describing
the type of conductivity model.

\fort{DataArray\_t} data structures may be used to store data associated
with the conductivity model.
Recommended data-name identifiers are listed in \autoref{t:id_EM}.

\begin{table}[htbp]
\centering
\caption[Data-Name Identifiers for Electromagnetics Models]{\textbf{Data-Name Identifiers for Electromagnetics Models}}
\label{t:id_EM}
\begin{tabular}{>{\ttfamily}l >{\quad}l >{\quad}c}
\\ \hline\hline \\*[-2ex]
\bold{Data-Name Identifier} & \bold{Description} & \bold{Units}
\\*[1ex] \hline\hline \\*[-2ex]
Voltage              & Voltage                                 & $\M \L^2/\T \I$ \\
ElectricFieldX       & $x$-component of electric field vector  & $\M \L/\T \I$ \\
ElectricFieldY       & $y$-component of electric field vector  & $\M \L/\T \I$ \\
ElectricFieldZ       & $z$-component of electric field vector  & $\M \L/\T \I$ \\
MagneticFieldX       & $x$-component of magnetic field vector  & $\I/\L$ \\
MagneticFieldY       & $y$-component of magnetic field vector  & $\I/\L$ \\
MagneticFieldZ       & $z$-component of magnetic field vector  & $\I/\L$ \\
CurrentDensityX      & $x$-component of current density vector & $\I/\L^2$ \\
CurrentDensityY      & $y$-component of current density vector & $\I/\L^2$ \\
CurrentDensityZ      & $z$-component of current density vector & $\I/\L^2$ \\
ElectricConductivity & Electrical conductivity                 & $\M \L/\T^3 \I^2$ \\
LorentzForceX        & $x$-component of Lorentz force vector   & $\M \L/\T^2$ \\
LorentzForceY        & $y$-component of Lorentz force vector   & $\M \L/\T^2$ \\
LorentzForceZ        & $z$-component of Lorentz force vector   & $\M \L/\T^2 $ \\
JouleHeating         & Joule heating                           & $\M \L^2/\T^2$
\\*[1ex] \hline\hline
\end{tabular}
\end{table}

The dimensional units are defined as follows: $\M$ is mass, $\L$ is
length, $\T$ is time, and $\I$ is electric current.
These are further described in
\hyperref[s:dataname]{Appendix~\ref*{s:dataname}}.

\fort{DataClass} defines the default for the class of data contained in
the conductivity model.
For any data that is dimensional, \fort{DimensionalUnits} may be used to
describe the system of dimensional units employed.
If present, these two entities take precedence of all corresponding
entities at higher levels of the hierarchy, following the standard
precedence rules.

The \fort{UserDefinedData\_t} data structure allows arbitrary
user-defined data to be stored in \fort{Descriptor\_t} and
\fort{DataArray\_t} children without the restrictions or implicit
meanings imposed on these node types at other node locations.

\subsection{Flow Equation Examples}
\label{s:flowexample}

This section presents two examples of flow-equation sets.  The first is
an inviscid case and the second is a turbulent case with a one-equation
turbulence model.

\begin{example}{3-D Compressible Euler}
\label{e:eqn_Euler}

3-D compressible Euler with a perfect gas assumption for a monatomic gas:
\begin{alltt}
  FlowEquationSet\_t<3> EulerEquations = 
    \{\{
    int EquationDimension = 3 ;
    
    GoverningEquations\_t<3> GoverningEquations =
      \{\{
      GoverningEquationsType\_t GoverningEquationsType = Euler ;
      \}\} ;
      
    GasModel\_t GasModel =
      \{\{
      GasModelType\_t GasModelType = CaloricallyPerfect ;
      
      DataArray\_t<real, 1, 1> SpecificHeatRatio =
        \{\{
        Data(real, 1, 1) = 1.667 ;

        DataClass\_t DataClass = NondimensionalParameter ;
        \}\} ;
      \}\} ;
    \}\} ;
\end{alltt}
\end{example}

\begin{example}{3-D Compressible Navier-Stokes}
\label{e:eqn_NS}

3-D compressible Navier-Stokes for a structured grid, with the S-A
turbulence model, a perfect gas assumption, Sutherland's law for the
molecular viscosity, a constant Prandtl-number assumption, and
inclusion of the full Navier-Stokes diffusion terms; all models assume
air:
\begin{alltt}
  FlowEquationSet\_t<3> NSEquations = 
    \{\{
    int EquationDimension = 3 ;
    
    GoverningEquations\_t<3> GoverningEquations =
      \{\{
      GoverningEquationsType\_t GoverningEquationsType = NSTurbulent ;
      
      int[6] DiffusionModel = [1,1,1,1,1,1] ;
      \}\} ;
      
    GasModel\_t GasModel =
      \{\{
      GasModelType\_t GasModelType = CaloricallyPerfect ;
      
      DataArray\_t<real, 1, 1> SpecificHeatRatio = \{\{ 1.4 \}\} ;
      \}\} ;

    ViscosityModel\_t ViscosityModel =
      \{\{
      ViscosityModelType\_t ViscosityModelType = SutherlandLaw ;
      
      DataArray\_t<real, 1, 1> SutherlandLawConstant = 
        \{\{ 
        Data(real, 1, 1) = 110.6 \}\} ;
      
        DataClass\_t DataClass = Dimensional ;
        DimensionalUnits\_t DimensionalUnits = \{\{ TemperatureUnits = Kelvin \}\} ;
        \}\} ;
      \}\} ;

    ThermalConductivityModel\_t ThermalConductivityModel =
      \{\{
      ThermalConductivityModelType\_t ThermalConductivityModelType =
         ConstantPrandtl ;
      
      DataArray\_t<real, 1, 1> Prandtl = \{\{ 0.72 \}\} ;
      \}\} ;

    TurbulenceClosure\_t<3> TurbulenceClosure =
      \{\{
      TurbulenceClosureType\_t TurbulenceClosureType = EddyViscosity ;
      
      DataArray\_t<real, 1, 1> PrandtlTurbulent = \{\{ 0.90 \}\} ;
      \}\} ;
      
    TurbulenceModel\_t<3> TurbulenceModel =
      \{\{
      TurbulenceModelType\_t TurbulenceModelType = OneEquation\_SpalartAllmaras ;
      
      int[6] DiffusionModel = [1,1,1,1,1,1] ;
      \}\} ;      
    \}\} ;
\end{alltt}
Note that all \fort{DataArray\_t} entities are abbreviated except
\fort{SutherlandLawConstant}.
\end{example}
