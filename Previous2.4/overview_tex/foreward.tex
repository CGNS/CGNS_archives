\pdfbookmark[1]{Foreward}{foreward}
\addtocontents{toc}{\protect\contentsline {section}{Foreward}{\thepage}{foreward}}
\hypertarget{foreward}{}
\section*{Foreward}
\thispagestyle{plain}

This document describes the structure of CGNS (CFD General Notation
System), its software, and its documentation.
The Overview, which was written primarily for new and prospective users
of CGNS, (1) introduces terminology, (2) identifies the elements of the
system and their relationships, and (3) describes the various documents
that elaborate the details.
Reading the material on the purpose (\autoref{s:purpose}) and general
description (\autoref{s:description}) of CGNS should help users
determine whether CGNS will meet their needs.
Those wanting a bit more detail should read \autoref{s:elements}, which
describes the various elements of CGNS.
For those interested only in understanding the scope and capabilities of
CGNS, or for the end user unconcerned with the internal workings of the
system, the Overview may prove sufficient documentation by itself.

The Overview also includes certain information that is current as of the
document date but which may change with time.
All information on CGNS compatible ``applications'' software (i.e.,
external programs such as grid generators, flow codes, or postprocessors
in \autoref{s:applications}) is of this type.
Also subject to change is the information on the acquisition of the
software and documentation in \autoref{s:acquiring}, and the current
status of CGNS in \autoref{s:history}.
