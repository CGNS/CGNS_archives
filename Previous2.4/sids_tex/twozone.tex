\section{Structured Two-Zone Flat Plate Example}
\label{s:twozone}
\thispagestyle{plain}

This section describes a complete database for a sample test case.
The test case is compressible turbulent flow past a flat plat at
zero incidence.  The domain is divided into two zones as shown in
\autoref{f:flatplate}.  The interface between the two zones is 1-to-1.

\begin{figure}[h]
\begin{center}
\setlength{\unitlength}{1.0in}
\begin{picture}(4.0,1.25)
\put(0,0){ \framebox(2.0,1.0){\shortstack{
           Zone 1\rule[-0.5ex]{0pt}{1ex} \\
           $25 \times 65 \times 3$\rule[-0.5ex]{0pt}{1ex}}} }
\put(2,0){ \framebox(2.0,1.0){\shortstack{
           Zone 2\rule[-0.5ex]{0pt}{1ex} \\
           $49 \times 65 \times 3$\rule[-0.5ex]{0pt}{1ex}}} }

\put(0,0){ \vector(1,0){0.3} }
\put(0,0){ \vector(0,1){0.3} }
\put( 0.30,-0.05){ \makebox(0,0)[t]{$x,\, i$} }
\put(-0.05, 0.30){ \makebox(0,0)[r]{$y,\, j$} }

\put(1,-0.1){ \makebox(0,0)[t]{Symmetry} }
\put(3,-0.1){ \makebox(0,0)[t]{Solid wall} }
\put(1, 1.1){ \makebox(0,0)[b]{Outflow} }
\put(3, 1.1){ \makebox(0,0)[b]{Outflow} }

\put(-0.1,0.5){ \makebox(0,0)[r]{Inflow} }
\put( 4.1,0.5){ \makebox(0,0)[l]{Outflow} }

\thicklines
\put(2, 0.00){ \line(1,0){2} }
\end{picture}
\end{center}
\caption{Two-Zone Flat Plate Test Case}
\label{f:flatplate}
\end{figure}

The database description includes the following:
\begin{itemize}
\item range of indices within each zone
\item grid coordinates of vertices
\item flowfield solution at cell centers including a row of ghost-cells
      along each boundary; the flowfield includes the conservation
      variables and a turbulent transport variable
\item multizone interface connectivity information
\item boundary condition information
\item reference state
\item description of the compressible Navier-Stokes equations including 
      one-equation turbulence model
\end{itemize}
Each of these items is described in separate sections to make the
information more readable.  The same database is presented in each
section, but only that information needed for the particular focus
is included.  The overall layout of the database is presented in
\autoref{s:flatplate_layout}.

All data for this test case is nondimensional and is normalized
consistently by the following (dimensional) quantities: plate length
$L$, freestream static density $\rho_\infty$, freestream static speed
of sound $c_\infty$, and freestream static temperature $T_\infty$.  The
fact that the database is completely nondimensional is reflected in the
value of the globally set data class.

\subsection{Overall Layout}
\label{s:flatplate_layout}

This section describes the overall layout of the database.  Included are
the cell dimension and physical dimension of the grid, the globally set
data class, the global reference state and flow-equations description,
and data pertaining to each zone.  Each zone contains the grid size,
grid coordinates, flow solution, multizone interfaces and boundary
conditions.  All entities given by |{{*}}| are expanded in subsequent
sections.  Note that because this example contains structured zones,
\fort{IndexDimension} = \fort{CellDimension} = 3 in each zone.
\begin{alltt}
CGNSBase\_t TwoZoneCase =
  \{\{
  int CellDimension = 3 ;
  int PhysicalDimension = 3 ;

  DataClass\_t DataClass = NormalizedByUnknownDimensional ;

  ReferenceState\_t ReferenceState = \{\{*\}\} ;
  
  FlowEquationSet\_t<3> FlowEquationSet = \{\{*\}\} ;

  !  CellDimension = 3, PhysicalDimension = 3
  Zone\_t<3,3> Zone1 =
    \{\{
    int VertexSize = [25,65,3] ;
    int CellSize   = [24,64,2] ;
    int VertexSizeBoundary = [0,0,0];

    ZoneType\_t ZoneType = Structured;
    
    !  IndexDimension = 3
    GridCoordinates\_t<3,VertexSize> GridCoordinates = \{\{*\}\} ;
    
    FlowSolution\_t<3,VertexSize,CellSize> FlowSolution = \{\{*\}\} ;
    
    ZoneGridConnectivity\_t<3,3> ZoneGridConnectivity = \{\{*\}\} ;
  
    ZoneBC\_t<3,3> ZoneBC = \{\{*\}\} ;
    \}\} ;        ! end Zone1
  
  !  CellDimension = 3, PhysicalDimension = 3
  Zone\_t<3,3> Zone2 =
    \{\{
    int VertexSize = [49,65,3] ;
    int CellSize   = [48,64,2] ;
    int VertexSizeBoundary = [0,0,0];

    ZoneType\_t ZoneType = Structured;
    
    !  IndexDimension = 3
    GridCoordinates\_t<3,VertexSize> GridCoordinates = \{\{*\}\} ;
    
    FlowSolution\_t<3,VertexSize,CellSize> FlowSolution = \{\{*\}\} ;
    
    ZoneGridConnectivity\_t<3,3> ZoneGridConnectivity = \{\{*\}\} ;
  
    ZoneBC\_t<3,3> ZoneBC = \{\{*\}\} ;
    \}\} ;        ! end Zone2
  \}\} ;          ! end TwoZoneCase
\end{alltt}

\subsection{Grid Coordinates}

This section describes the grid-coordinate entities for each
zone.  Since the coordinates are all nondimensional, the individual
|DataArray_t| entities do not include a data-class qualifier; instead,
this information is derived from the globally set data class.  The
grid-coordinate entities for zone 2 are abbreviated.
\begin{alltt}
CGNSBase\_t TwoZoneCase =
  \{\{
  int CellDimension = 3 ;
  int PhysicalDimension = 3 ;

  DataClass\_t DataClass = NormalizedByUnknownDimensional ;

  !  CellDimension = 3, PhysicalDimension = 3
  Zone\_t<3,3> Zone1 =
    \{\{
    int VertexSize = [25,65,3] ;
    int CellSize   = [24,64,2] ;
    int VertexSizeBoundary = [0,0,0];

    ZoneType\_t ZoneType = Structured;

    !  IndexDimension = 3
    !  VertexSize = [25,65,3]
    GridCoordinates\_t<3, [25,65,3]> GridCoordinates =
      \{\{
      DataArray\_t<real, 3, [25,65,3]> CoordinateX =
        \{\{
        Data(real, 3, [25,65,3]) = (((x(i,j,k), i=1,25), j=1,65), k=1,3) ;
        \}\} ;

      DataArray\_t<real, 3, [25,65,3]> CoordinateY =
        \{\{
        Data(real, 3, [25,65,3]) = (((y(i,j,k), i=1,25), j=1,65), k=1,3) ;
        \}\} ;

      DataArray\_t<real, 3, [25,65,3]> CoordinateZ =
        \{\{
        Data(real, 3, [25,65,3]) = (((z(i,j,k), i=1,25), j=1,65), k=1,3) ;
        \}\} ;
      \}\} ;      ! end Zone1/GridCoordinates
    \}\} ;        ! end Zone1

  !  CellDimension = 3, PhysicalDimension = 3
  Zone\_t<3,3> Zone2 =
    \{\{
    int VertexSize = [49,65,3] ;
    int CellSize   = [48,64,2] ;
    int VertexSizeBoundary = [0,0,0];

    ZoneType\_t ZoneType = Structured;

    !  IndexDimension = 3
    !  VertexSize = [49,65,3]
    GridCoordinates\_t<3, [49,65,3]> GridCoordinates =
      \{\{
      DataArray\_t<real, 3, [49,65,3]> CoordinateX = \{\{*\}\} ;
      DataArray\_t<real, 3, [49,65,3]> CoordinateY = \{\{*\}\} ;
      DataArray\_t<real, 3, [49,65,3]> CoordinateZ = \{\{*\}\} ;
      \}\} ;      ! end Zone2/GridCoordinates
    \}\} ;        ! end Zone2
  \}\} ;          ! end TwoZoneCase
\end{alltt}

\subsection{Flowfield Solution}

This section provides a description of the flowfield solution including
the conservation variables and the Spalart-Allmaras turbulent-transport
quantity ($\tilde{\nu}$).  The flowfield solution is given at cell
centers with a single row of ghost-cell values along each boundary.

As with the case for grid coordinates, the flow solution is
nondimensional, and this fact is derived from the globally set data
class.  The normalizations for each flow variable are,
$$
 \rho'_{ijk} = {\rho_{ijk} \over \rho_\infty}, \qquad
 (\rho u)'_{ijk} = {(\rho u)_{ijk} \over \rho_\infty c_\infty}, \qquad
 (\rho e_0)'_{ijk} = {(\rho e_0)_{ijk} \over \rho_\infty c_\infty^2}, \qquad
 \tilde{\nu}'_{ijk} = {\tilde{\nu}_{ijk} \over c_\infty L},
$$
where primed quantities are nondimensional and all others are dimensional.

Only the |Density| entity for zone 1 is fully described in the following.
The momentum, energy and turbulence solution are abbreviated.
The entire flow-solution data for zone 2 is also abbreviated.
\begin{alltt}
CGNSBase\_t TwoZoneCase =
  \{\{
  int CellDimension = 3 ;
  int PhysicalDimension = 3 ;

  DataClass\_t DataClass = NormalizedByUnknownDimensional ;

  !  CellDimension = 3, PhysicalDimension = 3
  Zone\_t<3,3> Zone1 =
    \{\{
    int VertexSize = [25,65,3] ;
    int CellSize   = [24,64,2] ;
    int VertexSizeBoundary = [0,0,0];

    ZoneType\_t ZoneType = Structured;

    !  IndexDimension = 3
    !  VertexSize = [25,65,3]
    !  CellSize   = [24,64,2]
    FlowSolution\_t<3, [25,65,3], [24,64,2]> FlowSolution =
      \{\{
      GridLocation\_t GridLocation = CellCenter ;

      !  IndexDimension = 3
      Rind\_t<3> Rind = 
        \{\{
        int[6] RindPlanes = [1,1,1,1,1,1] ;
        \}\} ;

      !  IndexDimension = 3
      !  DataSize = CellSize + [2,2,2] = [26,66,4]
      DataArray\_t<real, 3, [26,66,4]> Density =
        \{\{
        Data(real, 3, [26,66,4]) = (((rho(i,j,k), i=0,25), j=0,65), k=0,3) ;
        \}\} ;

      DataArray\_t<real, 3, [26,66,4]> MomentumX = \{\{*\}\} ;
      DataArray\_t<real, 3, [26,66,4]> MomentumY = \{\{*\}\} ;
      DataArray\_t<real, 3, [26,66,4]> MomentumZ = \{\{*\}\} ;
      DataArray\_t<real, 3, [26,66,4]> EnergyStagnationDensity = \{\{*\}\} ;
      DataArray\_t<real, 3, [26,66,4]> TurbulentSANutilde = \{\{*\}\} ;
      \}\} ;      ! end Zone1/FlowSolution
    \}\} ;        ! end Zone1

  Zone\_t<3,3> Zone2 = \{\{*\}\} ;
  \}\} ;          ! end TwoZoneCase
\end{alltt}

\subsection{Interface Connectivity}

This section describes the interface connectivity between zones 1 and
2; it also includes the $k$-plane periodicity for each zone (which is
essentially an interface connectivity of a zone onto itself).  Each
interface entity is repeated with the receiver and donor-zone roles
reversed; this includes the periodic $k$-plane interfaces.  Since each
interface is a complete zone face, the |GridConnectivity1to1_t| entities
are named after the face.

Because of the orientation of the zones, the index transformation
matrices (|Transform|) for all interfaces are diagonal.  This means that
each matrix is its own inverse, and the value of |Transform| is the same
for every pair of interface entities.
\begin{alltt}
CGNSBase\_t TwoZoneCase =
  \{\{
  int CellDimension = 3 ;
  int PhysicalDimension = 3 ;


  !  -----  ZONE 1 Interfaces  ------

  !  CellDimension = 3, PhysicalDimension = 3
  Zone\_t<3,3> Zone1 =
    \{\{
    int VertexSize = [25,65,3] ;
    int CellSize   = [24,64,2] ;
    int VertexSizeBoundary = [0,0,0];

    ZoneType\_t ZoneType = Structured;

    !  IndexDimension = 3, CellDimension = 3
    ZoneGridConnectivity\_t<3,3> ZoneGridConnectivity =
      \{\{
 
      !  IndexDimension = 3
      GridConnectivity1to1\_t<3> IMax =                ! ZONE 1 IMax
        \{\{
        int[3] Transform = [1,2,3] ;
        IndexRange\_t<3> PointRange =
          \{\{
          int[3] Begin = [25,1 ,1] ;
          int[3] End   = [25,65,3] ;
          \}\} ;
        IndexRange\_t<3> PointRangeDonor =
          \{\{
          int[3] Begin = [1,1 ,1] ;
          int[3] End   = [1,65,3] ;
          \}\} ;
        Identifier(Zone\_t) ZoneDonorName = Zone2 ;
        \}\} ;

      GridConnectivity1to1\_t<3> KMin =                ! ZONE 1 KMin 
        \{\{
        int[3] Transform = [1,2,-3] ;
        IndexRange\_t<3> PointRange =
          \{\{
          int[3] Begin = [1 ,1 ,1] ;
          int[3] End   = [25,65,1] ;
          \}\} ;
        IndexRange\_t<3> PointRangeDonor =
          \{\{
          int[3] Begin = [1 ,1 ,3] ;
          int[3] End   = [25,65,3] ;
          \}\} ;
        Identifier(Zone\_t) ZoneDonorName = Zone1 ;
        \}\} ;

      GridConnectivity1to1\_t<3> KMax =                ! ZONE 1 KMax 
        \{\{
        int[3] Transform = [1,2,-3] ;
        IndexRange\_t<3> PointRange =
          \{\{
          int[3] Begin = [1 ,1 ,3] ;
          int[3] End   = [25,65,3] ;
          \}\} ;
        IndexRange\_t<3> PointRangeDonor =
          \{\{
          int[3] Begin = [1 ,1 ,1] ;
          int[3] End   = [25,65,1] ;
          \}\} ;
        Identifier(Zone\_t) ZoneDonorName = Zone1 ;
        \}\} ;
      \}\} ;      ! end Zone1/ZoneGridConnectivity
    \}\} ;        ! end Zone1


   !  -----  ZONE 2 Interfaces  ------

  !  CellDimension = 3, PhysicalDimension = 3
   Zone\_t<3,3> Zone2 =
    \{\{
    int VertexSize = [49,65,3] ;
    int CellSize   = [48,64,2] ;
    int VertexSizeBoundary = [0,0,0];

    ZoneType\_t ZoneType = Structured;

    !  IndexDimension = 3, CellDimension = 3
    ZoneGridConnectivity\_t<3,3> ZoneGridConnectivity =
      \{\{
 
     !  IndexDimension = 3
      GridConnectivity1to1\_t<3> IMin =                ! ZONE 2 IMin
        \{\{
        int[3] Transform = [1,2,3] ;
        IndexRange\_t<3> PointRange =
          \{\{
          int[3] Begin = [1,1 ,1] ;
          int[3] End   = [1,65,3] ;
          \}\} ;
        IndexRange\_t<3> PointRangeDonor =
          \{\{
          int[3] Begin = [25,1 ,1] ;
          int[3] End   = [25,65,3] ;
          \}\} ;
        Identifier(Zone\_t) ZoneDonorName = Zone1 ;
        \}\} ;

      GridConnectivity1to1\_t<3> KMin =                ! ZONE 2 KMin 
        \{\{
        int[3] Transform = [1,2,-3] ;
        IndexRange\_t<3> PointRange =
          \{\{
          int[3] Begin = [1 ,1 ,1] ;
          int[3] End   = [49,65,1] ;
          \}\} ;
        IndexRange\_t<3> PointRangeDonor =
          \{\{
          int[3] Begin = [1 ,1 ,3] ;
          int[3] End   = [49,65,3] ;
          \}\} ;
        Identifier(Zone\_t) ZoneDonorName = Zone2 ;
        \}\} ;

      GridConnectivity1to1\_t<3> KMax =                ! ZONE 2 KMax 
        \{\{
        int[3] Transform = [1,2,-3] ;
        IndexRange\_t<3> PointRange =
          \{\{
          int[3] Begin = [1 ,1 ,3] ;
          int[3] End   = [49,65,3] ;
          \}\} ;
        IndexRange\_t<3> PointRangeDonor =
          \{\{
          int[3] Begin = [1 ,1 ,1] ;
          int[3] End   = [49,65,1] ;
          \}\} ;
        Identifier(Zone\_t) ZoneDonorName = Zone2 ;
        \}\} ;
      \}\} ;      ! end Zone2/ZoneGridConnectivity
    \}\} ;        ! end Zone2
  \}\} ;          ! end TwoZoneCase
\end{alltt}

\subsection{Boundary Conditions}

Boundary conditions for the flat plate case are described in this
section.  The minimal information necessary is included in each boundary
condition; this includes the boundary-condition type and BC-patch
specification.  The lone exception is the viscous wall, which is
isothermal and has an imposed temperature profile (given by the array
|temperatureprofile()|).  For all other boundary conditions a flow
solver is free to impose appropriate BC-data since none is provided in
the following.  The imposed BC-data for all cases should be evaluated at
the globally set reference state, since no other reference states have
been specified.

No boundary condition descriptions are provided for the multizone
interface or for the $k$-plane periodicity in each zone.  All
relevant information is provided for these interfaces in the
\fort{GridConnectivity1to1\_t} entities of the previous section.

The practice of naming |BC_t| entities after the face is followed.
\begin{alltt}
CGNSBase\_t TwoZoneCase =
  \{\{
  int CellDimension = 3 ;
  int PhysicalDimension = 3 ;

  DataClass\_t DataClass = NormalizedByUnknownDimensional ;

  !  -----  ZONE 1 BC's  ------

  !  CellDimension = 3, PhysicalDimension = 3
  Zone\_t<3,3> Zone1 =
    \{\{
    int VertexSize = [25,65,3] ;
    int CellSize   = [24,64,2] ;
    int VertexSizeBoundary = [0,0,0];

    ZoneType\_t ZoneType = Structured;

    !  IndexDimension = 3, PhysicalDimension = 3
    ZoneBC\_t<3,3> ZoneBC =
      \{\{
 
      !  IndexDimension = 3, PhysicalDimension = 3
      BC\_t<3,3> IMin =                                  !  ZONE 1 IMin
        \{\{
        BCType\_t BCType = BCInflowSubsonic ;
        IndexRange\_t<3> PointRange = 
          \{\{
          int[3] Begin = [1,1 ,1] ;
          int[3] End   = [1,65,3] ;
          \}\} ;
        \}\} ;

      BC\_t<3,3> JMin =                                  !  ZONE 1 JMin
        \{\{
        BCType\_t BCType = BCSymmetryPlane ;
        IndexRange\_t<3> PointRange = 
          \{\{
          int[3] Begin = [1 ,1,1] ;
          int[3] End   = [25,1,3] ;
          \}\} ;
        \}\} ;

      BC\_t<3,3> JMax =                                  !  ZONE 1 JMax
        \{\{
        BCType\_t BCType = BCOutFlowSubsonic ;
        IndexRange\_t<3> PointRange = 
          \{\{
          int[3] Begin = [1 ,65,1] ;
          int[3] End   = [25,65,3] ;
          \}\} ;
        \}\} ;
      \}\} ;      ! end Zone1/ZoneBC
    \}\} ;        ! end Zone1
    
 
  !  -----  ZONE 2 BC's  ------

  !  CellDimension = 3, PhysicalDimension = 3
  Zone\_t<3,3> Zone2 =
    \{\{
    int VertexSize = [49,65,3] ;
    int CellSize   = [48,64,2] ;
    int VertexSizeBoundary = [0,0,0];

    ZoneType\_t ZoneType = Structured;

    !  IndexDimension = 3, PhysicalDimension = 3
    ZoneBC\_t<3,3> ZoneBC =
      \{\{
 
      !  IndexDimension = 3, PhysicalDimension = 3
      BC\_t<3,3> IMax =                                  !  ZONE 2 IMax
        \{\{
        BCType\_t BCType = BCOutflowSubsonic ;
        IndexRange\_t<3> PointRange = 
          \{\{
          int[3] Begin = [49,1 ,1] ;
          int[3] End   = [49,65,3] ;
          \}\} ;
        \}\} ;    ! end Zone2/ZoneBC/IMax

      BC\_t<3,3> JMin =                                  !  ZONE 2 JMin
        \{\{
        BCType\_t BCType = BCWallViscous ;
        IndexRange\_t<3> PointRange = 
          \{\{
          int[3] Begin = [1 ,1,1] ;
          int[3] End   = [49,1,3] ;
          \}\} ;
        
        !  ListLength = 49*3 = 147
        BCDataSet<147> BCDataSet =
          \{\{
          BCTypeSimple\_t BCTypeSimple = BCWallViscousIsothermal ;

          !  Data array length = ListLength = 147
          BCData\_t<147> DirichletData = 
            \{\{
            DataArray\_t<real, 1, 147> Temperature =
              \{\{
              Data(real, 1, 147) = (temperatureprofile(n), n=1,147) ;
              \}\} ;
            \}\} ;
          \}\} ;
        \}\} ;    ! end Zone2/ZoneBC/JMin

      BC\_t<3,3> JMax =                                  !  ZONE 2 JMax
        \{\{
        BCType\_t BCType = BCOutFlowSubsonic ;
        IndexRange\_t<3> PointRange = 
          \{\{
          int[3] Begin = [1 ,65,1] ;
          int[3] End   = [49,65,3] ;
          \}\} ;
        \}\} ;    ! end Zone2/ZoneBC/JMax

      \}\} ;      ! end Zone2/ZoneBC
    \}\} ;        ! end Zone2
  \}\} ;          ! end TwoZoneCase
\end{alltt}

\subsection{Global Reference State}

This section provides a description of the freestream reference
state.  As previously stated, all data is nondimensional including
all reference state quantities.  The dimensional plate length $L$ and
freestream scales $\rho_\infty$, $c_\infty$ and $T_\infty$ are used for
normalization.

The freestream Mach number is 0.5 and the Reynolds number is 10\tsup{6}
based on freestream velocity and kinematic viscosity and the plate
length.
These are the only nondimensional parameters included in the reference
state.
The defining scales for each parameter are also included; these defining
scales are nondimensional.

Using consistent normalization, the following nondimensional freestream
quantities are defined:
$$
\begin{array}{l@{\qquad}l@{\qquad}l}
 \rho'_\infty = 1 & 
 (\rho_0)'_\infty = \rho'_\infty \Gamma^{1/(\gamma - 1)} &
 L' = 1 \\
 c'_\infty = 1 & 
 (c_0)'_\infty = c'_\infty \Gamma^{1/2} &
 u'_\infty = M_\infty = 0.5 \\
 T'_\infty = 1 & 
 (T_0)'_\infty = T'_\infty \Gamma &
 v'_\infty = 0 \\
 p'_\infty = 1/\gamma & 
 (p_0)'_\infty = p'_\infty \Gamma^{\gamma/(\gamma - 1)} &
 w'_\infty = 0 \\
 e'_\infty = 1/\gamma(\gamma - 1) & 
 (e_0)'_\infty = e'_\infty \Gamma &
 \nu'_\infty = u'_\infty L'/Re = 5\!\times\!10^{-7} \\
 h'_\infty = 1/(\gamma - 1) & 
 (h_0)'_\infty = h'_\infty \Gamma &
 \tilde{s}'_\infty = p'_\infty/(\rho'_\infty)^\gamma = 1/\gamma 
\end{array}
$$
where $\Gamma \equiv 1 + {\gamma - 1 \over 2} M_\infty^2$ based on $M_\infty = 0.5$
and $\gamma = 1.4$.

Except for the nondimensional parameters Mach number and Reynolds number,
all |DataArray_t| entities are abbreviated.
\begin{alltt}
CGNSBase\_t TwoZoneCase =
  \{\{

  DataClass\_t DataClass = NormalizedByUnknownDimensional ;

  ReferenceState\_t ReferenceState = 
    \{\{
    Descriptor\_t ReferenceStateDescription =
      \{\{
      Data(char, 1, 10) = "Freestream" ;
      \}\} ;

    DataArray\_t<real, 1, 1> Mach =
      \{\{
      Data(real, 1, 1) = 0.5 ;
      DataClass\_t DataClass = NondimensionalParameter ;
      \}\} ;
    DataArray\_t<real, 1, 1> Mach\_Velocity      = \{\{ 0.5 \}\} ;
    DataArray\_t<real, 1, 1> Mach\_VelocitySound = \{\{ 1 \}\} ;

    DataArray\_t<real, 1, 1> Reynolds =
      \{\{
      Data(real, 1, 1) = 1.0e+06 ;
      DataClass\_t DataClass = NondimensionalParameter ;
      \}\} ;
    DataArray\_t<real, 1, 1> Reynolds\_Velocity           = \{\{ 0.5 \}\} ;
    DataArray\_t<real, 1, 1> Reynolds\_Length             = \{\{ 1. \}\} ;
    DataArray\_t<real, 1, 1> Reynolds\_ViscosityKinematic = \{\{ 5.0E-07 \}\} ;

    DataArray\_t<real, 1, 1> Density                 = \{\{ 1. \}\} ;
    DataArray\_t<real, 1, 1> LengthReference         = \{\{ 1. \}\} ;
    DataArray\_t<real, 1, 1> VelocitySound           = \{\{ 1. \}\} ;
    DataArray\_t<real, 1, 1> VelocityX               = \{\{ 0.5 \}\} ;
    DataArray\_t<real, 1, 1> VelocityY               = \{\{ 0 \}\};
    DataArray\_t<real, 1, 1> VelocityZ               = \{\{ 0 \}\} ;
    DataArray\_t<real, 1, 1> Pressure                = \{\{ 0.714286 \}\} ;
    DataArray\_t<real, 1, 1> Temperature             = \{\{ 1. \}\} ;
    DataArray\_t<real, 1, 1> EnergyInternal          = \{\{ 1.785714 \}\} ;
    DataArray\_t<real, 1, 1> Enthalpy                = \{\{ 2.5 \}\} ;
    DataArray\_t<real, 1, 1> EntropyApprox           = \{\{ 0.714286 \}\} ;

    DataArray\_t<real, 1, 1> DensityStagnation       = \{\{ 1.129726 \}\} ;
    DataArray\_t<real, 1, 1> PressureStagnation      = \{\{ 0.847295 \}\} ;
    DataArray\_t<real, 1, 1> EnergyStagnation        = \{\{ 1.875 \}\} ;
    DataArray\_t<real, 1, 1> EnthalpyStagnation      = \{\{ 2.625 \}\} ;
    DataArray\_t<real, 1, 1> TemperatureStagnation   = \{\{ 1.05 \}\} ;
    DataArray\_t<real, 1, 1> VelocitySoundStagnation = \{\{ 1.024695 \}\} ;

    DataArray\_t<real, 1, 1> ViscosityKinematic      = \{\{ 5.0E-07 \}\} ;
    \}\} ;
  \}\} ;          ! end TwoZoneCase
\end{alltt}

\subsection{Equation Description}

This section provides a description of the flow equations used to solve
the problem.  The flow equation set is turbulent, compressible 3-D
Navier-Stokes with the Spalart-Allmaras (S-A) one-equation turbulence
model.  The thin-layer Navier-Stokes diffusion terms are modeled; only
diffusion in the $j$-coordinate direction is included.

A perfect gas assumption is made with $\gamma = 1.4$; based on the
normalization used in this database, the nondimensional scales defining
$\gamma$ are $(c_p)' = 1/(\gamma - 1)$ and $(c_v)' = 1/\gamma(\gamma - 1)$.
The molecular viscosity is obtained from Sutherland's Law.
In order to nondimensionalize the viscosity formula, standard atmospheric
conditions are assumed (i.e., $T_\infty = 288.15$ K).
A constant Prandtl number assumption is made for the thermal conductivity
coefficient; $Pr = 0.72$.
The defining scales of $Pr$ are evaluated at freestream conditions; the
nondimensional thermal conductivity is $k'_\infty = \mu'_\infty (c_p)' / Pr$.

The Navier-Stokes equations are closed with an eddy viscosity assumption 
using the S-A model.
A turbulent Prandtl number of $Pr_t = 0.9$ is prescribed.
All parameters not provided are defaulted.

Except for the nondimensional parameters $\gamma$ and $Pr$, all
|DataArray_t| entities are abbreviated.
\begin{alltt}
CGNSBase\_t TwoZoneCase =
  \{\{
  int CellDimension = 3 ;
  int PhysicalDimension = 3 ;

  DataClass\_t DataClass = NormalizedByUnknownDimensional ;

  !  CellDimension = 3
  FlowEquationSet\_t<3> FlowEquationSet = 
    \{\{
    int EquationDimension = 3
    
    !  CellDimension = 3 ;
    GoverningEquations\_t<3> GoverningEquations =
      \{\{
      GoverningEquationsType\_t GoverningEquationsType = NSTurbulent ;
      
      int[6] DiffusionModel = [0,1,0,0,0,0] ;
      \}\} ;
      
    GasModel\_t GasModel =
      \{\{
      GasModelType\_t GasModelType = CaloricallyPerfect ;
      
      DataArray\_t<real, 1, 1> SpecificHeatRatio =
        \{\{
        Data(real, 1, 1) = 1.4 ;
        DataClass\_t DataClass = NondimensionalParameter ;
        \}\} ;
      DataArray\_t<real, 1, 1> SpecificHeatRatio\_Pressure = \{\{ 2.5 \}\} ;
      DataArray\_t<real, 1, 1> SpecificHeatRatio\_Volume   = \{\{ 1.785714 \}\} ;
      \}\} ;

    ViscosityModel\_t ViscosityModel =
      \{\{
      ViscosityModelType\_t ViscosityModelType = SutherLandLaw ;
      
      DataArray\_t<real, 1, 1> SutherLandLawConstant       = \{\{ 0.38383 \}\} ;
      DataArray\_t<real, 1, 1> TemperatureReference        = \{\{ 1.05491 \}\} ;
      DataArray\_t<real, 1, 1> ViscosityMolecularReference = \{\{ 5.0E-07 \}\} ;
      \}\} ;

    ThermalConductivityModel\_t ThermalConductivityModel =
      \{\{
      ThermalConductivityModelType\_t ThermalConductivityModelType = 
        ConstantPrandtl ;
      
      DataArray\_t<real, 1, 1> Prandtl = 
        \{\{
        Data(real, 1, 1) = 0.72 ;
        DataClass\_t DataClass = NondimensionalParameter ;
        \}\} ;
      DataArray\_t<real, 1, 1> Prandtl\_ThermalConductivity  = \{\{ 1.73611E-0.6 \}\} ;
      DataArray\_t<real, 1, 1> Prandtl\_ViscosityMolecular   = \{\{ 5.0E-0.7 \}\} ;
      DataArray\_t<real, 1, 1> Prandtl\_SpecificHeatPressure = \{\{ 2.5 \}\} ;
      \}\} ;

    TurbulenceClosure\_t TurbulenceClosure =
      \{\{
      TurbulenceClosureType\_t TurbulenceClosureType = EddyViscosity ;
      
      DataArray<real, 1, 1> PrandtlTurbulent = \{\{ 0.90 \}\} ;
      \}\} ;
      
    TurbulenceModel\_t<3> TurbulenceModel =
      \{\{
      TurbulenceModelType\_t TurbulenceModelType = 
        OneEquation\_SpalartAllmaras ;
      
      int[6] DiffusionModel = [0,1,0,0,0,0] ;
      \}\} ;      
    \}\} ;        ! end FlowEquationSet
  \}\} ;          ! TwoZoneCase
\end{alltt}
