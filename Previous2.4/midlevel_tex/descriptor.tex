\section{Descriptors}
\label{s:descriptors}
\thispagestyle{plain}

\subsection{Descriptive Text}
\label{s:descriptor}

\noindent
\textit{Node}: \fort{Descriptor\_t}

\begin{fctbox}
\textcolor{output}{\textit{ier}} = cg\_descriptor\_write(\textcolor{input}{char *name}, \textcolor{input}{char *text}); & - w m \\
\textcolor{output}{\textit{ier}} = cg\_ndescriptors(\textcolor{output}{\textit{int *ndescriptors}}); & r - m \\
\textcolor{output}{\textit{ier}} = cg\_descriptor\_read(\textcolor{input}{int D}, \textcolor{output}{\textit{char *name}}, \textcolor{output}{\textit{char **text}}); & r - m \\
\hline
call cg\_descriptor\_write\_f(\textcolor{input}{name}, \textcolor{input}{text}, \textcolor{output}{\textit{ier}}) & - w m \\
call cg\_ndescriptors\_f(\textcolor{output}{\textit{ndescriptors}}, \textcolor{output}{\textit{ier}}) & r - m \\
call cg\_descriptor\_size\_f(\textcolor{input}{D}, \textcolor{output}{\textit{size}}, \textcolor{output}{\textit{ier}}) & r - m \\
call cg\_descriptor\_read\_f(\textcolor{input}{D}, \textcolor{output}{\textit{name}}, \textcolor{output}{\textit{text}}, \textcolor{output}{\textit{ier}}) & r - m \\
\end{fctbox}

\noindent
\textbf{\textcolor{input}{Input}/\textcolor{output}{\textit{Output}}}

\begin{Ventryi}{\fort{ndescriptors}}\raggedright
\item [\fort{ndescriptors}]
      Number of \fort{Descriptor\_t} nodes under the current node.
      (\textcolor{output}{Output})
\item [\fort{D}]
      Descriptor index number, where $1 \leq \text{\fort{D}} \leq \text{\fort{ndescriptors}}$.
      (\textcolor{input}{Input})
\item [\fort{name}]
      Name of the \fort{Descriptor\_t} node.
      (\textcolor{input}{Input} for \fort{cg\_descriptor\_write};
      \textcolor{output}{\textit{output}} for \fort{cg\_descriptor\_read})
\item [\fort{text}]
      Description held in the \fort{Descriptor\_t} node.
      (\textcolor{input}{Input} for \fort{cg\_descriptor\_write};
      \textcolor{output}{\textit{output}} for \fort{cg\_descriptor\_read})
\item [\fort{size}]
      Size of the descriptor data (Fortran interface only).
      (\textcolor{output}{\textit{Output}})
\item [\fort{ier}]
      Error status.
      (\textcolor{output}{\textit{Output}})
\end{Ventryi}

Note that with \texttt{cg\_descriptor\_read} the memory for the descriptor
character string, \texttt{text}, will be allocated by the Mid-Level
Library.
The application code is responsible for releasing this memory when it is
no longer needed by calling \texttt{cg\_free(text)}, described in
\autoref{s:free}.

\subsection{Ordinal Value}
\label{s:ordinal}

\noindent
\textit{Node}: \fort{Ordinal\_t}

\begin{fctbox}
\textcolor{output}{\textit{ier}} = cg\_ordinal\_write(\textcolor{input}{int Ordinal}); & - w m \\
\textcolor{output}{\textit{ier}} = cg\_ordinal\_read(\textcolor{output}{\textit{int *Ordinal}}); & r - m \\
\hline
call cg\_ordinal\_write\_f(\textcolor{input}{Ordinal}, \textcolor{output}{\textit{ier}}) & - w m \\
call cg\_ordinal\_read\_f(\textcolor{output}{\textit{Ordinal}}, \textcolor{output}{\textit{ier}}) & r - m \\
\end{fctbox}

\noindent
\textbf{\textcolor{input}{Input}/\textcolor{output}{\textit{Output}}}

\begin{Ventryi}{\fort{Ordinal}}\raggedright
\item [\fort{Ordinal}]
      Any integer value.
      (\textcolor{input}{Input} for \fort{cg\_ordinal\_write};
      \textcolor{output}{\textit{output}} for \fort{cg\_ordinal\_read})
\item [\fort{ier}]
      Error status.
      (\textcolor{output}{\textit{Output}})
\end{Ventryi}
