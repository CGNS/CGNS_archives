\section{Detailed CGNS Node Descriptions}
\label{s:nodes}
\thispagestyle{plain}

This section, together with the figures in
\hyperref[s:figures]{Appendix~\ref*{s:figures}}, constitutes a complete
description of the CGNS database structure, together with detailed
descriptions of the contents of each attribute of each node. It is
intended to be suitable as a reference for anyone implementing CGNS
using the \SLL, but should also be of interest to those wishing to
understand exactly where information is stored within a CGNS database.

Note that it is the advertised purpose of the CGNS Mid-Level Library to
store and retrieve information in conformity with the mapping herein
described. Therefore, anyone accessing a CGNS database through the
CGNS routines alone does not need a detailed understanding of the file
mapping per se. However, this document should still prove useful in
ascertaining the meaning of some of the arguments which must be supplied
to the Library routines. Further, it will be necessary to consult the
SIDS themselves to determine some of the naming conventions.

The node descriptions in this section are closely
coupled to and cross-referenced with the figures in
\hyperref[s:figures]{Appendix~\ref*{s:figures}}, which show all the
nodes defined in the SIDS that have child nodes.
In the current section, the ``\textbf{Children:}'' entry in the list of
\textit{Node Attributes} is a reference to the figure showing that node
with its children.

The nodal hierarchy of the CGNS database directly reflects that of the
SIDS. Certain sections of the SIDS, notably Sections 4, 5, and 12,
describe basic data structures which appear repeatedly as children
of nodes representing more complex structures. In order to simplify
the presentation and avoid the introduction of undefined terms, this
section has been divided into two parts. \autoref{s:basicnodes}
defines a number of basic types which recur often in the structure,
and \autoref{s:specializednodes} describes higher level nodes of more
specific function built from those in \autoref{s:basicnodes}.\footnote{For
convenience, we group the node descriptions according to the type of
data the nodes contain. These groupings only roughly correspond to the
chapters in the SIDS.}

\subsection{Basic CGNS Nodes}
\label{s:basicnodes}

In this section we describe CGNS nodes which hold fundamental
\emph{types} of information. Their structure and function, which are the
same everywhere, are described here. However, the \emph{meaning} of the
data they hold at any particular point in a CGNS database depends on the
context, i.e., the parent node. Therefore, where necessary, any special
context-dependent meaning is elaborated in the paragraph devoted to the
parent.

\subsubsection{Descriptor Group}

These are user-assigned nodes designed to further describe the user's
intent. Their data is meant for human perusal or other user-designated
purposes.

\paragraph{\texttt{Descriptor\_t}}

The purpose of a node of type \fort{Descriptor\_t} is to store textual
data. It is not intended to be read by any system software, except to
return the text for human examination.

The intent of the \fort{Descriptor\_t} node is to hold comment sufficient
to allow someone other than the originator to understand what the
node contains. This may consist of a problem description, reference
documents, personal notes, etc.

Any node may have zero to many \fort{Descriptor\_t} children, with
the uses differentiated by the Name field. At this time, there are \emph{no}
conventions for these names or for the form of the associated text. It
is expected that a standard set such as \fort{README}, \fort{TimeStamp},
etc. will evolve as a matter of practice.

\textit{\uline{Node Attributes}}
\begin{Ventryic}{\textbf{Dimension Values:}}
\item [\textbf{Name:}]
      User defined
\item [\textbf{Label:}]
      \fort{Descriptor\_t}
\item [\textbf{DataType:}]
      \fort{C1}
\item [\textbf{Dimension:}]
      1
\item [\textbf{Dimension Values:}]
      Length of string to be stored, including any carriage control or
      null bytes.
\item [\textbf{Data:}]
      String - Since line terminators can be stored within the data,
      the user could conceptually store an entire document in this
      area, read it into a program and then print it out. For example,
      an entire PostScript document describing the problem (and maybe
      results) could be stored in the Data field, read by a program and
      then sent to a printer.
\item [\textbf{Children:}]
      None
\item [\textbf{Cardinality:}]
      0,$N$
\end{Ventryic}

\paragraph{\texttt{Ordinal\_t}}

Because there is no notion of order among children, there is
occasionally a desire to order children in a way that survives from one
opening of a CGNS database to another. The current CGNS Library provides
means of doing this. However, another early method was to place the node
``number'' in a child of type \fort{Ordinal\_t}.

Like \fort{Descriptor\_t}, the \fort{Ordinal\_t} node is completely
under the control of the user, who takes full responsibility for its
content. Unlike \fort{Descriptor\_t}, CGNS conventions do not encourage
the use of \fort{Ordinal\_t}, as it usually encodes information which
is redundant with the name. It is not read or written by standard
CGNS software, and so there is no assurance that sibling nodes
will be differently, consecutively, or consistently numbered by
\fort{Ordinal\_t}. Clearly, if \fort{Ordinal\_t} must be used, no node
should have more than one \fort{Ordinal\_t} child, and no two siblings
should have \fort{Ordinal\_t} children containing the same data.

It is worth noting that, if consistent numbering is desired, one way of
achieving it is to make the desired integer either the name or part of
the name. In fact, if, for example, individual zones are left unnamed,
the default convention will provide names of \fort{Zone1}, \fort{Zone2},
etc. Alternatively, the character strings ``\fort{1}'', ``\fort{2}'',
\ldots, are legal names. \SLL~ or CGNS software, of course, will return
these as strings. This may necessitate type conversion or parsing before
the names can be used as integer indices.

\textit{\uline{Node Attributes}}
\begin{Ventryic}{\textbf{Dimension Values:}}
\item [\textbf{Name:}]
      \fort{Ordinal}
\item [\textbf{Label:}]
      \fort{Ordinal\_t}
\item [\textbf{DataType:}]
      \fort{I4}
\item [\textbf{Dimension:}]
      1
\item [\textbf{Dimension Values:}]
      1
\item [\textbf{Data:}]
      The user-defined ordinal number (an integer).
\item [\textbf{Children:}]
      None
\item [\textbf{Cardinality:}]
      0,1
\end{Ventryic}

\subsubsection{Physical Data Group}

\paragraph{\texttt{DataClass\_t}}

A \fort{DataClass\_t} node specifies the dimensional nature of the
data in and below its parent. It overrides any \fort{DataClass\_t}
information higher up in the tree. There are six recognized string
values. It is necessary to consult the SIDS to determine the precise
meaning.

\textit{\uline{Node Attributes}}
\begin{Ventryic}{\textbf{Dimension Values:}}
\raggedright
\item [\textbf{Name:}]
      \fort{DataClass}
\item [\textbf{Label:}]
      \fort{DataClass\_t}
\item [\textbf{DataType:}]
      \fort{C1}
\item [\textbf{Dimension:}]
      1
\item [\textbf{Dimension Values:}]
      The length of the string
\item [\textbf{Data:}]
      One of: \fort{DataClassNull}, \fort{DataClassUserDefined},
      \fort{Dimensional}, \fort{NormalizedByDimensional},
      \fort{NormalizedByUnknownDimensional},
      \fort{NondimensionalParameter}, or \fort{DimensionlessConstant}
\item [\textbf{Children:}]
      None
\item [\textbf{Cardinality:}]
      0,1
\end{Ventryic}

\paragraph{\texttt{DimensionalUnits\_t}}

A \fort{DimensionalUnits\_t} node specifies dimensional units
which apply to data in and below its parent.  It overrides any
\fort{DimensionalUnits\_t} information higher up in the tree. There are
five strings to specify, corresponding, respectively, to units for mass,
length, time, temperature, and angular measure. The number of recognized
string values varies with the physical property.

Units for three additional types of data are specified in a child node,
\fort{AdditionalUnits\_t}.

\textit{\uline{Node Attributes}}
\begin{Ventryic}{\textbf{Dimension Values:}}
\item [\textbf{Name:}]
      \fort{DimensionalUnits}
\item [\textbf{Label:}]
      \fort{DimensionalUnits\_t}
\item [\textbf{DataType:}]
      \fort{C1}
\item [\textbf{Dimension:}]
      2
\item [\textbf{Dimension Values:}]
      (32,5)
\item [\textbf{Data:}]
      \begin{Ventryc}{For temperature, one of:}
      \raggedright
      \item [For mass, one of:]
            \fort{MassUnitsNull}, \fort{MassUnitsUserDefined},
            \fort{Kilogram}, \fort{Gram}, \fort{Slug}, \fort{Pound-Mass}
      \item [For length, one of:]
            \fort{LengthUnitsNull}, \fort{LengthUnitsUserDefined},
            \fort{Meter}, \fort{Centimeter}, \fort{Millimeter}, \fort{Foot},
            \fort{Inch}
      \item [For time, one of:]
            \fort{TimeUnitsNull}, \fort{TimeUnitsUserDefined}, \fort{Second}
      \item [For temperature, one of:]
            \fort{TemperatureUnitsNull}, \fort{TemperatureUnitsUserDefined},
            \fort{Kelvin}, \fort{Celsius}, \fort{Rankine}, \fort{Fahrenheit}
      \item [For angles, one of:]
            \fort{AngleUnitsNull}, \fort{AngleUnitsUserDefined},
            \fort{Degree}, \fort{Radian}
      \end{Ventryc}
\item [\textbf{Children:}]
      See \autoref{f:DimensionalUnits}
\item [\textbf{Cardinality:}]
      0,1
\end{Ventryic}

\paragraph{\texttt{AdditionalUnits\_t}}

An \fort{AdditionalUnits\_t} node specifies dimensional units for
additional types of data.
To maintain compatibility with earlier CGNS versions, this is an
optional child node of \fort{DimensionalUnits\_t}.
The specified units apply to data in and below the parent of the
corresponding \fort{DimensionalUnits\_t} node, and override any
\fort{AdditionalUnits\_t} information higher up in the tree.
There are three strings to specify, corresponding, respectively, to
units for electric current, substance amount, and luminous intensity.
The number of recognized string values varies with the physical
property.

\textit{\uline{Node Attributes}}
\begin{Ventryic}{\textbf{Dimension Values:}}
\item [\textbf{Name:}]
      \fort{AdditionalUnits}
\item [\textbf{Label:}]
      \fort{AdditionalUnits\_t}
\item [\textbf{DataType:}]
      \fort{C1}
\item [\textbf{Dimension:}]
      2
\item [\textbf{Dimension Values:}]
      (32,3)
\item [\textbf{Data:}]
      \begin{Ventryc}{For luminous intensity, one of:}
      \raggedright
      \item [For electric current, one of:]
            \fort{ElectricCurrentUnitsNull},
            \fort{ElectricCurrentUnitsUserDefined}, \fort{Ampere},
            \fort{Abampere}, \fort{Statampere}, \fort{Edison},
            \fort{auCurrent}
      \item [For substance amount, one of:]
            \fort{SubstanceAmountUnitsNull},
            \fort{SubstanceAmountUnitsUserDefined}, \fort{Mole},
            \fort{Entities}, \fort{StandardCubicFoot},
            \fort{StandardCubicMeter}
      \item [For luminous intensity, one of:]
            \fort{LuminousIntensityUnitsNull},
            \fort{LuminousIntensityUnitsUserDefined}, \fort{Candela},
            \fort{Candle}, \fort{Carcel}, \fort{Hefner},
            \fort{Violle}
      \end{Ventryc}
\item [\textbf{Children:}]
      None
\item [\textbf{Cardinality:}]
      0,1
\end{Ventryic}

\paragraph{\texttt{DataConversion\_t}}

A \fort{DataConversion\_t} node specifies a non-homogeneous linear
function which converts non-dimensional data in its parent to
raw dimensional data. Although in principle it overrides any
\fort{DataConversion\_t} information higher up in the tree, it is
generally not meaningful for it to apply to more than one kind of
physical data. Therefore, CGNS specifies its use only as a child of a
node which actually contains a single type of real data.

There are two values to specify, corresponding to the scale factor and
offset. The SIDS contain the exact conversion formula.

\textit{\uline{Node Attributes}}
\begin{Ventryic}{\textbf{Dimension Values:}}
\item [\textbf{Name:}]
      \fort{DataConversion}
\item [\textbf{Label:}]
      \fort{DataConversion\_t}
\item [\textbf{DataType:}]
      \fort{R4} or \fort{R8}
\item [\textbf{Dimension:}]
      1
\item [\textbf{Dimension Values:}]
      2
\item [\textbf{Data:}]
      \fort{ConversionScale}, \fort{ConversionOffset}
\item [\textbf{Children:}]
      None
\item [\textbf{Cardinality:}]
      0,1
\end{Ventryic}

\paragraph{\texttt{DimensionalExponents\_t}}

A \fort{DimensionalExponents\_t} node specifies the powers of mass,
length, time, temperature, and angular measure which characterize
dimensional data in its parent. Although in principle it overrides any
\fort{DimensionalExponents\_t} information higher up in the tree, it
is generally not meaningful for it to apply to more than one kind of
physical data. Therefore, CGNS specifies its use only as a child of
a node which actually contains a single type of real data.
There are five values to specify, corresponding to the five types of
units specified using \fort{DimensionalUnits\_t}.
The data type is real, not integer.

Exponents for three additional types of data are specified in a child
node, \fort{AdditionalExponents\_t}.

\textit{\uline{Node Attributes}}
\begin{Ventryic}{\textbf{Dimension Values:}}
\raggedright
\item [\textbf{Name:}]
      \fort{DimensionalExponents}
\item [\textbf{Label:}]
      \fort{DimensionalExponents\_t}
\item [\textbf{DataType:}]
      \fort{R4} or \fort{R8}
\item [\textbf{Dimension:}]
      1
\item [\textbf{Dimension Values:}]
      5
\item [\textbf{Data:}]
      \fort{MassExponent}, \fort{LengthExponent}, \fort{TimeExponent},
      \fort{TemperatureExponent}, \fort{AngleExponent}
\item [\textbf{Children:}]
      See \autoref{f:DimensionalExponents}
\item [\textbf{Cardinality:}]
      0,1
\end{Ventryic}

\paragraph{\texttt{AdditionalExponents\_t}}

An \fort{AdditionalExponents\_t} node specifies the powers of the units
for additional types of data, which characterize the corresponding
dimensional data.
There are three values to specify, corresponding to the three types of
units specified using \fort{AdditionalUnits\_t}.
The data type is real, not integer.

\textit{\uline{Node Attributes}}
\begin{Ventryic}{\textbf{Dimension Values:}}
\raggedright
\item [\textbf{Name:}]
      \fort{AdditionalExponents}
\item [\textbf{Label:}]
      \fort{AdditionalExponents\_t}
\item [\textbf{DataType:}]
      \fort{R4} or \fort{R8}
\item [\textbf{Dimension:}]
      1
\item [\textbf{Dimension Values:}]
      3
\item [\textbf{Data:}]
      \fort{ElectricCurrentExponent}, \fort{SubstanceAmountExponent},
      \fort{LuminousIntensityExponent}
\item [\textbf{Children:}]
      None
\item [\textbf{Cardinality:}]
      0,1
\end{Ventryic}

\paragraph{\texttt{DataArray\_t}}

A \fort{DataArray\_t} node is a very general type of node meant to hold
large arrays of data, such as grids and flowfields. Often, some of the
attributes of a \fort{DataArray\_t} node depend on the context in which
the node is found; that is, they are structure parameters.

For example, the SIDS specify that the Data Type of \fort{DataArray\_t}
is a structure parameter, ``\fort{DataType}'', which may assume
any of the values ``\fort{I\textit{n}}'', ``\fort{R\textit{n}}'',
``\fort{C\textit{n}}'', or ``\fort{bit}''.

The other two attributes of \fort{DataArray\_t}, \fort{Dimensions}
and \fort{DataSize}, also depend on the context where they are being used.
\fort{Dimensions} is a function of the underlying dimensionality of
the data being described (often \fort{IndexDimension}, defined in the
\fort{CGNSBase\_t} node), and the \fort{DataSize} may be inferred from
detailed descriptions of the grid.

A node may have any number of \fort{DataArray\_t} children. The
meaning of their contents is differentiated by Name, often
according to conventions specified by the SIDS. SIDS names are
usually precise and descriptive, such as \fort{CoordinateTheta} or
\fort{EnergyInternal}. (For a current list of sanctioned names, see the
SIDS, Appendix A.) Conversely, quantities not specified by the SIDS can
be stored in \fort{DataArray\_t} nodes, but should be given names other
than those specified in the SIDS. In other words, to comply with the
SIDS requires that one give a quantity the SIDS-defined name \emph{if
and only if} it is one of the SIDS-defined quantities.

\textit{\uline{Node Attributes}}
\begin{Ventryic}{\textbf{Dimension Values:}}
\item [\textbf{Name:}]
      Context dependent
\item [\textbf{Label:}]
      \fort{DataArray\_t}
\item [\textbf{DataType:}]
      Context dependent
\item [\textbf{Dimension:}]
      Context dependent
\item [\textbf{Dimension Values:}]
      Context dependent
\item [\textbf{Data:}]
      The array of data values
\item [\textbf{Children:}]
      See \autoref{f:DataArray}
\item [\textbf{Cardinality:}]
      0,$N$
\item [\textbf{Parameters:}]
      \fort{DataType}, dimension of the data, size of the data
\end{Ventryic}

\paragraph{Integer Arrays}

Integer array nodes perform the same function as nodes of type
\fort{DataArray\_t}, but store integer instead of real arrays. They are
always of type \fort{int[]}, with the dimensions and values given either
explicitly in the appropriate fields, or as parameters or functions.

\textit{\uline{Node Attributes}}
\begin{Ventryic}{\textbf{Dimension Values:}}
\raggedright
\item [\textbf{Name:}]
      Context dependent
\item [\textbf{Label:}]
      \fort{int}, \fort{int[IndexDimension]}, \fort{int[2*IndexDimension]},
      or \fort{int[1 + ... + IndexDimension]}
\item [\textbf{DataType:}]
      \fort{I4}
\item [\textbf{Dimension:}]
      1
\item [\textbf{Dimension Values:}]
      1, \fort{IndexDimension}, \fort{2*IndexDimension}, or
      \fort{(1 + ... + IndexDimension)}
\item [\textbf{Data:}]
      The array of integer values
\item [\textbf{Cardinality:}]
      0,1
\item [\textbf{Children:}]
      None
\item [\textbf{Parameters:}]
      \fort{IndexDimension} or none (context dependent)
\end{Ventryic}

\subsubsection{Location and Position Group}

\paragraph{\texttt{GridLocation\_t}}

A \fort{GridLocation\_t} node specifies the physical location, with
respect to the underlying grid, with which the field data below
its parent is associated. The value (data field) is a character
string of enumeration type, i.e., it must take one of a number of
predefined values. These values are: \fort{Vertex}, \fort{CellCenter},
\fort{FaceCenter}, \fort{IFaceCenter}, \fort{JFaceCenter},
\fort{KFaceCenter}, or \fort{EdgeCenter}. The strings are case
sensitive, and an exact match is required. The \fort{GridLocation\_t}
node is optional, and the default is \fort{Vertex}.

\newpage
\textit{\uline{Node Attributes}}
\begin{Ventryic}{\textbf{Dimension Values:}}
\raggedright
\item [\textbf{Name:}]
      \fort{GridLocation}
\item [\textbf{Label:}]
      \fort{GridLocation\_t}
\item [\textbf{DataType:}]
      \fort{C1}
\item [\textbf{Dimension:}]
      1
\item [\textbf{Dimension Values:}]
      Length of the string value
\item [\textbf{Data:}]
      \fort{Vertex}, \fort{CellCenter}, \fort{FaceCenter}, \fort{IFaceCenter},
      \fort{JFaceCenter}, \fort{KFaceCenter}, or \fort{EdgeCenter}
\item [\textbf{Children:}]
      None
\item [\textbf{Cardinality:}]
      0,1
\end{Ventryic}

\paragraph{\texttt{Rind\_t}}

The presence of a \fort{Rind\_t} node indicates that field data stored
below its parent includes values associated with a spatial extent beyond
that of the basic underlying grid. Such data often arise from the use of
ghost cells, or from the copying of information from adjacent zones.

Within a single zone, the size of the basic grid is found in the data
field of the \fort{Zone\_t} node (see \autoref{s:Zone}).
The data field of a \fort{Rind\_t} node contains integers specifying
the number of planes (for structured grids) or number of rind points or
elements (for unstructured grids) of included extra data.
The planes for structured grids correspond to the low and high values in
the $i$-direction, low and high values in the $j$-direction, and low and
high values in the $k$-direction (if needed), in that order.
Note that the actual size of the field data, which is stored in a
\fort{DataArray\_t} sibling node, is a \fort{DataSize} structure
parameter which depends on the basic grid size, the \fort{GridLocation},
and the \fort{Rind}.

The \fort{Rind\_t} node is optional, and the default is no rind.

\textit{\uline{Node Attributes}}
\begin{Ventryic}{\textbf{Dimension Values:}}
\item [\textbf{Name:}]
      \fort{Rind}
\item [\textbf{Label:}]
      \fort{Rind\_t}
\item [\textbf{DataType:}]
      \fort{I4}
\item [\textbf{Dimension:}]
      1
\item [\textbf{Dimension Values:}]
      \fort{2*IndexDimension}
\item [\textbf{Data:}]
      Number of planes of extra data in low $i$, high $i$, low $j$, high
      $j$, etc. (for structured grids) or number of points or elements
      of extra data (for unstructured grids)
\item [\textbf{Children:}]
      None
\item [\textbf{Cardinality:}]
      0,1
\item [\textbf{Parameters:}]
      \fort{IndexDimension}
\end{Ventryic}

\paragraph{\texttt{IndexRange\_t}}

An \fort{IndexRange\_t} node describes a subregion of a zone. This may
be, for example, a sub-block or a portion of a face of a zone. It may be
used to describe the locations of boundary condition patches and holes
for overset grids.

\textit{\uline{Node Attributes}}
\begin{Ventryic}{\textbf{Dimension Values:}}
\item [\textbf{Name:}]
      \fort{PointRange}, \fort{PointRangeDonor}, \fort{ElementRange},
      or user defined
\item [\textbf{Label:}]
      \fort{IndexRange\_t}
\item [\textbf{DataType:}]
      \fort{I4}
\item [\textbf{Dimension:}]
      2
\item [\textbf{Dimension Values:}]
      \fort{IndexDimension}, 2
\item [\textbf{Data:}]
      First indices, last indices
\item [\textbf{Children:}]
      None
\item [\textbf{Cardinality:}]
      Context dependent
\item [\textbf{Parameters:}]
      \fort{IndexDimension}
\end{Ventryic}

\paragraph{\texttt{IndexArray\_t}}

An \fort{IndexArray\_t} node describes a general subregion of a
zone. Unlike \fort{In\-dex\-Range\_t}, it lists all the elements of the
subregion, rather then only the first and last ones. Its use is similar
to \fort{IndexRange\_t}.

\textit{\uline{Node Attributes}}
\begin{Ventryic}{\textbf{Dimension Values:}}
\raggedright
\item [\textbf{Name:}]
      \fort{PointList}, \fort{PointListDonor}, \fort{CellListDonor},
      or \fort{InwardNormalList}
\item [\textbf{Label:}]
      \fort{IndexArray\_t}
\item [\textbf{DataType:}]
      \fort{I4}, or (for \fort{InwardNormalList}) \fort{R4} or \fort{R8}
\item [\textbf{Dimension:}]
      2
\item [\textbf{Dimension Values:}]
      \fort{IndexDimension}, number of items in the list;
      or (for \fort{InwardNormalList}) \fort{PhysicalDimension}, number
      of items in the list
\item [\textbf{Data:}]
      Index coordinates of each point or element in the list, or
      (for \fort{InwardNormalList}) physical-space normal vectors
      at each point or element in the list
\item [\textbf{Children:}]
      None
\item [\textbf{Cardinality:}]
      0,1
\item [\textbf{Parameters:}]
      \fort{IndexDimension}, either \fort{PointListSize} or
      \fort{ListLength}, and \fort{DataType};
      or (for \fort{InwardNormalList})
      \fort{PhysicalDimension}, \fort{ListLength}, and \fort{DataType}
\end{Ventryic}

\subsubsection{Auxiliary Data Group}

\paragraph{\texttt{ReferenceState\_t}}

The appearance of a \fort{ReferenceState\_t} node is optional. It is
used to specify the values of flow quantities at reference conditions,
e.g., at freestream or stagnation. This is typically done for the whole
database, in which case the \fort{ReferenceState\_t} node is a child of
the \fort{CGNSBase\_t} node.

\fort{ReferenceState\_t} nodes follow the usual convention that
information specified lower in the tree overrides higher level
specifications. Such overrides are therefore specified if a
\fort{ReferenceState\_t} node appears as a child of a \fort{Zone\_t},
\fort{ZoneBC\_t}, or \fort{BCDataSet\_t} node.

The actual values are stored in one or more \fort{DataArray\_t} children
whose names identify the quantities being stored. If present, the
units specified in the \fort{DimensionalUnits\_t} child apply to all
\fort{DataArray\_t} children, subject to the usual override convention.
( I.e., if one of the \fort{DataArray\_t} children itself has a
\fort{DimensionalUnits\_t} child, it takes precedence over the higher
level specification.)

\textit{\uline{Node Attributes}}
\begin{Ventryic}{\textbf{Dimension Values:}}
\item [\textbf{Name:}]
      \fort{ReferenceState}
\item [\textbf{Label:}]
      \fort{ReferenceState\_t}
\item [\textbf{DataType:}]
      \fort{MT}
\item [\textbf{Children:}]
      See \autoref{f:ReferenceState}
\item [\textbf{Cardinality:}]
      0,1
\end{Ventryic}

\paragraph{\texttt{ConvergenceHistory\_t}}

\fort{ConvergenceHistory\_t} nodes are intended for the storage of lists
of quantities accumulated during calculations associated with either the
entire CGNS database or with a single zone.

In the former case, they are called Global convergence histories, and
appear as children of the \fort{CGNSBase\_t} node. In the latter, they
are called Local and stored below, with the zones to which they correspond.

Each \fort{ConvergenceHistory\_t} node is a parent of a collection
of one-dimensional \fort{DataArray\_t} nodes, each of which contains
a list of values of a quantity defined by the user. These quantities
are differentiated by their user-assigned Names. User definitions of
the names are recorded in a \fort{Descriptor\_t} child node with Name
\fort{NormDefinitions}. Children of types \fort{DataClass\_t} and
\fort{DimensionalUnits\_t} modify the meaning of the \fort{DataArray\_t}
children in the usual manner.

\textit{\uline{Node Attributes}}
\begin{Ventryic}{\textbf{Dimension Values:}}
\raggedright
\item [\textbf{Name:}]
      \fort{GlobalConvergenceHistory} if under a \fort{CGNSBase\_t} node;
      \fort{ZoneConvergenceHistory} if under a \fort{Zone\_t} node
\item [\textbf{Label:}]
      \fort{ConvergenceHistory\_t}
\item [\textbf{DataType:}]
      \fort{I4}
\item [\textbf{Dimension:}]
      1
\item [\textbf{Dimension Values:}]
      1
\item [\textbf{Data:}]
      Number of iterations
\item [\textbf{Children:}]
      See \autoref{f:ConvergenceHistory}
\item [\textbf{Cardinality:}]
      0,1
\end{Ventryic}

\paragraph{\texttt{IntegralData\_t}}

\fort{IntegralData\_t} nodes are intended for the storage of
integrated flow quantities such as mass flows, forces and
moments. These are kept in \fort{DataArray\_t} children just as in the
\fort{ConvergenceHistory\_t} nodes, except that these nodes hold only
one real number each.

\textit{\uline{Node Attributes}}
\begin{Ventryic}{\textbf{Dimension Values:}}
\item [\textbf{Name:}]
      User defined
\item [\textbf{Label:}]
      \fort{IntegralData\_t}
\item [\textbf{DataType:}]
      \fort{MT}
\item [\textbf{Children:}]
      See \autoref{f:IntegralData}
\item [\textbf{Cardinality:}]
      0,$N$
\end{Ventryic}

\paragraph{\texttt{UserDefinedData\_t}}

\fort{UserDefinedData\_t} nodes are intended as a means of
storing arbitrary user-defined data in \fort{Descriptor\_t} and
\fort{DataArray\_t} children without the restrictions or implicit
meanings imposed on these node types at other node locations.

Multiple \fort{Descriptor\_t} and \fort{DataArray\_t} children
may be stored below a \fort{UserDefinedData\_t} node, and the
\fort{DataArray\_t} children may be of any dimension and size.

\textit{\uline{Node Attributes}}
\begin{Ventryic}{\textbf{Dimension Values:}}
\item [\textbf{Name:}]
      User defined
\item [\textbf{Label:}]
      \fort{UserDefinedData\_t}
\item [\textbf{DataType:}]
      \fort{MT}
\item [\textbf{Children:}]
      See \autoref{f:UserDefinedData}
\item [\textbf{Cardinality:}]
      0,$N$
\end{Ventryic}

\paragraph{\texttt{Gravity\_t}}

An optional \fort{Gravity\_t} node may be used to define the
gravitational vector.

\textit{\uline{Node Attributes}}
\begin{Ventryic}{\textbf{Dimension Values:}}
\item [\textbf{Name:}]
      \fort{Gravity}
\item [\textbf{Label:}]
      \fort{Gravity\_t}
\item [\textbf{DataType:}]
      \fort{MT}
\item [\textbf{Children:}]
      See \autoref{f:Gravity}
\item [\textbf{Cardinality:}]
      0,1
\end{Ventryic}

\subsection{Specialized Nodes}
\label{s:specializednodes}

In this section we describe nodes whose use is specialized to certain
types of CFD-related data. Although these nodes may appear in multiple
places in a CGNS DataBase, they play a single role in the description of
the data.

\subsubsection{Grid Specification}

CGNS recognizes the notion of a collection of subdomains called zones,
within each of which there is a single structured or unstructured
grid. Mathematically, the grid is an assignment of a location in
physical space to each element in a discrete computational space. An
essential feature of the grid is the connection structure it inherits
from the underlying computational space.

It is possible, given a grid, to create others from it, by translation
to cell centers, for example. However, CGNS views these as new
field structures associated with the original grid, and the File
Mapping specifies that they be stored as \fort{FlowSolution\_t} or
\fort{DiscreteData\_t} nodes (see \autoref{s:fieldspecification}).

\paragraph{\texttt{GridCoordinates\_t}}

A \fort{GridCoordinates\_t} node describes a grid associated with
a single zone. For a structured zone, the connection structure of
the underlying computational space is that of a rectangular array,
and its dimension is the \fort{IndexDimension}, that is, the number
of integers required to identify a point in the grid. The physical
dimension is the number of real coordinates assigned at each grid point
and need not be the same. Thus CGNS can store a grid, for example, with
\fort{IndexDimension} equal to two and a physical dimension of three,
that is, a structured grid on a curved surface.

\fort{IndexDimension} is a zone dependent parameter. For an unstructured
grid, it always equals one, meaning that a unique index is required to
specified a node location. For a structured grid, \fort{IndexDimension}
varries with the \fort{CellDimension} of the mesh. For a mesh composed
of 3D cells, \fort{IndexDimension} equals 3, while for a mesh composed
of surface or shell elements, \fort{IndexDimension} equals 2. The
values of the physical coordinates of the grid points are stored
in \fort{DataArray\_t} children of \fort{GridCoordinates\_t}. The
names of the coordinates are stored in the Name field of the
corresponding \fort{DataArray\_t} node. For common coordinate systems,
i.e., Cartesian, polar, cylindrical, and spherical, the names are
specified by the SIDS.

Unlike \fort{FlowSolution\_t} and \fort{DiscreteData\_t} nodes (see
\autoref{s:fieldspecification}), \fort{GridCoordinates\_t} nodes are
not permitted to have \fort{GridLocation\_t} children, because all grid
points are at vertices by definition.

Coordinate arrays may also contain rind data. If
they do, the \fort{GridCoordinates\_t} node must have a \fort{Rind\_t}
child node describing the amount of rind. All \fort{DataArray\_t} nodes
under \fort{GridCoordinates\_t} must have the same size. Because the
number of field quantities to be stored depends on the number of rind,
the actual dimension values are functions, specified in this document by
the generic term \fort{DataSize[]}.

Under each node of type \fort{Zone\_t}, the original grid is
contained in a node named \fort{GridCoordinates}.  Additional
\fort{GridCoordinates\_t} data structures are allowed, with user-defined
names, to store the grid at multiple time steps or iterations.

\textit{\uline{Node Attributes}}
\begin{Ventryic}{\textbf{Dimension Values:}}
\item [\textbf{Name:}]
      \fort{GridCoordinates} or user defined
\item [\textbf{Label:}]
      \fort{GridCoordinates\_t}
\item [\textbf{DataType:}]
      \fort{MT}
\item [\textbf{Children:}]
      See \autoref{f:GridCoordinates}
\item [\textbf{Cardinality:}]
      0,$N$
\item [\textbf{Parameters:}]
      \fort{IndexDimension}, \fort{VertexSize}
\item [\textbf{Functions:}]
      \fort{DataSize}
\end{Ventryic}

\paragraph{\texttt{Elements\_t}}

The \fort{Elements\_t} data structure is required for unstructured
zones, and contains the element connectivity data, the element type,
the element range, the parent elements data, and the number of boundary
elements.

\textit{\uline{Node Attributes}}
\begin{Ventryic}{\textbf{Dimension Values:}}
\item [\textbf{Name:}]
      User defined
\item [\textbf{Label:}]
      \fort{Elements\_t}
\item [\textbf{DataType:}]
      \fort{I4}
\item [\textbf{Dimension:}]
      1
\item [\textbf{Dimension Values:}]
      2
\item [\textbf{Data:}]
      \fort{ElementType} value, \fort{ElementSizeBoundary}
\item [\textbf{Children:}]
      See \autoref{f:Elements}
\item [\textbf{Cardinality:}]
      0,$N$
\end{Ventryic}

\paragraph{\texttt{Axisymmetry\_t}}

The \fort{Axisymmetry\_t} data structure may be included as a child
of the \fort{CGNSBase\_t} node to record the axis of rotation and the
angle of rotation around this axis for an axisymmetric database.

\textit{\uline{Node Attributes}}
\begin{Ventryic}{\textbf{Dimension Values:}}
\item [\textbf{Name:}]
      \fort{Axisymmetry}
\item [\textbf{Label:}]
      \fort{Axisymmetry\_t}
\item [\textbf{DataType:}]
      \fort{MT}
\item [\textbf{Children:}]
      See \autoref{f:Axisymmetry}
\item [\textbf{Cardinality:}]
      0,1
\end{Ventryic}

\paragraph{\texttt{RotatingCoordinates\_t}}

The \fort{RotatingCoordinates\_t} data structure may be included as a
child of either the \fort{CGNSBase\_t} node or a \fort{Zone\_t} node
to record the center of rotation and the rotation rate vector for a
rotating coordinate system.

\textit{\uline{Node Attributes}}
\begin{Ventryic}{\textbf{Dimension Values:}}
\item [\textbf{Name:}]
      \fort{RotatingCoordinates}
\item [\textbf{Label:}]
      \fort{RotatingCoordinates\_t}
\item [\textbf{DataType:}]
      \fort{MT}
\item [\textbf{Children:}]
      See \autoref{f:RotatingCoordinates}
\item [\textbf{Cardinality:}]
      0,1
\end{Ventryic}

\subsubsection{Field Specification}
\label{s:fieldspecification}

The object of computational field physics is to compute fields
of physical data associated with points in space.

\paragraph{\texttt{FlowSolution\_t}}

A \fort{FlowSolution\_t} node describes a field of physical data
associated with the grid for a single zone. It is intended for
the storage of computed flowfield data such as densities and
pressures. There is no convention as to how many or what kind
of quantities must or may be stored. In particular, it is not
specified that the quantities need in any sense be either complete or
non-redundant.

The data are stored in \fort{DataArray\_t} children of
\fort{FlowSolution\_t}. These \fort{DataArray\_t} nodes are dimensioned
by the same underlying \fort{IndexDimension} parameter as the grid, and
the order of storage within the \fort{DataArray\_t} nodes is presumed
the same as it is for the grid. The names of the physical quantities are
stored in the Name field of the corresponding \fort{DataArray\_t}
node. For common fluid dynamic quantities the names are specified by the
SIDS.

The relationship between the locations of the field quantities and the
vertices of the grid is specified by a \fort{GridLocation\_t} child
node. If this node is absent, the field quantities are assumed to be
associated with the grid vertices. Field arrays
may also contain rind data. If they do, the \fort{FlowSolution\_t} node
must have a \fort{Rind\_t} child node describing the amount of rind. All
\fort{DataArray\_t} nodes under a single \fort{FlowSolution\_t} must
have the same size. Field arrays containing different numbers of rind
must be stored under different \fort{FlowSolution\_t} nodes. There
may be any number of nodes of type \fort{FlowSolution\_t} under a
\fort{Zone\_t}.

Because the number of field quantities to be stored depends on the
number of rind and on the location with respect to the grid, the actual
dimension values are functions, specified in this document by the
generic term \fort{DataSize[]}.

The meaning of the field arrays is modified in the usual way by any
\fort{DataClass\_t} or \fort{DimensionalUnits\_t} children of the
\fort{FlowSolution\_t} node.

\textit{\uline{Node Attributes}}
\begin{Ventryic}{\textbf{Dimension Values:}}
\item [\textbf{Name:}]
      User defined
\item [\textbf{Label:}]
      \fort{FlowSolution\_t}
\item [\textbf{DataType:}]
      \fort{MT}
\item [\textbf{Children:}]
      See \autoref{f:FlowSolution}
\item [\textbf{Cardinality:}]
      0,$N$
\item [\textbf{Parameters:}]
      \fort{IndexDimension}, \fort{VertexSize}, \fort{CellSize}
\item [\textbf{Functions:}]
      \fort{DataSize}
\end{Ventryic}

\paragraph{\texttt{DiscreteData\_t}}

\fort{DiscreteData\_t} nodes are identical to \fort{FlowSolution\_t}
nodes, but are intended for the storage of fields of real data not
usually identified as part of the field solution, such as cell-centered
grids.

\textit{\uline{Node Attributes}}
\begin{Ventryic}{\textbf{Dimension Values:}}
\item [\textbf{Name:}]
      User defined
\item [\textbf{Label:}]
      \fort{DiscreteData\_t}
\item [\textbf{DataType:}]
      \fort{MT}
\item [\textbf{Children:}]
      See \autoref{f:FlowSolution}
\item [\textbf{Cardinality:}]
      0,$N$
\item [\textbf{Parameters:}]
      \fort{IndexDimension}, \fort{VertexSize}, \fort{CellSize}
\item [\textbf{Functions:}]
      \fort{DataSize}
\end{Ventryic}

\subsubsection{Connectivity Group}

\paragraph{\texttt{Transform} Node}

The \fort{Transform} node is a node of type \fort{int[]} which is
identified by its name rather than its label. Thus the name must be
``\fort{Transform}''. It appears only as a child of a node of type
\fort{GridConnectivity1to1\_t}.

This node stores the transformation matrix relating the indices of two
adjacent zones. Its data field contains a list of \fort{IndexDimension}
signed integers, each within the range \fort{\mbox{[-IndexDimension},
..., +IndexDimension]}, and no two of which have the same absolute
value. Thus in 3-D allowed components are 0, $\pm 1$, $\pm 2$, and
$\pm 3$. Each component of the array shows the image in the adjacent zone
of a positive index increment in the current zone. The SIDS contain
complete details.

\textit{\uline{Node Attributes}}
\begin{Ventryic}{\textbf{Dimension Values:}}
\item [\textbf{Name:}]
      \fort{Transform}
\item [\textbf{Label:}]
      ``\fort{int[IndexDimension]}''
\item [\textbf{DataType:}]
      \fort{I4}
\item [\textbf{Dimension:}]
      1
\item [\textbf{Dimension Values:}]
      \fort{IndexDimension}
\item [\textbf{Data:}]
      Transformation matrix (shorthand)
\item [\textbf{Children:}]
      None
\item [\textbf{Cardinality:}]
      0,1
\item [\textbf{Parameters:}]
      \fort{IndexDimension}
\end{Ventryic}

\paragraph{\texttt{GridConnectivityType\_t}}

The purpose of this node is to describe the type of zone-to-zone
connectivity specified by its parent, which is always a
\fort{GridConnectivity\_t} node. The connectivity type is given in the
data field as a character string which may take one of three specific
values: \fort{Abutting}, \fort{Abutting1to1}, or \fort{Overset}.

There is a shorthand form of the \fort{GridConnectivity\_t}
node, namely, \fort{GridConnectivity1to1\_t}, which
incorporates the assumption that the connection is
\fort{Abutting1to1}. Nodes of type \fort{GridConnectivity1to1\_t}
do not have \fort{GridConnectivityType\_t} subnodes. However,
\fort{GridConnectivity1to1\_t} nodes can only be used to specify
zone-to-zone connections on rectangular subregions between two
structured zones. So the use of \fort{GridConnectivityType\_t} subnodes
to specify \fort{Abutting1to1} is required if the connecting regions
are not rectangular, or if the connectivity involves a least one
unstructured zone.

\textit{\uline{Node Attributes}}
\begin{Ventryic}{\textbf{Dimension Values:}}
\item [\textbf{Name:}]
      \fort{GridConnectivityType}
\item [\textbf{Label:}]
      \fort{GridConnectivityType\_t}
\item [\textbf{DataType:}]
      \fort{C1}
\item [\textbf{Dimension:}]
      1
\item [\textbf{Dimension Values:}]
      Length of string
\item [\textbf{Data:}]
      \fort{Abutting}, \fort{Abutting1to1}, or \fort{Overset}
\item [\textbf{Children:}]
      None
\item [\textbf{Cardinality:}]
      0,1
\end{Ventryic}

\paragraph{\texttt{GridConnectivity1to1\_t}}

This node is a shorthand format of \fort{GridConnectivity\_t}
(see \autoref{s:GridConnectivity}) capable of describing only
\fort{Abutting1to1} connections between two structured zones. The
underlying subregion must have rectangular data structure.

Each \fort{GridConnectivity1to1\_t} node describes a subregion
of a face of a zone whose vertices are coincident in a 1-to-1
fashion with those of a corresponding subregion of a face of another
zone. Each \fort{ZoneGridConnectivity\_t} node may have as many
\fort{GridConnectivity1to1\_t} (or \fort{GridConnectivity\_t}) children
as are required to describe the connection structure.

The location of the connected subregion of a face of the current zone
is given in a single child of type \fort{IndexRange\_t}, whose name
is specified by the mapping as ``\fort{PointRange}''. The location of
the corresponding subregion on a face of the other zone is given in
a single child of type \fort{IndexRange\_t}, whose name is specified
by the mapping as ``\fort{PointRangeDonor}''. The first (i.e.,
beginning) points in these \fort{IndexRange\_t} nodes are presumed to be
coincident. The specification of the correspondence is completed by the
inclusion of a \fort{Transform} child node which describes the relative
orientation of the two systems of indices. The second (i.e., end) point
of the \fort{PointRange} subnode specifies the extant of the connection.

In general, the File Mapping seeks to avoid the storage of
redundant data. However, there are two redundancies associated with
\fort{GridConnectivity1to1\_t}. First, for the sake of symmetry, the
information recorded here is duplicated (in reverse) in a corresponding
node under the donor zone. It is expected that these two specifications
will agree.

Second, the end point of the \fort{PointRangeDonor} can be
calculated from the other three points specified, along with the
transform. However, the transform cannot be inferred from the four
points. Therefore, the end point of the \fort{PointRangeDonor} is
considered to be redundant, and the three points and the transform are
designated as the primary specification.

\textit{\uline{Node Attributes}}
\begin{Ventryic}{\textbf{Dimension Values:}}
\item [\textbf{Name:}]
      User defined
\item [\textbf{Label:}]
      \fort{GridConnectivity1to1\_t}
\item [\textbf{DataType:}]
      \fort{C1}
\item [\textbf{Dimension:}]
      1
\item [\textbf{Dimension Values:}]
      Length of string
\item [\textbf{Data:}]
      \fort{ZoneDonorName}
\item [\textbf{Children:}]
      See \autoref{f:GridConnectivity1to1}
\item [\textbf{Cardinality:}]
      0,$N$
\item [\textbf{Parameters:}]
      \fort{IndexDimension}
\end{Ventryic}

\paragraph{\texttt{GridConnectivity\_t}}
\label{s:GridConnectivity}

The \fort{GridConnectivity\_t} node is the most general format for
describing grid connectivity. It can describe one-to-one, mismatched,
and overset connectivity, and the underlying subregions of the
connecting zones need not be rectangular.

Each \fort{GridConnectivity\_t} node describes a subregion
of a zone which corresponds to a subregion of another
zone. Each \fort{ZoneGridConnectivity\_t} node may have as many
\fort{GridConnectivity\_t} (or \fort{GridConnectivity1to1\_t}) children
as are required to describe the connection structure.

The location of the connected subregion of the current zone is
given in a single child of type either \fort{IndexRange\_t} or
\fort{IndexArray\_t}, whose name is specified by the mapping as
``\fort{PointRange}'' or ``\fort{PointList}'', respectively.

If the grid connectivity is one-to-one, the corresponding subregion is
defined with a single child of type \fort{IndexArray\_t}, whose name is
specified by the mapping as ``\fort{PointListDonor}''.  Otherwise, the
corresponding subregion is defined by two child nodes, one defining the
cells and the other the interpolation factors within the cells.
See the SIDS for the complete description.

\textit{\uline{Node Attributes}}
\begin{Ventryic}{\textbf{Dimension Values:}}
\item [\textbf{Name:}]
      User defined
\item [\textbf{Label:}]
      \fort{GridConnectivity\_t}
\item [\textbf{DataType:}]
      \fort{C1}
\item [\textbf{Dimension:}]
      1
\item [\textbf{Dimension Values:}]
      Length of string
\item [\textbf{Data:}]
      \fort{ZoneDonorName}
\item [\textbf{Children:}]
      See \autoref{f:GridConnectivity}
\item [\textbf{Cardinality:}]
      0,$N$
\item [\textbf{Parameters:}]
      \fort{IndexDimension}, \fort{CellDimension}
\item [\textbf{Functions:}]
      \fort{PointListSize}
\end{Ventryic}

\paragraph{\texttt{GridConnectivityProperty\_t}}

An optional \fort{GridConnectivityProperty\_t} node may be used to
record special properties associated with particular connectivity
patches.

\textit{\uline{Node Attributes}}
\begin{Ventryic}{\textbf{Dimension Values:}}
\item [\textbf{Name:}]
      \fort{GridConnectivityProperty}
\item [\textbf{Label:}]
      \fort{GridConnectivityProperty\_t}
\item [\textbf{DataType:}]
      \fort{MT}
\item [\textbf{Children:}]
      See \autoref{f:GridConnectivityProperty}
\item [\textbf{Cardinality:}]
      0,1
\end{Ventryic}

\paragraph{\texttt{Periodic\_t}}

A \fort{Periodic\_t} node may be used as a child of
\fort{GridConnectivityProperty\_t} to record data associated with a
periodic interface.

\textit{\uline{Node Attributes}}
\begin{Ventryic}{\textbf{Dimension Values:}}
\item [\textbf{Name:}]
      \fort{Periodic}
\item [\textbf{Label:}]
      \fort{Periodic\_t}
\item [\textbf{DataType:}]
      \fort{MT}
\item [\textbf{Children:}]
      See \autoref{f:Periodic}
\item [\textbf{Cardinality:}]
      0,1
\end{Ventryic}

\paragraph{\texttt{AverageInterface\_t}}

An \fort{AverageInterface\_t} node is used as a child of
\fort{GridConnectivityProperty\_t} when data at the current connectivity
interface is to be averaged in some way prior to passing it to a
neighboring interface.

\textit{\uline{Node Attributes}}
\begin{Ventryic}{\textbf{Dimension Values:}}
\item [\textbf{Name:}]
      \fort{AverageInterface}
\item [\textbf{Label:}]
      \fort{AverageInterface\_t}
\item [\textbf{DataType:}]
      \fort{MT}
\item [\textbf{Children:}]
      See \autoref{f:AverageInterface}
\item [\textbf{Cardinality:}]
      0,1
\end{Ventryic}

\paragraph{\texttt{OversetHoles\_t}}

A node of type \fort{OversetHoles\_t} describes a region in a grid in
which solution values are to be ignored because the data in the region
is to be represented by values associated with other ``overlapping''
zones (equivalent to that specified by \fort{IBLANK} $= 0$ in the PLOT3D
format). Each \fort{ZoneGridConnectivity\_t} node may have as many
\fort{OversetHoles\_t} children as are required to describe the affected
region.

Each hole is described either by a single child of type \fort{IndexArray\_t} or
by any number of children of type \fort{IndexRange\_t}. The latter is provided
as a means of specifying holes which are unions of small numbers of
logically rectangular subregions. However, if the region is irregular,
the intent is that it should be specified by a single child of type
\fort{IndexArray\_t} which lists the points.

\textit{\uline{Node Attributes}}
\begin{Ventryic}{\textbf{Dimension Values:}}
\item [\textbf{Name:}]
      User defined
\item [\textbf{Label:}]
      \fort{OversetHoles\_t}
\item [\textbf{DataType:}]
      \fort{MT}
\item [\textbf{Children:}]
      See \autoref{f:OversetHoles}
\item [\textbf{Cardinality:}]
      0,$N$
\item [\textbf{Parameters:}]
      \fort{IndexDimension}
\end{Ventryic}

\paragraph{\texttt{ZoneGridConnectivity\_t}}

Each \fort{Zone\_t} node may have at most one child of type
\fort{ZoneGridConnectivity}. It holds no data, but serves as the point
below which all connectivity data associated with the zone can be found.

\textit{\uline{Node Attributes}}
\begin{Ventryic}{\textbf{Dimension Values:}}
\item [\textbf{Name:}]
      \fort{ZoneGridConnectivity}
\item [\textbf{Label:}]
      \fort{ZoneGridConnectivity\_t}
\item [\textbf{DataType:}]
      \fort{MT}
\item [\textbf{Children:}]
      See \autoref{f:ZoneGridConnectivity}
\item [\textbf{Cardinality:}]
      0,1
\item [\textbf{Parameters:}]
      \fort{IndexDimension}, \fort{CellDimension}
\end{Ventryic}

\subsubsection{Boundary Condition Group}

Nodes in this group are used to specify the physical boundary
conditions. Each boundary condition is associated with a subregion of
a zone. For brevity below, we use the word ``domain'' to refer to the
region on which a boundary condition is to be enforced.

The domain is usually, but not necessarily, a subregion of a face of the
zone. The mapping is sufficiently general to permit the description of
internal boundary conditions and boundary conditions which do not lie on
a constant coordinate plane.

Mathematical boundary conditions are generally applied on subregions of
physical dimension one less than the corresponding field problem. This
condition, however, is neither defined nor enforced by the File Mapping.

A large number of standard boundary condition types are named by the
SIDS. In addition, it is possible to define new types as collections
of Dirichlet and Neumann conditions. It is not possible to describe
the entire array of possibilities within this document, and the reader
should consult the SIDS for a full description.

\paragraph{\texttt{InwardNormalIndex}}

An \fort{InwardNormalIndex} node is a node of type
\fort{int[IndexDimension]} which is identified by its Name.
It applies to structured grids only, and its function is to specify on
which side of the domain the condition is to be enforced.

\fort{InwardNormalIndex} may have only one nonzero element, whose
sign indicates the computational-coordinate direction of the boundary
condition patch normal; this normal points into the interior of the zone.
For example, if the domain lies on the face of a three-dimensional zone
where the second index is a maximum, the inward normal index values are
$[0,-1,0]$.

The \fort{InwardNormalIndex} node must apply to the entire domain
of the boundary condition.

For a boundary condition on a face of a zone, the
\fort{InwardNormalIndex} can be calculated from other data and need not
be specified. Its purpose is to define the normal direction for internal
boundary conditions and other cases where the direction is ambiguous.

\textit{\uline{Node Attributes}}
\begin{Ventryic}{\textbf{Dimension Values:}}
\item [\textbf{Name:}]
      \fort{InwardNormalIndex}
\item [\textbf{Label:}]
      ``\fort{int[IndexDimension]}''
\item [\textbf{DataType:}]
      \fort{I4}
\item [\textbf{Dimension:}]
      1
\item [\textbf{Dimension Values:}]
      \fort{IndexDimension}
\item [\textbf{Data:}]
      Index of inward normal
\item [\textbf{Children:}]
      None
\item [\textbf{Cardinality:}]
      0,1
\item [\textbf{Parameters:}]
      \fort{IndexDimension}
\end{Ventryic}

\paragraph{\texttt{InwardNormalList}}

An \fort{InwardNormalList} node is a node of type \fort{IndexArray\_t}
identified by its Name. Its data field contains an array of
physical (real) vectors which point into the region on which the
boundary condition is to be applied.  It may be used for boundary
conditions on complex domains for which \fort{InwardNormalIndex} is not
defined, or to store vectors orthogonal to the domain of the boundary
condition where these are not easily calculated from the domain itself.

\textit{\uline{Node Attributes}}
\begin{Ventryic}{\textbf{Dimension Values:}}
\item [\textbf{Name:}]
      \fort{InwardNormalList}
\item [\textbf{Label:}]
      \fort{IndexArray\_t}
\item [\textbf{DataType:}]
      \fort{R4} or \fort{R8}
\item [\textbf{Dimension:}]
      2
\item [\textbf{Dimension Values:}]
      \fort{PhysicalDimension}, \fort{ListLength}
\item [\textbf{Data:}]
      Inward normal vectors
\item [\textbf{Children:}]
      None
\item [\textbf{Cardinality:}]
      0,1
\item [\textbf{Parameters:}]
      \fort{PhysicalDimension}, \fort{ListLength}
\end{Ventryic}

\paragraph{\texttt{BCData\_t}}

When global or local Dirichlet or Neumann boundary conditions are
defined, a node of type \fort{BCData\_t} is introduced to store the
numerical data. For global data, this consists of a single quantity kept
in a \fort{DataArray\_t} child. For local
data, e.g., a pressure profile, it is a vector of quantities stored in
an order corresponding to that defining the domain and kept in a child
node of type \fort{DataArray\_t}.

\textit{\uline{Node Attributes}}
\begin{Ventryic}{\textbf{Dimension Values:}}
\item [\textbf{Name:}]
      \fort{DirichletData} or \fort{NeumannData}
\item [\textbf{Label:}]
      \fort{BCData\_t}
\item [\textbf{DataType:}]
      \fort{MT}
\item [\textbf{Children:}]
      See \autoref{f:BCData}
\item [\textbf{Cardinality:}]
      0,1
\item [\textbf{Parameters:}]
      \fort{ListLength}
\end{Ventryic}

\paragraph{\texttt{BCDataSet\_t}}

The function of a \fort{BCDataSet\_t} node is to specify the equations
to be applied at the boundary, including any actual data values which
may be required. The type of the equation is specified by the SIDS and
recorded in the data field. For some types, the data is implicit or
empty. For others, the data is specified in \fort{BCData\_t} children.

If the locations at which the boundary conditions are to be applied
are specified in \fort{BCDataSet\_t}, using \fort{PointRange} or
\fort{PointList}, the structure function \fort{ListLength} is used.
Otherwise, the structure parameter \fort{ListLength} is required.

\textit{\uline{Node Attributes}}
\begin{Ventryic}{\textbf{Dimension Values:}}
\item [\textbf{Name:}]
      User defined
\item [\textbf{Label:}]
      \fort{BCDataSet\_t}
\item [\textbf{DataType:}]
      \fort{C1}
\item [\textbf{Dimension:}]
      1
\item [\textbf{Dimension Values:}]
      Length of string
\item [\textbf{Data:}]
      \fort{BCTypeSimple} value
\item [\textbf{Children:}]
      See \autoref{f:BCDataSet}
\item [\textbf{Cardinality:}]
      0,$N$
\item [\textbf{Functions:}]
      \fort{ListLength}
\item [\textbf{Parameters:}]
      \fort{ListLength}
\end{Ventryic}

\paragraph{\texttt{BC\_t}}

A \fort{BC\_t} node specifies a single boundary condition to be applied
on a single zone. It specifies the domain on which the condition is
to be applied and the equations to be enforced. All the \fort{BC\_t}
nodes for a single zone are found under that zone's \fort{ZoneBC\_t}
node. A \fort{ZoneBC\_t} node may have as many \fort{BC\_t} children
as are required to describe the physical boundary conditions on the
corresponding zone.

The domain on which the boundary condition is to be enforced is
specified by a single node of type either \fort{IndexRange\_t} or
\fort{IndexArray\_t}. The equations are specified in one or
more \fort{BCDataSet\_t} children.

The type of the boundary condition, which may be either simple or
compound, is specified in the data field.  For a complete description,
it is necessary to consult the SIDS.

\textit{\uline{Node Attributes}}
\begin{Ventryic}{\textbf{Dimension Values:}}
\item [\textbf{Name:}]
      User defined
\item [\textbf{Label:}]
      \fort{BC\_t}
\item [\textbf{DataType:}]
      \fort{C1}
\item [\textbf{Dimension:}]
      1
\item [\textbf{Dimension Values:}]
      Length of string
\item [\textbf{Data:}]
      \fort{BCType} value
\item [\textbf{Children:}]
      See \autoref{f:BC}
\item [\textbf{Cardinality:}]
      0,$N$
\item [\textbf{Parameters:}]
      \fort{IndexDimension}, \fort{PhysicalDimension}
\end{Ventryic}

\paragraph{\texttt{ZoneBC\_t}}

The \fort{ZoneBC\_t} node occurs at most once for each zone and serves
as the location under which all boundary conditions on that zone are
collected.

\textit{\uline{Node Attributes}}
\begin{Ventryic}{\textbf{Dimension Values:}}
\item [\textbf{Name:}]
      \fort{ZoneBC}
\item [\textbf{Label:}]
      \fort{ZoneBC\_t}
\item [\textbf{DataType:}]
      \fort{MT}
\item [\textbf{Children:}]
      See \autoref{f:ZoneBC}
\item [\textbf{Cardinality:}]
      0,1
\item [\textbf{Parameters:}]
      \fort{IndexDimension}, \fort{PhysicalDimension}
\end{Ventryic}

\paragraph{\texttt{BCProperty\_t}}

An optional \fort{BCProperty\_t} node may be used to record special
properties associated with particular boundary condition patches.

\textit{\uline{Node Attributes}}
\begin{Ventryic}{\textbf{Dimension Values:}}
\item [\textbf{Name:}]
      \fort{BCProperty}
\item [\textbf{Label:}]
      \fort{BCProperty\_t}
\item [\textbf{DataType:}]
      \fort{MT}
\item [\textbf{Children:}]
      See \autoref{f:BCProperty}
\item [\textbf{Cardinality:}]
      0,1
\end{Ventryic}

\paragraph{\texttt{WallFunction\_t}}

A \fort{WallFunction\_t} node may be used as a child of
\fort{BCProperty\_t} to record data associated with the use of wall
function boundary conditions.

\textit{\uline{Node Attributes}}
\begin{Ventryic}{\textbf{Dimension Values:}}
\item [\textbf{Name:}]
      \fort{WallFunction}
\item [\textbf{Label:}]
      \fort{WallFunction\_t}
\item [\textbf{DataType:}]
      \fort{MT}
\item [\textbf{Children:}]
      See \autoref{f:WallFunction}
\item [\textbf{Cardinality:}]
      0,1
\end{Ventryic}

\paragraph{\texttt{Area\_t}}

An \fort{Area\_t} node may be used as a child of \fort{BCProperty\_t}
to record data associated with area-related boundary conditions such as
bleed.

\textit{\uline{Node Attributes}}
\begin{Ventryic}{\textbf{Dimension Values:}}
\item [\textbf{Name:}]
      \fort{Area}
\item [\textbf{Label:}]
      \fort{Area\_t}
\item [\textbf{DataType:}]
      \fort{MT}
\item [\textbf{Children:}]
      See \autoref{f:Area}
\item [\textbf{Cardinality:}]
      0,1
\end{Ventryic}

\subsubsection{Equation Specification Group}

Nodes in this group serve to identify the physical model associated
with the data being recorded. Nearly always, the data is of enumeration
type and is selected from a collection of terms defined in detail in
the SIDS.  The names are largely self explanatory, and the detailed
definitions will not be repeated here. Numerical values associated with
the physical model depend on the type of modeling being chosen and are
generally stored in child nodes of type \fort{DataArray\_t}.

\paragraph{\texttt{GoverningEquations\_t}}

This node names the equation set being solved, for example,
\fort{FullPotential} or \fort{NSTurbulent}. If Navier-Stokes, the
diffusion terms retained may be specified in a \fort{DiffusionModel}
subnode.

\textit{\uline{Node Attributes}}
\begin{Ventryic}{\textbf{Dimension Values:}}
\item [\textbf{Name:}]
      \fort{GoverningEquations}
\item [\textbf{Label:}]
      \fort{GoverningEquations\_t}
\item [\textbf{DataType:}]
      \fort{C1}
\item [\textbf{Dimension:}]
      1
\item [\textbf{Dimension Values:}]
      Length of string
\item [\textbf{Data:}]
      \fort{GoverningEquationsType} value
\item [\textbf{Children:}]
      See \autoref{f:GoverningEquations}
\item [\textbf{Cardinality:}]
      0,1
\item [\textbf{Parameters:}]
      \fort{CellDimension}
\end{Ventryic}

\paragraph{\texttt{GasModel\_t}}

A node of type \fort{GasModel\_t} names the gas model used, for example,
\fort{Ideal} or \fort{VanderWaals}.

\textit{\uline{Node Attributes}}
\begin{Ventryic}{\textbf{Dimension Values:}}
\item [\textbf{Name:}]
      \fort{GasModel}
\item [\textbf{Label:}]
      \fort{GasModel\_t}
\item [\textbf{DataType:}]
      \fort{C1}
\item [\textbf{Dimension:}]
      1
\item [\textbf{Dimension Values:}]
      Length of string
\item [\textbf{Data:}]
      \fort{GasModelType} value
\item [\textbf{Children:}]
      See \autoref{f:GasModel}
\item [\textbf{Cardinality:}]
      0,1
\end{Ventryic}

\paragraph{\texttt{ViscosityModel\_t}}

A node of type \fort{ViscosityModel\_t} names the molecular viscosity
model used to relate the viscosity to the temperature, for example,
\fort{PowerLaw} or \fort{SutherlandLaw}.

\textit{\uline{Node Attributes}}
\begin{Ventryic}{\textbf{Dimension Values:}}
\item [\textbf{Name:}]
      \fort{ViscosityModel}
\item [\textbf{Label:}]
      \fort{ViscosityModel\_t}
\item [\textbf{DataType:}]
      \fort{C1}
\item [\textbf{Dimension:}]
      1
\item [\textbf{Dimension Values:}]
      Length of string
\item [\textbf{Data:}]
      \fort{ViscosityModelType} value
\item [\textbf{Children:}]
      See \autoref{f:ViscosityModel}
\item [\textbf{Cardinality:}]
      0,1
\end{Ventryic}

\paragraph{\texttt{EquationDimension}}

A node named \fort{EquationDimension}, of type \fort{int[]}, gives the
number of dependent variables required for a complete solution
description, or the number of equations being solved. For example, for
\fort{NSTurbulent} with the $k$-$\epsilon$ turbulence model in three
dimensions, it is 7.

\textit{\uline{Node Attributes}}
\begin{Ventryic}{\textbf{Dimension Values:}}
\item [\textbf{Name:}]
      \fort{EquationDimension}
\item [\textbf{Label:}]
      ``\fort{int}''
\item [\textbf{DataType:}]
      \fort{I4}
\item [\textbf{Dimension:}]
      1
\item [\textbf{Dimension Values:}]
      1
\item [\textbf{Data:}]
      \fort{EquationDimension} value
\item [\textbf{Children:}]
      None
\item [\textbf{Cardinality:}]
      0,1
\end{Ventryic}

\paragraph{\texttt{ThermalConductivityModel\_t}}

A node of type \fort{ThermalConductivityModel\_t} names the model used
to relate the thermal conductivity to the temperature, for example,
\fort{ConstantPrandtl}, \fort{PowerLaw}, or \fort{SutherlandLaw}. These
closely parallel the viscosity model.

\textit{\uline{Node Attributes}}
\begin{Ventryic}{\textbf{Dimension Values:}}
\item [\textbf{Name:}]
      \fort{ThermalConductivityModel}
\item [\textbf{Label:}]
      \fort{ThermalConductivityModel\_t}
\item [\textbf{DataType:}]
      \fort{C1}
\item [\textbf{Dimension:}]
      1
\item [\textbf{Dimension Values:}]
      Length of string
\item [\textbf{Data:}]
      \fort{ThermalConductivityModelType} value
\item [\textbf{Children:}]
      See \autoref{f:ThermalConductivityModel}
\item [\textbf{Cardinality:}]
      0,1
\end{Ventryic}

\paragraph{\texttt{TurbulenceClosure\_t}}

\enlargethispage{\baselineskip}%
A node of type \fort{TurbulenceClosure\_t} names the method of closing
the Reynolds stress equations when the governing equations are
turbulent, for example, \fort{EddyViscosity} or
\fort{ReynoldsStressAlgebraic}.

\textit{\uline{Node Attributes}}
\begin{Ventryic}{\textbf{Dimension Values:}}
\item [\textbf{Name:}]
      \fort{TurbulenceClosure}
\item [\textbf{Label:}]
      \fort{TurbulenceClosure\_t}
\item [\textbf{DataType:}]
      \fort{C1}
\item [\textbf{Dimension:}]
      1
\item [\textbf{Dimension Values:}]
      Length of string
\item [\textbf{Data:}]
      \fort{TurbulenceClosureType} value
\item [\textbf{Children:}]
      See \autoref{f:TurbulenceClosure}
\item [\textbf{Cardinality:}]
      0,1
\end{Ventryic}

\paragraph{\texttt{TurbulenceModel\_t}}

A node of type \fort{TurbulenceModel\_t} names the equation
set used to model the turbulence quantities, for example,
\fort{Algebraic\_Bald\-winLomax} or \fort{OneEquation\_Spa\-lartAll\-maras}.

\textit{\uline{Node Attributes}}
\begin{Ventryic}{\textbf{Dimension Values:}}
\item [\textbf{Name:}]
      \fort{TurbulenceModel}
\item [\textbf{Label:}]
      \fort{TurbulenceModel\_t}
\item [\textbf{DataType:}]
      \fort{C1}
\item [\textbf{Dimension:}]
      1
\item [\textbf{Dimension Values:}]
      Length of string
\item [\textbf{Data:}]
      \fort{TurbulenceModelType} value
\item [\textbf{Children:}]
      See \autoref{f:TurbulenceModel}
\item [\textbf{Cardinality:}]
      0,1
\item [\textbf{Parameters:}]
      \fort{CellDimension}
\end{Ventryic}

\paragraph{\texttt{ThermalRelaxationModel\_t}}

A node of type \fort{ThermalRelaxationModel\_t} names the equation
set used to model the thermal relaxation quantities, for example,
\fort{Frozen} or \fort{ThermalEquilib}.

\textit{\uline{Node Attributes}}
\begin{Ventryic}{\textbf{Dimension Values:}}
\item [\textbf{Name:}]
      \fort{ThermalRelaxationModel}
\item [\textbf{Label:}]
      \fort{ThermalRelaxationModel\_t}
\item [\textbf{DataType:}]
      \fort{C1}
\item [\textbf{Dimension:}]
      1
\item [\textbf{Dimension Values:}]
      Length of string
\item [\textbf{Data:}]
      \fort{ThermalRelaxationModelType} value
\item [\textbf{Children:}]
      See \autoref{f:ThermalRelaxationModel}
\item [\textbf{Cardinality:}]
      0,1
\end{Ventryic}

\paragraph{\texttt{ChemicalKineticsModel\_t}}

A node of type \fort{ChemicalKineticsModel\_t} names the equation
set used to model the chemical kinetics quantities, for example,
\fort{Frozen} or \fort{ChemicalEquilibCurveFit}.

\textit{\uline{Node Attributes}}
\begin{Ventryic}{\textbf{Dimension Values:}}
\item [\textbf{Name:}]
      \fort{ChemicalKineticsModel}
\item [\textbf{Label:}]
      \fort{ChemicalKineticsModel\_t}
\item [\textbf{DataType:}]
      \fort{C1}
\item [\textbf{Dimension:}]
      1
\item [\textbf{Dimension Values:}]
      Length of string
\item [\textbf{Data:}]
      \fort{ChemicalKineticsModelType} value
\item [\textbf{Children:}]
      See \autoref{f:ChemicalKineticsModel}
\item [\textbf{Cardinality:}]
      0,1
\end{Ventryic}

\paragraph{\texttt{EMElectricFieldModel\_t}}

A node of type \fort{EMElectricFieldModel\_t} names the electric
field model used for electromagnetic flows, for example,
\fort{Constant} or \fort{Voltage}.

\textit{\uline{Node Attributes}}
\begin{Ventryic}{\textbf{Dimension Values:}}
\item [\textbf{Name:}]
      \fort{EMElectricFieldModel}
\item [\textbf{Label:}]
      \fort{EMElectricFieldModel\_t}
\item [\textbf{DataType:}]
      \fort{C1}
\item [\textbf{Dimension:}]
      1
\item [\textbf{Dimension Values:}]
      Length of string
\item [\textbf{Data:}]
      \fort{EMElectricFieldModelType} value
\item [\textbf{Children:}]
      See \autoref{f:EMElectricFieldModel}
\item [\textbf{Cardinality:}]
      0,1
\end{Ventryic}

\paragraph{\texttt{EMMagneticFieldModel\_t}}

A node of type \fort{EMMagneticFieldModel\_t} names the magnetic
field model used for electromagnetic flows, for example,
\fort{Constant} or \fort{Interpolated}.

\textit{\uline{Node Attributes}}
\begin{Ventryic}{\textbf{Dimension Values:}}
\item [\textbf{Name:}]
      \fort{EMMagneticFieldModel}
\item [\textbf{Label:}]
      \fort{EMMagneticFieldModel\_t}
\item [\textbf{DataType:}]
      \fort{C1}
\item [\textbf{Dimension:}]
      1
\item [\textbf{Dimension Values:}]
      Length of string
\item [\textbf{Data:}]
      \fort{EMMagneticFieldModelType} value
\item [\textbf{Children:}]
      See \autoref{f:EMMagneticFieldModel}
\item [\textbf{Cardinality:}]
      0,1
\end{Ventryic}

\paragraph{\texttt{EMConductivityModel\_t}}

A node of type \fort{EMConductivityModel\_t} names the conductivity
model used for electromagnetic flows, for example,
\fort{Constant} or \fort{Equilibrium\_LinRessler}.

\textit{\uline{Node Attributes}}
\begin{Ventryic}{\textbf{Dimension Values:}}
\item [\textbf{Name:}]
      \fort{EMConductivityModel}
\item [\textbf{Label:}]
      \fort{EMConductivityModel\_t}
\item [\textbf{DataType:}]
      \fort{C1}
\item [\textbf{Dimension:}]
      1
\item [\textbf{Dimension Values:}]
      Length of string
\item [\textbf{Data:}]
      \fort{EMConductivityModelType} value
\item [\textbf{Children:}]
      See \autoref{f:EMConductivityModel}
\item [\textbf{Cardinality:}]
      0,1
\end{Ventryic}

\paragraph{\texttt{FlowEquationSet\_t}}

A node of type \fort{FlowEquationSet\_t} appears either at the highest
level of the tree (under \fort{CGNSBase\_t}), to indicate the equation
set whose solution is recorded throughout the database, or below a
\fort{Zone\_t} node, to indicate the set of equations solved in that
zone. The usual convention applies, i.e., specifications at the local
(zone) level override global specifications.

\textit{\uline{Node Attributes}}
\begin{Ventryic}{\textbf{Dimension Values:}}
\item [\textbf{Name:}]
      \fort{FlowEquationSet}
\item [\textbf{Label:}]
      \fort{FlowEquationSet\_t}
\item [\textbf{DataType:}]
      \fort{MT}
\item [\textbf{Children:}]
      See \autoref{f:FlowEquationSet}
\item [\textbf{Cardinality:}]
      0,1
\item [\textbf{Parameters:}]
      \fort{CellDimension}
\end{Ventryic}

\subsubsection{Family Group}

Because there is rarely a 1-to-1 connection between mesh regions and
geometric entities, it is often desirable to set geometric associations
indirectly in a CGNS file.
That is, rather than setting the geometry data for each mesh entity
(nodes, edges, and faces), it's useful to associate them with
intermediate objects.
The intermediate objects are in turn linked to nodal regions of the
computational mesh.
This intermediate object is defined as a \emph{CFD family}.

Each mesh surface may linked to the geometric entities of one or more CAD
databases by a user-defined CFD family name.
The CFD family corresponds to one or more CAD geometric entities on which
the mesh face is projected.
Each one of these geometric entities is described in a CAD file and is
not redefined within the CGNS file.

\paragraph{\texttt{Family\_t}}

This node, a child of the \fort{CGNSBase\_t} node, contains the
definition of a single CFD family.
Multiple \texttt{Family\_t} nodes are allowed.

\textit{\uline{Node Attributes}}
\begin{Ventryic}{\textbf{Dimension Values:}}
\item [\textbf{Name:}]
      User defined
\item [\textbf{Label:}]
      \fort{Family\_t}
\item [\textbf{DataType:}]
      \fort{MT}
\item [\textbf{Children:}]
      See \autoref{f:Family}
\item [\textbf{Cardinality:}]
      0,$N$
\end{Ventryic}

\paragraph{\texttt{FamilyName\_t}}

This node is used to identify a family to which
a particular zone or boundary belongs.
Note that the name of the family is defined by the ``Name''
of the \fort{Family\_t} node, and is stored as data in the
\fort{FamilyName\_t} node.

\textit{\uline{Node Attributes}}
\begin{Ventryic}{\textbf{Dimension Values:}}
\item [\textbf{Name:}]
      \fort{FamilyName}
\item [\textbf{Label:}]
      \fort{FamilyName\_t}
\item [\textbf{DataType:}]
      \fort{C1}
\item [\textbf{Dimension:}]
      1
\item [\textbf{Dimension Values:}]
      Length of string
\item [\textbf{Data:}]
      Name of CFD family
\item [\textbf{Children:}]
      None
\item [\textbf{Cardinality:}]
      0,1
\end{Ventryic}

\paragraph{\texttt{FamilyBC\_t}}

This node contains a boundary condition type for a
particular CFD family.

\textit{\uline{Node Attributes}}
\begin{Ventryic}{\textbf{Dimension Values:}}
\item [\textbf{Name:}]
      \fort{FamilyBC}
\item [\textbf{Label:}]
      \fort{FamilyBC\_t}
\item [\textbf{DataType:}]
      \fort{C1}
\item [\textbf{Dimension:}]
      1
\item [\textbf{Dimension Values:}]
      Length of string
\item [\textbf{Data:}]
      \fort{BCType} value
\item [\textbf{Children:}]
      See \autoref{f:FamilyBC}
\item [\textbf{Cardinality:}]
      0,1
\end{Ventryic}

\paragraph{\texttt{GeometryReference\_t}}

\texttt{GeometryReference\_t} nodes are used to associate
a CFD family with one or more CAD databases.

\textit{\uline{Node Attributes}}
\begin{Ventryic}{\textbf{Dimension Values:}}
\item [\textbf{Name:}]
      User defined
\item [\textbf{Label:}]
      \fort{GeometryReference\_t}
\item [\textbf{DataType:}]
      \fort{MT}
\item [\textbf{Children:}]
      See \autoref{f:GeometryReference}
\item [\textbf{Cardinality:}]
      0,$N$
\end{Ventryic}

\paragraph{\texttt{GeometryFile\_t}}

This node contains the name of the CAD geometry file.

\textit{\uline{Node Attributes}}
\begin{Ventryic}{\textbf{Dimension Values:}}
\item [\textbf{Name:}]
      \fort{GeometryFile}
\item [\textbf{Label:}]
      \fort{GeometryFile\_t}
\item [\textbf{DataType:}]
      \fort{C1}
\item [\textbf{Dimension:}]
      1
\item [\textbf{Dimension Values:}]
      Length of string
\item [\textbf{Data:}]
      Name of the CAD geometry file
\item [\textbf{Children:}]
      None
\item [\textbf{Cardinality:}]
      1
\end{Ventryic}

\paragraph{\texttt{GeometryFormat\_t}}

This enumeration node defines the format of the CAD geometry file.

\textit{\uline{Node Attributes}}
\begin{Ventryic}{\textbf{Dimension Values:}}
\item [\textbf{Name:}]
      \fort{GeometryFormat}
\item [\textbf{Label:}]
      \fort{GeometryFormat\_t}
\item [\textbf{DataType:}]
      \fort{C1}
\item [\textbf{Dimension:}]
      1
\item [\textbf{Dimension Values:}]
      Length of string
\item [\textbf{Data:}]
      Name of the CAD geometry format
\item [\textbf{Children:}]
      None
\item [\textbf{Cardinality:}]
      1
\end{Ventryic}

\paragraph{\texttt{GeometryEntity\_t}}

\texttt{GeometryEntity\_t} nodes define the names of the
entities in CAD geometry file that make up a CFD family.

\textit{\uline{Node Attributes}}
\begin{Ventryic}{\textbf{Dimension Values:}}
\item [\textbf{Name:}]
      User defined
\item [\textbf{Label:}]
      \fort{GeometryEntity\_t}
\item [\textbf{DataType:}]
      \fort{MT}
\item [\textbf{Children:}]
      None
\item [\textbf{Cardinality:}]
      0,$N$
\end{Ventryic}

\subsubsection{Time-Dependent Group}
\label{s:timedep}

Nodes in this section are used for information related to time-dependent
flows, and include specification of grid motion and storage of
time-dependent or iterative data.

\paragraph{\texttt{BaseIterativeData\_t}}

Located directly under the \fort{CGNSBase\_t} node, the
\fort{BaseIterativeData\_t} node contains information about the number
of time steps or iterations being recorded, and the time and/or
iteration values at each step.
In addition, it may include the list of zones and families for each step
of the simulation, if these vary throughout the simulation.

\textit{\uline{Node Attributes}}
\begin{Ventryic}{\textbf{Dimension Values:}}
\item [\textbf{Name:}]
      User defined
\item [\textbf{Label:}]
      \fort{BaseIterativeData\_t}
\item [\textbf{DataType:}]
      \fort{I4}
\item [\textbf{Dimension:}]
      1
\item [\textbf{Dimension Values:}]
      1
\item [\textbf{Data:}]
      \fort{NumberOfSteps}
\item [\textbf{Children:}]
      See \autoref{f:BaseIterativeData}
\item [\textbf{Cardinality:}]
      0,1
\end{Ventryic}

\paragraph{\texttt{ZoneIterativeData\_t}}

The \fort{ZoneIterativeData\_t} node is a child of the \fort{Zone\_t}
node, and is used to store pointers to zonal data for each recorded
step of the simulation.

\textit{\uline{Node Attributes}}
\begin{Ventryic}{\textbf{Dimension Values:}}
\item [\textbf{Name:}]
      User defined
\item [\textbf{Label:}]
      \fort{ZoneIterativeData\_t}
\item [\textbf{DataType:}]
      \fort{MT}
\item [\textbf{Children:}]
      See \autoref{f:ZoneIterativeData}
\item [\textbf{Cardinality:}]
      0,1
\item [\textbf{Parameters:}]
      \fort{NumberOfSteps}
\end{Ventryic}

\paragraph{\texttt{RigidGridMotion\_t}}

\fort{RigidGridMotion\_t} nodes are used to store data
defining rigid translation and/or rotation of the grid coordinates.
Multiple \fort{RigidGridMotion\_t} nodes may be associated with different
iterations or time steps in the computation.
This association is recorded under the \fort{ZoneIterativeData\_t} node.

\textit{\uline{Node Attributes}}
\begin{Ventryic}{\textbf{Dimension Values:}}
\item [\textbf{Name:}]
      User defined
\item [\textbf{Label:}]
      \fort{RigidGridMotion\_t}
\item [\textbf{DataType:}]
      \fort{C1}
\item [\textbf{Dimension:}]
      1
\item [\textbf{Dimension Values:}]
      Length of string
\item [\textbf{Data:}]
      \fort{RigidGridMotionType} value
\item [\textbf{Children:}]
      See \autoref{f:RigidGridMotion}
\item [\textbf{Cardinality:}]
      0,$N$
\end{Ventryic}

\paragraph{\texttt{ArbitraryGridMotion\_t}}

\fort{ArbitraryGridMotion\_t} nodes are used to store grid
velocities for each grid point in a zone (i.e., for deforming grids).
Multiple \fort{ArbitraryGridMotion\_t} nodes may be associated with
different iterations or time steps in the computation.
This association is recorded under the \fort{ZoneIterativeData\_t} node.

Note that instantaneous grid coordinates at different iterations or time
steps may be recorded using multiple \fort{GridCoordinates\_t} nodes
under a \fort{Zone\_t} node.

\textit{\uline{Node Attributes}}
\begin{Ventryic}{\textbf{Dimension Values:}}
\item [\textbf{Name:}]
      User defined
\item [\textbf{Label:}]
      \fort{ArbitraryGridMotion\_t}
\item [\textbf{DataType:}]
      \fort{C1}
\item [\textbf{Dimension:}]
      1
\item [\textbf{Dimension Values:}]
      Length of string
\item [\textbf{Data:}]
      \fort{ArbitraryGridMotionType} value
\item [\textbf{Children:}]
      See \autoref{f:ArbitraryGridMotion}
\item [\textbf{Cardinality:}]
      0,$N$
\item [\textbf{Parameters:}]
      \fort{IndexDimension}, \fort{VertexSize}, \fort{CellSize}
\item [\textbf{Functions:}]
      \fort{DataSize}
\end{Ventryic}

\subsubsection{Structural Nodes}

In this section we describe the highest levels of the hierarchy. Nodes
in this section store only quantities which refer to all the entities
below them. Therir primary function is to provide organization to the
data below.

\paragraph{\texttt{Zone\_t}}
\label{s:Zone}

Directly below the highest level node in the database, which is
by definition of type \fort{CGNSBase\_t}, are found nodes of type
\fort{Zone\_t} providing entry into the data specific to each
zone. There are as many \fort{Zone\_t} nodes as there are zones. Their
children, in turn, record grid, field, connectivity, and boundary
conditions, and a variety of auxiliary data.

\textit{\uline{Node Attributes}}
\begin{Ventryic}{\textbf{Dimension Values:}}
\item [\textbf{Name:}]
      User defined
\item [\textbf{Label:}]
      \fort{Zone\_t}
\item [\textbf{DataType:}]
      \fort{I4}
\item [\textbf{Dimension:}]
      2
\item [\textbf{Dimension Values:}]
      \fort{IndexDimension}, 3
\item [\textbf{Data:}]
      \fort{VertexSize[IndexDimension]}, \fort{CellSize[IndexDimension]},
      \fort{VertexSizeBoundary[IndexDimension]}
\item [\textbf{Children:}]
      See \autoref{f:Zone}
\item [\textbf{Cardinality:}]
      0,$N$
\item [\textbf{Parameters:}]
      \fort{CellDimension}, \fort{PhysicalDimension}
\end{Ventryic}

\paragraph{\texttt{CGNSBase\_t}}

The \fort{CGNSBase\_t} node is by definition the highest level node in
the database, and is located directly below the root node\footnote{This
root node depends on the underlaying physical layer.
It is known as the \texttt{ADF Mother node} with the ADF system, and as
\texttt{HDF Mother node} with \HDF.}.
It provides entry into all other data.
Multiple \fort{CGNSBase\_t} nodes are allowed in a \SLL~  file.
The particular database being accessed is determined by the name
of the \fort{CGNSBase\_t} node.

The only data stored in the node itself are \fort{CellDimension}, the
dimensionality of a cell in the mesh (i.e., 3 for a volume cell and 2
for a face cell), and \fort{PhysicalDimension}, the number of indices
required to specify a unique physical location in the field data being
recorded. However, a variety of global information concerning the entire
database may be stored in children of the \fort{CGNSBase\_t} node. In
particular, a \fort{Descriptor\_t} node at this level can store user
commentary on the entire history of the development of the database.

Other information typically stored directly below the \fort{CGNSBase\_t}
node includes convergence histories, reference states, dimensional
units, integrated quantities, and information on the flow equations
being solved.

\textit{\uline{Node Attributes}}
\begin{Ventryic}{\textbf{Dimension Values:}}
\item [\textbf{Name:}]
      User defined
\item [\textbf{Label:}]
      \fort{CGNSBase\_t}
\item [\textbf{DataType:}]
      \fort{I4}
\item [\textbf{Dimension:}]
      1
\item [\textbf{Dimension Values:}]
      2
\item [\textbf{Data:}]
      \fort{CellDimension}, \fort{PhysicalDimension}
\item [\textbf{Children:}]
      See \autoref{f:CGNSBase}
\item [\textbf{Cardinality:}]
      0,$N$
\end{Ventryic}

\paragraph{\texttt{SimulationType\_t}}

This enumeration-type node is a child of the \fort{CGNSBase\_t} node,
and specifies whether or not the data below \fort{CGNSBase\_t} is
time-accurate.
Nodes for describing time-dependent data are presented in
\autoref{s:timedep}.

\textit{\uline{Node Attributes}}
\begin{Ventryic}{\textbf{Dimension Values:}}
\item [\textbf{Name:}]
      \fort{SimulationType}
\item [\textbf{Label:}]
      \fort{SimulationType\_t}
\item [\textbf{DataType:}]
      \fort{C1}
\item [\textbf{Dimension:}]
      1
\item [\textbf{Dimension Values:}]
      Length of string
\item [\textbf{Data:}]
      \fort{TimeAccurate} or \fort{NonTimeAccurate}
\item [\textbf{Children:}]
      None
\item [\textbf{Cardinality:}]
      0,1
\end{Ventryic}

\paragraph{\texttt{ZoneType\_t}}

This enumeration-type node is a required child of the \fort{Zone\_t}
node, and specifies whether the grid in that zone is structured or
unstructured.

\textit{\uline{Node Attributes}}
\begin{Ventryic}{\textbf{Dimension Values:}}
\item [\textbf{Name:}]
      \fort{ZoneType}
\item [\textbf{Label:}]
      \fort{ZoneType\_t}
\item [\textbf{DataType:}]
      \fort{C1}
\item [\textbf{Dimension:}]
      1
\item [\textbf{Dimension Values:}]
      Length of string
\item [\textbf{Data:}]
      \fort{Structured} or \fort{Unstructured}
\item [\textbf{Children:}]
      None
\item [\textbf{Cardinality:}]
      1
\end{Ventryic}

\paragraph{\texttt{CGNSLibraryVersion\_t}}

A \SLL~ file containing a CGNS database also contains, directly below the
\SLL~ root node, a \fort{CGNSLibraryVersion\_t} node.
This node contains the version number of the CGNS standard with which
the file is consistent, and is created automatically when the file is
created or modified using the CGNS Mid-Level Library.
Note that this node is not actually part of the CGNS database, since
it is not located below a \fort{CGNSBase\_t} node.

Note that a \SLL~ file may contain multiple CGNS databases, but there is
only one \fort{CGNSLibraryVersion\_t} node.
It is assumed that the version number in the \fort{CGNSLibraryVersion\_t}
node is applicable to all the CGNS databases in the file.

Note also that some CGNS nodes may actually be links to CGNS nodes in
other files.
In this case, it is assumed that the \fort{CGNSLibraryVersion\_t} node
in the ``top-level'' file is applicable to the file(s) containing the
linked nodes.

\textit{\uline{Node Attributes}}
\begin{Ventryic}{\textbf{Dimension Values:}}
\item [\textbf{Name:}]
      \fort{CGNSLibraryVersion}
\item [\textbf{Label:}]
      \fort{CGNSLibraryVersion\_t}
\item [\textbf{DataType:}]
      \fort{R4}
\item [\textbf{Dimension:}]
      1
\item [\textbf{Dimension Values:}]
      1
\item [\textbf{Data:}]
      CGNS version number
\item [\textbf{Children:}]
      None
\item [\textbf{Cardinality:}]
      1
\end{Ventryic}
