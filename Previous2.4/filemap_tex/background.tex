\section{CGNS Background}
\label{s:background}
\thispagestyle{plain}

The information in this section is supplied for the sake of
completeness. It is identical with the information found in the
Overview.

\subsection{Purpose}

The purpose of CGNS is to provide a standard for recording and
recovering computer data associated with the numerical solution of the
equations of fluid dynamics. The format implemented by this standard is
(1) general, (2) portable, (3) expandable, and (4) durable.

The CGNS system consists of a collection of conventions, and software
implementing those conventions, for the storage and retrieval of
CFD (computational fluid dynamics) data. The system consists of two
parts: (1) a standard format for recording the data, and (2) software
that reads, writes and modifies data in that format.  The format is a
conceptual entity established by the documentation; the software is a
physical product supplied to enable developers to access and produce
data recorded in that format. The CGNS standard, applied through the use
of the supplied software, is intended to do the following:

\begin{itemize*}
\item facilitate the exchange of CFD data
      \begin{itemize*}
      \item between sites.
      \item between applications codes.
      \item across computing platforms.
      \end{itemize*}
\item stabilize the archiving of CFD data.
\end{itemize*}

\subsection{Participation and Brief History}

The CGNS project originated around 1994--1995 through a series of
meetings between Boeing and NASA that addressed improved means for
transferring NASA technology to industrial use. It was held that
a principal impediment to technology transfer was the disparity
in I/O formats employed by various flow codes, grid generators,
and so forth. The CGNS system was conceived as a means to promote
``plug-and-play'' CFD.

Agreement was reached to develop CGNS at Boeing, under NASA Contract
NAS1-20267, with active participation by a team of CFD researchers from

\begin{itemize*}
\item NASA Langley Research Center
\item NASA Glenn Research Center
\item NASA Ames Research Center
\item Boeing St-Louis (McDonnell-Douglas Corporation).
\item Boeing Commercial Airplane Group Aerodynamics
\item Boeing Commercial Airplane Group Propulsion
\item ICEM CFD Engineering Corporation of Berkeley, California
\end{itemize*}
Also participating in the discussions at various times have been
researchers from
\begin{itemize*}
\item Defense and Space Group and Environmental Systems
\item Arnold Engineering Development Center, representing the NPARC Alliance
\item Wright-Patterson Air Force Base
\end{itemize*}

\subsection{Scope}

The principal target of CGNS is the data normally associated with
compressible viscous flow (i.e., the Navier-Stokes equations), but the
standard is also applicable to subclasses such as Euler and potential
flows.

CGNS Version 1.0, released 5/15/98, was limited to problems described
by multiblock structured grids. Version 1.1 addresses grids,
flowfields, boundary conditions, and block-to-block connection
information. Also included are a number of auxiliary items,
including nondimensionalization, reference state, and equation set
specifications. The extension to time-dependent flows and unstructured
grids is addressed in Version 2. Also included are links between CGNS
data and CAD geometry. Any mix of the following types of field data can
be recorded:

\begin{itemize*}
\item nodal.
\item edge-centered.
\item face-centered.
\item cell-centered.
\end{itemize*}
Block connections can be of the following types:
\begin{itemize*}
\item contiguous (one-to-one).
\item abutting (patched mismatched).
\item overlapping (Chimera).
\end{itemize*}

Much of the standard and the software is applicable to computational
field physics in general. Disciplines other than fluid dynamics would
need to augment the data definitions and storage conventions, but the
fundamental database software, which provides platform independence, is
not specific to fluid dynamics.
